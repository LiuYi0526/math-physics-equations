\chapter{调和方程}

\section{建立方程,定解条件}

\begin{exercise}
  设 $u(x_1,\cdots,x_n)=f(r)$ (其中 $r=\sqrt{x_1^2+\cdots+x_n^2}$)
  是 $n$ 维调和函数 $\Bigl( \text{即满足方程\ }
  \frac{\partial^2u}{\partial x_1^2}+\cdots+\frac{\partial^2u}{\partial x_n^2}=0\Bigr)$,
  试证明
  \[f(r) = c_1+\frac{c_2}{r^{n-2}}\quad (n\geq 2),\]
  \[f(r) = c_1+c_2\ln\frac{1}{r}\quad (n=2),\]
  其中 $c_1$, $c_2$ 为任意常数.
\end{exercise}

\begin{proof}
  因为 $\frac{\partial u}{\partial x_i}=\frac{\diff f}{\diff r}\frac{x_i}{r}$,所以
  \[\frac{\partial^2u}{\partial x_i^2} =
    \frac{\diff^2f}{\diff r^2}\frac{x_i^2}{r^2}
    + \frac{\diff f}{\diff r}\frac{r^2-x_i^2}{r^3},\]
  故
  \[\sum_{i=1}^n\frac{\partial^2u}{\partial x_i^2}
    = \frac{\diff^2f}{\diff r^2}+\frac{\diff f}{\diff r}\frac{n-1}{r}
    = f''(r)+\frac{n-1}{r}f'(r)=0.\]
  由上式得
  \[f'=cr^{1-n}.\]
  \begin{enumerate}[(i)]
    \item $n = 2$ 时, $f(r)=c_1+c_2\ln r=c_1+c_2\ln\frac{1}{r}$,
    \item $n\neq 2$ 时, $f(r)=c_1+\frac{c_2}{2-n}r^{2-n}=c_1+\frac{c_2}{r^{n-2}}$,
  \end{enumerate}
  其中 $c_1$, $c_2$ 为任意常数.
\end{proof}


\begin{exercise}
  证明: 拉普拉斯算子在球坐标$(r,\theta,\varphi)$下可以写成
  \[\Delta u
    = \frac{1}{r^2}\frac{\partial}{\partial r}\left(r^2\frac{\partial u}{\partial r}\right)
    + \frac{1}{r^2\sin\theta}\frac{\partial}{\partial\theta}
      \left(\sin\theta\frac{\partial u}{\partial\theta}\right)
    +\frac{1}{r^2\sin^2\theta}\frac{\partial^2u}{\partial\varphi^2}.\]
\end{exercise}

\begin{proof}
  球坐标变换及其逆变换为
  \[\begin{cases}
    x = r\sin\theta\cos\varphi, \\
    y = r\sin\theta\sin\varphi, \\
    z = r\cos\theta;
  \end{cases}\Rightarrow
  \begin{cases}
    r = \sqrt{x^2+y^2+z^2}, \\
    \theta = \arccos\frac{z}{\sqrt{x^2+y^2+z^2}}, \\
    \varphi = \arctan\frac{y}{x}.
  \end{cases}\]
  通过链式法则可得
  \[\frac{\partial u}{\partial x}
    = \frac{\partial u}{\partial r}\frac{\partial r}{\partial x}
      + \frac{\partial u}{\partial\theta}\frac{\partial\theta}{\partial x}
      + \frac{\partial u}{\partial\varphi}\frac{\partial\varphi}{\partial x}.\]
  \[\frac{\partial^2u}{\partial x^2}
    = \frac{\partial^2u}{\partial r^2}\left(\frac{\partial r}{\partial x}\right)^2
      + \frac{\partial u}{\partial r}\frac{\partial^2r}{\partial x^2}
      + \frac{\partial^2u}{\partial\theta^2}\left(\frac{\partial\theta}{\partial x}\right)^2
      + \frac{\partial u}{\partial\theta}\frac{\partial^2\theta}{\partial x^2}
      + \frac{\partial^2u}{\partial\varphi^2}\left(\frac{\partial\varphi}{\partial x}\right)^2
      + \frac{\partial u}{\partial\varphi}\frac{\partial^2\varphi}{\partial x^2}.\]
  由逆变换公式求得 (所有求导项并未在下面完全列出, 因为很多项的形式是一样的)
  \[\frac{\partial r}{\partial x}
    = \frac{x}{r},\quad
    \frac{\partial^2r}{\partial x^2} = \frac{r^2-x^2}{r^3}.\]
  \[\frac{\partial\theta}{\partial x}
    = \frac{zx}{r^2\sqrt{x^2+y^2}},\quad
    \frac{\partial\theta}{\partial z}
    = \frac{-\sqrt{x^2+y^2}}{r^2}.\]
  \[\frac{\partial^2\theta}{\partial x^2}
    = \frac{zr^2y^2-2zx^2(x^2+y^2)}{r^4(x^2+y^2)^{3/2}},\quad
    \frac{\partial^2\theta}{\partial z^2}=\frac{2z\sqrt{x^2+y^2}}{r^4}\]
  \[\frac{\partial\varphi}{\partial x} = \frac{-y}{x^2+y^2},\quad
    \frac{\partial\varphi}{\partial y} = \frac{x}{x^2+y^2},\quad
    \frac{\partial\varphi}{\partial z}=0.\]
  \[\frac{\partial^2\varphi}{\partial x^2} = \frac{2xy}{(x^2+y^2)^2},\quad
    \frac{\partial^2\varphi}{\partial y^2}=\frac{-2xy}{(x^2+y^2)^2}.\]
  故
  \[\begin{split}
  \Delta u
    & = \frac{\partial^2u}{\partial r^2}\left[\left(\frac{\partial r}{\partial x}\right)^2
        + \left(\frac{\partial r}{\partial y}\right)^2
        + \left(\frac{\partial r}{\partial z}\right)^2\right]
        + \frac{\partial u}{\partial r}\left(\frac{\partial^2r}{\partial x^2}
        + \frac{\partial^2r}{\partial y^2}
        + \frac{\partial^2r}{\partial z^2}\right) \\
    & + \frac{\partial^2u}{\partial\theta^2}
        \left[\left(\frac{\partial\theta}{\partial x}\right)^2
        + \left(\frac{\partial\theta}{\partial y}\right)^2
        + \left(\frac{\partial\theta}{\partial z}\right)^2\right]
        + \frac{\partial u}{\partial\theta}\left(\frac{\partial^2\theta}{\partial x^2}
        + \frac{\partial^2\theta}{\partial y^2}+\frac{\partial^2\theta}{\partial z^2}\right) \\
    & + \frac{\partial^2u}{\partial\varphi^2}
        \left[\left(\frac{\partial\varphi}{\partial x}\right)^2
        + \left(\frac{\partial\varphi}{\partial y}\right)^2
        + \left(\frac{\partial\varphi}{\partial z}\right)^2\right]
        + \frac{\partial u}{\partial\varphi}\left(\frac{\partial^2\varphi}{\partial x^2}
        + \frac{\partial^2\varphi}{\partial y^2}
        + \frac{\partial^2\varphi}{\partial z^2}\right) \\
    & = \frac{\partial^2u}{\partial r^2}+\frac{\partial u}{\partial r}\frac{2}{r}
        + \frac{\partial^2u}{\partial\theta^2}\frac{1}{r^2}
        + \frac{\partial u}{\partial\theta}\frac{z}{r^2\sqrt{x^2+y^2}}
        + \frac{\partial^2u}{\partial\varphi^2}\frac{1}{x^2+y^2} \\
    & = \frac{1}{r^2}\frac{\partial}{\partial r}\left(r^2\frac{\partial u}{\partial r}\right)
        + \frac{1}{r^2\sin\theta}\frac{\partial}{\partial\theta}
        \left(\sin\theta\frac{\partial u}{\partial\theta}\right)
        + \frac{1}{r^2\sin^2\theta}\frac{\partial^2u}{\partial\varphi^2}.
  \end{split}\]
  证毕.
\end{proof}



\begin{exercise}
  证明: 拉普拉斯算子在柱坐标$(r,\theta,z)$下可以写成
  \[\Delta u
    = \frac{1}{r}\frac{\partial}{\partial r}\left(r\frac{\partial u}{\partial r}\right)
    + \frac{1}{r^2}\frac{\partial^2u}{\partial\theta^2}+\frac{\partial^2u}{\partial z^2}.\]
\end{exercise}

\begin{proof}
  柱坐标变换为
  \[\begin{cases}
    x = r\cos\theta, \\
    y = r\sin\theta, \\
    z = z.
  \end{cases}\]
  故
  \[\frac{\partial u}{\partial r}
    = \frac{\partial u}{\partial x}\cos\theta
      + \frac{\partial u}{\partial y}\sin\theta,\quad
    \frac{\partial u}{\partial\theta}
    = -\frac{\partial u}{\partial x}r\sin\theta
      + \frac{\partial u}{\partial y}r\cos\theta.\]
  \[\begin{split}
  \frac{\partial^2u}{\partial r^2}
    ={} & \left[\frac{\partial}{\partial x}\left(\frac{\partial u}{\partial x}\right)\cos\theta
          + \frac{\partial}{\partial y}\left(\frac{\partial u}{\partial x}\right)\sin\theta\right]\cos\theta \\
        & + \left[\frac{\partial}{\partial x}\left(\frac{\partial u}{\partial y}\right)\cos\theta
          + \frac{\partial}{\partial y}\left(\frac{\partial u}{\partial y}\right)\sin\theta\right]\sin\theta \\
    ={} & \frac{\partial^2u}{\partial x^2}\cos^2\theta
          + 2\frac{\partial^2u}{\partial x\partial y}\sin\theta\cos\theta
          + \frac{\partial^2u}{\partial y^2}\sin^2\theta.
  \end{split}\]
  \[\begin{split}
  \frac{\partial^2u}{\partial\theta^2}=
    & -\frac{\partial u}{\partial x}r\cos\theta-\frac{\partial u}{\partial y}r\sin\theta
      - \left[\frac{\partial}{\partial x}\left(\frac{\partial u}{\partial x}\right)(-r\sin\theta)
      + \frac{\partial}{\partial y}\left(\frac{\partial u}{\partial x}\right)r\cos\theta\right]r\sin\theta \\
    & + \left[\frac{\partial}{\partial x}\left(\frac{\partial u}{\partial y}\right)(-r\sin\theta)
      + \frac{\partial}{\partial y}\left(\frac{\partial u}{\partial y}\right)r\cos\theta\right]r\cos\theta \\
    ={} & -\frac{\partial u}{\partial x}r\cos\theta-\frac{\partial u}{\partial y}r\sin\theta
      + \frac{\partial^2u}{\partial\theta^2}r^2\sin^2\theta-2\frac{\partial^2u}{\partial x\partial y}r^2\sin\theta\cos\theta+\frac{\partial^2u}{\partial y^2}r^2\cos^2\theta.
  \end{split}\]
  故
  \begin{align*}
    \frac{1}{r}\frac{\partial}{\partial r}\left(r\frac{\partial u}{\partial r}\right)
      + \frac{1}{r^2}\frac{\partial^2u}{\partial\theta^2}
      + \frac{\partial^2u}{\partial z^2}
    & = \frac{1}{r}\frac{\partial u}{\partial r}+\frac{\partial^2u}{\partial r^2}
      + \frac{1}{r^2}\frac{\partial^2u}{\partial\theta^2} + \frac{\partial^2u}{\partial z^2} \\
    & = \frac{\partial^2u}{\partial x^2}+\frac{\partial^2u}{\partial y^2}
      + \frac{\partial^2u}{\partial z^2}=\Delta u. \qedhere
  \end{align*}
\end{proof}



\begin{exercise}
  证明下列函数都是调和函数:
  \begin{enumerate}[(1)]
    \item $ax+by+c$ ($a,b,c$ 为常数);
    \item $x^2-y^2$和$2xy$;
    \item $x^3-3xy^2$和$3x^2y-y^2$;
  \end{enumerate}
\end{exercise}

\begin{proof}
  直接验证即可.
\end{proof}



\begin{exercise}
  证明用极坐标表示的下列函数都满足调和方程:
  \begin{enumerate}[(1)]
    \item $\ln r$ 和 $\theta$;
    \item $r^n\cos n\theta$ 和 $r^n\sin n\theta$ ($n$为常数);
    \item $r\ln r\cos\theta-r\theta\sin\theta$ 和 $r\ln r\sin\theta+r\theta\cos\theta$.
  \end{enumerate}
\end{exercise}

\begin{proof}
  极坐标下的 Laplace 算子为
  \[\Delta u = \frac{1}{r}\frac{\partial}{\partial r}\left(r\frac{\partial u}{\partial r}\right)
    + \frac{1}{r^2}\frac{\partial^2u}{\partial\theta^2}.\]
  代入验证 $\Delta u=0$ 即可.
\end{proof}


\begin{exercise}
  用分离变量法求解由下述调和方程的第一边值问题所描述的矩阵平板 ($0\leq x\leq a,0\leq y\leq b$)
  上的稳定温度分布:
  \[\begin{cases}
    \frac{\partial^2u}{\partial x^2}+\frac{\partial^2u}{\partial y^2}=0,\\
    u(0,y)=u(a,y)=0,\\
    u(x,0)=\sin\frac{\pi x}{a},u(x,b)=0.
  \end{cases}\]
\end{exercise}

\begin{solve}
  令 $u(x,y)=X(x)Y(y)$, 代入 $\Delta u=0$ 得
  \[\frac{X''(x)}{X(x)} = -\frac{Y''(y)}{Y(y)} = \lambda.\]
  由于 $u(x,0) = X(x)Y(0) = \sin\frac{\pi x}{a}$, 故 $X(x) = C\sin\frac{\pi x}{a}$ 且求导得
  \[\frac{X''(x)}{X(x)} = \lambda = -\biggl(\frac{\pi}{a}\biggr)^2.\]
  所以
  \[Y''(y) - \biggl(\frac{\pi}{a}\biggr)^2 Y(y) = 0.\]
  解得
  \[Y(y) = C_1 \e^{\frac{\pi}{a}y} + C_2 \e^{-\frac{\pi}{a}y},\]
  结合边界条件 $Y(b) = 0$ 得
  \[Y(y) = C_3 \Bigl(\e^{\frac{(y-b)\pi}{a}} - \e^{\frac{(b-y)\pi}{a}}\Bigr)
    = 2C_3 \sinh\frac{(y-b)\pi}{a}.\]
  于是
  \[u(x,y) = X(x)Y(y) = C_4 \sinh\frac{(y-b)\pi}{a} \sin\frac{\pi x}{a}.\]
  结合 $u(x,0) = \sin\frac{\pi x}{a}$, 得
  \[u(x,y) = \frac{\sinh \frac{(y-b)\pi}{a}}{\sinh \frac{-b\pi}{a}} \sin\frac{\pi x}{a}
    = \frac{\sinh (b-y)\pi/a}{\sinh b\pi/a} \sin\frac{\pi x}{a}.\qedhere\]
\end{solve}


\begin{exercise}
  在膜型扁壳渠道闸门的设计中, 为了考察闸门在水压力作用下的受力情况, 要在矩形区域
  $0\leq x\leq a$, $0\leq y\leq b$ 上求解如下的非齐次调和方程的边值问题:
  \[\begin{cases}
    \Delta u = py+q\quad (p<0, q>0\text{\ 常数}), \\
    \displaystyle\frac{\partial u}{\partial x}\biggm|_{x=0} = 0,\ u|_{x=a} = 0, \\
    u|_{y=0,y=b} = 0.
  \end{cases}\]
  试求解之.
\end{exercise}

\begin{solution}
  令 $v = u + (x^2-a^2)(fy+g)$, 通过选取 $f = -p/2$, $g = -q/2$, 则 $v$ 满足方程
  \[\begin{cases}
    \Delta v = 0, \\
    v_x|_{x=0} = v|_{x=a} = 0, \\
    v|_{y=0} = -\frac{q}{2}(x^2-a^2) = \alpha(x), \\
    v|_{y=b} = -\frac{1}{2}(bp+q)(x^2-a^2) = \beta(x).
  \end{cases}\]
  再利用分离变量法求解即可.
\end{solution}


\begin{exercise}
  举例说明在二维 Laplace 方程的 Dirichlet 外问题中, 如对解 $u(x,y)$ 不加在无穷远处
  为有界的限制, 那么定解问题的解不是唯一的.
\end{exercise}

\begin{proof}
  考虑区域 $\varOmega = \{(x,y)\mid x^2+y^2 > 1\}$ 以及相应的 Dirichlet 外问题
  \[\Delta u = 0\text{ in }\varOmega,\quad u = 1\text{ on }\partial\varOmega.\]
  显然 $u\equiv 1$ 和 $u = c\ln\frac{1}{r}+1$ 都为对应的解.
\end{proof}


\begin{exercise}
  设
  \[J(u) = \iiint_{\varOmega} \frac12 \biggl[\biggl(\frac{\partial u}{\partial x}\biggr)^2
    + \biggl(\frac{\partial u}{\partial y}\biggr)^2
    + \biggl(\frac{\partial u}{\partial z}\biggr)^2\biggr] \diff x\diff y\diff z
    + \iint_{\varGamma} \biggl(\frac12 \sigma u^2 - gu\biggr) \diff s,\]
  变分问题的提法为: 求 $u\in V$, 使
  \[J(u) = \min_{v\in V} J(v),\]
  其中 $V = C^2(\varOmega)\cap C^1(\closure{\varOmega})$.
  试导出与此变分问题等价的边值问题, 并证明它们的等价性.
\end{exercise}

\begin{proof}
  The equivalent BVP is
  \[\begin{cases}
    \Delta u = 0, \\
    \bigl(\frac{\partial u}{\partial \bm{n}} + \sigma u\bigr)\big|_{\varGamma} = g.
  \end{cases}\]
  For notational simplicity, we use $\int$ instand of $\iint,\iiint$ and use $\d x$
  instand of $\d x\d y\d z$.

  (Minimizer $\Rightarrow$ Solution) Suppose that $J(u) = \min_{v\in V} J(v)$.
  Define for any $v\in V$,
  \begin{equation}\label{eq:e8}
    I(\epsilon) = J(u+\epsilon v)
      = \int_\varOmega \frac12 |\nabla(u+\epsilon v)|^2 \d x
        + \int_\varGamma \biggl(\frac12 \sigma (u+\epsilon v)^2 - g(u+\epsilon v)\biggr) \d s.
  \end{equation}
  Then $I(\epsilon)$ attains its minimum at $\epsilon=0$ and so $I'(0)=0$.
  Direct computation yields that
  \begin{equation}\label{eq:e9}
    \begin{aligned}
      I'(0)
      & = \int_\varOmega \nabla u\cdot \nabla v \d x + \int_\varGamma (\sigma u-g)v \d s \\
      & = - \int_\varOmega \Delta u v \d x
          + \int_\varGamma \biggl(\frac{\partial u}{\partial \bm{n}} + \sigma u - g\biggr)v \d s = 0.
    \end{aligned}
  \end{equation}
  Since $v$ is arbitrary, we choose $v$ such that $v=0$ on $\varGamma$, then
  \begin{equation}\label{eq:e10}
    \int_\varOmega \Delta u v \d x = 0,
  \end{equation}
  which means $\Delta u = 0$ in $\varOmega$. Substituting $\Delta u = 0$ back into~\eqref{eq:e9},
  we now get
  \begin{equation}\label{eq:e11}
    \int_\varGamma \biggl(\frac{\partial u}{\partial \bm{n}} + \sigma u - g\biggr)v \d s = 0.
  \end{equation}
  Choosing $v = \frac{\partial u}{\partial \bm{n}} + \sigma u - g$ on $\varGamma$, we get
  \[ \frac{\partial u}{\partial \bm{n}} + \sigma u = g \quad \text{on}\ \varGamma. \]

  (Solution $\Rightarrow$ Minimizer) Suppose $u$ is the solution of the following BVP
  \[\begin{cases}
    \Delta u = 0, \\
    \bigl(\frac{\partial u}{\partial \bm{n}} + \sigma u\bigr)\big|_{\varGamma} = g.
  \end{cases}\]
  Since
  \begin{equation}\label{eq:e12}
    \begin{aligned}
      J(u+v)
      & = \int_\varOmega \frac12 |\nabla(u+v)|^2 \d x
          + \int_\varGamma \biggl(\frac12 \sigma (u+v)^2 - g(u+v)\biggr) \d s \\
      & = J(u) + \int_\varOmega \frac12 |\nabla v|^2 + \nabla u\cdot\nabla v \d x
          + \int_\varGamma \biggl(\frac12 \sigma v^2 + \sigma uv - gv\biggr) \d s \\
      & = J(u) + \int_\varOmega \frac12 |\nabla v|^2 \d x
          + \int_\varGamma \frac12 \sigma v^2 \d s,
    \end{aligned}
  \end{equation}
  we have that $J(u+v)\geq J(u)$ for any $v\in V$, so $J(u) = \min_{v\in V} J(v)$.
\end{proof}



\section{格林公式及其应用}

\begin{exercise}
  证明 (2.7) 式对于 $M_0$ 在 $\varOmega$ 外与 $\varGamma$ 上的情形成立.
\end{exercise}

\begin{proof}
  当 $M_0$ 在 $\varOmega$ 外时, $v=\frac{1}{r_{M_0M}}$ 在区域 $\varOmega$ 内无奇异点, 故由格林第二公式得
  \[\iiint_{\varOmega} \left(u\Delta\frac{1}{r}-\frac{1}{r}\Delta u\right)\diff V
    = \iint_{\varGamma}\left(u\frac{\partial}{\partial\vec{n}}\left(\frac{1}{r}\right)
      - \frac{1}{r}\frac{\partial u}{\partial\vec{n}}\right) \diff S.\]
  在 $\varOmega$ 内 $\Delta u=0$, $\Delta\frac{1}{r}=0$, 故
  \[-\iint_{\varGamma}\left(u\frac{\partial}{\partial\vec{n}}\left(\frac{1}{r}\right)
    - \frac{1}{r}\frac{\partial u}{\partial\vec{n}}\right)\diff S = 0.\]

  当 $M_0$ 在 $\varGamma$ 上时, 将以 $M_0$ 为球心, 
  以充分小正数 $\varepsilon$ 为半径的球与 $\varOmega$ 相交的部分记为 $K_{\varepsilon}$,
  将 $K_{\varepsilon}$ 的包含于 $\varOmega$ 内的边界记为 $\varGamma_{\varepsilon}$,
  且记 $\partial(\varOmega\setminus K_{\varepsilon})-\varGamma_{\varepsilon}=\varGamma_{\varepsilon}'$,
  则由格林第二公式得
  \[0 = \iiint_{\varOmega\backslash K_{\varepsilon}}
    \left(u\Delta\frac{1}{r} - \frac{1}{r}\Delta u\right)\diff V
    = \iint_{\varGamma_{\varepsilon}\bigcup\varGamma_{\varepsilon}'} \left(u\frac{\partial}{\partial\vec{n}}
      \left(\frac{1}{r}\right)-\frac{1}{r}\frac{\partial u}{\partial\vec{n}}\right) \diff S.\]
  故
  \begin{equation}\label{eq:e13}
    -\iint_{\varGamma_{\varepsilon}'} \left(u\frac{\partial}{\partial\vec{n}}\left(\frac{1}{r}\right)
    -\frac{1}{r}\frac{\partial u}{\partial\vec{n}}\right) \diff S
    = \iint_{\varGamma_{\varepsilon}}\left(u\frac{\partial}{\partial\vec{n}}
      \left(\frac{1}{r}\right) - \frac{1}{r}\frac{\partial u}{\partial\vec{n}}\right)\diff S.
  \end{equation}
  而
  \[\iint_{\varGamma_{\varepsilon}} \left(u\frac{\partial}{\partial\vec{n}}\left(\frac{1}{r}\right)
    -\frac{1}{r}\frac{\partial u}{\partial\vec{n}}\right) \diff S
    = \iint_{\varGamma_{\varepsilon}}\left(\frac{u}{\varepsilon^2}
      - \frac{1}{\varepsilon}\frac{\partial u}{\partial\vec{n}}\right) \diff S.\]
  令 $\varepsilon\to 0$, 注意到 $\varGamma$ 充分光滑
  (这意味着面积 $S(\varGamma_{\varepsilon})\to 2\pi\varepsilon^2$), 所以有
  \begin{equation}\label{eq:e14}
    \lim_{\varepsilon\to 0} \iint_{\varGamma_{\varepsilon}}
    \left(\frac{u}{\varepsilon^2}-\frac{1}{\varepsilon}\frac{\partial u}{\partial\vec{n}}\right)\diff S
    = 2\pi u(M_0).
  \end{equation}
  同时 $\lim_{\varepsilon\to0}\varGamma_{\varepsilon}'=\varGamma$,
  因此由 \eqref{eq:e13} 和 \eqref{eq:e14} 得
  \[-\iint_{\varGamma} \left(u\frac{\partial}{\partial\vec{n}}\left(\frac{1}{r}\right)-\frac{1}{r}
    \frac{\partial u}{\partial\vec{n}}\right)\diff S = 2\pi u(M_0). \qedhere\]
\end{proof}


\begin{exercise}
  若函数 $u(x,y)$ 是单位圆周上的调和函数, 又它在单位圆周上的数值已知为 $u=\sin\theta$,
  其中 $\theta$ 表示极角, 问函数 $u$ 在原点之值等于多少?
\end{exercise}

\begin{proof}
  由平均值公式知原点之值为
  \[u(O) = \frac{1}{2\pi}\int_{\varGamma}\sin\theta\diff s
    = \frac{1}{2\pi}\int_0^{2\pi}\sin\theta\diff\theta = 0.\qedhere\]
\end{proof}


\begin{exercise}[4]
  证明: 当 $u(M)$ 在闭曲面 $\varGamma$ 的外部调和, 并且在无穷远处成立
  \[u(M) = O\biggl(\frac{1}{r_{OM}}\biggr),\quad
    \frac{\partial u}{\partial r} = O\biggl(\frac{1}{r_{OM}^2}\biggr)\quad
    (r_{OM}\to\infty),\]
  而 $M_0$ 是 $\varGamma$ 外任意一点, 则公式 (2.6) 仍成立.
\end{exercise}

\begin{proof}
  取以 $M_0$ 为球心, 以 $R$ (充分大)为半径的球 $K_R$ 使其包含曲面 $\varGamma$,
  并记该球去掉闭曲面 $\varGamma$ 内部区域后得到的部分为 $\varOmega_R$. 将 $K_R$ 的边界记为 $\varGamma_R$,
  再取以 $M_0$ 为球心, 以 $\epsilon$ 为半径的球 $K_{\epsilon}$ 使其完全包含在区域 $\varOmega_R$中,
  将 $K_{\epsilon}$ 的边界记为 $\varGamma_{\epsilon}$.
  取 $r=r_{MM_0}$, 则由格林第二公式得
  \begin{equation}
    \begin{aligned}
    0 & = \iiint_{\varOmega_R\setminus K_{\epsilon}}\left(u\Delta\frac{1}{r}
        - \frac{1}{r}\Delta u\right)\diff V \\
      & = \iint_{\varGamma\bigcup\varGamma_R\bigcup\varGamma_{\epsilon}}\left(u\frac{\partial}{\partial\vec{n}}\left(\frac{1}{r}\right)-\frac{1}{r}\frac{\partial u}{\partial\vec{n}}\right)\diff S.
    \end{aligned} \tag{$\star$}
  \end{equation}
  因为 $u(M)=O(\frac{1}{r})$, $\frac{\partial u}{\partial r}=O(\frac{1}{r^2})$ $(r\to\infty)$,
  所以当 $R\to +\infty$ 时,
  \[\iint_{\varGamma_R}\left(u\frac{\partial}{\partial\vec{n}}\left(\frac{1}{r}\right)
    - \frac{1}{r}\frac{\partial u}{\partial\vec{n}}\right)\diff S
    = \iint_{\varGamma_R}\left(\frac{-u}{R^2}-\frac{1}{R}\frac{\partial u}{\partial r}\right)\diff S\to 0.\]
  又因为当 $\epsilon\to 0$ 时,
  \[\iint_{\varGamma_{\epsilon}}\left(u\frac{\partial}{\partial\vec{n}}\left(\frac{1}{r}\right)
    - \frac{1}{r}\frac{\partial u}{\partial\vec{n}}\right)\diff S
    = \iint_{\varGamma_{\epsilon}}\left(\frac{u}{\epsilon^2}
    + \frac{1}{\epsilon}\frac{\partial u}{\partial r}\right)\diff S
    \to 4\pi u(M_0).\]
  在 ($\star$) 式中令 $R\to +\infty$, $\epsilon\to 0$, 即得
  \[u(M_0)
  =-\frac{1}{4\pi}\iint_{\varGamma}\left(u\frac{\partial}{\partial\vec{n}}\left(\frac{1}{r}\right)
  -\frac{1}{r}\frac{\partial u}{\partial\vec{n}}\right)\diff S. \qedhere\]
\end{proof}


\begin{exercise}
  证明调和方程 Dirichlet 外问题解的稳定性.
\end{exercise}

\begin{proof}
  在闭曲面 $\varGamma$ 上给定两个函数 $f,f^*$, 并且在$\varGamma$上满足$|f-f^*|\leq\epsilon$,
  设 $u,u^*$ 是相应的狄利克雷外问题的解, 以 $\varGamma_R$ 表示半径为 $R$ 的球面, 令$v=u-u^*$, 因为
  \[\lim_{r\to\infty}v(x,y,z)=0.\]
  所以存在 $R_0$, 使得在 $\varGamma_{R_0}$ 及其外部满足 $|v|\leq\epsilon$,
  在 $\varGamma$ 和 $\varGamma_{R_0}$ 围成的有界区域中,
  利用极值原理知 $|v|\leq\epsilon$, 故在 $\varGamma$ 的外部满足 $|v|\leq\epsilon$,
  由此证明了狄利克雷外问题的解是稳定的.
\end{proof}


\begin{exercise}
  对于二阶偏微分方程
  \[\sum_{i,j=1}^n a_{ij} u_{x_ix_j} + \sum_{i=1}^n b_i u_{x_i} + cu = 0,\]
  其中 $a_{ij}$, $b_i$, $c$ ($i,j=1,\dots,n$) 均为常数. 假设存在常数 $\lambda>0$, 使得
  \[\sum_{i,j=1}^n a_{ij}\xi_i\xi_j \geq \lambda |\xi|^2,\quad \forall\xi\in\mathbb{R}^n.\]
  又设 $c<0$, 证明极值原理: 若 $u$ 在 $\varOmega$ 中满足方程, 在 $\varOmega\cup\varGamma$ 上连续,
  则 $u$ 不能在 $\varOmega$ 的内部达到正的最大值或负的最小值.
\end{exercise}

\begin{proof}
  Suppose $u$ attains its positive maximum value at some point $x_0\in\varOmega$, then
  \[\nabla u(x_0) = 0,\qquad D^2 u(x_0)\leq 0.\]
  Thus
  \[\sum_{i,j=1}^n a_{ij} u_{x_ix_j}(x_0) \leq 0,
    \quad \sum_{i=1}^n b_i u_{x_i}(x_0) = 0,\quad cu(x_0)<0.\]
  It follows that
  \[\sum_{i,j=1}^n a_{ij} u_{x_ix_j}(x_0) + \sum_{i=1}^n b_i u_{x_i}(x_0) + cu(x_0) <0,\]
  which is contradictory to the equation satisfied by $u$.

  The proof for the case when $u$ attains its negative minimum value in the interior of
  $\varOmega$ is similar.
\end{proof}


\begin{exercise}
  证明第 6 题中讨论的椭圆型方程的第一边值问题的唯一性与稳定性.
\end{exercise}


\begin{exercise}
  举例说明对于方程 $\frac{\partial^2u}{\partial x^2} + \frac{\partial^2 u}{\partial y^2} + cu = 0$
  ($c>0$), 不成立极值原理.
\end{exercise}

\begin{solution}
  Consider the function
  \[ u(x,y) = \sin\sqrt{\frac{c}{2}}x \sin\sqrt{\frac{c}{2}}y, \]
  which satisfies $\Delta u + cu = 0$ in the domain
  $\varOmega = \bigl[-\sqrt{\frac{2}{c}}\pi,\sqrt{\frac{2}{c}}\pi\bigr]^2$.
  However, it attains its maximum in the interior of $\varOmega$.
\end{solution}


\section{格林函数}

\[u(M_0) = \iint_{\varGamma}\left[\frac{1}{4\pi r_{M_0M}}\frac{\partial u}{\partial\vec{n}}-u\frac{\partial}{\partial\vec{n}}\frac{1}{4\pi r_{M_0M}}\right]\diff S_M\]
\[\iint_{\varGamma}\left(g\frac{\partial u}{\partial\vec{n}}-u\frac{\partial g}{\partial\vec{n}}\right)\diff S_M=0\]
相减得
\[u(M_0)=\iint_{\varGamma}\left(G\frac{\partial u}{\partial\vec{n}}-u\frac{\partial G}{\partial\vec{n}}\right)\diff S_M,\mbox{其中}G(M,M_0)=\frac{1}{4\pi r_{M_0M}}-g(M,M_0)\]

\begin{note}
  格林函数的五点性质的证明.

  \begin{property}
    格林函数 $G(M, M_0)$ 除 $M=M_0$ 一点外处处满足方程 (1.1), 而当 $M\to M_0$ 时,
    $G(M, M_0)$ 趋于无穷大, 其阶数和 $\frac{1}{4\pi r_{M_0M}}$ 相同.
  \end{property}

  \begin{proof}
    除了点$M=M_0$外,$\frac{1}{4\pi r_{M_0M}}$调和,又因为$g(M,M_0)$在$\varOmega$内调和,故$G(M,M_0)$除了$M=M_0$外处处调和,由极值原理知$g(M,M_0)$在$\varOmega$上有界,故
    \[\lim_{M\to M_0}G(M,M_0)=\lim_{M\to M_0}\left(\frac{1}{4\pi r_{M_0M}}-g(M,M_0)\right)=\infty\mbox{且和}\frac{1}{4\pi r_{M_0M}}\mbox{同阶}\]
  \end{proof}

  \begin{property}
    在边界上格林函数 $G(M, M_0)$ 恒等于零.
  \end{property}

  \begin{proof}
    由$g(M,M_0)$的定义知$G(M,M_0)|_{\varGamma}=0$.
  \end{proof}

  \begin{property}
    在区域 $\varOmega$ 上成立着不等式:
    \[0 < G(M, M_0) < \frac{1}{4\pi r_{M_0M}}.\]
  \end{property}
  
  \begin{proof}
    \[0<G(M,M_0)<\frac{1}{4\pi r_{M_0M}}\Leftrightarrow 0<g(M,M_0)<\frac{1}{4\pi r_{M_0M}}\]
    由极值原理知$g(M,M_0)>0$是显然的,下面证明$g(M,M_0)<\frac{1}{4\pi r_{M_0M}}$:

    取$\delta$足够小使得$\frac{1}{4\pi r_{M_0M}}>g(M,M_0)$在$B(M_0,\delta)$上成立(这样的$\delta$显然是可以取到的),记$D=\varOmega\backslash\overline{B(M_0,\delta)}$,则$\frac{1}{4\pi r_{M_0M}}-g(M,M_0)$在$D$上调和,且
    \[\min_{\partial D}\left(\frac{1}{4\pi r_{M_0M}}-g(M,M_0)\right)=0\]
    故由极值原理知在$D$上成立$\frac{1}{4\pi r_{M_0M}}>g(M,M_0)$,从而在$\varOmega$上成立$\frac{1}{4\pi r_{M_0M}}>g(M,M_0)$.
  \end{proof}

  \begin{property}
    格林函数 $G(M, M_0)$ 在自变量 $M$ 及参变量 $M_0$ 之间具有对称性, 即设
    $M_1$, $M_2$ 为区域中的亮两点, 则
    \[G(M_1, M_2) = G(M_2, M_1).\]
  \end{property}
  
  \begin{proof}
    记$D_{\epsilon}=\varOmega\backslash(B(M_1,\epsilon)\bigcup B(M_2,\epsilon))$,
    再记$w(M)=G(M,M_2),v(M)=G(M,M_1)$,则我们要证明$w(M_1)=v(M_2)$,
    由定义知$w(M)$和$v(M)$都是$D_{\epsilon}$上的调和函数,故由格林公式得
    \[\iint_{\partial D_{\epsilon}}\left(w\frac{\partial v}{\partial\vec{n}}-v\frac{\partial w}{\partial\vec{n}}\right)\diff S=0\]
    故
    \[\iint_{\partial B(M_1,\epsilon)}\left(w\frac{\partial v}{\partial\vec{n}}-v\frac{\partial w}{\partial\vec{n}}\right)\diff S=\iint_{\partial B(M_2,\epsilon)}\left(v\frac{\partial w}{\partial\vec{n}}-w\frac{\partial v}{\partial\vec{n}}\right)\diff S\]
    当$\epsilon\to0+$时,在$\partial B(M_1,\epsilon)$上,$v=O(\frac{1}{\epsilon}),\frac{\partial v}{\partial\vec{n}}=O(\frac{1}{\epsilon^2}),\frac{\partial v}{\partial\vec{n}}=-\frac{1}{4\pi\epsilon^2}+\mbox{有界量}$,故
    \[\iint_{\partial B(M_1,\epsilon)}w\frac{\partial v}{\partial\vec{n}}\diff S=\iint_{\partial B(M_1,\epsilon)}-\frac{1}{4\pi\epsilon^2}w\diff S+\iint_{\partial B(M_1,\epsilon)}\mbox{有界量}\cdot w\diff S\to-w(M_1)\]取极限$\epsilon\to0+$,即得$w(M_1)=v(M_2)$.
  \end{proof}

  \begin{property}
    $\iint_{\varGamma} \frac{\partial G(M, M_0)}{\partial n} \diff S_M = -1$.
  \end{property}
  
  \begin{proof}
    设 $\varGamma_{\epsilon}$ 是以 $M_0$为球心,
    以 $\epsilon$ 为半径的球面, 并且其包含在 $\varOmega$ 当中, 则
    \[\begin{split}
      \iint_{\varGamma}\frac{\partial G(M,M_0)}{\partial\vec{n}}\diff S_M
      & = \iint_{\varGamma}\frac{\partial}{\partial\vec{n}}\left(\frac{1}{4\pi r_{M_0M}}\right)\diff S_M-\iint_{\varGamma}\frac{\partial g(M,M_0)}{\partial\vec{n}}\diff S_M \\
      & = \iint_{\varGamma}\frac{\partial}{\partial\vec{n}}
          \left(\frac{1}{4\pi r_{M_0M}}\right)\diff S_M\quad (\text{Theorem 2.1}) \\
      & = \iint_{\varGamma_{\epsilon}}\frac{\partial}{\partial\vec{n}}
          \left(\frac{1}{4\pi r_{M_0M}}\right)\diff S_M\quad (\text{Theorem 2.1}) \\
      & = \iint_{\varGamma_{\epsilon}}\frac{-1}{4\pi\epsilon^2}\diff S_M = -1.
    \end{split}\]
    另法:考虑定解问题(老师上课时解法)
    \[\begin{cases}
    \Delta u=0(in\;\varOmega)\\u|_{\varGamma=1}
    \end{cases}\]
    由极值原理知解为$u\equiv1$,代入(3.4)式即得结论
  \end{proof}

  二维情形圆的格林函数取为$G(M,M_0)=\frac{1}{2\pi}\left(\ln\frac{1}{r_{M_0M}}-\ln\frac{R}{\rho_0}\frac{1}{r_{M_1M}}\right)$也是为了使得$\int_{\varGamma}\frac{\partial G(M,M_0)}{\partial\vec{n}}\diff s=-1$成立,证明思路同上.
\end{note}



\begin{exercise}[5]
  求半圆区域上狄利克雷问题的格林函数.
\end{exercise}

\begin{solve}
  设上半圆区域为 $D$, 半径为$R$, $A(x_0,y_0)\in D,M(x,y)\in D$,
  点 $A$ 关于边界圆的反演点为 $A_1(R^2\frac{x_0}{\sqrt{x_0^2+y_0^2}}$,
  $R^2\frac{y_0}{\sqrt{x_0^2+y_0^2}})$,
  记 $A$ 与 $A_1$ 关于 $x$ 轴的对称点分别为 $A_2$, $A_3$,
  再记 $MA=r$, $MA_1=r_1$, $MA_2=r_2$, $MA_3=r_3$, 则格林函数为
  \[G(M,M_0)=\frac{1}{2\pi}\ln\frac{r_1r_2}{rr_3}.\qedhere\]
\end{solve}


\begin{exercise}[7]
  证明二维调和函数的可去奇点定理:
  若 $A$ 是调和函数 $u(M)$ 的孤立奇点, 在 $A$ 点邻域中成立着
  \[u(M) = o\biggl(\ln\frac{1}{r_{AM}}\biggr),\]
  则此时可以重新定义 $u(M)$ 在 $M=A$ 的值, 使它在 $A$ 点也是调和的.
\end{exercise}

\begin{proof}
  设 $K$ 是以 $A$ 为圆心, 以$R$为半径的圆,并且其完全包含在$A$的那个所考察的邻域中,
  以 $u$ 在 $K$ 上的值为边界条件, 在 $K$ 内求拉普拉斯方程的解,记为 $v$,
  下面证明在整个 $K$ 内除了点 $A$ 外 $u=v$, 令 $w=u-v$, 则
  \[\lim_{M\to A}\frac{w(M)}{\ln\frac{1}{r_{AM}}}=0.\]
  此外在圆 $K$ 的边界上 $w=0$. 作函数
  \[w_{\epsilon}(M) = \epsilon\left(\ln\frac{1}{r_{AM}}-\ln\frac{1}{R}\right).\]
  具有如下的性质: (1)在$K$的边界上$w_{\epsilon}(M)=0$;在$K$内$w_{\epsilon}(M)>0$.
  (2) 在$r=\delta$和$r=R$所包围的同心圆环上是调和函数,这里$\delta$是任意小的正数.
  因为
  \[\lim_{M\to A}\frac{w_{\epsilon}(M)}{\ln\frac{1}{r_{AM}}}=\epsilon,\]
  所以存在 $\delta'$, 使得当 $r_{AM}<\delta'$时
  \[\frac{w_{\epsilon}(M)}{\ln\frac{1}{r_{AM}}}>\frac{\epsilon}{2}.\]
  又因为
  \[\lim_{M\to A}\frac{w(M)}{\ln\frac{1}{r_{AM}}}=0,\]
  所以存在 $\delta''$,使得当$r_{AM}<\delta''$时
  \[\left|\frac{w(M)}{\ln\frac{1}{r_{AM}}}\right|<\frac{\epsilon}{2}\]
  令$\delta=\min(\delta',\delta'')$,则在圆$r_{AM}=\delta$上有
  \[\left|\frac{w(M)}{\ln\frac{1}{r_{AM}}}\right|
    \leq\frac{w_{\epsilon}(M)}{\ln\frac{1}{r_{AM}}}\Rightarrow|w(M)|\leq w_{\epsilon}(M).\]
  故由极值原理推论2知对于 $r=\delta$ 和 $r=R$ 所包围的同心圆环中的任意一点 $M$, 成立
  \[|w(M)|\leq w_{\epsilon}(M).\]
  令 $\epsilon\to0$, 即得 $w(M)=0,\forall M\in K\backslash A$.
\end{proof}


\begin{exercise}
  证明: 如果三维调和函数 $u(M)$ 在奇点 $A$ 附近表示成 $\frac{N}{r_{AM}^{\alpha}}$,
  其中常数 $0<\alpha\leq 1$, 而 $N$ 是不为零的光滑函数, 则当 $M\to A$ 时它趋于
  无穷大的阶数必与 $\frac{1}{r_{AM}}$ 同阶, 即 $\alpha=1$.
\end{exercise}

\begin{proof}
  假设$\alpha<1$,则
  \[\lim_{M\to A}r_{AM}u(M) = \lim_{M\to A}Nr_{AM}^{1-\alpha} = 0.\]
  由可去奇点定理知 $A$ 是可去奇点, 矛盾, 故 $\alpha=1$.
\end{proof}


\begin{exercise}
  Try to find a function $u$ such that it is harmonic in a circle of radius $\alpha$
  and takes the following values on the boundary $C$:
  \begin{enumerate}[(1)]
    \item $u|_C = A\cos\varphi$;
    \item $u|_C = A + B\sin\varphi$.
  \end{enumerate}
\end{exercise}

9.解:利用泊松公式(3.13)式
\[u(\rho_0,\varphi_0)=\frac{1}{2\pi}\int_0^{2\pi}\frac{(R^2-\rho_0^2)f(\varphi)}{R^2+\rho_0^2-2R\rho_0\cos(\varphi-\varphi_0)}\diff\varphi\]
(1)\[\begin{split}
u(\rho_0,\varphi_0)&=\frac{1}{2\pi}\int_0^{2\pi}\frac{(R^2-\rho_0^2)A\cos(\varphi)}{R^2+\rho_0^2-2R\rho_0\cos(\varphi-\varphi_0)}\diff\varphi\\
&=\frac{A(R^2-\rho_0^2)}{2\pi}\int_0^{2\pi}\frac{\cos(\varphi)}{R^2+\rho_0^2-2R\rho_0\cos(\varphi-\varphi_0)}\diff\varphi\\
&=\frac{A(R^2-\rho_0^2)}{2\pi}\int_{0}^{2\pi}\frac{\cos(\theta+\varphi_0)}{R^2+\rho_0^2-2R\rho_0\cos\theta}\diff\theta(\mbox{换元}\theta=\varphi-\varphi_0)\\
&=\frac{A(R^2-\rho_0^2)}{2\pi}\int_{-\pi}^{\pi}\frac{\cos\theta\cos\varphi_0-\sin\theta\sin\varphi_0}{R^2+\rho_0^2-2R\rho_0\cos\theta}\diff\theta\\
&=\frac{A(R^2-\rho_0^2)}{2\pi}\int_{-\pi}^{\pi}\frac{\cos\theta\cos\varphi_0}{R^2+\rho_0^2-2R\rho_0\cos\theta}\diff\theta\\
&=\frac{2A(R^2-\rho_0^2)\cos\varphi_0}{2\pi}\int_0^{\pi}\frac{\cos\theta}{R^2+\rho_0^2-2R\rho_0\cos\theta}\diff\theta\\
&=\frac{2A(R^2-\rho_0^2)\cos\varphi_0}{2\pi}\frac{\pi\rho_0}{R(R^2-\rho_0^2)}\\
&=\frac{A}{R}\rho_0\cos\varphi_0=\frac{A}{\alpha}\rho_0\cos\varphi_0
\end{split}\]
故\[u(\rho,\varphi)=\frac{A}{\alpha}\rho\cos\varphi\]
(2)因为$u|_C=A+B\sin\varphi=A+B\cos(\varphi-\frac{\pi}{2})$,故由叠加原理及(1)中结果知
\[u(\rho,\varphi)=A+\frac{B}{\alpha}\rho\cos\left(\varphi-\frac{\pi}{2}\right)=A+\frac{B}{\alpha}\rho\sin\varphi\]


\begin{exercise}
  Derive the solution of the Dirichlet problem of 2-dimensional Laplace
  equation in the half plane:
  \[\begin{cases}
    \Delta u = u_{xx} + u_{yy} = 0, \quad y>0, \\
    u|_{y=0} = f(x).
  \end{cases}\]
\end{exercise}

\begin{solve}
  \[G(M,M_0)=\frac{1}{2\pi}\left[\ln\frac{1}{\sqrt{(x-x_0)^2+(y-y_0)^2}}-\ln\frac{1}{\sqrt{(x-x_0)^2+(y+y_0)^2}}\right]\]
  注意到$\frac{\partial}{\partial\vec{n}}=-\frac{\partial}{\partial y}$,故
  \[\begin{split}
  u(x_0,y_0)&=-\int_{\varGamma}f(x)\frac{\partial G(M,M_0)}{\partial\vec{n}}\diff x\\
  &=\frac{1}{2\pi}\int_{-\infty}^{\infty}f(x)\frac{\partial}{\partial y}\left[\ln\frac{1}{\sqrt{(x-x_0)^2+(y-y_0)^2}}-\ln\frac{1}{\sqrt{(x-x_0)^2+(y+y_0)^2}}\right]\Bigg|_{y=0}\diff x\\
  &=\frac{1}{2\pi}\int_{-\infty}^{\infty}f(x)\left(-\frac{1}{2}\frac{2(y-y_0)}{(x-x_0)^2+(y-y_0)^2}+\frac{1}{2}\frac{2(y+y_0)}{(x-x_0)^2+(y+y_0)^2}\right)\bigg|_{y=0}\diff x\\
  &=\frac{1}{\pi}\int_{-\infty}^{\infty}f(x)\frac{y_0}{(x-x_0)^2+y_0^2}\diff x
  \end{split}\]
  因此二维调和方程在半平面上的狄利克雷问题的解为:
  \[u(x,y)=\frac{1}{\pi}\int_{-\infty}^{\infty}\frac{f(\xi)y}{(\xi-x)^2+y^2}\diff\xi. \qedhere\]
\end{solve}


\section{强极值原理,第二边值问题解的唯一性}

\begin{exercise}
  试用强极值原理来证明极值原理: 对不恒等于常数的调和函数$u(x,y,z)$,
  其在区域 $\varOmega$ 的任何内点上的值不可能达到它在 $\varOmega$ 上的上界或下界.
\end{exercise}

\begin{proof}
  假设调和函数 $u(x,y,z)$ 不恒等于常数, 且在区域 $\varOmega$ 内部某点达最小值 $m$, 记
  \[E = \{ M\in\varOmega \mid u(M) = m\}.\]
  则由 $u$ 的连续性知 $E$ 是相对闭集, 由于 $u$ 不恒为常数, 故 $\varOmega\setminus E$ 为非空开集,
  取点 $M_0\in\varOmega\setminus E$ 使得 $\dist(M_0, E) < \dist(M_0, \partial\varOmega)$.
  取以点 $M_0$ 为球心, 以 $\dist(M_0, E)$ 为半径的球 $B$, 取 $M_1\in \partial B\cap E$.
  对于 $B$ 内任一点 $M$ 均有 $u(M)>u(M_1)$, 故由强极值原理
  \[\frac{\partial u}{\partial\vec{\nu}}\bigg|_{M_1} > 0,\]
  其中 $\vec{\nu}$ 与 $B$ 在点 $M_1$ 处的内法线方向成锐角,
  但是由于 $M_1$ 是 $\varOmega$ 内部的最小值点, 故对于任意方向 $\vec{l}$ 均有
  \[\frac{\partial u}{\partial\vec{l}}\bigg|_{M_1} = 0.\]
  矛盾.
\end{proof}


\begin{exercise}
  利用极值原理和强极值原理证明: 当区域 $\varOmega$ 的边界 $\varGamma$
  满足定理 4.2 中的条件时, 调和方程第三边值问题
  \[\biggl(\frac{\partial u}{\partial \bm{n}} + \sigma u\biggr)\bigg|_{\varGamma}
    = f \quad (\sigma>0)\]
  的解的唯一性.
\end{exercise}

\begin{proof}
  只需要证明满足边界条件 $(\frac{\partial u}{\partial\bm{n}}+\sigma u)|_{\varGamma}=0$ 的只有零解即可,
  下面分两种情况讨论.

  对于第三边值问题的内问题: 假设 $u$ 不恒为常数, 则由极值原理知 $u$ 在 $\varGamma$ 上取得最大值和最小值,
  记在 $M_1$ 处取到最小值, 在 $M_2$ 处取得最大值, 则
  \[\frac{\partial u}{\partial\bm{n}}(M_1) + \sigma u(M_1) = 0
    \Rightarrow u(M_1) = -\frac{1}{\sigma}\frac{\partial u}{\partial\bm{n}}(M_1)>0.\]
  \[\frac{\partial u}{\partial\bm{n}}(M_2) + \sigma u(M_2) = 0
    \Rightarrow u(M_2) = -\frac{1}{\sigma}\frac{\partial u}{\partial\bm{n}}(M_2)<0.\]
  故 $u(M_2)<u(M_1)$, 矛盾, 故假设不成立, 所以
  \[u\equiv C\Rightarrow\frac{\partial u}{\partial\bm{n}}\bigg|_{\varGamma} = 0
    \Rightarrow u|_{\varGamma}=0\Rightarrow u\equiv 0.\]
  
  对于第三边值问题的外问题: 记边界 $\varGamma$ 的外部为 $\varOmega'$,
  假设存在 $M_0\in\varOmega'$, 使得 $u(M_0)>0$, 由于 $\lim_{M\to\infty}u(M)=0$,
  故存在充分大的 $R$, 使得在 $\varGamma_R = \{r=R\}$ 上成立 $|u|<u(M_0)$,
  则由极值原理知 $u$ 的最大值只能在 $\varGamma$ 上取, 设最大值点为 $M_1\in\varGamma$, 则
  \[\frac{\partial u}{\partial\bm{n}}\bigg|_{M_1}+\sigma u(M_1)>0,\]
  与边界条件相矛盾, 假设 $u(M_0)<0$ 同样可以导出矛盾, 故$u\equiv 0$.
\end{proof}


\begin{exercise}
  说明在证明强极值原理过程中, 不可能作出一个满足条件 (1) 和 (3) 的辅助函数 $v(x,y,z)$,
  使它在整个球 $\closure{B}_R = \{x^2+y^2+z^2 \leq R^2\}$ 内满足 $\Delta v>0$.
\end{exercise}

\begin{proof}
  若在 $B_R$ 上有 $\Delta v>0$, 则
  \[\max_{\closure{B}_R} v = \max_{\partial B_R} v = 0.\]
  又因为 $\frac{\partial v}{\partial r} < 0$, 所以
  \[\min_{\closure{B}_R} v = \min_{\partial B_R} v = 0.\]
  结合二者即得 $v\equiv 0$, 与 $\Delta v>0$ 矛盾.
\end{proof}


\begin{exercise}
  设 $\varOmega$ 为 $\mathbb{R}^3$ 中有界区域, 边界为 $\varGamma$, $u$ 为定解问题
  \[\begin{cases}
    - \Delta u + cu = f, \\
    \biggl(\frac{\partial u}{\partial \bm{n}} + \sigma u\biggr)\biggm|_{\partial\varOmega} = g
  \end{cases}\]
  的解, 其中 $c,f,g,\sigma>0$, 证明在 $\closure{\varOmega}$ 上 $u>0$.
\end{exercise}


\begin{exercise}[6]
  对于一般的椭圆型方程
  \[\sum_{i,j=1}^n a_{ij}\frac{\partial^2u}{\partial x_i\partial x_j}
    + \sum_{i=1}^n b_i \frac{\partial u}{\partial x_i} + cu = 0,\]
  其中矩阵 $(a_{ij})$ 正定, 即存在常数 $\alpha>0$ 使得
  \[\sum_{i,j=1}^n a_{ij}\xi_i\xi_j \geq \alpha|\xi|^2\quad
    \forall\xi\in \mathbb{R}^n.\]
  又设 $c\leq 0$, 试证明它的解也成立着强极值原理.
  也就是说, 如果 $u(M)$ 在球 $|x|<R$ 内满足上述方程, 在闭球 $|x|\leq R$ 上连续,
  在球面上一点 $M_0$ 处取到非正的最小值, 且在该点沿 $\bm{\nu}$ 方向的方向导数
  $\frac{\partial u}{\partial \bm{\nu}}$ 存在, 其中 $\bm{\nu}$ 与球的内法线方向成锐角,
  则在 $M_0$ 点有 $\frac{\partial u}{\partial \bm{\nu}}>0$.
\end{exercise}

\begin{proof}
  若 $u$ 在球面上一点 $M_0$ 取非正的最小值, 即 $u(M_0)\leq 0$,
  且对球内任一点 $M$ 有 $u(M)>u(M_0)$, 因此在 $M_0$ 点有
  \[\frac{\partial u}{\partial\bm{\nu}}\geq 0.\]
  现在需要证明上式中的等号不能成立, 构造函数
  \[v(x) = \e^{-a\sum_{i=1}^nx_i^2} - \e^{-aR^2} = \e^{-a|x|^2} - \e^{-aR^2},\]
  其中 $a$ 为待定的正常数, 则 $v$ 满足如下性质:
  \begin{enumerate}[(1)]
    \item 在球面 $|x|=R$ 上 $v=0$;
    \item 通过适当选取 $a$, 在区域 $D=\{ R/2 \leq |x| \leq R\}$ 内
      \[Lv = \sum_{i,j=1}^na_{ij}\frac{\partial^2v}{\partial x_i\partial x_j}
        + \sum_{i=1}^nb_i\frac{\partial v}{\partial x_i}+cv > 0.\]
      事实上,因为
      \[\frac{\partial v}{\partial x_i} = -2a x_i \e^{-a|x|^2},\]
      \[\frac{\partial^2v}{\partial x_i\partial x_j} = 4a^2 x_i x_j 
        \e^{-a|x|^2},\quad i\neq j,\]
      \[\frac{\partial^2v}{\partial x_i^2} = 4a^2 x_i^2 \e^{-a|x|^2}
        -2a \e^{-a|x|^2}.\]
      所以
      \begin{align*}
      Lv
      & = 4a^2\left(\sum_{i,j=1}^na_{ij}x_ix_j\right) \e^{-a|x|^2}
        - 2a \sum_{i=1}^n (b_ix_i + a_{ii}) \e^{-a|x|^2} \\
      & \quad + c \left(\e^{-a|x|^2} - \e^{-aR^2}\right) \\
      & = \e^{-a|x|^2} \biggl\{ 4a^2\sum_{i,j=1}^n a_{ij} x_i x_j
        - 2a\sum_{i=1}^n (b_i x_i + a_{ii}) + c \left(1 - \e^{-a(R^2-|x|^2)}\right)\biggr\}
      \end{align*}
      因为 $\sum_{i,j=1}^n a_{ij} x_i x_j \geq \alpha |x|^2 \geq \frac{\alpha R^2}{4} > 0$,
      故当 $a$ 充分大时在区域 $D$ 内 $Lv > 0$.
    \item $v$ 沿球的半径方向$\frac{\partial v}{\partial r}<0$.
        于是 $\frac{\partial v}{\partial\bm{\nu}} > 0$.
  \end{enumerate}

  作函数
  \[\tilde{u}(M)=\varepsilon v(M)+u(M_0).\]
  在 $M_0$ 点有 $\frac{\partial\tilde{u}}{\partial\bm{\nu}}
  = \varepsilon\frac{\partial v}{\partial\bm{\nu}}>0$,
  令函数
  \[w(M) := u(M)-\tilde{u}(M) = u(M)-\varepsilon v(M)-u(M_0).\]
  
  在区域 $D$ 上考察 $w(M)$:
  \begin{enumerate}[(1)]
    \item $Lw=Lu-\varepsilon Lv-Lu(M_0)=-\varepsilon Lv-cu(M_0)<0$;
    \item 在 $|x| = R/2$ 上由于 $u(M)>u(M_0)$,
      取 $\varepsilon$ 足够小可使得 $w(M)>0$;
    \item 在 $|x| = R$ 上 $v=0$, $u(M)>u(M_0)$, 故 $w(M)\geq 0$.
  \end{enumerate}
  现在证明在整个区域 $D$ 上 $w\geq 0$, 假设存在 $M_1\in D$, 使得 $w(M_1) < 0$, 于是
  \[cw(M_1)\geq 0,\quad \frac{\partial w}{\partial x_i}\bigg|_{M_1} = 0,
    \quad\left(\frac{\partial^2w}{\partial x_i\partial x_j}\right)\bigg|_{M_1}\text{ 非负定}.\]
  又 $a_{ij}=\sum_{r=1}^n g_{ri}g_{rj}$, 故
  \[\sum_{i,j=1}^na_{ij}\frac{\partial^2w}{\partial x_i\partial x_j}
    \bigg|_{M_1}=\sum_{r=1}^n\sum_{i,j=1}^n\frac{\partial^2w}{\partial x_i\partial x_j}
    \bigg|_{M_1}g_{ri}g_{rj}\geq 0,\]
  因此 $Lw|_{M_1}\geq 0$, 与(1)矛盾, 因此在 $D$ 内 $w(M)\geq w(M_0)$, 故
  \[\frac{\partial w}{\partial\bm{\nu}}\bigg|_{M_0} \geq 0.\]
  从而
  \[\frac{\partial u}{\partial\bm{\nu}} > 0. \qedhere\]
\end{proof}