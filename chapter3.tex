\chapter{调和方程}

\section{建立方程,定解条件}

\begin{exercise}
  设 $u(x_1,\cdots,x_n)=f(r)$ (其中 $r=\sqrt{x_1^2+\cdots+x_n^2}$)
  是 $n$ 维调和函数 (即满足方程
  $\displaystyle\frac{\partial^2u}{\partial x_1^2}+\cdots+\frac{\partial^2u}{\partial x_n^2}=0$),
  试证明
  \[f(r) = c_1+\frac{c_2}{r^{n-2}}\quad (n\geq 2),\]
  \[f(r) = c_1+c_2\ln\frac{1}{r}\quad (n=2),\]
  其中 $c_1$, $c_2$ 为任意常数.
\end{exercise}

\begin{proof}
  因为 $\displaystyle\frac{\partial u}{\partial x_i}=\frac{\d f}{\d r}\frac{x_i}{r}$,所以
  \[\frac{\partial^2u}{\partial x_i^2} =
    \frac{\d^2f}{\d r^2}\frac{x_i^2}{r^2}
    + \frac{\d f}{\d r}\frac{r^2-x_i^2}{r^3},\]
  故
  \[\sum_{i=1}^n\frac{\partial^2u}{\partial x_i^2}
    = \frac{\d^2f}{\d r^2}+\frac{\d f}{\d r}\frac{n-1}{r}
    = f''(r)+\frac{n-1}{r}f'(r)=0.\]
  由上式得
  \[f'=cr^{1-n}.\]
  \begin{enumerate}[(i)]
    \item $n = 2$ 时, $\displaystyle f(r)=c_1+c_2\ln r=c_1+c_2\ln\frac{1}{r}$,
    \item $n\neq 2$ 时, $\displaystyle f(r)=c_1+\frac{c_2}{2-n}r^{2-n}=c_1+\frac{c_2}{r^{n-2}}$,
  \end{enumerate}
  其中 $c_1$, $c_2$ 为任意常数.
\end{proof}


\begin{exercise}
  证明: 拉普拉斯算子在球坐标$(r,\theta,\varphi)$下可以写成
  \[\Delta u
    = \frac{1}{r^2}\frac{\partial}{\partial r}\left(r^2\frac{\partial u}{\partial r}\right)
    + \frac{1}{r^2\sin\theta}\frac{\partial}{\partial\theta}
      \left(\sin\theta\frac{\partial u}{\partial\theta}\right)
    +\frac{1}{r^2\sin^2\theta}\frac{\partial^2u}{\partial\varphi^2}.\]
\end{exercise}

\begin{proof}
  球坐标变换及其逆变换为
  \[\begin{cases}
    x = r\sin\theta\cos\varphi, \\
    y = r\sin\theta\sin\varphi, \\
    z = r\cos\theta;
  \end{cases}\Rightarrow
  \begin{cases}
    r = \sqrt{x^2+y^2+z^2}, \\
    \theta = \arccos\frac{z}{\sqrt{x^2+y^2+z^2}}, \\
    \varphi = \arctan\frac{y}{x}.
  \end{cases}\]
  通过链式法则可得
  \[\frac{\partial u}{\partial x}
    = \frac{\partial u}{\partial r}\frac{\partial r}{\partial x}
      + \frac{\partial u}{\partial\theta}\frac{\partial\theta}{\partial x}
      + \frac{\partial u}{\partial\varphi}\frac{\partial\varphi}{\partial x}.\]
  \[\frac{\partial^2u}{\partial x^2}
    = \frac{\partial^2u}{\partial r^2}\left(\frac{\partial r}{\partial x}\right)^2
      + \frac{\partial u}{\partial r}\frac{\partial^2r}{\partial x^2}
      + \frac{\partial^2u}{\partial\theta^2}\left(\frac{\partial\theta}{\partial x}\right)^2
      + \frac{\partial u}{\partial\theta}\frac{\partial^2\theta}{\partial x^2}
      + \frac{\partial^2u}{\partial\varphi^2}\left(\frac{\partial\varphi}{\partial x}\right)^2
      + \frac{\partial u}{\partial\varphi}\frac{\partial^2\varphi}{\partial x^2}.\]
  由逆变换公式求得 (所有求导项并未在下面完全列出, 因为很多项的形式是一样的)
  \[\frac{\partial r}{\partial x}
    = \frac{x}{r},\quad
    \frac{\partial^2r}{\partial x^2} = \frac{r^2-x^2}{r^3}.\]
  \[\frac{\partial\theta}{\partial x}
    = \frac{zx}{r^2\sqrt{x^2+y^2}},\quad
    \frac{\partial\theta}{\partial z}
    = \frac{-\sqrt{x^2+y^2}}{r^2}.\]
  \[\frac{\partial^2\theta}{\partial x^2}
    = \frac{zr^2y^2-2zx^2(x^2+y^2)}{r^4(x^2+y^2)^{3/2}},\quad
    \frac{\partial^2\theta}{\partial z^2}=\frac{2z\sqrt{x^2+y^2}}{r^4}\]
  \[\frac{\partial\varphi}{\partial x} = \frac{-y}{x^2+y^2},\quad
    \frac{\partial\varphi}{\partial y} = \frac{x}{x^2+y^2},\quad
    \frac{\partial\varphi}{\partial z}=0.\]
  \[\frac{\partial^2\varphi}{\partial x^2} = \frac{2xy}{(x^2+y^2)^2},\quad
    \frac{\partial^2\varphi}{\partial y^2}=\frac{-2xy}{(x^2+y^2)^2}.\]
  故
  \[\begin{split}
  \Delta u
    & = \frac{\partial^2u}{\partial r^2}\left[\left(\frac{\partial r}{\partial x}\right)^2
        + \left(\frac{\partial r}{\partial y}\right)^2
        + \left(\frac{\partial r}{\partial z}\right)^2\right]
        + \frac{\partial u}{\partial r}\left(\frac{\partial^2r}{\partial x^2}
        + \frac{\partial^2r}{\partial y^2}
        + \frac{\partial^2r}{\partial z^2}\right) \\
    & + \frac{\partial^2u}{\partial\theta^2}
        \left[\left(\frac{\partial\theta}{\partial x}\right)^2
        + \left(\frac{\partial\theta}{\partial y}\right)^2
        + \left(\frac{\partial\theta}{\partial z}\right)^2\right]
        + \frac{\partial u}{\partial\theta}\left(\frac{\partial^2\theta}{\partial x^2}
        + \frac{\partial^2\theta}{\partial y^2}+\frac{\partial^2\theta}{\partial z^2}\right) \\
    & + \frac{\partial^2u}{\partial\varphi^2}
        \left[\left(\frac{\partial\varphi}{\partial x}\right)^2
        + \left(\frac{\partial\varphi}{\partial y}\right)^2
        + \left(\frac{\partial\varphi}{\partial z}\right)^2\right]
        + \frac{\partial u}{\partial\varphi}\left(\frac{\partial^2\varphi}{\partial x^2}
        + \frac{\partial^2\varphi}{\partial y^2}
        + \frac{\partial^2\varphi}{\partial z^2}\right) \\
    & = \frac{\partial^2u}{\partial r^2}+\frac{\partial u}{\partial r}\frac{2}{r}
        + \frac{\partial^2u}{\partial\theta^2}\frac{1}{r^2}
        + \frac{\partial u}{\partial\theta}\frac{z}{r^2\sqrt{x^2+y^2}}
        + \frac{\partial^2u}{\partial\varphi^2}\frac{1}{x^2+y^2} \\
    & = \frac{1}{r^2}\frac{\partial}{\partial r}\left(r^2\frac{\partial u}{\partial r}\right)
        + \frac{1}{r^2\sin\theta}\frac{\partial}{\partial\theta}
        \left(\sin\theta\frac{\partial u}{\partial\theta}\right)
        + \frac{1}{r^2\sin^2\theta}\frac{\partial^2u}{\partial\varphi^2}.
  \end{split}\]
  证毕.
\end{proof}



\begin{exercise}
  证明: 拉普拉斯算子在柱坐标$(r,\theta,z)$下可以写成
  \[\Delta u
    = \frac{1}{r}\frac{\partial}{\partial r}\left(r\frac{\partial u}{\partial r}\right)
    + \frac{1}{r^2}\frac{\partial^2u}{\partial\theta^2}+\frac{\partial^2u}{\partial z^2}.\]
\end{exercise}

\begin{proof}
  柱坐标变换为
  \[\begin{cases}
    x = r\cos\theta, \\
    y = r\sin\theta, \\
    z = z.
  \end{cases}\]
  故
  \[\frac{\partial u}{\partial r}
    = \frac{\partial u}{\partial x}\cos\theta
      + \frac{\partial u}{\partial y}\sin\theta,\quad
    \frac{\partial u}{\partial\theta}
    = -\frac{\partial u}{\partial x}r\sin\theta
      + \frac{\partial u}{\partial y}r\cos\theta.\]
  \[\begin{split}
  \frac{\partial^2u}{\partial r^2}
    ={} & \left[\frac{\partial}{\partial x}\left(\frac{\partial u}{\partial x}\right)\cos\theta
          + \frac{\partial}{\partial y}\left(\frac{\partial u}{\partial x}\right)\sin\theta\right]\cos\theta \\
        & + \left[\frac{\partial}{\partial x}\left(\frac{\partial u}{\partial y}\right)\cos\theta
          + \frac{\partial}{\partial y}\left(\frac{\partial u}{\partial y}\right)\sin\theta\right]\sin\theta \\
    ={} & \frac{\partial^2u}{\partial x^2}\cos^2\theta
          + 2\frac{\partial^2u}{\partial x\partial y}\sin\theta\cos\theta
          + \frac{\partial^2u}{\partial y^2}\sin^2\theta.
  \end{split}\]
  \[\begin{split}
  \frac{\partial^2u}{\partial\theta^2}=
    & -\frac{\partial u}{\partial x}r\cos\theta-\frac{\partial u}{\partial y}r\sin\theta
      - \left[\frac{\partial}{\partial x}\left(\frac{\partial u}{\partial x}\right)(-r\sin\theta)
      + \frac{\partial}{\partial y}\left(\frac{\partial u}{\partial x}\right)r\cos\theta\right]r\sin\theta \\
    & + \left[\frac{\partial}{\partial x}\left(\frac{\partial u}{\partial y}\right)(-r\sin\theta)
      + \frac{\partial}{\partial y}\left(\frac{\partial u}{\partial y}\right)r\cos\theta\right]r\cos\theta \\
    ={} & -\frac{\partial u}{\partial x}r\cos\theta-\frac{\partial u}{\partial y}r\sin\theta
      + \frac{\partial^2u}{\partial\theta^2}r^2\sin^2\theta-2\frac{\partial^2u}{\partial x\partial y}r^2\sin\theta\cos\theta+\frac{\partial^2u}{\partial y^2}r^2\cos^2\theta.
  \end{split}\]
  故
  \begin{align*}
    \frac{1}{r}\frac{\partial}{\partial r}\left(r\frac{\partial u}{\partial r}\right)
      + \frac{1}{r^2}\frac{\partial^2u}{\partial\theta^2}
      + \frac{\partial^2u}{\partial z^2}
    & = \frac{1}{r}\frac{\partial u}{\partial r}+\frac{\partial^2u}{\partial r^2}
      + \frac{1}{r^2}\frac{\partial^2u}{\partial\theta^2} + \frac{\partial^2u}{\partial z^2} \\
    & = \frac{\partial^2u}{\partial x^2}+\frac{\partial^2u}{\partial y^2}
      + \frac{\partial^2u}{\partial z^2}=\Delta u. \qedhere
  \end{align*}
\end{proof}



\begin{exercise}
  证明下列函数都是调和函数:
  \begin{enumerate}[(1)]
    \item $ax+by+c$ ($a,b,c$ 为常数);
    \item $x^2-y^2$和$2xy$;
    \item $x^3-3xy^2$和$3x^2y-y^2$;
  \end{enumerate}
\end{exercise}

\begin{proof}
  直接验证即可.
\end{proof}



\begin{exercise}
  证明用极坐标表示的下列函数都满足调和方程:
  \begin{enumerate}[(1)]
    \item $\ln r$ 和 $\theta$;
    \item $r^n\cos n\theta$ 和 $r^n\sin n\theta$ ($n$为常数);
    \item $r\ln r\cos\theta-r\theta\sin\theta$ 和 $r\ln r\sin\theta+r\theta\cos\theta$.
  \end{enumerate}
\end{exercise}

\begin{proof}
  极坐标下的 Laplace 算子为
  \[\Delta u = \frac{1}{r}\frac{\partial}{\partial r}\left(r\frac{\partial u}{\partial r}\right)
    + \frac{1}{r^2}\frac{\partial^2u}{\partial\theta^2}.\]
  代入验证 $\Delta u=0$ 即可.
\end{proof}


\begin{exercise}
  用分离变量法求解由下述调和方程的第一边值问题所描述的矩阵平板 ($0\leq x\leq a,0\leq y\leq b$)
  上的稳定温度分布:
  \[\begin{cases}
    \displaystyle\frac{\partial^2u}{\partial x^2}+\frac{\partial^2u}{\partial y^2}=0,\\
    u(0,y)=u(a,y)=0,\\
    \displaystyle u(x,0)=\sin\frac{\pi x}{a},u(x,b)=0.
  \end{cases}\]
\end{exercise}

\begin{solve}
  令 $u(x,y)=X(x)Y(y)$, 代入 $\Delta u=0$ 得
  \[\frac{X''(x)}{X(x)} = -\frac{Y''(y)}{Y(y)} = \lambda.\]
  由于 $\displaystyle u(x,0) = X(x)Y(0) = \sin\frac{\pi x}{a}$, 故 $X(x) = C\sin\frac{\pi x}{a}$ 且求导得
  \[\frac{X''(x)}{X(x)} = \lambda = -\biggl(\frac{\pi}{a}\biggr)^2.\]
  所以
  \[Y''(y) - \biggl(\frac{\pi}{a}\biggr)^2 Y(y) = 0.\]
  解得
  \[Y(y) = C_1 \e^{\frac{\pi}{a}y} + C_2 \e^{-\frac{\pi}{a}y},\]
  结合边界条件 $Y(b) = 0$ 得
  \[Y(y) = C_3 \Bigl(\e^{\frac{(y-b)\pi}{a}} - \e^{\frac{(b-y)\pi}{a}}\Bigr)
    = 2C_3 \sinh\frac{(y-b)\pi}{a}.\]
  于是
  \[u(x,y) = X(x)Y(y) = C_4 \sinh\frac{(y-b)\pi}{a} \sin\frac{\pi x}{a}.\]
  结合 $\displaystyle u(x,0) = \sin\frac{\pi x}{a}$, 得
  \[u(x,y) = \frac{\sinh \frac{(y-b)\pi}{a}}{\sinh \frac{-b\pi}{a}} \sin\frac{\pi x}{a}
    = \frac{\sinh (b-y)\pi/a}{\sinh b\pi/a} \sin\frac{\pi x}{a}.\qedhere\]
\end{solve}


\begin{exercise}
  在膜型扁壳渠道闸门的设计中, 为了考察闸门在水压力作用下的受力情况, 要在矩形区域
  $0\leq x\leq a$, $0\leq y\leq b$ 上求解如下的非齐次调和方程的边值问题:
  \[\begin{cases}
    \Delta u = py+q\quad (p<0, q>0\text{\ 常数}), \\
    \displaystyle\frac{\partial u}{\partial x}\biggm|_{x=0} = 0,\ u|_{x=a} = 0, \\
    u|_{y=0,y=b} = 0.
  \end{cases}\]
  试求解之.
\end{exercise}

\begin{solution}
  令 $v = u + (x^2-a^2)(fy+g)$, 通过选取 $f = -p/2$, $g = -q/2$, 则 $v$ 满足方程
  \[\begin{cases}
    \Delta v = 0, \\
    v_x|_{x=0} = v|_{x=a} = 0, \\
    v|_{y=0} = -\frac{q}{2}(x^2-a^2) = \alpha(x), \\
    v|_{y=b} = -\frac{1}{2}(bp+q)(x^2-a^2) = \beta(x).
  \end{cases}\]
  再利用分离变量法求解即可.
\end{solution}


\begin{exercise}
  举例说明在二维 Laplace 方程的 Dirichlet 外问题中, 如对解 $u(x,y)$ 不加在无穷远处
  为有界的限制, 那么定解问题的解不是唯一的.
\end{exercise}

\begin{proof}
  考虑区域 $\varOmega = \{(x,y)\mid x^2+y^2 > 1\}$ 以及相应的 Dirichlet 外问题
  \[\Delta u = 0\text{ in }\varOmega,\quad u = 1\text{ on }\partial\varOmega.\]
  显然 $u\equiv 1$ 和 $\displaystyle u = c\ln\frac{1}{r}+1$ 都为对应的解.
\end{proof}


\begin{exercise}[11]
	证明: 若$u\in C^2(\varOmega)\cap C^1(\overline{\varOmega})$,使泛函
	$$J[v]=\frac{1}{2}\iint\limits_{\varOmega}(|\Delta v|^2+cv^2)\d x\d y-\iint\limits_{\Omega}Fv\d x\d y-\int\limits_{\partial\Omega}gv\d S$$
	取极小, 则它满足
	$$\begin{cases}
		-\Delta u+cu=F,\\
		\displaystyle\frac{\partial u}{\partial\bm{n}}\Big|_{\partial\varOmega}=g.
	\end{cases}$$
\end{exercise}

\begin{proof}
	任取$w\in C^2(\varOmega)\cap C^1(\overline{\varOmega})$,
	
	则$\forall\lambda\in\mathbb{R},J(u+\lambda w)\geq J(w)$.
	
	令$\varphi(\lambda)=J(u+\lambda w)$,则$\varphi(\lambda)\geq\varphi(0)$,可得$\varphi^\prime(0)=0$
	
	由 Green 第一公式,
	$$\begin{aligned}
		\varphi^{\prime}(0)&=\iint_{\varOmega}(u_{x}w_{x}+u_{y}w_{y})+cuw\d\varOmega-\iint_{\varOmega}Fw\d x\d y-\int_{\partial\varOmega}gw\d S\\
		&=\int_{\partial\Omega}w(\frac{\partial u}{\partial\bm{n}}-g)\d S+\iint_{\varOmega}w(cu-\Delta u-F)\d x\d y\\
		&=0.
	\end{aligned}$$
	
	由 $w$ 的任意性,可说明
	$$\begin{cases}
		-\Delta u+cu=F,\\
		\frac{\partial u}{\partial\bm{n}}\Big|_{\partial\varOmega}=g
	\end{cases}$$
	
	不妨取 $w$ 在$\partial\varOmega$上为 $0$, 若设 $cu-\Delta u-F$在 $M$ 点取值不为 $0$, 不妨设为正.
	
	由$cu-\Delta u-F$的连续性,其在 $M$ 的某个小邻域 $N$ 内均为正值
	
	则进一步取 $w$ 在 $N$ 内为正,$N$ 外为 $0$, 可得$\displaystyle\varphi^\prime(0)=\iint_\varOmega w(cu-\Delta u-F)\d x\d y>0$, 矛盾.

	故$cu-\Delta u-F\equiv0$,类似地, 可得$\displaystyle\frac{\partial u}{\partial\bm{n}}\bigg|_{\partial\varOmega}-g\equiv0$.
\end{proof}
\begin{exercise}[12]
  设
  \[J(u) = \iiint_{\varOmega} \frac12 \biggl[\biggl(\frac{\partial u}{\partial x}\biggr)^2
    + \biggl(\frac{\partial u}{\partial y}\biggr)^2
    + \biggl(\frac{\partial u}{\partial z}\biggr)^2\biggr] \d x\d y\d z
    + \iint_{\varGamma} \biggl(\frac12 \sigma u^2 - gu\biggr) \d s,\]
  变分问题的提法为: 求 $u\in V$, 使
  \[J(u) = \min_{v\in V} J(v),\]
  其中 $V = C^2(\varOmega)\cap C^1(\closure{\varOmega})$.
  试导出与此变分问题等价的边值问题, 并证明它们的等价性.
\end{exercise}

\begin{proof}
  The equivalent BVP is
  \[\begin{cases}
    \Delta u = 0, \\
    \bigl(\frac{\partial u}{\partial \bm{n}} + \sigma u\bigr)\big|_{\varGamma} = g.
  \end{cases}\]
  For notational simplicity, we use $\int$ instand of $\iint$, $\iiint$ and use $\d x$
  instand of $\d x\d y\d z$.
  \begin{description}
    \item[(Minimizer $\Rightarrow$ Solution)]
      Suppose that $J(u) = \min_{v\in V} J(v)$.
      Define for any $v\in V$,
      \begin{equation}\label{eq:e8}
        I(\varepsilon) = J(u+\varepsilon v)
          = \int_\varOmega \frac12 |\nabla(u+\varepsilon v)|^2 \d x
            + \int_\varGamma \biggl(\frac12 \sigma (u+\varepsilon v)^2 - g(u+\varepsilon v)\biggr) \d s.
      \end{equation}
      Then $I(\varepsilon)$ attains its minimum at $\varepsilon=0$ and so $I'(0)=0$.
      Direct computation yields that
      \begin{equation}\label{eq:e9}
        \begin{aligned}
          I'(0)
          & = \int_\varOmega \nabla u\cdot \nabla v \d x + \int_\varGamma (\sigma u-g)v \d s \\
          & = - \int_\varOmega \Delta u v \d x
              + \int_\varGamma \biggl(\frac{\partial u}{\partial \bm{n}} + \sigma u - g\biggr)v \d s = 0.
        \end{aligned}
      \end{equation}
      Since $v$ is arbitrary, we choose $v$ such that $v=0$ on $\varGamma$, then
      \begin{equation}\label{eq:e10}
        \int_\varOmega \Delta u v \d x = 0,
      \end{equation}
      which means $\Delta u = 0$ in $\varOmega$. Substituting $\Delta u = 0$ back into~\eqref{eq:e9},
      we now get
      \begin{equation}\label{eq:e11}
        \int_\varGamma \biggl(\frac{\partial u}{\partial \bm{n}} + \sigma u - g\biggr)v \d s = 0.
      \end{equation}
      Choosing $v = \frac{\partial u}{\partial \bm{n}} + \sigma u - g$ on $\varGamma$, we get
      \[ \frac{\partial u}{\partial \bm{n}} + \sigma u = g \quad \text{on}\ \varGamma. \]
    \item[(Solution $\Rightarrow$ Minimizer)]
      Suppose $u$ is the solution of the following BVP
      \[\begin{cases}
        \Delta u = 0, \\
        \bigl(\frac{\partial u}{\partial \bm{n}} + \sigma u\bigr)\big|_{\varGamma} = g.
      \end{cases}\]
      Since
      \begin{equation}\label{eq:e12}
        \begin{aligned}
          J(u+v)
          & = \int_\varOmega \frac12 |\nabla(u+v)|^2 \d x
              + \int_\varGamma \biggl(\frac12 \sigma (u+v)^2 - g(u+v)\biggr) \d s \\
          & = J(u) + \int_\varOmega \frac12 |\nabla v|^2 + \nabla u\cdot\nabla v \d x
              + \int_\varGamma \biggl(\frac12 \sigma v^2 + \sigma uv - gv\biggr) \d s \\
          & = J(u) + \int_\varOmega \frac12 |\nabla v|^2 \d x
              + \int_\varGamma \frac12 \sigma v^2 \d s,
        \end{aligned}
      \end{equation}
      we have that $J(u+v)\geq J(u)$ for any $v\in V$, so $J(u) = \min_{v\in V} J(v)$. \qedhere
  \end{description}
\end{proof}



\section{格林公式及其应用}

\begin{exercise}
  证明 (2.7) 式对于 $M_0$ 在 $\varOmega$ 外与 $\varGamma$ 上的情形成立.
\end{exercise}

\begin{proof}
  (1)当 $M_0$ 在 $\varOmega$ 外时, $\displaystyle v=\frac{1}{r_{M_0M}}$ 在区域 $\varOmega$ 内无奇异点, 故由Green第二公式得
  \[\iiint_{\varOmega} \left(u\Delta\frac{1}{r}-\frac{1}{r}\Delta u\right)\d V
    = \iint_{\varGamma}\left(u\frac{\partial}{\partial\bm{n}}\left(\frac{1}{r}\right)
      - \frac{1}{r}\frac{\partial u}{\partial\bm{n}}\right) \d S.\]
  在 $\varOmega$ 内 $\Delta u=0$, $\displaystyle\Delta\frac{1}{r}=0$, 故
  \[-\iint_{\varGamma}\left(u\frac{\partial}{\partial\bm{n}}\left(\frac{1}{r}\right)
    - \frac{1}{r}\frac{\partial u}{\partial\bm{n}}\right)\d S = 0.\]

  (2)当 $M_0$ 在 $\varGamma$ 上时, 将以 $M_0$ 为球心, 
  以充分小正数 $\varepsilon$ 为半径的球与 $\varOmega$ 相交的部分记为 $K_{\varepsilon}$,
  将 $K_{\varepsilon}$ 的包含于 $\varOmega$ 内的边界记为 $\varGamma_{\varepsilon}$,
  且记 $\partial(\varOmega\setminus K_{\varepsilon})-\varGamma_{\varepsilon}=\varGamma_{\varepsilon}'$,
  则由Green第二公式得
  \[0 = \iiint_{\varOmega\backslash K_{\varepsilon}}
    \left(u\Delta\frac{1}{r} - \frac{1}{r}\Delta u\right)\d V
    = \iint_{\varGamma_{\varepsilon}\bigcup\varGamma_{\varepsilon}'} \left(u\frac{\partial}{\partial\bm{n}}
      \left(\frac{1}{r}\right)-\frac{1}{r}\frac{\partial u}{\partial\bm{n}}\right) \d S.\]
  故
  \begin{equation}\label{eq:e13}
    -\iint_{\varGamma_{\varepsilon}'} \left(u\frac{\partial}{\partial\bm{n}}\left(\frac{1}{r}\right)
    -\frac{1}{r}\frac{\partial u}{\partial\bm{n}}\right) \d S
    = \iint_{\varGamma_{\varepsilon}}\left(u\frac{\partial}{\partial\bm{n}}
      \left(\frac{1}{r}\right) - \frac{1}{r}\frac{\partial u}{\partial\bm{n}}\right)\d S.
  \end{equation}
  而
  \[\iint_{\varGamma_{\varepsilon}} \left(u\frac{\partial}{\partial\bm{n}}\left(\frac{1}{r}\right)
    -\frac{1}{r}\frac{\partial u}{\partial\bm{n}}\right) \d S
    = \iint_{\varGamma_{\varepsilon}}\left(\frac{u}{\varepsilon^2}
      - \frac{1}{\varepsilon}\frac{\partial u}{\partial\bm{n}}\right) \d S.\]
  令 $\varepsilon\to 0$, 注意到 $\varGamma$ 充分光滑
  (这意味着面积 $S(\varGamma_{\varepsilon})\to 2\pi\varepsilon^2$), 所以有
  \begin{equation}\label{eq:e14}
    \lim_{\varepsilon\to 0} \iint_{\varGamma_{\varepsilon}}
    \left(\frac{u}{\varepsilon^2}-\frac{1}{\varepsilon}\frac{\partial u}{\partial\bm{n}}\right)\d S
    = 2\pi u(M_0).
  \end{equation}
  同时 $\displaystyle\lim_{\varepsilon\to0}\varGamma_{\varepsilon}'=\varGamma$,
  因此由 \eqref{eq:e13} 和 \eqref{eq:e14} 得
  \[-\iint_{\varGamma} \left(u\frac{\partial}{\partial\bm{n}}\left(\frac{1}{r}\right)-\frac{1}{r}
    \frac{\partial u}{\partial\bm{n}}\right)\d S = 2\pi u(M_0). \qedhere\]
\end{proof}


\begin{exercise}
  若函数 $u(x,y)$ 是单位圆周上的调和函数, 又它在单位圆周上的数值已知为 $u=\sin\theta$,
  其中 $\theta$ 表示极角, 问函数 $u$ 在原点之值等于多少?
\end{exercise}

\begin{proof}
  由平均值公式知原点之值为
  \[u(O) = \frac{1}{2\pi}\int_{\varGamma}\sin\theta\d s
    = \frac{1}{2\pi}\int_0^{2\pi}\sin\theta\d\theta = 0.\qedhere\]
\end{proof}


\begin{exercise}[4]
  证明: 当 $u(M)$ 在闭曲面 $\varGamma$ 的外部调和, 并且在无穷远处成立
  \[u(M) = O\biggl(\frac{1}{r_{OM}}\biggr),\quad
    \frac{\partial u}{\partial r} = O\biggl(\frac{1}{r_{OM}^2}\biggr)\quad
    (r_{OM}\to\infty),\]
  而 $M_0$ 是 $\varGamma$ 外任意一点, 则公式 (2.6) 仍成立.
\end{exercise}

\begin{proof}
  取以 $M_0$ 为球心, 以 $R$ (充分大)为半径的球 $K_R$ 使其包含曲面 $\varGamma$,
  并记该球去掉闭曲面 $\varGamma$ 内部区域后得到的部分为 $\varOmega_R$. 将 $K_R$ 的边界记为 $\varGamma_R$,
  再取以 $M_0$ 为球心, 以 $\varepsilon$ 为半径的球 $K_{\varepsilon}$ 使其完全包含在区域 $\varOmega_R$中,
  将 $K_{\varepsilon}$ 的边界记为 $\varGamma_{\varepsilon}$.
  取 $r=r_{MM_0}$, 则由格林第二公式得
  \begin{equation}
    \begin{aligned}
    0 & = \iiint_{\varOmega_R\setminus K_{\varepsilon}}\left(u\Delta\frac{1}{r}
        - \frac{1}{r}\Delta u\right)\d V \\
      & = \iint_{\varGamma\bigcup\varGamma_R\bigcup\varGamma_{\varepsilon}}\left(u\frac{\partial}{\partial\bm{n}}\left(\frac{1}{r}\right)-\frac{1}{r}\frac{\partial u}{\partial\bm{n}}\right)\d S.
    \end{aligned} \tag{$\star$}
  \end{equation}
  因为 $\displaystyle u(M)=O\left(\frac{1}{r}\right)$, $\displaystyle\frac{\partial u}{\partial r}=O\left(\frac{1}{r^2}\right)\quad(r\to\infty)$,
  所以当 $R\to +\infty$ 时,
  \[\iint_{\varGamma_R}\left(u\frac{\partial}{\partial\bm{n}}\left(\frac{1}{r}\right)
    - \frac{1}{r}\frac{\partial u}{\partial\bm{n}}\right)\d S
    = \iint_{\varGamma_R}\left(\frac{-u}{R^2}-\frac{1}{R}\frac{\partial u}{\partial r}\right)\d S\to 0.\]
  又因为当 $\varepsilon\to 0$ 时,
  \[\iint_{\varGamma_{\varepsilon}}\left(u\frac{\partial}{\partial\bm{n}}\left(\frac{1}{r}\right)
    - \frac{1}{r}\frac{\partial u}{\partial\bm{n}}\right)\d S
    = \iint_{\varGamma_{\varepsilon}}\left(\frac{u}{\varepsilon^2}
    + \frac{1}{\varepsilon}\frac{\partial u}{\partial r}\right)\d S
    \to 4\pi u(M_0).\]
  在 ($\star$) 式中令 $R\to +\infty$, $\varepsilon\to 0$, 即得
  \[u(M_0)
  =-\frac{1}{4\pi}\iint_{\varGamma}\left(u\frac{\partial}{\partial\bm{n}}\left(\frac{1}{r}\right)
  -\frac{1}{r}\frac{\partial u}{\partial\bm{n}}\right)\d S. \qedhere\]
\end{proof}


\begin{exercise}
  证明调和方程 Dirichlet 外问题解的稳定性.
\end{exercise}

\begin{proof}
  在闭曲面 $\varGamma$ 上给定两个函数 $f,f^*$, 并且在$\varGamma$上满足$|f-f^*|\leq\varepsilon$,
  设 $u,u^*$ 是相应的狄利克雷外问题的解, 以 $\varGamma_R$ 表示半径为 $R$ 的球面, 令$v=u-u^*$, 因为
  \[\lim_{r\to\infty}v(x,y,z)=0.\]
  所以存在 $R_0$, 使得在 $\varGamma_{R_0}$ 及其外部满足 $|v|\leq\varepsilon$,
  在 $\varGamma$ 和 $\varGamma_{R_0}$ 围成的有界区域中,
  利用极值原理知 $|v|\leq\varepsilon$, 故在 $\varGamma$ 的外部满足 $|v|\leq\varepsilon$,
  由此证明了狄利克雷外问题的解是稳定的.
\end{proof}


\begin{exercise}
  对于二阶偏微分方程
  \[\sum_{i,j=1}^n a_{ij} u_{x_ix_j} + \sum_{i=1}^n b_i u_{x_i} + cu = 0,\]
  其中 $a_{ij}$, $b_i$, $c$ ($i,j=1,\dots,n$) 均为常数. 假设存在常数 $\lambda>0$, 使得
  \[\sum_{i,j=1}^n a_{ij}\xi_i\xi_j \geq \lambda |\xi|^2,\quad \forall\xi\in\mathbb{R}^n.\]
  又设 $c<0$, 证明极值原理: 若 $u$ 在 $\varOmega$ 中满足方程, 在 $\varOmega\cup\varGamma$ 上连续,
  则 $u$ 不能在 $\varOmega$ 的内部达到正的最大值或负的最小值.
\end{exercise}

\begin{proof}
  Suppose $u$ attains its positive maximum value at some point $x_0\in\varOmega$, then
  \[\nabla u(x_0) = 0,\qquad D^2 u(x_0)\leq 0.\]
  Thus
  \[\sum_{i,j=1}^n a_{ij} u_{x_ix_j}(x_0) \leq 0,
    \quad \sum_{i=1}^n b_i u_{x_i}(x_0) = 0,\quad cu(x_0)<0.\]
  It follows that
  \[\sum_{i,j=1}^n a_{ij} u_{x_ix_j}(x_0) + \sum_{i=1}^n b_i u_{x_i}(x_0) + cu(x_0) <0,\]
  which is contradictory to the equation satisfied by $u$.

  The proof for the case when $u$ attains its negative minimum value in the interior of
  $\varOmega$ is similar.
\end{proof}


\begin{exercise}
  证明第 6 题中讨论的椭圆型方程的第一边值问题的唯一性与稳定性.
\end{exercise}

\begin{proof}
	唯一性.只要证明方程在齐次边值条件下只有零解.
	
	设$u$在$\varOmega$内满足方程,在边界$\varGamma$上$u|_\varGamma=0$. 因$u$在$\varOmega\cup\varGamma$上连续,故$u$是有界的.现证在$\varOmega$内$u\equiv0$.
	
	用反证法. 不妨设在$\varOmega$内$u>0$, 则$u$必在$\varOmega$内某一点达到正的最大值, 与第6题所述极值原理矛盾. 同理在$\Omega$内$u<0$也是不成立的, 故$u\equiv0$, 唯一性得证.
	
	稳定性. 只须证明当$f$充分小时, 满足方程及边值$u|_\Gamma=f$的函数$u$也充分小.
	
	对任意的$\varepsilon>0$,若$|f|<\varepsilon$,即$-\varepsilon<f<\varepsilon$, 根据极值原理知, 在$\varOmega$内部 必有
	\[-\varepsilon<u<\varepsilon,\]
	即$|u|<\varepsilon$,稳定性得证.
\end{proof}


\begin{exercise}
  举例说明对于方程 $\displaystyle\frac{\partial^2u}{\partial x^2} + \frac{\partial^2 u}{\partial y^2} + cu = 0(c>0)$, 不成立极值原理.
\end{exercise}

\begin{solution}
  Consider the function
  \[ u(x,y) = \sin\sqrt{\frac{c}{2}}x \sin\sqrt{\frac{c}{2}}y, \]
  which satisfies $\Delta u + cu = 0$ in the domain
  $\varOmega = \bigl[-\sqrt{\frac{2}{c}}\pi,\sqrt{\frac{2}{c}}\pi\bigr]^2$.
  However, it attains its maximum in the interior of $\varOmega$.
\end{solution}


\section{格林函数}

\begin{note}
	\[u(M_0) = \iint_{\varGamma}\left[\frac{1}{4\pi r_{M_0M}}\frac{\partial u}{\partial\bm{n}}-u\frac{\partial}{\partial\bm{n}}\frac{1}{4\pi r_{M_0M}}\right]\d S_M,\]
	\[\iint_{\varGamma}\left(g\frac{\partial u}{\partial\bm{n}}-u\frac{\partial g}{\partial\bm{n}}\right)\d S_M=0,\]
	相减得
	\[u(M_0)=\iint_{\varGamma}\left(G\frac{\partial u}{\partial\bm{n}}-u\frac{\partial G}{\partial\bm{n}}\right)\d S_M,\mbox{其中}G(M,M_0)=\frac{1}{4\pi r_{M_0M}}-g(M,M_0).\]
\end{note}


\begin{exercise}
  证明格林函数的性质 3 及性质 5.
\end{exercise}

\begin{proof}
  为了完整性,下面给出格林函数五点性质的全部证明.

  \begin{property}
    格林函数 $G(M, M_0)$ 除 $M=M_0$ 一点外处处满足方程 (1.1), 而当 $M\to M_0$ 时,
    $G(M, M_0)$ 趋于无穷大, 其阶数和 $\displaystyle \frac{1}{4\pi r_{M_0M}}$ 相同.
  \end{property}

  \begin{proof}
    除了点 $M=M_0$ 外, $\displaystyle\frac{1}{4\pi r_{M_0M}}$调和,
    又因为 $g(M,M_0)$ 在 $\varOmega$ 内调和, 故$G(M,M_0)$除了$M=M_0$外处处调和,
    由极值原理知 $g(M,M_0)$在$\varOmega$上有界,故
    \[\lim_{M\to M_0}G(M,M_0)=\lim_{M\to M_0}\left(\frac{1}{4\pi r_{M_0M}}-g(M,M_0)\right)=\infty,\]
    且和 $\displaystyle\frac{1}{4\pi r_{M_0M}}$ 同阶.
  \end{proof}

  \begin{property}
    在边界上格林函数 $G(M, M_0)$ 恒等于零.
  \end{property}

  \begin{proof}
    由 $g(M,M_0)$ 的定义知 $G(M,M_0)|_{\varGamma}=0$.
  \end{proof}

  \begin{property}
    在区域 $\varOmega$ 上成立着不等式:
    \[0 < G(M, M_0) < \frac{1}{4\pi r_{M_0M}}.\]
  \end{property}
  
  \begin{proof}
    注意到
    \[0<G(M,M_0)<\frac{1}{4\pi r_{M_0M}}\Leftrightarrow 0<g(M,M_0)<\frac{1}{4\pi r_{M_0M}}.\]
    由极值原理知$g(M,M_0)>0$是显然的,下面证明$\displaystyle g(M,M_0)<\frac{1}{4\pi r_{M_0M}}$:

    取 $\delta$ 足够小使得在 $B(M_0,\delta)$ 上成立 $\displaystyle\frac{1}{4\pi r_{M_0M}}>g(M,M_0)$,
    记 $D=\varOmega\setminus\overline{B(M_0,\delta)}$,
    则 $\displaystyle\frac{1}{4\pi r_{M_0M}}-g(M,M_0)$ 在 $D$ 上调和, 且
    \[\min_{\partial D}\left(\frac{1}{4\pi r_{M_0M}}-g(M,M_0)\right)=0.\]
    故由极值原理知在 $D$ 上成立 $\displaystyle\frac{1}{4\pi r_{M_0M}}>g(M,M_0)$,
    从而在 $\varOmega$ 上成立 $\displaystyle\frac{1}{4\pi r_{M_0M}}>g(M,M_0)$.
  \end{proof}

  \begin{property}
    格林函数 $G(M, M_0)$ 在自变量 $M$ 及参变量 $M_0$ 之间具有对称性, 即设
    $M_1$, $M_2$ 为区域中的两点, 则
    \[G(M_1, M_2) = G(M_2, M_1).\]
  \end{property}
  
  \begin{proof}
    取 $D_{\varepsilon}:=\varOmega\setminus(B(M_1,\varepsilon)\bigcup B(M_2,\varepsilon))$.
    令
    \[ w(M):=G(M,M_2), \qquad v(M):=G(M,M_1). \]
    我们只需证明 $w(M_1)=v(M_2)$.
    根据定义, $w(M)$ 和 $v(M)$ 在$D_{\varepsilon}$内调和.
    由Green第二公式可得
    \[ \iint_{\partial D_{\varepsilon}}\left(w\frac{\partial v}{\partial\bm{n}}
      -v\frac{\partial w}{\partial\bm{n}}\right)\d S = 0, \]
    即
    \begin{equation}\label{eq:3.1}
      \iint_{\partial B(M_1,\varepsilon)}
      \left(w\frac{\partial v}{\partial\bm{n}}-v\frac{\partial w}{\partial\bm{n}}\right)\d S
      = \iint_{\partial B(M_2,\varepsilon)}
        \left(v\frac{\partial w}{\partial\bm{n}}-w\frac{\partial v}{\partial\bm{n}}\right)\d S.
    \end{equation}
    我们仅考虑 \eqref{eq:3.1} 左端.
    在 $\partial B(M_1,\varepsilon)$上,
    \begin{align*}
      v(M) & = G(M,M_1) = \frac{1}{4\pi r_{MM_1}} -g(M,M_1) = O\Bigl(\frac{1}{\varepsilon}\Bigr), \\
      \frac{\partial v}{\partial\bm{n}} & = \frac{1}{4\pi\varepsilon^2} + O(1), \qquad
      w = \frac{\partial w}{\partial\bm{n}} = O(1).
    \end{align*}
    因此, 当$\varepsilon\to 0$时,
    \[ \iint_{\partial B(M_1,\varepsilon)}
      \left(w\frac{\partial v}{\partial\bm{n}}-v\frac{\partial w}{\partial\bm{n}}\right)\d S
      = \iint_{\partial B(M_1,\varepsilon)} \biggl(w\Bigl(\frac{1}{4\pi\varepsilon^2} + O(1)\Bigr)
          - O\Bigl(\frac{1}{\varepsilon}\Bigr)O(1)\biggr)
      \to w(M_1)\]
    同理, 当 $\varepsilon\to 0$ 时, \eqref{eq:3.1}右端趋于 $v(M_2)$ . 故得 $w(M_1)=v(M_2)$.
  \end{proof}

  \begin{property}
    $\displaystyle\iint_{\varGamma} \frac{\partial G(M, M_0)}{\partial n} \d S_M = -1$.

  \end{property}
  
  \begin{proof}
    设 $\varGamma_{\varepsilon}$ 是以 $M_0$为球心,
    以 $\varepsilon$ 为半径的球面, 并且其包含在 $\varOmega$ 当中, 则
    \[\begin{split}
      \iint_{\varGamma}\frac{\partial G(M,M_0)}{\partial\bm{n}}\d S_M
      & = \iint_{\varGamma}\frac{\partial}{\partial\bm{n}}\left(\frac{1}{4\pi r_{M_0M}}\right)\d S_M-\iint_{\varGamma}\frac{\partial g(M,M_0)}{\partial\bm{n}}\d S_M \\
      & = \iint_{\varGamma}\frac{\partial}{\partial\bm{n}}
          \left(\frac{1}{4\pi r_{M_0M}}\right)\d S_M\quad (\text{Theorem 2.1}) \\
      & = \iint_{\varGamma_{\varepsilon}}\frac{\partial}{\partial\bm{n}}
          \left(\frac{1}{4\pi r_{M_0M}}\right)\d S_M\quad (\text{Theorem 2.1}) \\
      & = \iint_{\varGamma_{\varepsilon}}\frac{-1}{4\pi\varepsilon^2}\d S_M = -1.
    \end{split}\]
    另法: 考虑定解问题
    \[\begin{cases}
      \Delta u=0(in\;\varOmega) \\
      u|_{\varGamma=1}
    \end{cases}\]
    由极值原理知解为 $u\equiv1$, 代入教材 (3.4) 式即得结论.
  \end{proof}

  二维情形圆的格林函数取为 $\displaystyle G(M,M_0)=\frac{1}{2\pi}\left(\ln\frac{1}{r_{M_0M}}-\ln\frac{R}{\rho_0}\frac{1}{r_{M_1M}}\right)$
  也是为了使得
  \[ \int_{\varGamma}\frac{\partial G(M,M_0)}{\partial\bm{n}}\d s = -1 \]
  成立, 证明思路同上.
\end{proof}


\begin{exercise}
  证明格林函数的对称性: $G(M_1,M_2) = G(M_2,M_1)$.
\end{exercise}

\begin{proof}
  见第一题.
\end{proof}


\begin{exercise}
  写出球的外部区域的格林函数, 并由此导出对调和方程求解球的 Dirichlet 外问题的 Poisson 公式.
\end{exercise}

\begin{solve}
	先计算球外区域的格林函数. 与狄利克雷问题的作法相仿, 设$M_0$为球外一点,$M_1$为$M_0$关于球面$\partial B$的反演点. 设$B$的半径为$R$, 以球心为坐标原点$O$, 建立球坐标系. 设$OM_0=\rho _0$, $OM_1=\rho _1$, 则
	\[\rho_1\rho_0=R^2.\]
	$M_1,M_0$的坐标分别为:$\displaystyle M_1\left(\frac{R^2}{\rho_0},\theta_0,\varphi_0\right),M_0(\rho_0,\theta_0,\varphi_0)$, 对于球面上任一点$P$, 有 $\displaystyle r_{M_1P}=\frac R{\rho_{0}}r_{M_{0}P}$, 故
	$$\begin{aligned}
		G(M,M_{0})&=\frac{1}{4\pi}\left(\frac{1}{r_{M_{0}M}}-\frac{R}{\rho_0 r_{M_{1}M}}\right)\\
		&=\frac1{4\pi}\left(\frac1{\sqrt{\rho_0^2+\rho^2-2\rho_0\rho\cos\gamma}}-\frac R{\rho_0\sqrt{\rho_1^2+\rho^2-2\rho_1\rho\cos\gamma}}\right),
	\end{aligned}$$
	其中$\rho=OM$, $\gamma$是$OM$与$OM_{0}$的夹角.
	
	再求外问题的泊松公式. 根据第二节习题4题知, 若在无穷远处
	\[u(M)=O\left(\frac{1}{r_{OM}}\right),\quad\frac{\partial u}{\partial r}=O\left(\frac{1}{r_{OM}^{2}}\right)\quad (r_{OM}\to0),\]
	则调和函数基本积分公式(2.6)仍成立, 其中$\bm{n}$指向$\partial B$的内部. 故与内问题解的积分表达式的推导完全一样, 利用格林函数得
	\[u(M_0)=-\iint_{\partial B}\frac{\partial G(M,M_0)}{\partial n}f(M)dS,\]
	其中$f$为边值, 且$\bm{n}$指向球心.
	
	因此与内问题推导完全一样 可求出$\displaystyle\frac{\partial G}{\partial \bm{n}}$, 只是法线方向相反, 即
	$$\begin{aligned}
		\frac{\partial G}{\partial \bm{n}}\bigg|_{\rho=R}
		&=-\frac{\partial G}{\partial\rho}\bigg|_{\rho=R}\\
		&=\frac{1}{4\pi}\left[\frac{\rho-\rho_{0}\cos\gamma}{(\rho_{0}^{2}+\rho^{2}-2\rho_{0}\rho\cos\gamma)^{\frac{3}{2}}}-\frac{(\rho-\rho_{1}\cos\gamma)R}{\rho_{0}(\rho_{1}^{2}+\rho^{2}-2\rho_{1}\rho\cos\gamma)^{\frac{3}{2}}}\right]\bigg|_{\rho=R}\\
		&=\frac{1}{4\pi R}\frac{R^2-\rho_0^2}{(\rho_0^2+R^2-2\rho_0R\cos\gamma)^{\frac{3}{2}}}.
	\end{aligned}$$
	所以,
	$$u(M_0)=-\frac{1}{4\pi R}\iint_{\partial B}\frac{R^2-\rho_0^2}{(\rho_0^2+R^2-2\rho_0R\cos\gamma)^{\frac{3}{2}}}f(M)\d S,$$
	或写为球面坐标形式
	$$u(M_{0})=-\frac{R}{4\pi}\int_{0}^{2\pi}\int_{0}^{2\pi}\frac{R^{2}-\rho_{0}^{2}}{(\rho_{0}^{2}+R^{2}-2\rho_{0}R\cos\gamma)^{\frac{3}{2}}}f(R,\theta,\varphi)\sin\theta \d\theta\d\varphi.$$
	其中 $\cos\gamma=\cos\theta\cos\theta_0+\sin\theta\sin\theta_0\cos(\varphi-\varphi_0)$.
	
	因为
	$$(\rho_0^2+R^2-2\rho_0R\cos\gamma)^{\frac{3}{2}}\geq(R-\rho_0)^3=(R-\rho_0)^3,$$
	所以,
	$$|u(\rho_{0},\theta_{0},\varphi_{0})|\leq-\frac{RM}{4\pi}\frac{R+\rho_{0}}{(R-\rho_{0})^{2}}\int_{0}^{2\pi}\d\varphi\int_{0}^{2\pi}\sin\theta\d\theta=RM\cdot\frac{R+\rho_{0}}{(R-\rho_0)^2}.$$
	其中取$M$,使$\left|f(R,\theta,\varphi)\right|\leq M$. 由此知,当$\rho_0\to0$时,$\displaystyle u=O\left(\frac1{\rho_0}\right)$.
	
	同理,对$r$求导后,可验证$\displaystyle\frac{\partial u}{\partial r}=O\left(\frac1{\rho_0^2}\right)$,即在无穷远处所要求的条件是满足的.
\end{solve}


\begin{exercise}
  试用格林函数法导出调和方程第二边值问题解的表达式.
\end{exercise}

\begin{solution}
  We need to solve 
  \[\begin{cases}
    \Delta u = 0, & \text{in}\ \varOmega, \\
    \displaystyle\frac{\partial u}{\partial\bm{n}} = f, & \text{on}\ \varGamma.
  \end{cases}\]
  To this end, introduce the function $g(M,M_0)$
  such that it is harmonic with respect to $M$ in $\varOmega$ and satisfies
  \[ \frac{\partial g}{\partial\bm{n}}
    = \frac{\partial}{\partial\bm{n}} \biggl(\frac{1}{4\pi r_{M_0M}}\biggr)
    \quad\text{on}\ \varGamma. \]
  Hence
  \[ u(M_0) = \iint_\varGamma \biggl(G\frac{\partial u}{\partial\bm{n}}
      - u \frac{\partial G}{\partial\bm{n}}\biggr) \d S_M \]
  is reduced to
  \[ u(M_0) = \iint_\varGamma fG \d S_M, \]
  where $\displaystyle G(M,M_0) = \frac{1}{4\pi r_{M_0M}} - g(M,M_0)$ is the Green function.
\end{solution}


\begin{exercise}
  求半圆区域上狄利克雷问题的格林函数.
\end{exercise}

\begin{solve}
  \begin{figure}
    \centering
    \includegraphics[width=0.7\textwidth]{figures/fig3.5.pdf}
    \caption{Green's function for the semicircle $D$}
    \label{fig:3.1}
  \end{figure}
  See figure~\ref{fig:3.1} for the geometric interpretation.
    Let $D$ be the semicircle centered at $O$ with radius $R$ above the $x$-axis.
  Fix any $M_0\in D$ and let $M_1$ be the dual point of $M_0$, i.e.,
  $|OM_0|\cdot|OM_1|=R^2$.
  And let $M_2$, $M_3$ be the symmetric points of $M_0$, $M_1$
  with respect to the $x$-axis, respectively.
  For simplicity, denote
  \[ r = |MM_0|,\quad r_i = |MM_i|\ \text{for}\ 1\leq i\leq 3. \]
  
  Now we verify that
  \[ G(M,M_0) := \frac{1}{2\pi} \ln\frac{r_1r_2}{rr_3} \]
  is the Green's function on $D$.
  Firstly, $G(M,M_0)$ is harmonic in $D$ except the single point $M_0$.
  On the other hand, $G(M,M_0)$ vanishes on $\partial D$,
  which is obvious on $\partial D\cap \{y=0\}$
  since on this line segment $r=r_2$ and $r_1=r_3$.
  For the curved part of $\partial D$, we should notice that
  \begin{equation}\label{eq:3.9}
  	\triangle OM_0M \sim \triangle OMM_1
  \end{equation}
  and
  \begin{equation}\label{eq:3.10}
  	\triangle OM_2M \sim \triangle OMM_3.
  \end{equation}
  Therefore by \eqref{eq:3.9} and \eqref{eq:3.10},
  \begin{equation}\label{eq:3.11}
  	\mathcolor{red}{\frac{|MM_1|}{|MM_0|}}
  	= \frac{|OM_1|}{|OM|}
  	= \frac{|OM_3|}{|OM|} 
  	= \mathcolor{red}{\frac{|MM_3|}{|MM_2|}},
  \end{equation}
  i.e., $\displaystyle \frac{r_1}{r} = \frac{r_3}{r_2}$, from which we see that $G(M,M_0) = 0$.
\end{solve}


\begin{exercise}
  利用泊松公式求边值问题
  \[\begin{cases}
    u_{xx} + u_{yy} + u_{zz} = 0, \quad x^2 + y^2 + z^2 < 1, \\
    u(r,\theta,\varphi)|_{r=1} = 3\cos2\theta +1
  \end{cases}\]
  的解.
\end{exercise}

\begin{solution}
  令 $G(x,t)$ 是 Legendre 多项式 $(P_n(x))$ 的生成函数, 即
  \begin{equation}\label{eq:3.2}
    G(x,t) = \frac{1}{\sqrt{1-2xt+t^2}} = \sum_{n=0}^\infty P_n(x) t^n.
  \end{equation}
  \eqref{eq:3.2}中对 $t$ 求导, 有
  \begin{equation}\label{eq:3.3}
    \frac{x-t}{(1-2xt+t^2)^\frac32} = \sum_{n=1}^\infty n P_n(x) t^{n-1}.
  \end{equation}
  故
  \begin{equation}\label{eq:3.4}
    \begin{aligned}
      \frac{1-t^2}{(1-2xt+t^2)^\frac32}
      & = \frac{1-2xt+t^2+2xt-2t^2}{(1-2xt+t^2)^\frac32} \\
      & = \frac{1}{\sqrt{1-2xt+t^2}} + 2t \frac{x-t}{(1-2xt+t^2)^\frac32} \\
      & = \sum_{n=0}^\infty (2n+1) P_n(x) t^n.
    \end{aligned}
  \end{equation}
  所以
  \[\frac{1-{\rho_0}^2}{(1+{\rho_0}^2-2\rho_0\cos\gamma)^\frac32} = \sum_{n=0}^\infty (2n+1) P_n(\cos\gamma) t^n,\]
  其中
  \[ \cos\gamma = \cos\theta\cos\theta_0 + \sin\theta\sin\theta_0\cos(\varphi-\varphi_0).\]
  由 Legendre 多项式的加法公式得
  \begin{equation}\label{eq:3.5}
    P_n(\cos\gamma) = \sum_{m=-n}^n \frac{(n-m)!}{(n+m)!} P_n^m(\cos\theta)
      P_n^m(\cos\theta_0) \e^{\upi m(\varphi-\varphi_0)},
  \end{equation}
  其中 $P_n^m$ 是连带 Legendre 多项式. 由 $\displaystyle P_2(x)=\frac12(3x^2-1)$ 可得
  \begin{equation}\label{eq:3.6}
    4P_2(\cos\theta) = 3\cos2\theta+1.
  \end{equation}
  由 \eqref{eq:3.4}, \eqref{eq:3.5}, \eqref{eq:3.6} 和 $\mathbb{R}^3$ 上的 Poisson 公式,
  我们得到
  \begin{equation}\label{eq:3.7}
    \begin{aligned}
      u(\rho_0,\theta_0,\varphi_0)
      & = \frac{1}{4\pi} \int_0^{2\pi} \int_0^\pi
          \frac{(1-\rho_0^2)(3\cos2\theta+1)\sin\theta}{(1+\rho_0^2-2\rho_0\cos\gamma)^{3/2}}
          \d\theta\d\varphi \\
      & = \frac{1}{4\pi} \int_0^{2\pi} \int_0^\pi
          \sum_{n=0}^\infty (2n+1) P_n(\cos\gamma)\rho_0^n (3\cos2\theta+1) \sin\theta
          \d\theta\d\varphi \\
      & = \frac{1}{\pi} \int_0^{2\pi} \int_0^\pi
          \sum_{n=0}^\infty (2n+1) \biggl(\sum_{m=-n}^n \frac{(n-m)!}{(n+m)!} P_n^m(\cos\theta)
          P_n^m(\cos\theta_0) \e^{\upi m(\varphi-\varphi_0)}\biggr) \\
      & \qquad\qquad \cdot\rho_0^n P_2(\cos\theta) \sin\theta \d\theta\d\varphi.
    \end{aligned}
  \end{equation}
  利用
  \[ \int_0^{2\pi} \e^{\upi m(\varphi-\varphi_0)} \d\varphi
      = \begin{cases}
        0,    & m\neq 0, \\
        2\pi, & m=0
      \end{cases}\]
  和 $P_n^0=P_n$ 得
  \begin{equation}\label{eq:3.8}
  	\begin{aligned}
  		u(\rho_0,\theta_0,\varphi_0)
  		& = 2 \int_0^\pi \sum_{n=0}^\infty (2n+1) P_n(\cos\theta) P_n(\cos\theta_0)
  		\rho_0^n P_2(\cos\theta) \sin\theta \d\theta \\
  		& = 2 \sum_{n=0}^\infty (2n+1) P_n(\cos\theta_0) \rho_0^n
  		\int_0^\pi P_n(\cos\theta) P_2(\cos\theta) \sin\theta \d\theta \\
  		& = 2 \sum_{n=0}^\infty (2n+1) P_n(\cos\theta_0) \rho_0^n
  		\int_{-1}^1 P_n(x) P_2(x) \d x.
  		\end{aligned}
  \end{equation}
  由 Legendre 多项式的正交性
  \[ \int_{-1}^1 P_m(x)P_n(x) \d x = \frac{2}{2n+1}\delta_{mn}\]
  可得
  \[u(\rho_0,\theta_0,\varphi_0) = 4 P_2(\cos\theta_0) \rho_0^2 = \rho_0^2 (3\cos2\theta_0+1).\qedhere\]
\end{solution}


\begin{exercise}
	求泊松方程狄利克雷问题
	\[\begin{cases}
		\Delta u = x^2y, \quad &x^2 + y^2 < a^2, \\
		u=0, \quad &x^2 + y^2 = a^2
	\end{cases}\]
	的解.
\end{exercise}

\begin{solve}
	设$u(x,y)=v(x,y)+w(x,y)$, 令$\displaystyle v(x,y)=\frac{1}{12}x^4y$, 有
	\[\Delta v=x^2y.\]
	于是
	\[v(\rho,\theta)=\frac{1}{12}\rho^5\cos^4\theta\sin\theta,\]
	当$\rho=a$时
	\[v(a,\theta)=\frac{a^5}{12}\cos^4\theta\sin\theta=\frac{a^5}{12}(\sin\theta-2\sin^3\theta+\sin^5\theta)=\frac{a^5}{192}(\sin5\theta+3\sin3\theta+2\sin\theta).\]
	则$w(\rho,\theta)$满足
	\[\begin{cases}
		\Delta w = 0, \quad &\rho < a, \\
		\displaystyle w=-\frac{a^5}{192}(\sin5\theta+3\sin3\theta+2\sin\theta), \quad &\rho = a.
	\end{cases}\]
	令$\displaystyle\varepsilon=\frac{\rho_0}{a}$, 圆上Dirichlet问题解的表达式为
	$$\begin{aligned}
		w(\rho_{0},\theta_{0})&=\frac{1}{2\pi}\int_{0}^{2\pi}\frac{(a^2-\rho_{0}^{2})f(\theta)}{a^2-2a\rho_{0}\cos(\theta-\theta_0)+\rho_0^2}\d\theta\\
		&=\frac{1}{2\pi}\int_{0}^{2\pi}\frac{(1-\varepsilon^{2})f(\theta)}{1-2\varepsilon\mathrm{cos}(\theta-\theta_0)+\varepsilon^{2}}\d\theta.
	\end{aligned}$$
	由
	$$\frac{1-\varepsilon^2}{1-2\varepsilon\cos(\theta-\theta_0)+\varepsilon^2}=1+2\sum_{n=1}^\infty\varepsilon^n\cos n(\theta-\theta_0)$$
	可得
	$$\begin{aligned}
		w(\rho_{0},\varphi_{0})&=\frac{1}{2\pi} \int_{0}^{2\pi}f(\varphi) \left[1+2\sum_{n=1}^{\infty}\varepsilon^{n}\cos n(\theta-\theta_{0}) \right]\d\varphi \\
		&=\sum_{n=0}^\infty\left(\frac{\rho_0}a\right)^n\left[\cos n\theta_0\cdot\frac1\pi\int_0^{2\pi}f(\theta)\cos n\theta\d\theta+\sin n\theta_0\cdot\frac1\pi\int_0^{2\pi}f(\theta)\sin n\theta\d\theta\right] \\
		&=\sum_{n=0}^\infty(A_n\cos n\theta_0+B_n\sin n\theta_0)\left(\frac{\rho_0}a\right)^n,
	\end{aligned}$$
	其中
	$$A_n=\frac1{\pi}\int_0^{2\pi}f(\theta)\cos n\theta\d\theta,\quad B_n=\frac1{\pi }\int_0^{2\pi}f(\theta)\sin n\theta\d\theta.$$
	代入初始条件,可得
	\[w|_{\rho=a}=\sum_{n=0}^\infty\left(A_n\cos n\varphi+B_n\sin n\varphi\right)=-\frac{a^5}{192}(\sin5\theta+3\sin3\theta+2\sin\theta).\]
	比较系数,得到
	\[\begin{cases}
		C_0=0, \\
		A_n=0, \\
		\displaystyle B_1=-\frac{2a^5}{192},B_3=-\frac{3a^5}{192},B_5=-\frac{a^5}{192},B_n=0(n\neq 1,3,5),
	\end{cases}\]
	于是
	\[w=-\frac{1}{192}(\rho^5\sin5\theta+3a^2\rho^3\sin3\theta+2a^4\rho\sin\theta).\]
	由叠加原理,
	\[u(\rho,\theta)=\frac{1}{192}\left[3(\rho^2-a^2)\rho^3\sin3\theta+2(\rho^4-a^4)\rho\sin\theta\right].\]
	或者写成直角坐标形式:
	\[u(x,y)=y(11x^4+10x^2y^2-y^4-9a^2x^2+3a^2y^2-2a^4).\qedhere\]
\end{solve}


\begin{exercise}[10]
  试求一函数 $u$, 使其在半径为 $\alpha$ 的圆的内部是调和的,
  而且在圆周 $C$ 上取下列的值:
  \begin{enumerate}[(1)]
    \item $u|_C = A\cos\varphi$;
    \item $u|_C = A + B\sin\varphi$.
  \end{enumerate}
  其中 $A$, $B$ 都是常数.
\end{exercise}


\begin{solution}
  利用泊松公式(3.13)式
  \[u(\rho_0,\varphi_0)=\frac{1}{2\pi}\int_0^{2\pi}\frac{(R^2-\rho_0^2)f(\varphi)}{R^2+\rho_0^2-2R\rho_0\cos(\varphi-\varphi_0)}\d\varphi\]
  (1)
  \begin{align*}
    u(\rho_0,\varphi_0)
    & = \frac{1}{2\pi}\int_0^{2\pi}\frac{(R^2-\rho_0^2)A\cos(\varphi)}{R^2+\rho_0^2-2R\rho_0\cos(\varphi-\varphi_0)}\d\varphi \\
    & = \frac{A(R^2-\rho_0^2)}{2\pi}\int_0^{2\pi}\frac{\cos(\varphi)}{R^2+\rho_0^2-2R\rho_0\cos(\varphi-\varphi_0)}\d\varphi \\
    & = \frac{A(R^2-\rho_0^2)}{2\pi}\int_{0}^{2\pi}\frac{\cos(\theta+\varphi_0)}{R^2+\rho_0^2-2R\rho_0\cos\theta}\d\theta \qquad (\text{Let}\ \theta=\varphi-\varphi_0) \\
    & = \frac{A(R^2-\rho_0^2)}{2\pi}\int_{-\pi}^{\pi}\frac{\cos\theta\cos\varphi_0-\sin\theta\sin\varphi_0}{R^2+\rho_0^2-2R\rho_0\cos\theta}\d\theta \\
    & = \frac{A(R^2-\rho_0^2)}{2\pi}\int_{-\pi}^{\pi}\frac{\cos\theta\cos\varphi_0}{R^2+\rho_0^2-2R\rho_0\cos\theta}\d\theta \\
    & = \frac{2A(R^2-\rho_0^2)\cos\varphi_0}{2\pi}\int_0^{\pi}\frac{\cos\theta}{R^2+\rho_0^2-2R\rho_0\cos\theta}\d\theta \\
    & = \frac{2A(R^2-\rho_0^2)\cos\varphi_0}{2\pi}\frac{\pi\rho_0}{R(R^2-\rho_0^2)} \\
    & = \frac{A}{R}\rho_0\cos\varphi_0=\frac{A}{\alpha}\rho_0\cos\varphi_0.
  \end{align*}
  故
  \[u(\rho,\varphi)=\frac{A}{\alpha}\rho\cos\varphi.\]

  (2)因为$\displaystyle u|_C=A+B\sin\varphi=A+B\cos(\varphi-\frac{\pi}{2})$,故由叠加原理及(1)中结果知
  \[u(\rho,\varphi)=A+\frac{B}{\alpha}\rho\cos\left(\varphi-\frac{\pi}{2}\right)=A+\frac{B}{\alpha}\rho\sin\varphi.\qedhere\]
\end{solution}


\begin{exercise}
  试用静电源像法导出二维调和方程在半平面上的 Dirichlet 问题:
  \[\begin{cases}
    \Delta u = u_{xx} + u_{yy} = 0, \quad y>0, \\
    u|_{y=0} = f(x).
  \end{cases}\]
  的解, 其中 $f$ 是有界连续函数.
\end{exercise}

\begin{solve}
  注意点$M_0(x_0,y_0)$的对称点为$M_1(x_0,-y_0)$, 其中$y_0>0$. 所以, Green 函数为
  \[G(M,M_0) = \frac{1}{2\pi} \left[\ln\frac{1}{\sqrt{(x-x_0)^2+(y-y_0)^2}}
      - \ln\frac{1}{\sqrt{(x-x_0)^2+(y+y_0)^2}}\right]. \]
  对于半平面$y>0$来讲, 直线$y=0$的外法线方向是与$y$轴相反的方向, 即 $\displaystyle\frac{\partial}{\partial\bm{n}} = - \frac{\partial}{\partial y}$, 故解的表达式为
  \begin{align*}
  & \phantom{=} u(x_0,y_0) \\
  & = -\int_{\varGamma}f(x) \frac{\partial G(M,M_0)}{\partial\bm{n}} \d x \\
  & = \frac{1}{2\pi} \int_{-\infty}^{\infty}f(x) \frac{\partial}{\partial y}
      \left[\ln\frac{1}{\sqrt{(x-x_0)^2+(y-y_0)^2}}-\ln\frac{1}{\sqrt{(x-x_0)^2+(y+y_0)^2}}\right]
      \Bigg|_{y=0} \d x \\
  & = \frac{1}{2\pi} \int_{-\infty}^{\infty} f(x)
      \left(-\frac{1}{2} \frac{2(y-y_0)}{(x-x_0)^2+(y-y_0)^2}
        + \frac{1}{2} \frac{2(y+y_0)}{(x-x_0)^2+(y+y_0)^2}\right)\bigg|_{y=0} \d x \\
  & = \frac{y_0}{\pi}\int_{-\infty}^{\infty}\frac{f(x)}{(x-x_0)^2+y_0^2} \d x.\qedhere
  \end{align*}
\end{solve}


\section{调和函数的基本性质}

\begin{exercise}[2]
	证明二维调和函数的可去奇点定理:
	若 $A$ 是调和函数 $u(M)$ 的孤立奇点, 在 $A$ 点邻域中成立着
	\begin{equation}
		u(M) = o\biggl(\ln\frac{1}{r_{AM}}\biggr), \tag{$\star$}
	\end{equation}
	则此时可以重新定义 $u(M)$ 在 $M=A$ 的值, 使它在 $A$ 点也是调和的.
\end{exercise}

\begin{proof}
	Without loss of generality, we may assume that $A$ is the origin point
	and that $(\star)$ holds in $B(0,1)$.
	Let $v$ be the solution of the Dirichlet problem
	\[\begin{cases}
		\Delta v = 0, & |x|<1, \\
		v(x) = u(x), &  |x|=1.
	\end{cases}\]
	The existence of $v$ is guaranteed by the Poisson integral formula.
	Set $M = \max_{\partial B_1} |u|$. By maximum principle, $|v|\leq M$ in $B_1$.
	To complete the proof, it suffices to show that $v = u$ in $B_1\setminus \{0\}$.
	Let $w = v-u$ in $B_1\setminus \{0\}$ and $M_r:=\max_{\partial B_r} |w|$ for
	$\forall r<1$. First, we have
	\begin{equation}\label{eq:7}
		- M_r \frac{\log |x|}{\log r} \leq w(x) \leq M_r \frac{\log |x|}{\log r}
	\end{equation}
	for any $x\in \partial B_r\cup\partial B_1$.
	Note that $w(x)$ and $\log |x|$ are harmonic in $B_1\setminus B_r$,
	so the maximum principle implies that
	\[ - M_r \frac{\log |x|}{\log r} \leq w(x) \leq M_r \frac{\log |x|}{\log r}, \]
	and hence
	\begin{equation}\label{eq:8}
		|w(x)| \leq M_r \frac{\log |x|}{\log r},
	\end{equation}
	for all $x\in B_1\setminus B_r$. With
	\begin{equation}\label{eq:9}
		M_r = \max_{\partial B_r} |v-w| \leq \max_{\partial B_r} |v|
		+ \max_{\partial B_r} |w|
		\leq M + \max_{\partial B_r} |u|,
	\end{equation}
	we then have
	\begin{equation}\label{eq:10}
		|w(x)| \leq M \frac{\log |x|}{\log r}
		+ \log |x| \cdot \frac{\max_{\partial B_r} |u|}{\log r},
	\end{equation}
	for all $x\in B_1\setminus B_r$. For each fixed $x\in B_1\setminus \{0\}$,
	we take $r<|x|$ and let $r\to 0$ in \eqref{eq:10}
	to obtain $w(x) = 0$ since $u(x) = o(\log |x|)$ as $|x|\to 0$.
\end{proof}


\begin{exercise}[3]
	证明: 如果三维调和函数 $u(M)$ 在奇点 $A$ 附近表示成 $\frac{N}{r_{AM}^{\alpha}}$,
	其中常数 $0<\alpha\leq 1$, 而 $N$ 是不为零的光滑函数, 则当 $M\to A$ 时它趋于
	无穷大的阶数必与 $\frac{1}{r_{AM}}$ 同阶, 即 $\alpha=1$.
\end{exercise}

\begin{proof}
	Suppose that $\alpha<1$, then
	\[\lim_{M\to A}r_{AM}u(M) = \lim_{M\to A}Nr_{AM}^{1-\alpha} = 0.\]
	By removable singularity theorem we know that $A$ is a removable singularity,
	which leads to a contradiction.
\end{proof}


\begin{exercise}[4]
  设区域 $\varOmega$ 整个地包含在以原点 $O$ 为球心、$R$ 为半径的球 $K$ 中,
  $u(r,\theta,\varphi)$ 是此区域中的调和函数, 其中 $(r,\theta,\varphi)$
  表示 $\varOmega$ 中动点 $M$ 的球坐标. 设 $\displaystyle r_1=\frac{R^2}{r}$, 则点
  $M_1=(r_1,\theta,\varphi)$ 就是点 $M$ 关于球 $K$ 的反演点, 从 $M(r,\theta,\varphi)$
  到 $M_1(r_1,\theta,\varphi)$ 的变换称为\textbf{\heiti 逆矢径变换}或\textbf{\heiti 反演变换}. 以 $\varOmega_1$
  表示 $\varOmega$ 的反演区域, 证明:函数
  \[ v(r_1,\theta,\varphi) = \frac{R}{r_1} u\biggl(\frac{R^2}{r_1},\theta,\varphi\biggr)\]
  是区域 $\varOmega_1$ 中的调和函数 (无穷远处除外).
\end{exercise}

\begin{proof}
  Recall that in spherical coordinate system, the Laplacian can be expressed as
  \[ \Delta u = \frac{\partial^2 u}{\partial r^2} + \frac{2}{r}\frac{\partial u}{\partial r}
      + \frac{1}{r^2\sin\theta} \frac{\partial}{\partial\theta}\biggl(\sin\theta \frac{\partial u}{\partial \theta}\biggr)
      + \frac{1}{r^2\sin^2\theta} \frac{\partial^2 u}{\partial\varphi^2}. \]
  Let
  \[ v(r,\theta,\varphi) := \frac{R}{r} u\biggl(\frac{R^2}{r},\theta,\varphi\biggr), \]
  and denote $x = (R^2/r,\theta,\varphi)$, then we can compute directly to get
  \begin{align*}
    \Delta v(r,\theta,\varphi)
    & = \frac{R^5}{r^5} u_{rr}(x) + \frac{2R^3}{r^4} u_r(x)
        + \frac{1}{r^2\sin\theta} \frac{\partial}{\partial\theta}\biggl(\sin\theta \frac{R}{r} \frac{\partial u}{\partial \theta}\biggr)
        + \frac{1}{r^2\sin^2\theta} \frac{R}{r} \frac{\partial^2 u}{\partial\varphi^2} \\
    & = \frac{R^5}{r^5} \biggl(u_{rr}(x) + \frac{2r}{R^2} u_r(x)
          + \frac{r^2}{R^4\sin\theta} \frac{\partial}{\partial\theta}\biggl(\sin\theta\frac{\partial u}{\partial\theta}\biggr)
          + \frac{r^2}{R^4\sin^2\theta} \frac{\partial^2 u}{\partial \varphi^2}\biggr) \\
    & = \Bigl(\frac{R}{r}\Bigr)^5 \Delta u(x) = 0. \qedhere
  \end{align*}
\end{proof}


\begin{exercise}[5]
  利用开尔文变换及奇点可去性定理把 Dirichlet 外问题化为 Dirichlet 内问题.
\end{exercise}

\begin{proof}
  Let $\varOmega\subset\mathbb{R}^3$ be a bounded domain and
  $\varOmega\supset B(0,R)$ for some $R>0$.
  Consider the following Dirichlet outer problem
  \begin{equation}\label{eq:3.12}
    \begin{cases}
      \Delta u = 0, & \text{in}\ \varOmega^c, \\
      u = f, & \text{on}\ \partial\varOmega.
    \end{cases}
\end{equation}

  Let $v$ be the Kelvin transform of $u$.
  Suppose $\lim_{r\to\infty} u(r,\theta,\varphi)=0$, then
  \[ \lim_{r\to 0} r\cdot v(r,\theta,\varphi)
      = \lim_{r\to 0} R u\biggl(\frac{R^2}{r},\theta,\varphi\biggr) = 0. \]
  By removable singularity theorem we know $v$ is harmonic in $\widetilde{\varOmega^c}$,
  which is the reflection area of $\varOmega^c$. Hence problem~\eqref{eq:3.12}
  can be converted into
  \begin{equation}\label{eq:3.13}
    \begin{cases}
      \Delta v = 0, & \text{in}\ \widetilde{\varOmega^c}, \\
      v = \mathcal{K}[f], & \text{on}\ \partial\widetilde{\varOmega^c}.
    \end{cases}
  \end{equation}
\end{proof}


\begin{exercise}
  证明在空间一有界区域外定义的调和函数 $u$ 在无穷远处趋于零, 那么它趋于零的阶数
  至少为 $\displaystyle O\left(\frac{1}{r}\right)$.
\end{exercise}

\begin{proof}
  Let $v$ be the Kelvin transform of $u$.
  Since $u$ tends to zero at infinity, we know that the origin point is
  a removable singularity of $v$, so
  \[ \lim_{r\to 0} v = \lim_{r\to 0} \frac{R}{r} u\biggl(\frac{R^2}{r},\theta,\varphi\biggr) <\infty, \]
  which means
  \[ u\biggl(\frac{R^2}{r},\theta,\varphi\biggr) = O(r) \qquad \text{as}\ r\to 0. \]
  In other words,
  \[ u(r,\theta,\varphi) = O(1/r) \qquad \text{as}\ r\to\infty. \qedhere \]
\end{proof}

\begin{exercise}
  证明处处满足平均值公式 (2.13) 的连续函数一定是调和函数.
\end{exercise}

\begin{proof}
  See \cite[Theorem~1.8]{han_elliptic_2011}.
  Choose a test function $\varphi(x)=\varphi(|x|)\in C_0^\infty(B_1(0))$ and such that
  $\displaystyle\int_{B_1(0)} \varphi =1$. Define
  \[ \varphi_\varepsilon(x) = \frac{1}{\varepsilon^n} \varphi\biggl(\frac{x}{\varepsilon}\biggr)
      \in C_0^\infty(B_{\varepsilon}(0)). \]
  For any $x\in\Omega$, take $\varepsilon<\dist(x,\partial\Omega)$. Then we have
  \begin{align*}
    \int_\Omega u(y) \varphi_\varepsilon(y-x) \d y
    & = \int_{\Omega-\{x\}} u(x+y) \varphi_\varepsilon(y) \d y \\
    & = \int_{|y|<\varepsilon} u(x+y) \varphi_\varepsilon(y) \d y \\
    & = \int_{|y|<\varepsilon} \frac{1}{\varepsilon^n} u(x+y)
        \varphi\biggl(\frac{y}{\varepsilon}\biggr) \d y \\
    & = \int_{|z|<1} u(x+\varepsilon z) \varphi(z) \d z \\
    & = \int_0^1 \int_{\partial B_r} u(x+\varepsilon z) \varphi(z) \d S_z \d r \qquad (\text{let }z=rw)\\
    & = \int_0^1 r^{n-1} \int_{\partial B_1} u(x+\varepsilon rw) \varphi(rw) \d S_w \d r
        \qquad (\varphi\ \text{is radial}) \\
    & = \int_0^1 r^{n-1} \varphi(r) \int_{\partial B_1} u(x+\varepsilon rw) \d S_w \d r
        \qquad (u\ \text{satisfies MVP}) \\
    & = \int_0^1 r^{n-1} \varphi(r) \omega_n u(x) \d r \\
    & = u(x),
  \end{align*}
  i.e., $u(x) = (u*\varphi)(x)$ which means that $u$ is in fact a smooth function.
  Now using the MVP and divergence theorem we have
  \begin{align*}
    \int_{B_r(x)} \Delta u
    & = r^{n-1} \frac{\partial}{\partial r}\int_{|w|=1} u(x+rw) \d S_w \\
    & = r^{n-1} \frac{\partial}{\partial r}\bigl(\omega_n u(x)\bigr) = 0
  \end{align*}
  for any $B_r(x)\subset\Omega$. Hence $\Delta u = 0$.
\end{proof}


\section{霍普夫极值原理,第二边值问题解的唯一性}

\begin{exercise}
  试用霍普夫极值原理来证明极值原理: 对不恒等于常数的调和函数$u(x,y,z)$,
  其在区域 $\varOmega$ 的任何内点上的值不可能达到它在 $\varOmega$ 上的上界或下界.
\end{exercise}

\begin{proof}
  假设调和函数 $u(x,y,z)$ 不恒等于常数, 且在区域 $\varOmega$ 内部某点达最小值 $m$, 记
  \[E = \{ M\in\varOmega \mid u(M) = m\}.\]
  则由 $u$ 的连续性知 $E$ 是相对闭集, 由于 $u$ 不恒为常数, 故 $\varOmega\setminus E$ 为非空开集,
  取点 $M_0\in\varOmega\setminus E$ 使得 $\dist(M_0, E) < \dist(M_0, \partial\varOmega)$.
  取以点 $M_0$ 为球心, 以 $\dist(M_0, E)$ 为半径的球 $B$, 取 $M_1\in \partial B\cap E$.
  对于 $B$ 内任一点 $M$ 均有 $u(M)>u(M_1)$, 故由强极值原理
  \[\frac{\partial u}{\partial\bm{\nu}}\bigg|_{M_1} > 0,\]
  其中 $\bm{\nu}$ 与 $B$ 在点 $M_1$ 处的内法线方向成锐角,
  但是由于 $M_1$ 是 $\varOmega$ 内部的最小值点, 故对于任意方向 $\bm{l}$ 均有
  \[\frac{\partial u}{\partial\bm{l}}\bigg|_{M_1} = 0.\]
  矛盾.
\end{proof}


\begin{exercise}
  利用极值原理及霍普夫极值原理证明: 当区域 $\varOmega$ 的边界 $\varGamma$
  满足定理 5.2 中的条件时, 调和方程第三边值问题
  \[\biggl(\frac{\partial u}{\partial \bm{n}} + \sigma u\biggr)\bigg|_{\varGamma}
    = f \quad (\sigma>0)\]
  的解的唯一性.
\end{exercise}

\begin{proof}
  只需要证明满足边界条件 $(\frac{\partial u}{\partial\bm{n}}+\sigma u)|_{\varGamma}=0$ 的只有零解即可,
  下面分两种情况讨论.

  对于第三边值问题的内问题: 假设 $u$ 不恒为常数, 则由极值原理知 $u$ 在 $\varGamma$ 上取得最大值和最小值,
  记在 $M_1$ 处取到最小值, 在 $M_2$ 处取得最大值, 则
  \[\frac{\partial u}{\partial\bm{n}}(M_1) + \sigma u(M_1) = 0
    \Rightarrow u(M_1) = -\frac{1}{\sigma}\frac{\partial u}{\partial\bm{n}}(M_1)>0.\]
  \[\frac{\partial u}{\partial\bm{n}}(M_2) + \sigma u(M_2) = 0
    \Rightarrow u(M_2) = -\frac{1}{\sigma}\frac{\partial u}{\partial\bm{n}}(M_2)<0.\]
  故 $u(M_2)<u(M_1)$, 矛盾, 故假设不成立, 所以
  \[u\equiv C\Rightarrow\frac{\partial u}{\partial\bm{n}}\bigg|_{\varGamma} = 0
    \Rightarrow u|_{\varGamma}=0\Rightarrow u\equiv 0.\]
  
  对于第三边值问题的外问题: 记边界 $\varGamma$ 的外部为 $\varOmega'$,
  假设存在 $M_0\in\varOmega'$, 使得 $u(M_0)>0$, 由于 $\lim_{M\to\infty}u(M)=0$,
  故存在充分大的 $R$, 使得在 $\varGamma_R = \{r=R\}$ 上成立 $|u|<u(M_0)$,
  则由极值原理知 $u$ 的最大值只能在 $\varGamma$ 上取, 设最大值点为 $M_1\in\varGamma$, 则
  \[\frac{\partial u}{\partial\bm{n}}\bigg|_{M_1}+\sigma u(M_1)>0,\]
  与边界条件相矛盾, 假设 $u(M_0)<0$ 同样可以导出矛盾, 故$u\equiv 0$.
\end{proof}


\begin{exercise}
  说明在证明霍普夫极值原理过程中, 不可能作出一个满足条件 (1) 和 (3) 的辅助函数 $v(x,y,z)$,
  使它在整个球 $\closure{B}_R = \{x^2+y^2+z^2 \leq R^2\}$ 内满足 $\Delta v>0$.
\end{exercise}

\begin{proof}
  若在 $B_R$ 上有 $\Delta v>0$, 则
  \[\max_{\closure{B}_R} v = \max_{\partial B_R} v = 0.\]
  又因为 $\frac{\partial v}{\partial r} < 0$, 所以
  \[\min_{\closure{B}_R} v = \min_{\partial B_R} v = 0.\]
  结合二者即得 $v\equiv 0$, 与 $\Delta v>0$ 矛盾.
\end{proof}


\begin{exercise}
  设 $\varOmega$ 为 $\mathbb{R}^3$ 中有界区域, 边界为 $\varGamma$, $u$ 为定解问题
  \[\begin{cases}
    - \Delta u + cu = f, \\
    \displaystyle\biggl(\frac{\partial u}{\partial \bm{n}} + \sigma u\biggr)\biggm|_{\partial\varOmega} = g
  \end{cases}\]
  的解, 其中 $c,f,g,\sigma>0$, 证明在 $\closure{\varOmega}$ 上 $u>0$.
\end{exercise}

\begin{proof}
  Let $x_0$ be the minimum point of $u$ in $\overline{\varOmega}$.
  \begin{enumerate}[(1)]
    \item If $x_0\in\varOmega$ and $u(x_0)\leq 0$, then $-\Delta u(x_0)\leq 0$, $cu(x_0)\leq 0$,
      which is contradictory to $f>0$;
    \item If $x_0\in\partial\varOmega$ and $u(x_0)\leq 0$,
      then $\frac{\partial u(x_0)}{\partial\bm{n}}\leq 0$ and $\sigma u(x_0)\leq 0$,
      which is contradictory to $g>0$.
  \end{enumerate}
  Therefore $u>0$ in $\overline{\varOmega}$.
\end{proof}


\begin{exercise}
  举例说明: 当 $\sigma>0$ 不成立时 (但 $\sigma$ 不恒等于零), 调和方程满足边界条件
  $\displaystyle\bigl(\frac{\partial u}{\partial\bm{n}}+\sigma u\bigr)\big|_{\partial\varOmega}=g$
  的解可以不唯一.
\end{exercise}

\begin{proof}
  It suffices to show that there exists non-trivial solution to the following problem
  \[\begin{cases}
    \Delta u = 0, & \text{in}\ \varOmega, \\
    \frac{\partial u}{\partial \bm{n}}+\sigma u = 0, & \text{on}\ \partial\varOmega,
  \end{cases}\]
  where $\varOmega = B(0,R')\setminus\overline{B(0,R)}$ and $R'>R>0$ are constants.
  Let $u = 1/r$. Obviously $u$ is harmonic in $\varOmega$ and
  \[\frac{\partial u}{\partial \bm{n}} =
  \begin{cases}
     -\frac{1}{R'^2}, & \text{on}\ \partial B(0,R'), \\
     \frac{1}{R^2}, & \text{on}\ \partial B(0,R).
  \end{cases}\]
  Define
  \[ \sigma = \begin{cases}
    \frac{1}{R'}, & \text{on}\ \partial B(0,R'), \\
    -\frac{1}{R}, & \text{on}\ \partial B(0,R),
  \end{cases}\]
  then $\bigl(\frac{\partial u}{\partial \bm{n}}+\sigma u\bigr)\big|_{\partial\varOmega} = 0$.
\end{proof}


\begin{exercise}
  对于一般的椭圆型方程
  \[\sum_{i,j=1}^n a_{ij}\frac{\partial^2u}{\partial x_i\partial x_j}
    + \sum_{i=1}^n b_i \frac{\partial u}{\partial x_i} + cu = 0,\]
  其中矩阵 $(a_{ij})$ 正定, 即存在常数 $\alpha>0$ 使得
  \[\sum_{i,j=1}^n a_{ij}\xi_i\xi_j \geq \alpha|\xi|^2\quad
    \forall\xi\in \mathbb{R}^n.\]
  又设 $c\leq 0$, 试证明它的解也成立着霍普夫极值原理.
  也就是说, 如果 $u(M)$ 在球 $|x|<R$ 内满足上述方程, 在闭球 $|x|\leq R$ 上连续,
  在球面上一点 $M_0$ 处取到非正的最小值, 且在该点沿 $\bm{\nu}$ 方向的方向导数
  $\frac{\partial u}{\partial \bm{\nu}}$ 存在, 其中 $\bm{\nu}$ 与球的内法线方向成锐角,
  则在 $M_0$ 点有 $\frac{\partial u}{\partial \bm{\nu}}>0$.
\end{exercise}

\begin{proof}
  See \cite[Hopf's Lemma in \S6.4.2]{evans_partial_2010} and \cite[Theorem~2.5 \& Corollary~2.9]{han_elliptic_2011}.
  若 $u$ 在球面上一点 $M_0$ 取非正的最小值, 即 $u(M_0)\leq 0$,
  且对球内任一点 $M$ 有 $u(M)>u(M_0)$, 因此在 $M_0$ 点有
  \[\frac{\partial u}{\partial\bm{\nu}}\geq 0.\]
  现在需要证明上式中的等号不能成立, 构造函数
  \[v(x) = e^{-\lambda|x|^2} - e^{-\lambda R^2},\]
  其中 $\lambda$ 为待定的正常数, 则 $v$ 满足如下性质:
  \begin{enumerate}[(1)]
    \item 在球面 $|x|=R$ 上 $v=0$;
    \item 通过适当选取 $\lambda$, 可以使得在区域 $D=\{ R/2 \leq |x| \leq R\}$ 内
      \[Lv = \sum_{i,j=1}^na_{ij}\frac{\partial^2v}{\partial x_i\partial x_j}
        + \sum_{i=1}^nb_i\frac{\partial v}{\partial x_i}+cv > 0.\]
      事实上,因为
      \begin{align*}
        \frac{\partial v}{\partial x_i} & = -2\lambda x_i e^{-\lambda |x|^2}, \\
        \frac{\partial^2v}{\partial x_i\partial x_j} & = 4\lambda^2 x_i x_j 
          e^{-\lambda |x|^2},\quad i\neq j, \\
        \frac{\partial^2v}{\partial x_i^2} & = 4\lambda^2 x_i^2 e^{-\lambda |x|^2}
          -2\lambda e^{-\lambda |x|^2},
      \end{align*}
      所以
      \begin{align*}
      Lv
      & = 4\lambda^2\biggl(\sum_{i,j=1}^na_{ij}x_ix_j\biggr) e^{-\lambda |x|^2}
          - 2\lambda \sum_{i=1}^n (b_ix_i + a_{ii}) e^{-\lambda |x|^2} \\
      & \quad + c \Bigl(e^{-\lambda |x|^2} - e^{-\lambda R^2}\Bigr) \\
      & = e^{-\lambda |x|^2} \biggl( 4\lambda^2\sum_{i,j=1}^n a_{ij} x_i x_j
          - 2\lambda\sum_{i=1}^n (b_i x_i + a_{ii}) + c \left(1 - e^{-\lambda(R^2-|x|^2)}\right)\biggr).
      \end{align*}
      因为 $\sum_{i,j=1}^n a_{ij} x_i x_j \geq \alpha |x|^2 \geq \frac{\alpha R^2}{4} > 0$,
      故当 $\lambda$ 充分大时在区域 $D$ 内 $Lv > 0$.
    \item $v$ 沿球的半径方向$\frac{\partial v}{\partial r}<0$.
        于是 $\frac{\partial v}{\partial\bm{\nu}} > 0$.
  \end{enumerate}
  \medskip
  作函数
  \[\tilde{u}(M)=\varepsilon v(M)+u(M_0).\]
  在 $M_0$ 点有 $\frac{\partial\tilde{u}}{\partial\bm{\nu}}
  = \varepsilon\frac{\partial v}{\partial\bm{\nu}}>0$,
  令函数
  \begin{equation}\label{eq:3.14}
    w(M) := u(M)-\tilde{u}(M) = u(M)-\varepsilon v(M)-u(M_0).
  \end{equation}
  
  在区域 $D$ 上考察 $w(M)$:
  \medskip
  \begin{enumerate}[(1)]
    \item $Lw=Lu-\varepsilon Lv-Lu(M_0)=-\varepsilon Lv-cu(M_0)<0$;
    \item 在 $|x| = R/2$ 上由于 $u(M)>u(M_0)$,
      取 $\varepsilon$ 足够小可使得 $w(M)>0$;
    \item 在 $|x| = R$ 上 $v=0$, $u(M)>u(M_0)$, 故 $w(M)\geq 0$.
  \end{enumerate}
  \medskip
  现在证明在整个区域 $D$ 上 $w\geq 0$, 假设存在 $M_1\in D$, 使得 $w(M_1) < 0$, 于是
  \[cw(M_1)\geq 0,\quad \frac{\partial w}{\partial x_i}\bigg|_{M_1} = 0,
    \quad\left(\frac{\partial^2w}{\partial x_i\partial x_j}\right)\bigg|_{M_1}\text{ 非负定}.\]
  又 $a_{ij}=\sum_{r=1}^n g_{ri}g_{rj}$, 故
  \[\sum_{i,j=1}^na_{ij}\frac{\partial^2w}{\partial x_i\partial x_j}
    \bigg|_{M_1}=\sum_{r=1}^n\sum_{i,j=1}^n\frac{\partial^2w}{\partial x_i\partial x_j}
    \bigg|_{M_1}g_{ri}g_{rj}\geq 0,\]
  因此 $Lw|_{M_1}\geq 0$, 与(1)矛盾, 因此在 $D$ 内 $w(M)\geq w(M_0)$, 故
  \[\frac{\partial w}{\partial\bm{\nu}}\bigg|_{M_0} \geq 0.\]
  从而与 \eqref{eq:3.14} 结合得
  \[\frac{\partial u}{\partial\bm{\nu}}\bigg|_{M_0} > 0. \qedhere\]
\end{proof}


\begin{exercise}
	设$\varOmega$是具有光滑边界的有界区域, 边值问题
	\[\begin{cases}
		\Delta u-u=0,\quad\text{在}\varOmega\text{内},\\
		\displaystyle\frac{\partial u}{\partial\bm{n}}\bigg|_{\partial\varOmega}=0
	\end{cases}\]
	的解$u\in C^2(\varOmega)\cap C^1(\varOmega)$在$\varOmega$内是否可能是严格正的?
\end{exercise}

\begin{proof}
	结论: 不可能严格正
	
	(法1): 若设 $u$ 是严格正的, 对函数$u$和 $1$ 使用 Green 第一公式, 得到
	$$0=\iint_{\partial\Omega}\frac{\partial u}{\partial\bm{n}}\d s=\iiint_\varOmega\Delta u\d\varOmega=\iiint_\Omega u\d\varOmega>0$$
	得到矛盾
	
	(法2): 反设 $u$ 是严格正的, 若 $u$ 在内部达到最大值, 则在最大值点处$\Delta u<0$, 这与$\Delta u=u>0$矛盾
	
	若 $u$ 在边界达到最大值, 则由 Hopf 极值原理, $\frac{\partial u}{\partial\bm{n}}\big|_{\partial\Omega}>0$, 矛盾.
\end{proof}


\begin{exercise}
	设$\Omega$为平面上的椭圆环$\{(x,y)|1\leq x^2+2y^2\leq2\},u(x,y)\in C^2(\overline{\Omega})$是如下的边值问题的解:
	$$\begin{cases}
		\Delta u=0,\quad (x,y)\in\overline{\varOmega},\\
		u(x,y)=x+y,\quad x^2+2y^2=2,\\
		\displaystyle\frac{\partial u(x,y)}{\partial\bm{n}}+(1-x)u(x,y)=0,\quad x^2+2y^2=1.
	\end{cases}$$
	求$\displaystyle\max_{\overline{\Omega}}|u(x,y)|$.
\end{exercise}

\begin{proof}
	首先由极值原理, 最值不在$\varOmega$内部取到,
	设最大值在边界$x^2+2y^2=1$上一点$M_0$取到, 由 Hopf 极值原理, $\frac{\partial u}{\partial\bm{n}}|_{M_0}<0$
	
	则$\frac{\partial u(x,y)}{\partial\bm{n}}+(1-x)u(x,y)|_{M_{0}}<0$, 矛盾
	
	这说明$\max_{\overline{\Omega}}|u(x,y)|=\max_{x^{2}+2y^{2}=2}|x+y|=\sqrt{3}$.
\end{proof}