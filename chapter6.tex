\chapter{广义函数}
\section{基本空间}
1.\textit{Proof}:(反证法)若存在$x_0\in[a,b],s.t.f(x_0)\neq0$,不妨设$f(x_0)>0$,则由连续性知
\[\exists\delta>0,s.t.f(x)>0,\forall x\in(x_0-\delta,x_0+\delta)\cap[a,b]\]
考虑$[a,b]$内紧集$[x_0-\delta/2,x_0+\delta/2]\cap[a,b]$,知存在$\varphi\in C_0^{\infty}([a,b])$,使得$\varphi\geq0$且$\varphi$在该紧集上取1,在$(x_0-\delta,x_0+\delta)$之外取0.将$\varphi$限制在$(a,b)$上知$\varphi\in C_0^{\infty}((a,b))$,但是
\[\int f(x)\varphi(x)\diff x\geq\int_{x_0-\delta/2}^{x_0+\delta/2}f(x)\varphi(x)\diff x>0\]
矛盾,故假设不成立,故$f(x)\equiv0$\\
4.\textit{Proof}:因为
\[\begin{split}
\int|f_{\epsilon}(x)-f(x)|\diff x
&=\int\left|\int(f(x-\tau)-f(x))\varphi_{\epsilon}(\tau)\diff\tau\right|\diff x\\
&\leq\int\left|\int|f(x-\tau)-f(x)||\varphi_{\epsilon}(\tau)|\diff\tau\right|\diff x\\
&=\iint|f(x-\tau)-f(x)|\diff x|\varphi_{\epsilon}(\tau)|\diff\tau\\
&=\iint|f(x-\epsilon s)-f(x)|\diff x\frac{|\varphi(s)|}{\|\varphi\|_{L_1}}\diff s
\end{split}\]
又
\[\int|f(x-\epsilon s)-f(x)|\diff x\leq 2\|f\|_{L_1}\]
\[\lim_{\epsilon\to0}\int|f(x-\epsilon s)-f(x)|\diff x=0\]
由勒贝格控制收敛定理得$\lim_{\epsilon\to0}\|f_{\epsilon}(x)-f(x)\|_{L_1}=0$\\\\
5.\textit{Proof}:\\
以$y$代$x-ty$,则
\[\begin{split}
\partial_{x_i}(t^ku(x,t))
&=\partial_{x_i}\left(t^{k-1}\frac{1}{t^n}\int v(y)\varphi\left(\frac{x-y}{t}\right)\diff y\right)\\
&=t^{k-1}\int v(y)\partial_{x_i}\varphi\left(\frac{x-y}{t}\right)\diff y\\
&=t^{k-1}\int v(x-ty)\partial_{x_i}\varphi(y)\diff y
\end{split}\]
\[\begin{split}
\partial_t(t^ku(x,t))&=\partial_t\left(t^{k-1-n}\int v(y)\varphi\left(\frac{x-y}{t}\right)\diff y\right)\\
&=t^{k-1}\int v(x-ty)((k-n)\varphi(y)-\sum_iy_i\partial_{y_i}\varphi(y))\diff y
\end{split}\]
\newline
6.\textit{Proof}:因为
\[\partial^{\alpha}f_{\epsilon}(x)=\int\partial_x^{\alpha}\frac{1}{c\epsilon^n}\varphi\left(\frac{x-y}{t}\right)\diff y=\frac{1}{c\epsilon^n}\frac{1}{\epsilon^{|\alpha|}}\int\varphi^{(\alpha)}\left(\frac{x-y}{t}\right)\diff y\]
所以
\[|\partial^{\alpha}f_{\epsilon(x)}|\leq\frac{1}{c\epsilon^n}\frac{1}{\epsilon^{|\alpha|}}\int\left|\varphi^{(\alpha)}\left(\frac{x-y}{t}\right)\right|\diff y\]
可以命
\[C(\alpha,n)=\frac{1}{c\epsilon^n}\int\left|\varphi^{(\alpha)}\left(\frac{x-y}{t}\right)\right|\diff y=\frac{1}{c}\int\left|\varphi^{|\alpha|}{\xi}\right|\diff\xi\]
\section{广义函数及其基本运算}
1.\textit{Proof}:$\forall\varphi\in\mathscr{D}(\Omega)$,有
\[
\left\langle\frac{\partial^2u}{\partial x_i\partial x_j},\varphi\right\rangle=-\left\langle\frac{\partial u}{\partial x_i},\frac{\partial\varphi}{\partial x_j}\right\rangle
=\left\langle u,\frac{\partial^2\varphi}{\partial x_j\partial x_i}\right\rangle
=-\left\langle\frac{\partial u}{\partial x_j},\frac{\partial\varphi}{\partial x_i}\right\rangle
=\left\langle\frac{\partial^2u}{\partial x_j\partial x_i},\varphi\right\rangle
\]
故\[\frac{\partial^2u}{\partial x_i\partial x_j}=\frac{\partial^2u}{\partial x_j\partial x_i}\]
\newline
2.\textit{Proof}:$\forall\varphi\in\mathscr{D}(\Omega)$,有
\[\left\langle\frac{\partial}{\partial x_i}(au),\varphi\right\rangle=-\left\langle au,\frac{\partial\varphi}{\partial x_i}\right\rangle=-\left\langle u,a\frac{\partial\varphi}{\partial x_i}\right\rangle=\left\langle u,\varphi\frac{\partial a}{\partial x_i}-\frac{\partial(a\varphi)}{\partial x_i}\right\rangle\]
因为
\[-\left\langle u,\frac{\partial(a\varphi)}{\partial x_i}\right\rangle=\left\langle\frac{\partial u}{\partial x_i},a\varphi\right\rangle=\left\langle a\frac{\partial u}{\partial x_i},\varphi\right\rangle\]
所以
\[\left\langle\frac{\partial}{\partial x_i}(au),\varphi\right\rangle=\left\langle\frac{\partial a}{\partial x_i}u+a\frac{\partial u}{\partial x_i},\varphi\right\rangle\]
故
\[\frac{\partial}{\partial x_i}(au)=a\frac{\partial u}{\partial x_i}+\frac{\partial a}{\partial x_i}u\]
\newline
3.\textit{Proof}:\\
(1)$\forall\varphi\in\mathscr{D}(\Omega)$,有
\[\left\langle\check{\delta},\varphi\right\rangle=\left\langle\delta,\check{\varphi}\right\rangle=\check{\varphi}(0)=\varphi(0)=\left\langle\delta,\varphi\right\rangle\]
\[\left\langle\check{c},\varphi\right\rangle=\left\langle c,\check{\varphi}\right\rangle=\int c\varphi(-x)\diff x=\int c\varphi(x)\diff x=\left\langle c,\varphi\right\rangle\]
故$\delta$和$c$都是偶广义函数\\
(2)设$u$为偶广义函数,$\forall\varphi\in\mathscr{D}(\Omega)$,有
\[\left\langle\check{\left(\frac{\partial u}{\partial x_j}\right)},\varphi\right\rangle=\left\langle\frac{\partial u}{\partial x_j},\check{\varphi}\right\rangle=-\left\langle u,\frac{\partial\check{\varphi}}{\partial x_j}\right\rangle=\left\langle u,\frac{\partial\varphi}{\partial x_j}\right\rangle=-\left\langle\frac{\partial u}{\partial x_j},\varphi\right\rangle\]
故$\frac{\partial u}{\partial x_j}$为奇广义函数\\
(3)$\forall\varphi\in\mathscr{D}(\Omega)$,有
\[\left\langle\check{(u+\check{u})},\varphi\right\rangle\]