\chapter{一阶偏微分方程组}
\section{引言}
1.解:令$p=\frac{\partial u}{\partial t},q_1=\frac{\partial u}{\partial x},q_2=\frac{\partial u}{\partial y},q_3=\frac{\partial u}{\partial z}$,则
\[\begin{cases}
\frac{\partial p}{\partial t}=a^2\left(\frac{\partial q_1}{\partial x}+\frac{\partial q_2}{\partial y}+\frac{\partial q_3}{\partial z}\right)&(1)\\
\frac{\partial p}{\partial x}=\frac{\partial q_1}{\partial t},\frac{\partial p}{\partial y}=\frac{\partial q_2}{\partial t},\frac{\partial p}{\partial z}=\frac{\partial q_3}{\partial t}&(2)\\
t=0:p=\psi,q_1=\frac{\partial\varphi}{\partial x},q_2=\frac{\partial\varphi}{\partial y},q_3=\frac{\partial\varphi}{\partial z}&(3)
\end{cases}\]
原方程的解显然满足新方程,而如果新方程的解为$(p,q_1,q_2,q_3)$,则
\[u(x,y,z,t)=\varphi(x,y,z)+\int_{(0,0,0,0)}^{(x,y,x,t)}p\diff t+q_1\diff x+q_2\diff y+q_3\diff z\]
是原方程的解,其中条件(2)确保了积分与路径无关,故积分定义是合理的
\section{两个自变量的一阶线性偏微分方程组的特征理论}
1.解:特征线为$x=\int_0^ta(x,\tau)\diff\tau$\\\\
2.解:(1)令$x=\frac{1}{2}(y+z),t=\frac{1}{2}(y-z)$,则
\[u_y=u_xx_y+u_tt_y=\frac{1}{2}(u_x+u_t)=0\]
故$u=\Phi(z)=\Phi(x-t)$,令$t=0$得$\Phi(x)=\varphi(x)$,故解为$u=\varphi(x-t)$\\
(2)令$x=\frac{1}{2}(y+z),t=\frac{1}{2}(y-z)$,则
\[u_y=\frac{1}{2}(u_x+u_t)=\frac{1}{2}u\]
故$u=Ce^y+\Phi(z)=Ce^{x+t}+\Phi(x-t)$,令$t=0$得$Ce^x+\Phi(x)=\varphi(x)$,故解为
\[u=Ce^{x+t}-Ce^{x-t}+\varphi(x-t)\]
\newline
3.解:
\[A=\begin{pmatrix}
a&-b\\b&a
\end{pmatrix}\Rightarrow(\lambda-a)^2+b^2=0\]
故当$b=0$时为双曲型,当$b\neq0$时为椭圆型\\\\
4.解:\\
(1)\[A=\begin{pmatrix}
1+\sin x&2\\0&0\end{pmatrix}\Rightarrow\lambda=0\mbox{或}\lambda=1+\sin x\]
相应的特征向量为$(-2,1+\sin x)^T,(c,0)^T$,作变换
\[\begin{cases}
u=-2v_1+cv_2\\
v=(1+\sin x)v_1
\end{cases}\]
则得
\[\begin{cases}
(1+\sin x)\frac{\partial v_1}{\partial t}-2v_1+cv_2=0\\
c(1+\sin x)\frac{\partial v_2}{\partial t}+c(1+\sin x)^2\frac{\partial v_2}{\partial x}+x(1+\sin x)-4v_1+2cv_2=0
\end{cases}\]
(2)\[A=\begin{pmatrix}
x&1\\a^2&x
\end{pmatrix}\Rightarrow(\lambda-x)^2-a^2=0\Rightarrow\lambda=x\pm a\]
对应特征向量为$(-1,a)^T,(1,a)^T$,作变换
\[\begin{cases}
u=-v_1+v_2\\
v=av_1+av_2
\end{cases}\]
则得
\[\begin{cases}
\frac{\partial v_1}{\partial t}=(x-a)\frac{\partial v_1}{\partial x}\\
\frac{\partial v_2}{\partial t}=(x+a)\frac{\partial v_2}{\partial x}
\end{cases}\]
(3)\[A=
\begin{pmatrix}
6&5&0\\5&6&0\\-1&0&2
\end{pmatrix}\Rightarrow
\begin{vmatrix}
\lambda-6&-5&0\\-5&\lambda-6&0\\1&0&\lambda-2
\end{vmatrix}=0\Rightarrow(\lambda-2)(\lambda^2-12\lambda+11)=0\Rightarrow\lambda=1,2,11\]
对应特征向量为$(1,-1,1)^T,(0,0,1)^T,(9,9,-1)^T$,作变换
\[\begin{cases}
u_1=v_1+9v_3\\
u_2=-v_1+9v_3\\
u_3=v_1+v_2-v_3
\end{cases}\]
\[R=\begin{pmatrix}
1&0&9\\-1&0&9\\1&1&-1
\end{pmatrix}
\Rightarrow
R^{-1}=
\begin{pmatrix}
1/2&-1/2&0\\-4/9&5/9&1\\1/18&1/18&0
\end{pmatrix}\]
又因为
\[B=\begin{pmatrix}
0&0&0\\-2&0&0\\1&-2/3&-1
\end{pmatrix},C=\begin{pmatrix}
0\\0\\0
\end{pmatrix}\]
所以
\[R^{-1}BR=\begin{pmatrix}
1&0&9\\-4/9&-1&-6\\-1/9&0&-1
\end{pmatrix},R^{-1}C=\begin{pmatrix}
0\\0\\0
\end{pmatrix}\]
即得对角型方程组
\[\begin{cases}
\frac{\partial v_1}{\partial t}+\frac{\partial v_1}{\partial x}+v_1+9v_3=0\\
\frac{\partial v_2}{\partial t}+2\frac{\partial v_2}{\partial x}-\frac{4}{9}v_1-v_2-6v_3=0\\
\frac{\partial v_3}{\partial t}+11\frac{\partial v_3}{\partial x}-\frac{1}{9}v_1-v_3=0
\end{cases}\]\\\\
5.\textit{Proof}:原方程为
\[\frac{\partial U}{\partial t}+A\frac{\partial U}{\partial x}+BU+C=0\]
未知函数作可逆线性变换$U=RV$后,有
\[R\frac{\partial V}{\partial t}+AR\frac{\partial V}{\partial x}+\left(\frac{\partial R}{\partial t}+A\frac{\partial R}{\partial x}+BR\right)V+C=0\]
原特征方程为$\det(A-\lambda I)=0$,新的特征方程仍然不变,因此方程组在每一点的特征线方向保持不变