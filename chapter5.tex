\chapter{一阶偏微分方程组}

\section{引言}

\begin{exercise}
  把波动方程
  \[\frac{\partial^2u}{\partial t^2} = a^2 
    \biggl(\frac{\partial^2u}{\partial x^2} + \frac{\partial^2u}{\partial y^2}
    + \frac{\partial^2u}{\partial z^2}\biggr)\]
  带初始条件
  \[\begin{cases}
    u|_{t=0} = \varphi(x,y,z), \\
    \frac{\partial u}{\partial t}|_{t=0} = \psi(x,y,z)
  \end{cases}\]
  的柯西问题化为一个一阶方程组的柯西问题, 并证明其解的等价性.
\end{exercise}

\begin{proof}
  令 $p=\frac{\partial u}{\partial t}$, $q_1=\frac{\partial u}{\partial x}$,
  $q_2=\frac{\partial u}{\partial y}$, $q_3=\frac{\partial u}{\partial z}$, 则
  \[\begin{cases}
    \frac{\partial p}{\partial t} = a^2\left(\frac{\partial q_1}{\partial x}
      +\frac{\partial q_2}{\partial y}+\frac{\partial q_3}{\partial z}\right) & (1) \\
    \frac{\partial p}{\partial x} = \frac{\partial q_1}{\partial t},
    \frac{\partial p}{\partial y}=\frac{\partial q_2}{\partial t},
    \frac{\partial p}{\partial z}=\frac{\partial q_3}{\partial t} & (2) \\
    t = 0: p=\psi, q_1 = \frac{\partial\varphi}{\partial x},
                   q_2 = \frac{\partial\varphi}{\partial y},
                   q_3 = \frac{\partial\varphi}{\partial z} & (3)
  \end{cases}\]
  原方程的解显然满足新方程,而如果新方程的解为 $(p,q_1,q_2,q_3)$,则
  \[u(x,y,z,t) = \varphi(x,y,z)
    + \int_{(0,0,0,0)}^{(x,y,x,t)}p\d t+q_1\d x+q_2\d y+q_3\d z\]
  是原方程的解, 其中条件 (2) 确保了积分与路径无关, 故积分定义是合理的.
\end{proof}


\begin{exercise}
  把方程
  \[ u_{tt} = u_{x}^2 + u_{y}^2 \]
  带初始条件
  \[\begin{cases}
    u|_{t=0} = 0, \\
    u_t|_{t=0} = e^x \sin y
  \end{cases}\]
  的柯西问题化为一个一阶偏微分方程组的柯西问题.
\end{exercise}

\begin{solution}
  Let $p=u_t$, then
  \[\begin{cases}
    p = u_t, \\
    p_t = u_x^2 + u_y^2.
  \end{cases}\]
  And the initial value condition is
  \[\begin{cases}
    u|_{t=0} = 0, \\
    p|_{t=0} = e^x \sin y.
  \end{cases}\qedhere\]
\end{solution}


\begin{exercise}
  证明柯瓦列夫斯卡娅型方程 (1.9) 满足初始条件
  \begin{equation}
    t=0: u=\varphi_0(x),\ldots, \frac{\partial^{m-1}u}{\partial t^{m-1}} = \varphi_{m-1}(x) \tag{$\star$}
  \end{equation}
  的柯西问题可以化为一阶方程组的柯西问题, 并证明其解的等价性.
\end{exercise}

\begin{proof}
  Let $\alpha = (\alpha_0,\alpha_1,\ldots,\alpha_n)$ be the general multi-index
  and let 
  \begin{align*}
    \alpha^0 & := (\alpha_0+1,\alpha_1,\ldots,\alpha_n), \\
    \alpha^i & := (\alpha_0,\alpha_1,\ldots,\alpha_i+1,\ldots,\alpha_n).
  \end{align*}
  Decompose $\alpha$ as follows:
  \[ \alpha = \beta + \gamma, \]
  where $\beta=(\alpha_0,0,\ldots,0)$ and $\gamma=(0,\alpha_1,\ldots,\alpha_n)$.
  We denote the special multi-index $\alpha^* = (m-1,0,\ldots,0)$
  and introduce new functions by $u_\alpha = D^\alpha u$.
  Then the Kowalevskaya type equation can be transformed into
  the following first-order system
  \begin{equation}\label{eq:5.1}
    \begin{cases}
      \frac{\partial u_\alpha}{\partial t} = u_{\alpha^0}, \quad |\alpha|\leq m-1, \alpha_0\leq m-2, \\
      \frac{\partial u_\alpha}{\partial x_i} = u_{\alpha^i}, \quad |\alpha|\leq m-1, \\
      \frac{\partial u_{\alpha^*}}{\partial t} = F(t,x,u,u_\alpha,|\alpha|\leq m, \alpha_0\leq m-1),
    \end{cases}
  \end{equation}
  together with the following initial value conditions
  \begin{equation}\label{eq:5.2}
    \begin{cases}
      u|_{t=0} = \varphi_0(x), \\
      u_\alpha|_{t=0} = D^\gamma \varphi_{\alpha_0}(x),\quad 1\leq |\alpha|\leq m, \alpha_0\leq m-1. 
    \end{cases}
  \end{equation}

  Now we prove that equivalence of solutions. First of all,
  if $u$ is a solution to (1.9) in the textbook with initial value condition $(\star)$,
  it is straightforward to verify that $(u_\alpha)$ is the solution to~\eqref{eq:5.1} 
  and~\eqref{eq:5.2}.

  Conversely, let $(u_\alpha)$ be the solution to~\eqref{eq:5.1} and \eqref{eq:5.2}.
  Then for all $|\alpha|\leq m-1$ with $\alpha_0\leq m-2$ we have
  \begin{equation}\label{eq:5.3}
    \begin{cases}
      \frac{\partial u_\alpha}{\partial t} = u_{\alpha^0}, \\
      \frac{\partial u_{\alpha^0}}{\partial x_i} = \frac{\partial u_{\alpha^i}}{\partial t}.
    \end{cases}
  \end{equation}
  It follows that
  \begin{equation}\label{eq:5.4}
    \frac{\partial}{\partial t}\biggl(u_{\alpha^i} - \frac{\partial u_\alpha}{\partial x_i}\biggr)
      = 0.
  \end{equation}
  So $u_{\alpha^i} - \frac{\partial u_\alpha}{\partial x_i}$ is independent of $t$.
  From \eqref{eq:5.2} we know that it is equal to zero at $t=0$, thus
  \begin{equation}\label{eq:5.5}
    u_{\alpha^i} \equiv \frac{\partial u_\alpha}{\partial x_i} \quad \text{for all}\ t\geq 0.
  \end{equation}
  Plugging \eqref{eq:5.5} into \eqref{eq:5.1} we find that $u$ is a solution to (1.9)
  in the textbook with the given initial value condition.
\end{proof}

\section{两个自变量的一阶线性偏微分方程组的特征理论}

\begin{exercise}
  求一阶方程
  \begin{enumerate}[(1)]
    \item $\frac{\partial u}{\partial t} + a(x,t) \frac{\partial u}{\partial x} 
      + b(x,t)u + c(x,t) = 0$,
    \item $\frac{\partial u}{\partial t} + a(x,t) \frac{\partial u}{\partial x}
      + b(x,t,u) = 0$
  \end{enumerate}
  的特征线和解沿特征线应成立的关系式.
\end{exercise}

\begin{solve}
  (1) 特征线满足的方程为 $\frac{\d x}{\d t} = a(x,t)$,
  解得特征线为 $x = \int_0^t a(x,\tau) \d\tau$.
  在特征线上, $u(x,t) = u\Bigl(\int_0^t a(x, \tau) \d\tau, t\Bigr)$, 故
  \[\frac{\d u}{\d t} = \frac{\partial u}{\partial x} a(x,t)
    + \frac{\partial u}{\partial t}.\]
  故 $u$ 在特征线上满足关系式
  \[\frac{\d u}{\d t} + b(x,t)u + c(x,t) = 0.\]

  (2) 同理 $\frac{\d u}{\d t} + b(x,t,u) = 0$.
\end{solve}


\begin{exercise}
  求下列一阶方程带初始条件 $u|_{t=0} = \varphi(x)$ 的柯西问题的解:
  \begin{enumerate}[(1)]
    \item $\frac{\partial u}{\partial t} + \frac{\partial u}{\partial x} = 0$;
    \item $\frac{\partial u}{\partial t} + \frac{\partial u}{\partial x} = u$.
  \end{enumerate}
\end{exercise}

\begin{solve}
  (1) 特征线为 $x = t+C$, 在特征线上, $\frac{\d u}{\d t} = 
    \frac{\partial u}{\partial x} + \frac{\partial u}{\partial t} = 0$,
  故 $u$ 在特征线上为常数. 故
  \[u(x_0, t_0) = u(x_0-t_0, 0) = \varphi(x_0-t_0).\]
  因此 $u(x,t) = \varphi(x-t)$.

  (2) 特征线为 $x = t+C$, 在特征线上, $\frac{\d u}{\d t} = u$,
  故 $u = C\e^t$, 令 $t=0$, 得 $u|_{t=0} = C$. 故
  \[u(x_0, t_0) = u(x_0-t_0, 0) \e^{t_0} = \varphi(x_0-t_0)\e^{t_0}.\]
  因此
  \[u(x, t) = \varphi(x-t) \e^t. \qedhere\]
\end{solve}


\begin{exercise}
  判断方程组
  \begin{align*}
    & \frac{\partial u_1}{\partial t} = a(x,t) \frac{\partial u_1}{\partial x}
      - b(x,t) \frac{\partial u_2}{\partial x} + f_1, \\
    & \frac{\partial u_2}{\partial t} = b(x,t) \frac{\partial u_1}{\partial x}
      + a(x,t) \frac{\partial u_2}{\partial x} + f_2
  \end{align*}
  属于何种类型.
\end{exercise}

\begin{solve}
  \[A = \begin{pmatrix}
    a & -b \\
    b & a
  \end{pmatrix} \Rightarrow (\lambda-a)^2+b^2 = 0.\]
  故当 $b=0$ 时为双曲型, 当 $b\neq 0$ 时为椭圆型.
\end{solve}


\begin{exercise}
  将下列各方程组化为对角型方程组:
  \begin{enumerate}[(1)]
    \item $\begin{cases}
             \frac{\partial u}{\partial t} + (1+\sin x)\frac{\partial u}{\partial x}
                + 2 \frac{\partial v}{\partial x} + x = 0, \\
             \frac{\partial v}{\partial t} + u = 0;  
           \end{cases}$
    \item $\begin{cases}
             \frac{\partial u}{\partial t} = x \frac{\partial u}{\partial x}
               + \frac{\partial v}{\partial x}, \\
             \frac{\partial v}{\partial t} = a^2 \frac{\partial u}{\partial x}
               + x \frac{\partial v}{\partial x}\quad (a>0);
           \end{cases}$
    \item $\begin{cases}
             \frac{\partial u_1}{\partial t} + 6 \frac{\partial u_1}{\partial x}
               + 5 \frac{\partial u_2}{\partial x} = 0, \\
             \frac{\partial u_2}{\partial t} + 5 \frac{\partial u_1}{\partial x}
               + 6 \frac{\partial u_2}{\partial x} = 2u_1, \\
             3 \frac{\partial u_3}{\partial t} + 6 \frac{\partial u_3}{\partial x}
               - 3 \frac{\partial u_1}{\partial x} = 2u_2 + 3u_3 - 3u_1.
           \end{cases}$
  \end{enumerate}
\end{exercise}

\begin{solve}
  (1) \[A = \begin{pmatrix}
  1+\sin x&2\\0&0\end{pmatrix}\Rightarrow\lambda=0\text{\ 或\ }\lambda=1+\sin x.\]
  相应的特征向量为$(-2,1+\sin x)^T,(c,0)^T$,作变换
  \[\begin{cases}
  u=-2v_1+cv_2\\
  v=(1+\sin x)v_1
  \end{cases}\]
  则得
  \[\begin{cases}
  (1+\sin x)\frac{\partial v_1}{\partial t}-2v_1+cv_2=0, \\
  c(1+\sin x)\frac{\partial v_2}{\partial t}+c(1+\sin x)^2\frac{\partial v_2}{\partial x}
    + x(1+\sin x)-4v_1+2cv_2=0.
  \end{cases}\]

  (2) \[A=\begin{pmatrix}
        x & 1 \\ a^2 & x
  \end{pmatrix}\Rightarrow(\lambda-x)^2-a^2=0\Rightarrow\lambda=x\pm a\]
  对应特征向量为 $(-1,a)^T$, $(1,a)^T$,作变换
  \[\begin{cases}
  u=-v_1+v_2\\
  v=av_1+av_2
  \end{cases}\]
  则得
  \[\begin{cases}
  \frac{\partial v_1}{\partial t}=(x-a)\frac{\partial v_1}{\partial x}\\
  \frac{\partial v_2}{\partial t}=(x+a)\frac{\partial v_2}{\partial x}
  \end{cases}\]

  (3)\[A=
  \begin{pmatrix}
  6&5&0\\5&6&0\\-1&0&2
  \end{pmatrix}\Rightarrow
  \begin{vmatrix}
  \lambda-6&-5&0\\-5&\lambda-6&0\\1&0&\lambda-2
  \end{vmatrix}=0.\]
  解得 $(\lambda-2)(\lambda^2-12\lambda+11)=0\Rightarrow\lambda=1,2,11$,
  对应特征向量为 $(1,-1,1)^T$, $(0,0,1)^T$, $(9,9,-1)^T$,作变换
  \[\begin{cases}
  u_1=v_1+9v_3\\
  u_2=-v_1+9v_3\\
  u_3=v_1+v_2-v_3
  \end{cases}\]
  \[R=\begin{pmatrix}
  1&0&9\\-1&0&9\\1&1&-1
  \end{pmatrix}
  \Rightarrow
  R^{-1}=
  \begin{pmatrix}
  1/2&-1/2&0\\-4/9&5/9&1\\1/18&1/18&0
  \end{pmatrix}\]
  又因为
  \[B=\begin{pmatrix}
  0&0&0\\-2&0&0\\1&-2/3&-1
  \end{pmatrix},C=\begin{pmatrix}
  0\\0\\0
  \end{pmatrix}\]
  所以
  \[R^{-1}BR=\begin{pmatrix}
  1&0&9\\-4/9&-1&-6\\-1/9&0&-1
  \end{pmatrix},R^{-1}C=\begin{pmatrix}
  0\\0\\0
  \end{pmatrix}\]
  即得对角型方程组
  \[\begin{cases}
  \frac{\partial v_1}{\partial t}+\frac{\partial v_1}{\partial x}+v_1+9v_3=0\\
  \frac{\partial v_2}{\partial t}+2\frac{\partial v_2}{\partial x}-\frac{4}{9}v_1-v_2-6v_3=0\\
  \frac{\partial v_3}{\partial t}+11\frac{\partial v_3}{\partial x}-\frac{1}{9}v_1-v_3=0
  \end{cases}\]
\end{solve}


\begin{exercise}
  证明: 经过未知函数的任何实系数的可逆线性变换, 方程组 (2.1) 在每一点的特征线方向
  (或特征曲线) 保持不变, 因此也不会改变方程组 (2.1) 所属的类型.
\end{exercise}

\begin{proof}
  原方程为
  \[\frac{\partial U}{\partial t}+A\frac{\partial U}{\partial x}+BU+C = 0.\]
  未知函数作可逆线性变换 $U=RV$ 后, 有
  \[R\frac{\partial V}{\partial t}+AR\frac{\partial V}{\partial x}
    + \left(\frac{\partial R}{\partial t}+A\frac{\partial R}{\partial x}
    + BR\right)V + C = 0.\]
  两端左乘 $R^{-1}$ 得
  \[ \frac{\partial V}{\partial t} + A' \frac{\partial V}{\partial x}
      + R^{-1}\biggl(\frac{\partial R}{\partial t}
      + A \frac{\partial R}{\partial x} + BR\biggr)V + R^{-1}C = 0, \]
  其中 $A' = R^{-1}AR$. 由于
  \[\det(A'-\lambda I) = \det(R^{-1}AR - \lambda I)
    = \det(A - \lambda I),\]
  故方程的根保持不变, 特征方向不变, 特征线也不变.
\end{proof}


\begin{exercise}
  证明: 方程组 (2.1) 在每一点的特征线方向 (或特征曲线) 经过自变量的任何可逆变换后
  就变成变换后方程组在对应点的特征线方向 (或特征曲线), 即特征线方向 (或特征曲线)
  对可逆坐标变换具有不变性.
\end{exercise}
  
\begin{proof}
  Suppose the original characteristic curve satisfies
  \[ \frac{\d x}{\d t} = \lambda(x,t), \]
  where $\lambda(x,t)$ is the solution to
  \begin{equation}\label{eq:5.6}
    \det(A-\lambda I) = 0.
  \end{equation}
  Now make a change of variables
  \[ y = \xi(x,t),\qquad s = \eta(x,t). \]
  Then for the new variables $(y,s)$, the characteristic curve satisfies
  \begin{equation}\label{eq:5.9}
    \frac{\d y}{\d s} = \frac{\xi_x \d x + \xi_t \d t}{\eta_x \d x + \eta_t \d t}
        = \frac{\xi_x \lambda + \xi_t}{\eta_x \lambda + \eta_t}.
  \end{equation}
  Under the transform of variables, the original system
  \[ \frac{\partial U}{\partial t} + A\frac{\partial U}{\partial x} + BU + C = 0 \]
  can be turned into
  \[ \biggl(A \frac{\partial\eta}{\partial x} + \frac{\partial\eta}{\partial t}I\biggr)
      \frac{\partial U}{\partial s}
      + \biggl(A \frac{\partial\xi}{\partial x} + \frac{\partial\xi}{\partial t}I\biggr)
        \frac{\partial U}{\partial y} + BU + C = 0. \]
  Suppose its characteristic curve is
  \[ \frac{\d y}{\d s} = \tilde\lambda. \] 
  Then $\tilde\lambda$ must satisfy
  \[\det\biggl(A \frac{\partial\xi}{\partial x} + \frac{\partial\xi}{\partial t}I
      - \biggl(A \frac{\partial\eta}{\partial x} + \frac{\partial\eta}{\partial t}I\biggr)\tilde\lambda\biggr) = 0,\]
  i.e.,
  \begin{equation}\label{eq:5.7}
    \det\biggl(A - \frac{\tilde\lambda \eta_t - \xi_t}{\xi_x - \eta_x \tilde\lambda}I\biggr) = 0.
  \end{equation}
  By \eqref{eq:5.6} and \eqref{eq:5.7} we have that
  \[ \frac{\tilde\lambda \eta_t - \xi_t}{\xi_x - \eta_x \tilde\lambda} = \lambda, \]
  from which we get
  \begin{equation}\label{eq:5.8}
    \tilde\lambda = \frac{\xi_x\lambda + \xi_t}{\eta_x\lambda + \eta_t}.
  \end{equation}
  Combining with \eqref{eq:5.9}, the proof is finished.
\end{proof}


\section{两个自变量的线性双曲型方程组的柯西问题}

\begin{exercise}
  用逐次逼近法求方程组
  \[\begin{cases}
    u_t + u_x = v, \\
    v_t - v_x = u
  \end{cases}\]
  带下列初始条件的柯西问题的解:
  \begin{enumerate}[(1)]
    \item $u|_{t=0}=1$, $v|_{t=0}=0$;
    \item $u|_{t=0}=\sin x$, $v|_{t=0}=\cos x$.
  \end{enumerate}
\end{exercise}

\begin{solution}
  $\lambda_1(x,t) = 1$ and $\lambda_2(x,t)=-1$, so the first characteristic curve (straight line in fact) $l_1$
  passes through $(x,t)$ and $(x-t,0)$ while the second $l_2$ passes through $(x,t)$ and $(x+t,0)$.
  \begin{enumerate}[(1)]
    \item The equations are equivalent to
      \begin{align*}
        u(x,t) & = 1 + \int_{l_1} v \d\tau, \\
        v(x,t) & = \int_{l_2} u \d\tau.
      \end{align*}
      Let $u^{(0)}\equiv 1$ and $v^{(0)}\equiv 0$, then
      \begin{align*}
        u^{(1)} & = 1, & v^{(1)} & = t, \\
        u^{(2)} & = 1+\frac{t^2}{2}, & v^{(2)} & = t, \\
        u^{(3)} & = 1+\frac{t^2}{2}, & v^{(3)} & = t+\frac{t^3}{6}, \\
        u^{(4)} & = 1+\frac{t^2}{2}+\frac{t^4}{24}, & v^{(4)} & = t+\frac{t^3}{6}.
      \end{align*}
      By induction we have that
      \[ u(x,t) = \sum_{k=1}^\infty \frac{t^{2k}}{2k!} = \cosh t,
          \qquad v(x,t) = \sum_{k=1}^\infty \frac{t^{2k-1}}{(2k-1)!} = \sinh t. \]
    \item The equations are equivalent to
      \begin{align*}
        u(x,t) & = \sin(x-t) + \int_{l_1} v(x_1(\tau;x,t),\tau) \d\tau, \\
        v(x,t) & = \cos(x+t) + \int_{l_2} u(x_2(\tau;x,t),\tau) \d\tau,
      \end{align*}
      where $x_1(\tau;x,t) = x-t+\tau$ and $x_2(\tau;x,t) = x+t-\tau$.
      Let $u^{(0)} = \sin(x-t)$ and $v^{(0)} = \cos(x+t)$, then
      \begin{align*}
        u^{(1)}(x,t) & = \sin(x-t) + \int_0^t v^{(0)} (x-t+\tau,\tau) \d\tau \\
                     & = \frac{1}{2}[\sin(x+t) - \sin(x-t)], \\
        v^{(1)}(x,t) & = \cos(x+t) + \int_0^t u^{(0)} (x+t-\tau,\tau) \d\tau \\
                     & = \frac{1}{2} [\cos(x+t) + \cos(x-t)],
      \end{align*}
      Similarly, we calculate that
      \begin{align*}
        u^{(2)}(x,t) & = \frac34 \sin(x-t) + \frac14 \sin(x+t) + \frac12 t\cos(x-t), \\
        v^{(2)}(x,t) & = \frac34 \cos(x+t) + \frac14 \cos(x-t) + \frac12 t\sin(x+t), \\
        u^{(3)}(x,t) & = \frac12 [\sin(x+t)+\sin(x-t)] + \frac14 t[\cos(x-t)-\cos(x+t)], \\
        v^{(3)}(x,t) & = \frac12 [\cos(x+t)+\cos(x-t)] + \frac14 t[\sin(x+t)-\sin(x-t)], \\
        u^{(4)}(x,t) & = \frac{5}{16}\sin(x+t) + \frac{11}{16} \sin(x-t) + \frac{t}{2}\cos(x-t) \\
                     & \quad -\frac{t}{8}\cos(x+t) + \frac{t^2}{8}\sin(x-t), \\
        v^{(4)}(x,t) & = \frac{11}{16} \cos(x+t) + \frac{5}{16}\cos(x-t) + \frac{t}{2}\sin(x+t) \\
                     & \quad -\frac{t}{8}\sin(x-t) - \frac{t^2}{8}\cos(x+t).
      \end{align*}
      So $u = \sin x$ and $v = \cos x$.
  \end{enumerate}
\end{solution}


\begin{exercise}
  求解柯西问题:
  \[\begin{cases}
    u_t - u_x = (x+t)v, \\
    v_t + v_x + (x+t)u = 0, \\
    t=0: u=0, v=1.
  \end{cases}\]
\end{exercise}


\begin{exercise}
  证明用 (3.9) 式表示的函数序列 $\bigl\{\frac{\partial V_i^{(n)}}{\partial x}\bigr\}$
  ($i=1,\ldots,N$, $n=0,1,2,\ldots$) 在区域 $\overline{G}$ 上的一致收敛性.
\end{exercise}

\begin{proof}
  Let
  \begin{align*}
    A & := \max \biggl\{|\alpha_{ij}|, \biggl|\frac{\partial\alpha_{ij}}{\partial x}\biggr|\biggr\}, \\
    M & := \max \biggl\{|V_i^{(0)}|, |V_i^{(1)}|,
                  \biggl|\frac{\partial V_i^{(0)}}{\partial x}\biggr|,
                  \biggl|\frac{\partial V_i^{(1)}}{\partial x}\biggr|\biggr\}.
  \end{align*}
  Then
  \[ \biggl|\frac{\partial V_i^{(1)}}{\partial x} - \frac{\partial V_i^{(0)}}{\partial x}\biggr|
        \leq 2M. \]
  Suppose that 
  \begin{equation}\label{eq:5.10}
    \biggl|\frac{\partial V_i^{(n+1)}}{\partial x} - \frac{\partial V_i^{(n)}}{\partial x}\biggr|
      \leq 2(n+1)M\frac{(ANt)^n}{n!},
  \end{equation}
  then
  \begin{align*}
    \biggl|\frac{\partial V_j^{(n+2)}}{\partial x} - \frac{\partial V_j^{(n+1)}}{\partial x}\biggr|
    & \leq \int_0^t \sum_{j=1}^N \biggl|\frac{\partial\alpha_{ij}}{\partial x}\biggr|
        \cdot |V_j^{(n+1)} - V_j^{(n)}| \\
    & \quad + \sum_{j=1}^N |\alpha_{ij}|
        \cdot \biggl|\frac{\partial V_j^{(n+1)}}{\partial x}-\frac{\partial V_j^{(n)}}{\partial x}\biggr| \\
    & \leq 2(n+2)M \frac{(ANt)^{n+1}}{(n+1)!}.
  \end{align*}
  By the principle of induction, we obtain that~\eqref{eq:5.10} holds for all $n\geq 0$.
  Hence the proof is finished by the convergence of the sequence
  $2(n+1)M(ANt)^n/n!$.
\end{proof}


\begin{exercise}
  设 $V_i(x,t)$ ($i=1,\ldots,N$) 是方程组 (3.1) 带初始条件 (3.2)
  的柯西问题在区域 $\overline{G}$ 上的解, 且设
  \begin{gather*}
    \varphi_0 = \max_{\substack{(x,t)\in\overline{G} \\ 1\leq i\leq N}} |\varphi_i(x)|,
    \qquad
    \alpha = \max_{\substack{(x,t)\in\overline{G} \\ 1\leq i\leq N}} \sum_{j=1}^N |\alpha_{ij}(x,t)|, \\
    \beta(\tau) = \max_{\substack{t=\tau, (x,t)\in\overline{G} \\ 1\leq i\leq N}} |\beta_i(x,t)|,
  \end{gather*}
  则在区域 $\overline{G}$ 上成立着下面的 Haar 估计式
  \[ |V_i(x,t)| \leq \varphi_0 e^{\alpha t} + \int_0^t \beta(\tau) e^{\alpha(t-\tau)} \d\tau
      \quad (i=1,\ldots,N),\]
  并利用此估计式证明柯西问题 (3.1)--(3.2) 解的唯一性和对初始条件的连续依赖性.
\end{exercise}

\begin{proof}
  Use (3.6) and Gronwall's inequality.
\end{proof}