\chapter{一阶偏微分方程组}

\section{引言}

% Consider the following first-order system
% \begin{equation}\label{1}
%   \frac{\partial V_i}{\partial t} + \lambda_i(x,t) \frac{\partial V_i}{\partial x}
%     = \sum_{j=1}^n \alpha_{ij}(x,t)V_j + \beta_i(x,t)
% \end{equation}
% with the initial value condition
% \begin{equation}\label{2}
%   V_i(x,0) = \varphi_i(x) \quad i=1,\ldots,N.
% \end{equation}
% Suppose that the eigenvalues are mutually distinct, i.e.,
% \[ \lambda_1(x,t) < \lambda_2(x,t) < \cdots < \lambda_N(x,t). \]
% Then through every point there are $N$ characteristic curves $L_i$ ($i=1,\ldots,N$) pass and
% they are given by
% \[ \frac{\d x}{\d t} = \lambda_i(x,t). \]
% Suppose the IVC \eqref{2} is given on the interval $[a,b]$.
% Draw two characteristic curves $L_N$ and $L_1$ passing through $(a,0)$
% and $(b,0)$ respectively. And let $G$ be the domain enclosed by $L_N$,
% $L_1$, $x$-axis and $t=T$. 
% We suppose that in the closed domain $\overline{G}$, functions
% $\alpha_{ij}$, $\beta_i$ and $\varphi_i$ are all $C^1$.
% Fix any point $(x,t)\in\overline{G}$, draw a characteristic curve $L_i$
% which intersects $x$-axis at $(x_i,0)$. See the following figure.
% Note that the left-hand side of \eqref{eq:1} is the derivative of $V_i$
% with respect to $t$ along the curve $L_i$. Integrate \eqref{1} from $(x_i,0)$
% to $(x,t)$ with respect to $t$ along $l_i$, we have
% \begin{equation}\label{3}
%   V_i(x,t) - V_i(x_i,0) = \int_{l_i} \biggl(\sum_{j=1}^N \alpha_{ij} V_j + \beta_i\biggr) \d\tau,
% \end{equation}
% that is
% \begin{equation}\label{4}
%   V_i(x,t) = \varphi_i(x_i) + \int_{l_i} \biggl(\sum_{j=1}^N \alpha_{ij} V_j + \beta_i\biggr) \d\tau.
% \end{equation}
% Now we use \eqref{4} to construct the iteration sequence. Define for $n\geq 0$
% \begin{equation}\label{5}\tag{5}
%   V_i^{(n+1)}(x,t) = \varphi_i(x_i) + \int_{l_i} \biggl[\sum_{j=1}^N \alpha_{ij} V_j^{(n)}+\beta_i\biggr] \d\tau \quad (i=1,\ldots,N),
% \end{equation}
% or more precisely,
% \begin{equation}\label{6}\tag{6}
%   \begin{aligned}
%     V_i^{(n+1)}(x,t) & = \varphi_i(x_i(0;x,t)) + \int_0^t \biggl[\sum_{j=1}^N
%       \alpha_{ij}(x_i(\tau;x,t),\tau) V_j^{(n)}(x_i(\tau;x,t),\tau) \\
%                      & \quad + \beta_i(x_i(\tau;x,t),\tau)\biggr] \d\tau,
%   \end{aligned}
% \end{equation}
% where $x_i(\tau;x,t)$ is the $i$th characteristic curve passing through $(x,t)$.
% Let
% $$M := \max_{(x,t)\in\overline{G},\,i=1,\ldots,N} \bigl\{|V_i^{(0)}|, |V_i^{(1)}|\bigr\},$$
% $$A := \max_{(x,t)\in\overline{G},\,i,j=1,\ldots,N} |\alpha_{ij}|.$$
% By a simple induction argument, we can prove that
% \begin{equation}\label{7}\tag{7}
%   |V_i^{(n+1)}(x,t) - V_i^{(n)}(x,t)| \leq 2M \frac{(ANt)^n}{n!} \quad i=1,\ldots,N,
% \end{equation}
% which means that $(V_i^{(n)})$ converges uniformly to some function $V_i$ and it is the solution
% to \eqref{1}.

\begin{exercise}
  把波动方程
  \[\frac{\partial^2u}{\partial t^2} = a^2 
    \biggl(\frac{\partial^2u}{\partial x^2} + \frac{\partial^2u}{\partial y^2}
    + \frac{\partial^2u}{\partial z^2}\biggr)\]
  带初始条件
  \[\begin{cases}
    u|_{t=0} = \varphi(x,y,z), \\
    \frac{\partial u}{\partial t}|_{t=0} = \psi(x,y,z)
  \end{cases}\]
  的柯西问题化为一个一阶方程组的柯西问题, 并证明其解的等价性.
\end{exercise}

\begin{proof}
  令 $p=\frac{\partial u}{\partial t}$, $q_1=\frac{\partial u}{\partial x}$,
  $q_2=\frac{\partial u}{\partial y}$, $q_3=\frac{\partial u}{\partial z}$, 则
  \[\begin{cases}
    \frac{\partial p}{\partial t} = a^2\left(\frac{\partial q_1}{\partial x}
      +\frac{\partial q_2}{\partial y}+\frac{\partial q_3}{\partial z}\right) & (1) \\
    \frac{\partial p}{\partial x} = \frac{\partial q_1}{\partial t},
    \frac{\partial p}{\partial y}=\frac{\partial q_2}{\partial t},
    \frac{\partial p}{\partial z}=\frac{\partial q_3}{\partial t} & (2) \\
    t = 0: p=\psi, q_1 = \frac{\partial\varphi}{\partial x},
                   q_2 = \frac{\partial\varphi}{\partial y},
                   q_3 = \frac{\partial\varphi}{\partial z} & (3)
  \end{cases}\]
  原方程的解显然满足新方程,而如果新方程的解为 $(p,q_1,q_2,q_3)$,则
  \[u(x,y,z,t) = \varphi(x,y,z)
    + \int_{(0,0,0,0)}^{(x,y,x,t)}p\d t+q_1\d x+q_2\d y+q_3\d z\]
  是原方程的解, 其中条件 (2) 确保了积分与路径无关, 故积分定义是合理的.
\end{proof}


\begin{exercise}
  把方程
  \[ u_{tt} = u_{x}^2 + u_{y}^2 \]
  带初始条件
  \[\begin{cases}
    u|_{t=0} = 0, \\
    u_t|_{t=0} = e^x \sin y
  \end{cases}\]
  的柯西问题化为一个一阶偏微分方程组的柯西问题.
\end{exercise}

\begin{solution}
  Let $p=u_t$, then
  \[\begin{cases}
    p = u_t, \\
    p_t = u_x^2 + u_y^2.
  \end{cases}\]
  And the initial value condition is
  \[\begin{cases}
    u|_{t=0} = 0, \\
    p|_{t=0} = e^x \sin y.
  \end{cases}\qedhere\]
\end{solution}


\begin{exercise}
  证明柯瓦列夫斯卡娅型方程 (1.6) 满足初始条件
  \begin{equation}
    t=0: u=\varphi_0(x),\ldots, \frac{\partial^{m-1}u}{\partial t^{m-1}} = \varphi_{m-1}(x) \tag{$\star$}
  \end{equation}
  的柯西问题可以化为一阶方程组的柯西问题, 并证明其解的等价性.
\end{exercise}

\begin{proof}
  Let $\alpha = (\alpha_0,\alpha_1,\ldots,\alpha_n)$ be the general multi-index
  and let 
  \begin{align*}
    \alpha^0 & := (\alpha_0+1,\alpha_1,\ldots,\alpha_n), \\
    \alpha^i & := (\alpha_0,\alpha_1,\ldots,\alpha_i+1,\ldots,\alpha_n).
  \end{align*}
  Decompose $\alpha$ as follows:
  \[ \alpha = \beta + \gamma, \]
  where $\beta=(\alpha_0,0,\ldots,0)$ and $\gamma=(0,\alpha_1,\ldots,\alpha_n)$.
  We denote the special multi-index $\alpha^* = (m-1,0,\ldots,0)$
  and introduce new functions by $u_\alpha = D^\alpha u$.
  Then the Kowalevskaya type equation can be transformed into
  the following first-order system
  \begin{equation}\label{eq:5.1}
    \begin{cases}
      \frac{\partial u_\alpha}{\partial t} = u_{\alpha^0}, \quad |\alpha|\leq m-1, \alpha_0\leq m-2, \\
      \frac{\partial u_\alpha}{\partial x_i} = u_{\alpha^i}, \quad |\alpha|\leq m-1, \\
      \frac{\partial u_{\alpha^*}}{\partial t} = F(t,x,u,u_\alpha,|\alpha|\leq m, \alpha_0\leq m-1),
    \end{cases}
  \end{equation}
  together with the following initial value conditions
  \begin{equation}\label{eq:5.2}
    \begin{cases}
      u|_{t=0} = \varphi_0(x), \\
      u_\alpha|_{t=0} = D^\gamma \varphi_{\alpha_0}(x),\quad 1\leq |\alpha|\leq m, \alpha_0\leq m-1. 
    \end{cases}
  \end{equation}

  Now we prove that equivalence of solutions. First of all,
  if $u$ is a solution to (1.9) in the textbook with initial value condition $(\star)$,
  it is straightforward to verify that $(u_\alpha)$ is the solution to~\eqref{eq:5.1} 
  and~\eqref{eq:5.2}.

  Conversely, let $(u_\alpha)$ be the solution to~\eqref{eq:5.1} and \eqref{eq:5.2}.
  Then for all $|\alpha|\leq m-1$ with $\alpha_0\leq m-2$ we have
  \begin{equation}\label{eq:5.3}
    \begin{cases}
      \frac{\partial u_\alpha}{\partial t} = u_{\alpha^0}, \\
      \frac{\partial u_{\alpha^0}}{\partial x_i} = \frac{\partial u_{\alpha^i}}{\partial t}.
    \end{cases}
  \end{equation}
  It follows that
  \begin{equation}\label{eq:5.4}
    \frac{\partial}{\partial t}\biggl(u_{\alpha^i} - \frac{\partial u_\alpha}{\partial x_i}\biggr)
      = 0.
  \end{equation}
  So $u_{\alpha^i} - \frac{\partial u_\alpha}{\partial x_i}$ is independent of $t$.
  From \eqref{eq:5.2} we know that it is equal to zero at $t=0$, thus
  \begin{equation}\label{eq:5.5}
    u_{\alpha^i} \equiv \frac{\partial u_\alpha}{\partial x_i} \quad \text{for all}\ t\geq 0.
  \end{equation}
  Plugging \eqref{eq:5.5} into \eqref{eq:5.1} we find that $u$ is a solution to (1.9)
  in the textbook with the given initial value condition.
\end{proof}

\section{两个自变量的一阶线性偏微分方程组的特征理论}

\begin{exercise}
  求一阶方程
  \begin{enumerate}[(1)]
    \item $\displaystyle\frac{\partial u}{\partial t} + a(x,t) \frac{\partial u}{\partial x} + b(x,t)u + c(x,t) = 0$;
    \item $\displaystyle\frac{\partial u}{\partial t} + a(x,t) \frac{\partial u}{\partial x} + b(x,t,u) = 0$
  \end{enumerate}
  的特征线和解沿特征线应成立的关系式.
\end{exercise}

\begin{solve}
  (1) 特征线满足的方程为 $\frac{\d x}{\d t} = a(x,t)$,
  解得特征线为 $\displaystyle x = \int_0^t a(x,\tau) \d\tau$.
  在特征线上, $\displaystyle u(x,t) = u\Bigl(\int_0^t a(x, \tau) \d\tau, t\Bigr)$, 故
  \[\frac{\d u}{\d t} = \frac{\partial u}{\partial x} a(x,t)
    + \frac{\partial u}{\partial t}.\]
  故 $u$ 在特征线上满足关系式
  \[\frac{\d u}{\d t} + b(x,t)u + c(x,t) = 0.\]

  (2) 同理 $\frac{\d u}{\d t} + b(x,t,u) = 0$.
\end{solve}


\begin{exercise}
  求下列一阶方程带初始条件 $u|_{t=0} = \varphi(x)$ 的柯西问题的解:
  \begin{enumerate}[(1)]
    \item $\displaystyle\frac{\partial u}{\partial t} + \frac{\partial u}{\partial x} = 0$;
    \item $\displaystyle\frac{\partial u}{\partial t} + \frac{\partial u}{\partial x} = u$.
  \end{enumerate}
\end{exercise}

\begin{solve}
  (1) 特征线为 $x = t+C$, 在特征线上, $\frac{\d u}{\d t} = 
    \frac{\partial u}{\partial x} + \frac{\partial u}{\partial t} = 0$,
  故 $u$ 在特征线上为常数. 故
  \[u(x_0, t_0) = u(x_0-t_0, 0) = \varphi(x_0-t_0).\]
  因此 $u(x,t) = \varphi(x-t)$.

  (2) 特征线为 $x = t+C$, 在特征线上, $\frac{\d u}{\d t} = u$,
  故 $u = C\e^t$, 令 $t=0$, 得 $u|_{t=0} = C$. 故
  \[u(x_0, t_0) = u(x_0-t_0, 0) \e^{t_0} = \varphi(x_0-t_0)\e^{t_0}.\]
  因此
  \[u(x, t) = \varphi(x-t) \e^t. \qedhere\]
\end{solve}


\begin{exercise}
  判断方程组
  \begin{align*}
    & \frac{\partial u_1}{\partial t} = a(x,t) \frac{\partial u_1}{\partial x}
      - b(x,t) \frac{\partial u_2}{\partial x} + f_1, \\
    & \frac{\partial u_2}{\partial t} = b(x,t) \frac{\partial u_1}{\partial x}
      + a(x,t) \frac{\partial u_2}{\partial x} + f_2
  \end{align*}
  属于何种类型.
\end{exercise}

\begin{solve}
  \[A = \begin{pmatrix}
    a & -b \\
    b & a
  \end{pmatrix} \Rightarrow (\lambda-a)^2+b^2 = 0.\]
  故当 $b=0$ 时为双曲型, 当 $b\neq 0$ 时为椭圆型.
\end{solve}


\begin{exercise}
  将下列各方程组化为对角型方程组:
  \begin{enumerate}[(1)]
    \item $\begin{cases}
             \frac{\partial u}{\partial t} + (1+\sin x)\frac{\partial u}{\partial x}
                + 2 \frac{\partial v}{\partial x} + x = 0, \\
             \frac{\partial v}{\partial t} + u = 0;  
           \end{cases}$
    \item $\begin{cases}
             \frac{\partial u}{\partial t} = x \frac{\partial u}{\partial x}
               + \frac{\partial v}{\partial x}, \\
             \frac{\partial v}{\partial t} = a^2 \frac{\partial u}{\partial x}
               + x \frac{\partial v}{\partial x}\quad (a>0);
           \end{cases}$
    \item $\begin{cases}
             \frac{\partial u_1}{\partial t} + 6 \frac{\partial u_1}{\partial x}
               + 5 \frac{\partial u_2}{\partial x} = 0, \\
             \frac{\partial u_2}{\partial t} + 5 \frac{\partial u_1}{\partial x}
               + 6 \frac{\partial u_2}{\partial x} = 2u_1, \\
             3 \frac{\partial u_3}{\partial t} + 6 \frac{\partial u_3}{\partial x}
               - 3 \frac{\partial u_1}{\partial x} = 2u_2 + 3u_3 - 3u_1.
           \end{cases}$
  \end{enumerate}
\end{exercise}

\begin{solve}
  (1) \[A = \begin{pmatrix}
  1+\sin x&2\\0&0\end{pmatrix}\Rightarrow\lambda=0\text{\ 或\ }\lambda=1+\sin x.\]
  相应的特征向量为$(-2,1+\sin x)^T,(c,0)^T$,作变换
  \[\begin{cases}
  u=-2v_1+cv_2\\
  v=(1+\sin x)v_1
  \end{cases}\]
  则得
  \[\begin{cases}
  (1+\sin x)\frac{\partial v_1}{\partial t}-2v_1+cv_2=0, \\
  c(1+\sin x)\frac{\partial v_2}{\partial t}+c(1+\sin x)^2\frac{\partial v_2}{\partial x}
    + x(1+\sin x)-4v_1+2cv_2=0.
  \end{cases}\]

  (2) \[A=\begin{pmatrix}
        x & 1 \\ a^2 & x
  \end{pmatrix}\Rightarrow(\lambda-x)^2-a^2=0\Rightarrow\lambda=x\pm a\]
  对应特征向量为 $(-1,a)^T$, $(1,a)^T$,作变换
  \[\begin{cases}
  u=-v_1+v_2\\
  v=av_1+av_2
  \end{cases}\]
  则得
  \[\begin{cases}
  \frac{\partial v_1}{\partial t}=(x-a)\frac{\partial v_1}{\partial x}\\
  \frac{\partial v_2}{\partial t}=(x+a)\frac{\partial v_2}{\partial x}
  \end{cases}\]

  (3)\[A=
  \begin{pmatrix}
  6&5&0\\5&6&0\\-1&0&2
  \end{pmatrix}\Rightarrow
  \begin{vmatrix}
  \lambda-6&-5&0\\-5&\lambda-6&0\\1&0&\lambda-2
  \end{vmatrix}=0.\]
  解得 $(\lambda-2)(\lambda^2-12\lambda+11)=0\Rightarrow\lambda=1,2,11$,
  对应特征向量为 $(1,-1,1)^T$, $(0,0,1)^T$, $(9,9,-1)^T$,作变换
  \[\begin{cases}
  u_1=v_1+9v_3\\
  u_2=-v_1+9v_3\\
  u_3=v_1+v_2-v_3
  \end{cases}\]
  \[R=\begin{pmatrix}
  1&0&9\\-1&0&9\\1&1&-1
  \end{pmatrix}
  \Rightarrow
  R^{-1}=
  \begin{pmatrix}
  1/2&-1/2&0\\-4/9&5/9&1\\1/18&1/18&0
  \end{pmatrix}\]
  又因为
  \[B=\begin{pmatrix}
  0&0&0\\-2&0&0\\1&-2/3&-1
  \end{pmatrix},C=\begin{pmatrix}
  0\\0\\0
  \end{pmatrix}\]
  所以
  \[R^{-1}BR=\begin{pmatrix}
  1&0&9\\-4/9&-1&-6\\-1/9&0&-1
  \end{pmatrix},R^{-1}C=\begin{pmatrix}
  0\\0\\0
  \end{pmatrix}\]
  即得对角型方程组
  \[\begin{cases}
  \frac{\partial v_1}{\partial t}+\frac{\partial v_1}{\partial x}+v_1+9v_3=0\\
  \frac{\partial v_2}{\partial t}+2\frac{\partial v_2}{\partial x}-\frac{4}{9}v_1-v_2-6v_3=0\\
  \frac{\partial v_3}{\partial t}+11\frac{\partial v_3}{\partial x}-\frac{1}{9}v_1-v_3=0
  \end{cases}\]
\end{solve}


\begin{exercise}
  证明: 经过未知函数的任何实系数的可逆线性变换, 方程组 (2.1) 在每一点的特征线方向
  (或特征曲线) 保持不变, 因此也不会改变方程组 (2.1) 所属的类型.
\end{exercise}

\begin{proof}
  原方程为
  \[\frac{\partial U}{\partial t}+A\frac{\partial U}{\partial x}+BU+C = 0.\]
  未知函数作可逆线性变换 $U=RV$ 后, 有
  \[R\frac{\partial V}{\partial t}+AR\frac{\partial V}{\partial x}
    + \left(\frac{\partial R}{\partial t}+A\frac{\partial R}{\partial x}
    + BR\right)V + C = 0.\]
  两端左乘 $R^{-1}$ 得
  \[ \frac{\partial V}{\partial t} + A' \frac{\partial V}{\partial x}
      + R^{-1}\biggl(\frac{\partial R}{\partial t}
      + A \frac{\partial R}{\partial x} + BR\biggr)V + R^{-1}C = 0, \]
  其中 $A' = R^{-1}AR$. 由于
  \[\det(A'-\lambda I) = \det(R^{-1}AR - \lambda I)
    = \det(A - \lambda I),\]
  故方程的根保持不变, 特征方向不变, 特征线也不变.
\end{proof}


\begin{exercise}
  证明: 方程组 (2.1) 在每一点的特征线方向 (或特征曲线) 经过自变量的任何可逆变换后
  就变成变换后方程组在对应点的特征线方向 (或特征曲线), 即特征线方向 (或特征曲线)
  对可逆坐标变换具有不变性.
\end{exercise}
  
\begin{proof}
  Suppose the original characteristic curve satisfies
  \[ \frac{\d x}{\d t} = \lambda(x,t), \]
  where $\lambda(x,t)$ is the solution to
  \begin{equation}\label{eq:5.6}
    \det(A-\lambda I) = 0.
  \end{equation}
  Now make a change of variables
  \[ y = \xi(x,t),\qquad s = \eta(x,t). \]
  Then for the new variables $(y,s)$, the characteristic curve satisfies
  \begin{equation}\label{eq:5.9}
    \frac{\d y}{\d s} = \frac{\xi_x \d x + \xi_t \d t}{\eta_x \d x + \eta_t \d t}
        = \frac{\xi_x \lambda + \xi_t}{\eta_x \lambda + \eta_t}.
  \end{equation}
  Under the transform of variables, the original system
  \[ \frac{\partial U}{\partial t} + A\frac{\partial U}{\partial x} + BU + C = 0 \]
  can be turned into
  \[ \biggl(A \frac{\partial\eta}{\partial x} + \frac{\partial\eta}{\partial t}I\biggr)
      \frac{\partial U}{\partial s}
      + \biggl(A \frac{\partial\xi}{\partial x} + \frac{\partial\xi}{\partial t}I\biggr)
        \frac{\partial U}{\partial y} + BU + C = 0. \]
  Suppose its characteristic curve is
  \[ \frac{\d y}{\d s} = \tilde\lambda. \] 
  Then $\tilde\lambda$ must satisfy
  \[\det\biggl(A \frac{\partial\xi}{\partial x} + \frac{\partial\xi}{\partial t}I
      - \biggl(A \frac{\partial\eta}{\partial x} + \frac{\partial\eta}{\partial t}I\biggr)\tilde\lambda\biggr) = 0,\]
  i.e.,
  \begin{equation}\label{eq:5.7}
    \det\biggl(A - \frac{\tilde\lambda \eta_t - \xi_t}{\xi_x - \eta_x \tilde\lambda}I\biggr) = 0.
  \end{equation}
  By \eqref{eq:5.6} and \eqref{eq:5.7} we have that
  \[ \frac{\tilde\lambda \eta_t - \xi_t}{\xi_x - \eta_x \tilde\lambda} = \lambda, \]
  from which we get
  \begin{equation}\label{eq:5.8}
    \tilde\lambda = \frac{\xi_x\lambda + \xi_t}{\eta_x\lambda + \eta_t}.
  \end{equation}
  Combining with \eqref{eq:5.9}, the proof is finished.
\end{proof}


\begin{exercise}
	证明:对于两个未知数的任何线性椭圆方程组, 都可能通过未知函数的实系数的可逆线性变换, 化为下面的形式:
	$$\frac{\partial V_1}{\partial t}=a\frac{\partial V_1}{\partial x}-b\frac{\partial V_2}{\partial x}+f_1,$$
	$$\frac{\partial V_2}{\partial t}=b\frac{\partial V_1}{\partial x}+a\frac{\partial V_2}{\partial x}+f_2,$$
	其中$a,b$为$x,t$的函数, 且$b\neq0$,而$f_1,f_{2}$为未知函数的线性函数.
\end{exercise}

\begin{proof}
	设两个未知函数的线性椭圆型方程组为
	$$U_t+AU_x+BU+C=0,$$
	其中, $U=( U_1, U_2)^T$, $A$, $B$为二阶矩阵.
	
	设方程组为椭圆型方程组, 对应特征根为两个共轭复根$a\pm b\upi$.进一步,  由于矩阵$\begin{pmatrix}a&-b\\b&a\end{pmatrix}$的特征根也为$a\pm b\upi$, 那么矩阵$A$与矩阵$\begin{pmatrix}a&-b\\b&a\end{pmatrix}$为相似矩阵, 则存在实函数可逆矩阵$P$, 使得$P^{-1}AP=\begin{pmatrix}a&-b\\b&a\end{pmatrix}$.
	
	于是, 作实函数的可逆线性变换$(U_1,U_2)^T=P(V_1,V_2)^T$, 可使得方程组
	(0.43)化简为要求形式.
\end{proof}


\section{两个自变量的线性双曲型方程组的柯西问题}

\begin{exercise}
  用逐次逼近法求方程组
  \[\begin{cases}
    u_t + u_x = v, \\
    v_t - v_x = u
  \end{cases}\]
  带下列初始条件的柯西问题的解:
  \begin{enumerate}[(1)]
    \item $u|_{t=0}=1$, $v|_{t=0}=0$;
    \item $u|_{t=0}=\sin x$, $v|_{t=0}=\cos x$.
  \end{enumerate}
\end{exercise}

\begin{solution}
  $\lambda_1(x,t) = 1$ and $\lambda_2(x,t)=-1$, so the first characteristic curve (straight line in fact) $l_1$
  passes through $(x,t)$ and $(x-t,0)$ while the second $l_2$ passes through $(x,t)$ and $(x+t,0)$.
  \begin{enumerate}[(1)]
    \item The equations are equivalent to
      \begin{align*}
        u(x,t) & = 1 + \int_{l_1} v \d\tau, \\
        v(x,t) & = \int_{l_2} u \d\tau.
      \end{align*}
      Let $u^{(0)}\equiv 1$ and $v^{(0)}\equiv 0$, then
      \begin{align*}
        u^{(1)} & = 1, & v^{(1)} & = t, \\
        u^{(2)} & = 1+\frac{t^2}{2}, & v^{(2)} & = t, \\
        u^{(3)} & = 1+\frac{t^2}{2}, & v^{(3)} & = t+\frac{t^3}{6}, \\
        u^{(4)} & = 1+\frac{t^2}{2}+\frac{t^4}{24}, & v^{(4)} & = t+\frac{t^3}{6}.
      \end{align*}
      By induction we have that
      \[ u(x,t) = \sum_{k=1}^\infty \frac{t^{2k}}{2k!} = \cosh t,
          \qquad v(x,t) = \sum_{k=1}^\infty \frac{t^{2k-1}}{(2k-1)!} = \sinh t. \]
    \item The equations are equivalent to
      \begin{align*}
        u(x,t) & = \sin(x-t) + \int_{l_1} v(x_1(\tau;x,t),\tau) \d\tau, \\
        v(x,t) & = \cos(x+t) + \int_{l_2} u(x_2(\tau;x,t),\tau) \d\tau,
      \end{align*}
      where $x_1(\tau;x,t) = x-t+\tau$ and $x_2(\tau;x,t) = x+t-\tau$.
      Let $u^{(0)} = \sin(x-t)$ and $v^{(0)} = \cos(x+t)$, then
      \begin{align*}
        u^{(1)}(x,t) & = \sin(x-t) + \int_0^t v^{(0)} (x-t+\tau,\tau) \d\tau \\
                     & = \frac{1}{2}[\sin(x+t) - \sin(x-t)], \\
        v^{(1)}(x,t) & = \cos(x+t) + \int_0^t u^{(0)} (x+t-\tau,\tau) \d\tau \\
                     & = \frac{1}{2} [\cos(x+t) + \cos(x-t)],
      \end{align*}
      Similarly, we calculate that
      \begin{align*}
        u^{(2)}(x,t) & = \frac34 \sin(x-t) + \frac14 \sin(x+t) + \frac12 t\cos(x-t), \\
        v^{(2)}(x,t) & = \frac34 \cos(x+t) + \frac14 \cos(x-t) + \frac12 t\sin(x+t), \\
        u^{(3)}(x,t) & = \frac12 [\sin(x+t)+\sin(x-t)] + \frac14 t[\cos(x-t)-\cos(x+t)], \\
        v^{(3)}(x,t) & = \frac12 [\cos(x+t)+\cos(x-t)] + \frac14 t[\sin(x+t)-\sin(x-t)], \\
        u^{(4)}(x,t) & = \frac{5}{16}\sin(x+t) + \frac{11}{16} \sin(x-t) + \frac{t}{2}\cos(x-t) \\
                     & \quad -\frac{t}{8}\cos(x+t) + \frac{t^2}{8}\sin(x-t), \\
        v^{(4)}(x,t) & = \frac{11}{16} \cos(x+t) + \frac{5}{16}\cos(x-t) + \frac{t}{2}\sin(x+t) \\
                     & \quad -\frac{t}{8}\sin(x-t) - \frac{t^2}{8}\cos(x+t).
      \end{align*}
      So $u = \sin x$ and $v = \cos x$.
  \end{enumerate}
\end{solution}


\begin{exercise}
  求解柯西问题:
  \[\begin{cases}
    u_t - u_x = (x+t)v, \\
    v_t + v_x + (x+t)u = 0, \\
    t=0: u=0, v=1.
  \end{cases}\]
\end{exercise}


\begin{exercise}
  证明用 (3.9) 式表示的函数序列 $\bigl\{\frac{\partial V_i^{(n)}}{\partial x}\bigr\}$
  ($i=1,\ldots,N$, $n=0,1,2,\ldots$) 在区域 $\overline{G}$ 上的一致收敛性.
\end{exercise}

\begin{proof}
  Let
  \begin{align*}
    A & := \max \biggl\{|\alpha_{ij}|, \biggl|\frac{\partial\alpha_{ij}}{\partial x}\biggr|\biggr\}, \\
    M & := \max \biggl\{|V_i^{(0)}|, |V_i^{(1)}|,
                  \biggl|\frac{\partial V_i^{(0)}}{\partial x}\biggr|,
                  \biggl|\frac{\partial V_i^{(1)}}{\partial x}\biggr|\biggr\}.
  \end{align*}
  Then
  \[ \biggl|\frac{\partial V_i^{(1)}}{\partial x} - \frac{\partial V_i^{(0)}}{\partial x}\biggr|
        \leq 2M. \]
  Suppose that 
  \begin{equation}\label{eq:5.10}
    \biggl|\frac{\partial V_i^{(n+1)}}{\partial x} - \frac{\partial V_i^{(n)}}{\partial x}\biggr|
      \leq 2(n+1)M\frac{(ANt)^n}{n!},
  \end{equation}
  then
  \begin{align*}
    \biggl|\frac{\partial V_j^{(n+2)}}{\partial x} - \frac{\partial V_j^{(n+1)}}{\partial x}\biggr|
    & \leq \int_0^t \sum_{j=1}^N \biggl|\frac{\partial\alpha_{ij}}{\partial x}\biggr|
        \cdot |V_j^{(n+1)} - V_j^{(n)}| \\
    & \quad + \sum_{j=1}^N |\alpha_{ij}|
        \cdot \biggl|\frac{\partial V_j^{(n+1)}}{\partial x}-\frac{\partial V_j^{(n)}}{\partial x}\biggr| \\
    & \leq 2(n+2)M \frac{(ANt)^{n+1}}{(n+1)!}.
  \end{align*}
  By the principle of induction, we obtain that~\eqref{eq:5.10} holds for all $n\geq 0$.
  Hence the proof is finished by the convergence of the sequence
  $2(n+1)M(ANt)^n/n!$.
\end{proof}


\begin{exercise}
  设 $V_i(x,t)$ ($i=1,\ldots,N$) 是方程组 (3.1) 带初始条件 (3.2)
  的柯西问题在区域 $\overline{G}$ 上的解, 且设
  \begin{gather*}
    \varphi_0 = \max_{\substack{(x,t)\in\overline{G} \\ 1\leq i\leq N}} |\varphi_i(x)|,
    \qquad
    \alpha = \max_{\substack{(x,t)\in\overline{G} \\ 1\leq i\leq N}} \sum_{j=1}^N |\alpha_{ij}(x,t)|, \\
    \beta(\tau) = \max_{\substack{t=\tau, (x,t)\in\overline{G} \\ 1\leq i\leq N}} |\beta_i(x,t)|,
  \end{gather*}
  则在区域 $\overline{G}$ 上成立着下面的 Haar 估计式
  \[ |V_i(x,t)| \leq \varphi_0 e^{\alpha t} + \int_0^t \beta(\tau) e^{\alpha(t-\tau)} \d\tau
      \quad (i=1,\ldots,N),\]
  并利用此估计式证明柯西问题 (3.1)--(3.2) 解的唯一性和对初始条件的连续依赖性.
\end{exercise}

\begin{proof}
  Use (3.6) and Gronwall's inequality.
\end{proof}


\section{两个自变量的线性双曲型方程组的其他定解问题}


\begin{exercise}
	用逐次逼近法求方程组
	$$\begin{cases}
		\dfrac{\partial u}{\partial t}+\dfrac{\partial u}{\partial x}=v,\\
		\dfrac{\partial v}{\partial t}-\dfrac{\partial v}{\partial x}=u
	\end{cases}$$
	满足条件
	$$\begin{cases}
		&u|_{x=t}=\sin t,\\
		&v|_{x=-t}=\cos t
	\end{cases}$$
	的古尔萨问题的近似解, 并决定在特征线$x=t$上的函数$v$的数值及在特征
	线$x=-t$上函数$u$的数值.
\end{exercise}

\begin{solve}
	易知方程组的特征根为$\lambda_{1,2}=\pm$l,那么过$(x,t)$点的两条特征线方程分
	别为
	$$l_{1}:\quad x_{1}(\tau)=\tau+x-t,\\l_{2}:\quad x_{2}(\tau)=-\tau+x+t.$$
	那么, $A$点坐标为$\left(\frac{x-t}2,-\frac{x-t}2\right),B$点坐标为$\left(\frac {x+t}2,\frac{x+t}2\right)$. Goursat 问题的积分方程组为
	$$\begin{cases}
		\displaystyle u(x,t)=u\left(x_{2}\left(\frac{x+t}{2};x,t\right),\frac{x+t}{2}\right)+\int_{\frac{x+t}{2}}^{t}v(x_{2}(\tau;x,t),\tau)d\tau,\\
		\displaystyle v(x,t)=v\left(x_{1}\left(-\frac{x-t}{2};x,t\right),\frac{-(x-t)}{2}\right)+\int_{-\frac{x-t}{2}}^{t}u(x_{1}(\tau;x,t),\tau)d\tau.
	\end{cases}$$
	进一步, 构造积分方程组的逼近解序列为
	$$\begin{cases}
		u^{(n)}(x,t)=\sin\left(\frac{x+t}{2}\right)+\int_{\frac{v+t}{t}}^{t}v^{(n-1)}(-\tau+x+t,\tau)d\tau,\\
		v^{(n)}(x,t)=\cos\left(\frac{x-t}{2}\right)+\int_{-\frac{x-t}{2}}^{t}u^{(n-1)}(\tau+x-t,\tau)d\tau,
	\end{cases}$$
	其中 $u^{(0)}(x;t)=\sin\left(\dfrac{x+t}2\right),v^{(0)}(x;t)=\cos\left(\dfrac{x-t}2\right)$.
	
	那么
	$$\begin{cases}
		u^{(1)}(x,t)=u^{(2)}(x,t)=\cdots=u^{(n)}(x,t)=\sin\dfrac{x+t}{2},\\
		v^{(1)}(x,t)=v^{(2)}(x,t)=\cdots=v^{(n)}(x,t)=\cos\dfrac{x-t}{2}.
	\end{cases}$$
	于是, 近似解为
	\[u(x,t)=\sin\frac{x+t}{2},\quad v(x,t)=\cos\frac{x-t}{2},\\v|_{x=t}=1,\quad u|_{x=-t}=0.\qedhere\]
\end{solve}


\begin{exercise}
	证明:方程组(3.1)的任何广义柯西问题(即在某一处处不与特征方向相切的曲线$C:t=t(\sigma),x=x(\sigma)$上给定函数$u_i$的数值$u_i=\varphi_i(\sigma)$ $(i=$ $1,2,\cdots,N)$,要在曲线 $C$ 的某一侧求方程(3.1)满足此定解条件的解),都可通过适当的坐标变换化为普通的柯西问题来解决.
\end{exercise}

\begin{proof}
	作坐标变换
	$$\begin{cases}&s=t-t(\sigma),\\&y=x(\sigma).\end{cases}$$
	令$\bar{V}_i(y,s)=V_i(y,s+t(\sigma)),(i=1,2,\ldots,N)$,方程组 (3.1) 转换为
	$$\frac{\partial V_{i}}{\partial s}+\lambda_{i}(y,s+t(\sigma))\left(\frac{\partial V_{i}}{\partial y}+\frac{\partial V_{i}}{\partial s}\left(-\frac{t'(\sigma)}{x'(\sigma)}\right)\right)=\sum_{j=1}^{N}a_{ij}(y,s+t(\sigma))\bar{V_{j}}+\beta_{i}(y,s+t(\sigma)),$$
	即
	$$\left(1-\frac{t'(\sigma)}{x'(\sigma)}\lambda_{i}(y,s+t(\sigma))\right)\frac{\partial\bar{V}_{i}}{\partial s}+\lambda_{i}(y,s+t(\sigma))\frac{\partial\bar{V}_{i}}{\partial y}=\sum_{j=1}^{N}a_{ij}(y,s+t(\sigma))\bar{V}_{j}+\beta_{i}(y,s+t(\sigma)).$$
	由于曲线$C$与特征线处处不相切, 则$1-\dfrac{t^{\prime}(\sigma)}{x^{\prime}(\sigma)}\lambda_{i}(y,s+t(\sigma))$处处不为零。那么得到在区域$s>0$上的Cauthy问题
	$$\begin{cases}
		\displaystyle\frac{\partial\bar{V_i}}{\partial s}+\frac{x'(\sigma)\lambda_i(y,s+t(\sigma))}{x'(\sigma)-t'(\sigma)\lambda_i(y,s+t(\sigma))}\frac{\partial\bar{V_i}}{\partial y}\\
		=\displaystyle x^{\prime}(\sigma)\left(\sum_{j=1}^{N}\frac{a_{ij}(y,s+t(\sigma))}{x^{\prime}(\sigma)-t^{\prime}(\sigma)\lambda_{i}(y,s+t(\sigma))}\vec{V}_{j}+\frac{\beta_{i}(y,s+t(\sigma))}{x^{\prime}(\sigma)-t^{\prime}(\sigma)\lambda_{i}(y,s+t(\sigma))}\right),s>0\\
		\bar{V}_{i}(y,t(\sigma))=\varphi_{i}(\sigma).
	\end{cases}$$
	其中$,i=1,2,\ldots,N.$于是广义柯西问题变换为普通柯西问题。
\end{proof}

\begin{exercise}
	证明积分方程组(4.4)存在唯一的 $C^1$ 解, 并证明该解满足(4.2).
\end{exercise}

\begin{proof}
	使用逐次逼近法证明积分方程组解的存在唯一性.
	
	作逼近解
	$$\begin{pmatrix}u^{(0)}\\v^{(0)}\end{pmatrix}=\begin{pmatrix}\varphi(\xi_1(x,t))\\\psi(\xi_2(x,t))\end{pmatrix},$$
	$$\begin{pmatrix}u^{(n+1)}(x,t)\\v^{(n+1)}(x,t)\end{pmatrix}=\begin{pmatrix}\varphi(\xi_1(x,t))\\\psi(\xi_2(x,t))\end{pmatrix}+\begin{pmatrix}\int_{h(\xi_1(x,t))}^t\left(a_1u^{(n)}+b_1v^{(n)}+c_1\right)(x_1(\tau;x,t),\tau)d\tau\\\int_{h(\xi_2(x,t))}^t\left(a_2u^{(n)}+b_2v^{(n)}+c_2\right)(x_2(\tau;x,t),\tau)d\tau\end{pmatrix}$$
	其中, $n=0,1,2,\ldots,x=x_{i}(t;x^{0},t^{0})$表示经过点$(x^{0},t^{0})$的第$i$条特征线	$l_i,(i=1,2).$
	
	于是, 得到函数组序列 $\displaystyle\left\{\left(u^{(n)}(x,t), v^{(n)}(x,t)\right)^{T}\right\}_{n=1}^{\infty}$. 下面证明该序列在区域 $G$ 上为一致收敛.
	
	这相当于证明级数
	\begin{equation}\label{eq:5.11}
		\left(u^{(0)}(x,t),v^{(0)}(x,t)\right)^T+\sum_{n=0}^\infty\left[\left(u^{(n+1)}(x,t),v^{(n+1)}(x,t)\right)^T-\left(u^{(n)}(x,t),v^{(n)}(x,t)\right)^T\right]
	\end{equation}
	在$\bar{G}$上一致收敛.
	
	因为函数$u^{(0)}(x,t),v^{(0)}(x,t)$和$u^{(1)}(x,t),v^{(1)}(x,t)$在$\bar{G}$上连续, 故在$\bar{G}$中有界. 记
	$$M=\max_{(x,t)\in\bar{G}}\left\{\left|u^{(0)}(x,t)\right|,\:\left|v^{(0)}(x,t)\right|,\:\left|u^{(1)}(x,t)\right|,\:\left|v^{(1)}(x,t)\right|\right\},$$
	$$A=\max_{(x,t)\in\bar{G}}\left\{|a_{1}|,\:|a_{2}|,\:|b_{1}|,\:|b_{2}|\right\},\quad B=\min_{(x,t)\in\bar{G}}h(x),$$
	在$\bar{G}$上有
	$$\begin{aligned}
		&\left|u^{(1)}(x,t)-u^{(0)}(x,t)\right|\leq2M,\quad\left|v^{(1)}(x,t)-v^{(0)}(x,t)\right|\leq2M,\\
		&\left|u^{(2)}(x,t)-u^{(1)}(x,t)\right|\leq\int_{h(\xi_{1}(x,t))}^{t}|a_{1}|\left|u^{(1)}-u^{(0)}\right|+|b_{1}|\left|v^{(1)}-v^{(0)}\right|d\tau\leq4MA(t-B),\\
		&\left|v^{(2)}(x,t)-v^{(1)}(x,t)\right|\leq\int_{h(\xi_{2}(x,t))}^{t}|a_{2}|\left|u^{(1)}-u^{(0)}\right|+|b_{2}|\left|v^{(1)}-v^{(0)}\right|d\tau\leq4MA(t-B),\\
		&\cdots.
	\end{aligned}$$
	现在假定
	$$\left|u^{(n)}(x,t)-u^{(n-1)}(x,t)\right|\leq\frac{2M(2A(t-B))^{n-1}}{(n-1)!},$$
	及
	$$\left|v^{(n)}(x,t)-v^{(n-1)}(x,t)\right|\leq\frac{2M(2A(t-B))^{n-1}}{(n-1)!},$$
	成立, 则
	$$\begin{aligned}
		\left|u^{(n+1)}(x,t)-u^{(n)}(x,t)\right|&\leq\int_{h(\xi_{1}(x,t))}^{t}|a_{1}|\left|u^{(n)}-u^{(n-1)}\right|+|b_{1}|\left|v^{(n)}-v^{(n-1)}\right|d\tau\\
		&\leq\frac{1}{n!}2M(2A(t-B))^{n},
	\end{aligned}$$
	$$\begin{aligned}
		\left|v^{(n+1)}(x,t)-v^{(n)}(x,t)\right|&\leq\int_{h(\xi_{2}(x,t))}^{t}|a_{2}|\left|u^{(n)}-u^{(n-1)}\right|+|b_{2}|\left|v^{(n)}-v^{(n-1)}\right|d\tau\\
		&\leq\frac{1}{n!}2M(2A(t-B))^{n}.
	\end{aligned}$$
	由数学归纳法知, 对于任意$n$都有
	$$\begin{aligned}
		\left|u^{(n+1)}(x,t)-u^{(n)}(x,t)\right|\leq\frac{2M(2A(t-B))^{n}}{n!},\\
		\left|v^{(n+1)}(x,t)-v^{(n)}(x,t)\right|\leq\frac{2M(2A(t-B))^{n}}{n!}.
	\end{aligned}$$

	取定$\bar{G}$中最大的$t$值为$T_0$,则
	$$\left|u^{(n+1)}(x,t)-u^{(n)}(x,t)\right|\leq\frac{2M(2A(T_0-B))^n}{n!},$$
	$$\left|v^{(n+1)}(x,t)-v^{(n)}(x,t)\right|\leq\frac{2M(2A(T_0-B))^n}{n!}.$$
	
	由于数项级数$\dfrac{(2A(T_0-B))^n}{n!}$收敛, 则级数\eqref{eq:5.11}在$\bar{G}$中一致收敛, 那么积分方程组的解存在且连续.
	
	下面证明在$\bar{G}$中连续解$(u,v)^T$的唯一性.
	
	假定积分方程组有两个解$(u,v)^T$与$(\bar{u},\bar{v})^T$,代入方程组作差得
	$$\begin{cases}
		\displaystyle u(x,t)-\bar{u}(x,t)=\int_{h(\xi_1(x,t))}^ta_1(u-\bar{u})+b_1(v-\bar{v})d\tau,\\
		\displaystyle v(x,t)-\bar{v}(x,t)=\int_{h(\xi_2(x,t))}^ta_2(u-\bar{u})+b_2(v-\bar{v})d\tau.
	\end{cases}$$
	假定
	$$\max_{(x,t)\in\bar{G}}\{|u(x,t)-\bar{u}(x,t)|,\:|v(x,t)-\bar{v}(x,t)|\}=M_{1}>0,$$
	重复使用存在性使用的方法知, 对于任何$n$成立
	$$M_1\leq M_1\frac{(2A(T_0-B))^n}{n!}$$
	当$n$充分大时, 导致矛盾. 所以必须有$M_1=0.$
	
	因而, 在$\bar{G}$中
	$$u(x,t)\equiv\bar{u}(x,t),\quad v(x,t)\equiv\bar{v}(x,t).$$
	
	下面证明$(u,v)^T\in C^1(\bar{G}).$
	
	对逼近解两端对$x$求导, 得到
	$$\begin{aligned}
		&\frac{\partial u^{(n+1)}(x,t)}{\partial x}\\
		&=\varphi'(\xi_{1}(x,t))\frac{\partial\xi_{1}}{\partial x}+\int_{h(\xi_{1}(x,t))}^{t}\left(\frac{\partial a_{1}}{\partial x_{1}}u^{(n)}+\frac{\partial b_{1}}{\partial x_{1}}v^{(n)}+\frac{\partial c_{1}}{\partial x_{1}}\right)\frac{\partial x_{1}(\tau;x,t)}{\partial x}d\tau\\
		&+\int_{h(\xi_{1}(x,t))}^{t}\left(a_{1}\frac{\partial u^{(n)}}{\partial x_{1}}+b_{1}\frac{\partial v^{(n)}}{\partial x_{1}}\right)\frac{\partial x_{1}(\tau;x,t)}{\partial x}d\tau\\
		&+\left(a_1u^{(n)}+b_1v^{(n)}+c_1\right)(x_1(h(\xi_1(x,t));x,t),h(\xi_1(x,t)))-h^{\prime}(\xi_1(x,t))\frac{\partial\xi_1}{\partial x}.
	\end{aligned}$$

	由数学归纳法知$\left\{\dfrac{\partial u^{(n)}}{\partial x}\right\} (n=0, 1, 2, \ldots)$在$\bar{G}$上连续. 根据证明$\left\{u^{(n)}\right\}$一致收敛的方法可得$\left\{\dfrac{\partial u^{(n)}}{\partial x}\right\}$在$\bar G$上一致收敛, 其极限函数为$\dfrac{\partial u}{\partial x}$在$\bar{G}$中连续.
	
	同理可证$\dfrac{\partial u}{\partial t}$, $\dfrac{\partial v}{\partial x}$, $\dfrac{\partial v}{\partial t}$在$\bar{G}$中连续.
	
	综上积分方程组存在唯一的$C^1$解.
	
	在$\widehat{AB}$上, 即$t= h(x), (x,t)\in\widehat{AB}$, $\xi_1(x,t)=\xi_2(x,t)=x$,由逼近解序列
	$$\begin{cases}
		u^{(n)}(x,t)=\varphi(\xi_1(x,t))=\varphi(x),\\
		v^{(n)}(x,t)=\psi(\xi_1(x,t))=\psi(x),
	\end{cases}$$
	则有$u(x,t)=\varphi (x)$, $v(x,t)=\psi (x)$. 因此, $(u,v)^T$满足问题 (4.2).
\end{proof}


\begin{exercise}
	写出与古尔萨问题等价的积分方程组.
\end{exercise}

\begin{solve}
	Goursat 问题设$\widehat{PA},\widehat{PB}$分别是$P$点的第一和第二簇特征线, 给定
	$$V|_{\widehat{PA}}=\varphi(x),\quad u|_{\widehat{PB}}=\psi(x),$$
	求$\widehat{PA},\widehat{PB}$所夹的角状区域$G$中方程组的解.
	
	设$\widehat{PA},\widehat{PB}$的方程分别为$t=h_1(x), t=h_2(x)$,过区域$G$上的任何一点$(x,t)$都能作全落在$G$上的两条特征线$l_1,l_2$,分别与曲线$\widehat{PA}$和$\widehat{PB}$相交,  交点的横坐标分别由关系
	$$\xi=x_1(h_2(\xi);x,t)$$
	和
	$$\xi=x_2(h_1(\xi);x,t)$$
	确定.
	
	记交点横坐标分别为$\xi_1(x,t)$与$\xi_2(x,t)$,则 Goursat 问题等价于积分方
	程组
	$$\begin{cases}
		\displaystyle u(x,t)=\varphi(\xi_1(x,t))+\int_{h_2(\xi_1(x,t))}^t(a_1+b_1v+c_1)(x_1(\tau;x,t),\tau)d\tau,\\
		\displaystyle v(x,t)=\psi(\xi_2(x,t))+\int_{h_1(\xi_2(x,t))}^t(a_2u+b_2v+c_2)(x_2(\tau;x,t),\tau)d\tau.
	\end{cases}$$
\end{solve}