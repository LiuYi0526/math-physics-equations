\documentclass{report}
\usepackage{ctex}
\usepackage{amsmath}
\usepackage{amssymb}
\usepackage{amsthm}
\usepackage{bm}
\usepackage{mathrsfs}
\usepackage{bbm}
\usepackage{xcolor}
\usepackage[shortlabels]{enumitem}
\usepackage{geometry}
\usepackage[unicode=true,colorlinks=true,linkcolor=red]{hyperref}
\usepackage{fixdif}

\setlist{nosep}

\geometry{left=4cm,right=4cm}
\newcommand\diff{\mathop{}\!\mathrm{d}}
\newcommand{\e}{\mathrm{e}}
\newcommand{\upi}{\mathrm{i}}
\newcommand{\closure}[1]{\overline{#1}}
\DeclareMathOperator{\diag}{diag}
\DeclareMathOperator{\dist}{dist}

\setlist[enumerate,1]{left=\parindent}

\setCJKmainfont{FandolSong}

\theoremstyle{remark}
\newtheorem*{remark}{注}
\newtheoremstyle{PDE}{3pt}{3pt}{\rmfamily}{\parindent}{\bfseries}{}{.5em}{}
\theoremstyle{PDE}
\newtheorem{note}{Note}[section]
\newtheorem{property}{性质}

\makeatletter
\renewenvironment{proof}{\par
	\pushQED{\qed}%
	\normalfont \topsep1\p@\@plus6\p@\relax
	\trivlist
	\item\relax
	{\hspace*{\parindent}{\bfseries 证明}\@addpunct{:}}\hspace\labelsep\ignorespaces
}{%
	\popQED\endtrivlist\@endpefalse
}
\newenvironment{solve}{\par
	\pushQED{\qed}%
	\normalfont \topsep1\p@\@plus6\p@\relax
	\trivlist
	\item\relax
	{\hspace*{\parindent}{\bfseries 解}\@addpunct{:}}\hspace\labelsep\ignorespaces
}{%
	\popQED\endtrivlist\@endpefalse
}


\newenvironment{solution}{\par
	\pushQED{\qed}%
	\normalfont \topsep1\p@\@plus6\p@\relax
	\trivlist
	\item\relax
	{\hspace*{\parindent}{\bfseries 解}\@addpunct{:}}\hspace\labelsep\ignorespaces
}{%
	\popQED\endtrivlist\@endpefalse
}
\makeatother
\newcounter{exercise}[section]
\NewDocumentEnvironment{exercise}{o d<> +b}
  {%
    \IfNoValueTF{#1}
      {\stepcounter{exercise}}
      {\setcounter{exercise}{#1}}
    \par\textbf{\theexercise.}\hspace{.333em}%
	\IfNoValueTF{#2}
	  {#3}
	  {\textbf{(#2)}\hspace{.333em}#3}
  }
  {\ignorespacesafterend}
\renewcommand{\div}{\operatorname{div}}

\title{\textbf{数学物理方程习题解答}}
\author{2017301000046武思豫}

\begin{document}

\maketitle
\tableofcontents

\chapter{波动方程}

\section{方程的导出,定解条件}


\begin{exercise}
  细杆或弹簧受某种外界原因而产生纵向振动, 
  以 $u(x,t)$ 表示静止时在 $x$ 点处的点在时刻 $t$ 离开原来位置的偏移.
  假设振动过程中所发生的张力服从胡克定律, 试证明 $u(x,t)$ 满足方程:
  \[\frac{\partial}{\partial t}\left(\rho(x)\frac{\partial u}{\partial t}\right)
    = \frac{\partial}{\partial x}\left(E\frac{\partial u}{\partial x}\right),\]
  其中 $\rho$ 为杆的密度, $E$为杨氏模量%
  \footnote{杨氏模量$E=\frac{F/S}{\Delta L/L}$衡量的是一个各向同性弹性体的刚度(stiffness),
    定义为在胡克定律适用的范围内,单轴应力和单轴形变之间的比.}.
\end{exercise}

\begin{proof}
  记杆的横截面积为 $S$,取杆的左端截面的形心为原点, 杆轴为 $x$ 轴,
  任取静止时坐标为 $(x,x+\Delta x)$的一小段细杆$B$.
  在 $t$ 时刻, $B$的两段位移分别为$u(x,t)$和$u(x+\Delta x,t)$,
  $B$ 的伸长为 $\Delta L=u(x+\Delta x,t)-u(x,t)$, 故 $B$ 的应变为
  \[\frac{u(x+\Delta x,t)-u(x,t)}{\Delta x}\]
  令 $\Delta x\to 0$ 即得 $x$ 点在时刻 $t$ 的应变为 $\frac{\partial u}{\partial x}(x,t)$,
  因此 $B$ 两端的张力分别为 $ES\frac{\partial u}{\partial x}(x,t)$
  和 $ES\frac{\partial u}{\partial x}(x+\Delta x,t)$, 故 $B$ 的运动方程为
  \[ES\frac{\partial u}{\partial x}(x+\Delta x,t)-ES\frac{\partial u}{\partial x}(x,t)
    = S\rho(x)\Delta x\frac{\partial^2u}{\partial t^2}(\bar{x},t).\]
  消去 $S$ 并令 $\Delta x\to 0$ 即得所证.
\end{proof}


\begin{exercise}
  在杆纵向振动时, 假设 (1)端点固定, (2)端点自由, (3)端点固定在弹性支承上,
  试分别导出这三种情况下所对应的边界条件.
\end{exercise}

\begin{proof}
	内容...
\end{proof}


\begin{exercise}
  试证:圆锥形枢轴的纵振动方程为
  \[E\frac{\partial}{\partial x}\left[\left(1-\frac{x}{h}\right)^2\frac{\partial u}{\partial x}\right]=\rho\left(1-\frac{x}{h}\right)^2\frac{\partial^2u}{\partial t^2}\]
  其中$h$为圆锥的高.
\end{exercise}

\begin{proof}
	内容...
\end{proof}


\begin{exercise}
  绝对柔软而均匀的弦线有一端固定,在它本身重力的作用下,此线处于铅锤的平衡位置,试导出此线的微小横振动方程.
\end{exercise}

\begin{solve}
	内容...
\end{solve}


\begin{exercise}
  一柔软均匀的细弦, 一端固定, 另一段是弹性支承.
  设该弦在阻力与速度成正比的介质中作微小的横振动, 试写出弦的位移所满足的定解问题.
\end{exercise}

\begin{solve}
  由教材推导过程知此时满足:
  \[\int_t^{t+\Delta t}\int_x^{x+\Delta x}\left(T\frac{\partial^2u(x,t)}{\partial x^2}\
    - \rho\frac{\partial^2u(x,t)}{\partial t^2}
    - b\frac{\partial u(x,t)}{\partial t}\right)\diff x\diff t = 0.\]
  因此定解问题为(不妨设弹性支承在右端):
  \[\begin{cases}
    \frac{\partial^2u}{\partial t^2}
    - a^2\frac{\partial^2u}{\partial x^2}
    + \frac{b}{\rho}\frac{\partial u}{\partial t}
    = 0, \\
    u(x,0)
    = \varphi(x),\frac{\partial u(x,0)}{\partial t}
    = \psi(x)\\u|_{x=0}=0,\left(\frac{\partial u}{\partial x}
    + \sigma u\right)|_{x=l}=0.
  \end{cases}\qedhere\]
\end{solve}


\begin{exercise}
  若 $F(\xi)$, $G(\xi)$ 均为其变元的二次连续可导函数,
  验证 $F(x-at)$, $G(x+at)$ 均满足弦振动方程 (1.11).
\end{exercise}

\begin{proof}
  直接验证即可.
\end{proof}


\begin{exercise}
  验证
  \[u(x,y,t) = \frac{1}{\sqrt[]{t^2-x^2-y^2}}\]
  在锥 $t^2-x^2-y^2>0$ 中满足波动方程
  \[\frac{\partial^2 u}{\partial t^2}
    = \frac{\partial^2 u}{\partial x^2}
      + \frac{\partial^2 u}{\partial y^2}.\]
\end{exercise}

\begin{proof}
  直接计算
  \[\frac{\partial u}{\partial t}=-t(t^2-x^2-y^2)^{-3/2},\quad
    \frac{\partial^2u}{\partial t^2}=(t^2-x^2-y^2)^{-5/2}(2t^2+x^2+y^2),\]
  \[\frac{\partial u}{\partial x}=x(t^2-x^2-y^2)^{-3/2},\quad
    \frac{\partial^2u}{\partial x^2}=(t^2-x^2-y^2)^{-5/2}(t^2-2x^2+y^2).\]
  同理
  \[\frac{\partial u^2}{\partial y^2}=(t^2-x^2-y^2)^{-5/2}(t^2+2y^2-x^2).\]
  结合以上三式可得
  \[\frac{\partial^2u}{\partial t^2}
    = \frac{\partial^2u}{\partial x^2}+\frac{\partial^2u}{\partial y^2}.\qedhere\]
\end{proof}


\section{达朗贝尔公式、波的传播}

\begin{exercise}
  设 $h>0$ 为常数, 证明方程
  \[\frac{\partial}{\partial x}\biggl[\biggl(1-\frac{x}{h}\biggr)^2
    \frac{\partial u}{\partial x}\biggr]
    = \frac{1}{a^2}\biggl(1-\frac{x}{h}\biggr)^2 \frac{\partial^2u}{\partial t^2}\]
  的通解可以写成
  \[u = \frac{F(x-at)+G(x+at)}{h-x},\]
  其中 $F$, $G$ 为任意的具有二阶连续导数的单变量函数, 并由此求它满足初始条件
  \[t=0: u = \varphi(x), \frac{\partial u}{\partial t} = \psi(x)\]
  的初值问题的解.
\end{exercise}

\begin{solve}
  原方程等价于
  \[\frac{\partial^2}{\partial x^2}[(h-x)u]
    = \frac{1}{a^2}\frac{\partial^2}{\partial t^2}[(h-x)u].\]
  故存在函数 $F$, $G$ 使得
  \[(h-x)u = F(x-at) + G(x+at),\]
  也即
  \[u = \frac{F(x-at)+G(x+at)}{h-x}.\]
  初值问题的解为
  \begin{align*}
    u(x,t)
     = \frac{1}{h-x}\biggl[
      & \frac12 (h-x+at)\varphi(x-at) + \frac12 (h-x-at)\varphi(x+at) \\
      & + \frac{1}{2a}\int_{x-at}^{x+at} (h-\xi)\psi(\xi)\diff\xi\biggr]. \qedhere
  \end{align*}
\end{solve}


\begin{exercise}
  问初始条件 $\varphi(x)$ 与 $\psi(x)$ 满足怎样的条件时,
  齐次波动方程初值问题的解仅由右传播波组成?
\end{exercise}

\begin{proof}
  \[\varphi(x) + \frac{1}{a} \Psi(x) = C\]
  or
  \[\varphi'(x) + \frac{1}{a} \psi(x) = 0.\qedhere\]
\end{proof}


\begin{exercise}[3]
  利用传播波法, 求解波动方程的古尔萨 (Goursat) 问题
  \[\begin{cases}
    \frac{\partial^2u}{\partial t^2} = a^2 \frac{\partial^2u}{\partial x^2}, \\
    u|_{x-at=0} = \varphi(x), \\
    u|_{x+at=0} = \psi(x),\; \varphi(0) = \psi(0).
  \end{cases}\]
\end{exercise}

\begin{solve}
  设 $u(x,t)$ 具有行波解 $u(x,t) = F(x-at)+G(x+at)$, 由边界条件得
  \[F(0)+G(2x) = \varphi(x),\]
  \[F(2x)+G(0) = \psi(x).\]
  由上式得 $F(x) = \psi(x/2)-G(0)$, $G(x) = \varphi(x/2)-F(0)$.
  取 $(x,t) = (0,0)$, 得
  \[u(0,0) = F(0) + G(0) = \varphi(0) = \psi(0).\]
  从而
  \[ \boxed{u(x,t) = \psi\left(\frac{x-at}{2}\right)
    + \varphi\left(\frac{x+at}{2}\right)-\varphi(0).} \qedhere\]
\end{solve}


\begin{exercise}[5]
  Solve
  \[\begin{cases}
    u_{tt} - a^2 u_{xx} = 0,\quad x>0, t>0, \\
    u|_{t=0} = \varphi(x),\quad u_t|_{t=0} = 0, \\
    u_x - ku_t|_{x=0} = 0,
  \end{cases}\]
  in which $k$ is a constant.
\end{exercise}

\begin{solution}
  If $x-at\geq 0$, then
  \[ \boxed{u(x,t) = \frac12 \bigl(\varphi(x+at) + \varphi(x-at)\bigr).} \]
  If $x-at\leq 0$, then
  \[ \boxed{u(x,t) = \frac{1-ak}{2(1+ak)}\varphi(at-x)
    + \frac12\varphi(x+at) + \frac{ak}{1+ak}\varphi(0).}\qedhere \]
\end{solution}

\begin{exercise}[6]
  求解初边值问题
  \[\begin{cases}
    u_{tt} - u_{xx} = 0,\quad 0<t<kx, k>1, \\
    u|_{t=0} = \varphi_0(x),\quad x\geq 0, \\
    u_t|_{t=0} = \varphi_1(x),\quad x\geq 0, \\
    u|_{t=kx} = \psi(x),
  \end{cases}\]
  其中 $\varphi_0(0) = \psi(0)$.
\end{exercise}

\begin{solve}
  当$x-t\geq 0$时,由d'Alembert公式有
  \[u(x,t)=\frac{1}{2}(\varphi_0(x-t)+\varphi_0(x+t))+\frac{1}{2}\int_{x-t}^{x+t}\varphi_1(\xi)d\xi\]
  当$x-t<0$时,设$u=F(x-t)+G(x+t)$,利用边界条件得
  \[F((1-k)x)+G((1+k)x)=\psi(x)\cdots(*1)\]
  又\[F(0)+G(2x)=\frac{1}{2}(\varphi_0(0)+\varphi_0(2x))+\frac{1}{2}\int_0^{2x}\varphi_1(\xi)d\xi\cdots(*2)\]
  由$(*2)$有
  \[F(0)+G((1+k)x)=\frac{1}{2}(\varphi_0(0)+\varphi_0((1+k)x))+\frac{1}{2}\int_0^{(k+1)x}\varphi_1(\xi)d\xi\cdots(*3)\]
  $(*1)-(*3)$得\[F((1-k)x)-F(0)=\psi(x)-\frac{1}{2}(\varphi_0(0)+\varphi_0((1+k)x))-\frac{1}{2}\int_0^{(k+1)x}\varphi_1(\xi)d\xi\cdots(*4)\]
  故\[\begin{split}F(x-t)-F(0)=&\psi\left(\frac{x-t}{1-k}\right)-\frac{1}{2}\left(\varphi_0(0)+\varphi_0\left(\frac{1+k}{1-k}(x-t)\right)\right)\\&-\frac{1}{2}\int_0^{\frac{k+1}{1-k}(x-t)}\varphi_1(\xi)d\xi\cdots(*5)\end{split}\]
  又由$(*2)$有\[F(0)+G(x+t)=\frac{1}{2}(\varphi_0(0)+\varphi_0(x+t))+\frac{1}{2}\int_0^{x+t}\varphi_1(\xi)d\xi\cdots(*6)\]
  结合$(*5)(*6)$得\[\begin{split}u(x,t)&=F(x-t)+G(x+t)=\psi\left(\frac{x-t}{1-k}\right)+\frac{1}{2}\left[\varphi_0(x+t)-\varphi_0\left(\frac{1+k}{1-k}(x-t)\right)\right]\\&+\frac{1}{2}\int_{\frac{k+1}{1-k}(x-t)}^{x+t}\varphi_1(\xi)d\xi\end{split}\]
\end{solve}


\begin{exercise}
  Solve the following initial value problem
  \[\begin{cases}
    u_{tt} - u_{xx} = 0,\quad f(t) < x < t, \\
    u|_{x=t} = \varphi(t), \\
    u|_{x = f(t)} = \psi(t),
  \end{cases}\]
  where $\varphi(0)=\psi(0)=0$ and $x = f(t)$ is a smooth curve
  passing the origin point and between $x=t$ and $x=-t$,
  and $|f'(t)|\neq 1$ for all $t$.
\end{exercise}

\begin{solution}
  \[\boxed{u(x,t) = \varphi\biggl(\frac{x+t}{2}\biggr)
    - \varphi\biggl(g(x-t) + \frac{x-t}{2}\biggr) + \psi(g(x-t)),}\]
  where $t = g(h)$ is the inverse function of $h = f(t) - t$.
\end{solution}


\begin{exercise}[8]
  求解波动方程的初值问题
  \[
    \begin{cases}
      \frac{\partial^2u}{\partial t^2} - \frac{\partial^2u}{\partial x^2} = t\sin x, \\
      u|_{t=0} = 0,\; \frac{\partial u}{\partial t}\Big|_{t=0} = \sin x.
    \end{cases}
  \]
\end{exercise}

\begin{solve}
  \[u(x,t) = \frac{1}{2}\int_{x-t}^{x+t}\sin \xi\d\xi
      + \frac{1}{2}\int_0^t\int_{x-t+\tau}^{x+t-\tau}\tau\sin \xi\d\xi\d\tau 
      = t\sin x.\qedhere\]
\end{solve}


\begin{exercise}[9]
  求解波动方程的初值问题
  \[\begin{cases}
    u_{tt} = a^2u_{xx} + \frac{tx}{(1+x^2)^2}, \\
    u|_{t=0} = 0, \\
    u_t|_{t=0} = \frac{1}{1+x^2}.
  \end{cases}\]  
\end{exercise}


\begin{solve}
  \begin{align*}
    u(x,t)
    ={} & \frac{1}{2a}\int_{x-at}^{x+at}\frac{1}{1+\xi^2}\d\xi
      + \frac{1}{2a}\int_0^t\int_{x-a(t-\tau)}^{x+a(t-\tau)}
      \frac{\tau\xi}{(1+\xi^2)^2}\d\xi\d\tau \\
    ={} & -\frac{1}{4a^3}\bigg[\frac{1}{2}\ln\frac{1+(x-at)^2}{1+(x+at)^2}
          - 2at\arctan x \\
        & + (x+at-2a^2)\arctan (x+at) - (x-at-2a^2)\arctan (x-at)\bigg].\qedhere
  \end{align*}
\end{solve}


\section{初边值问题的分离变量法}

\begin{exercise}
  用分离变量法求下列问题的解:
  \begin{enumerate}[(1)]
    \item $\begin{cases}
      \frac{\partial^2u}{\partial t^2} = a^2 \frac{\partial^2u}{\partial x^2}, \\
      u(x,0) = \sin\frac{3\pi x}{l},\quad u_t(x,0) = x(l-x)\quad (0<x<l), \\
      u(0,t) = u(l,t) = 0.
    \end{cases}$
    \item $\begin{cases}
      \frac{\partial^2u}{\partial t^2} - a^2 \frac{\partial^2u}{\partial x^2} = 0, \\
      u(x,0) = \frac{h}{l}x, \quad u_t(x,0) = 0, \\
      u(0,t) = 0,\quad u_x(l,t) = 0.  
    \end{cases}$
  \end{enumerate}
\end{exercise}

\begin{solve}
  (2) 边界条件是线性的, 故设 $u(x,t)=X(x)T(t)$,由边界条件知 $X(x)$ 满足定解问题
  \[\begin{cases}
    X''(x)+\lambda X(x) = 0, \\
    X(0) = 0,\quad X'(l) = 0.
  \end{cases}\]
  根据 $\lambda$ 的符号分类讨论得
  \[X_k(x)=C_k\sin \frac{(2k+1)\pi}{2l}x,\quad k=0,1,2,\ldots\]
  其中相应的特征值 $\lambda_k=\left(\frac{2k+1}{2l}\pi\right)^2$.
  由 $T''(t)+\lambda_k a^2T(t)=0$ 解得
  \[T_k(t) = A_k\cos\frac{(2k+1)a\pi}{2l}t+B_k\sin \frac{(2k+1)a\pi}{2l}t.\]
  于是
  \[u(x,t) = \sum_{k=0}^{\infty}\left(A_k\cos\frac{(2k+1)\pi a}{2l}t
    + B_k\sin \frac{(2k+1)\pi a}{2l}t\right)\sin \frac{(2k+1)\pi}{2l}x.\]
  再根据初始条件得
  \[\sum_{k=0}^{\infty}A_k\sin \frac{(2k+1)\pi}{2l}x=\frac{h}{l}x,\]
  \[\sum_{k=0}^{\infty}B_k\frac{(2k+1)\pi a}{2l}\sin \frac{(2k+1)\pi}{2l}x = 0.\]
  利用三角函数序列 $\Bigl(\sin \frac{(2k+1)\pi}{2l}x\Bigr)_{k\geq 0}$
  在区间$[0,l]$上的正交性, 即
  \[\int_0^l\sin \frac{(2m+1)\pi}{2l}x\cdot\sin \frac{(2n+1)\pi}{2l}x\diff x
    = \frac{l}{2}\delta_{mn},\]
  得
  \[A_k=\frac{2}{l}\int_0^l\frac{h}{l}x\sin \frac{(2k+1)\pi}{2l}x\diff x
    =(-1)^k\frac{8h}{(2k+1)^2\pi^2},\quad B_k = 0.\]
  因此
  \[u(x,t) = \sum_{k=0}^{\infty}(-1)^k\frac{8h}{(2k+1)^2\pi^2}\cos\frac{(2k+1)\pi at}{2l}
    \sin\frac{(2k+1)\pi x}{2l}.\qedhere\]
\end{solve}

\begin{remark}
  从上述过程中可以看出, 给齐次 Dirichlet 边界条件和给齐次 Neumann 边界条件时的
  特征值是不一样的.
\end{remark}

\begin{remark}
  $\int_{-l}^l\sin\alpha x\sin\beta x\diff x=0$ 及 $\int_0^l\sin\alpha x\sin\beta x\diff x=0$
  的充要条件为 $(\alpha\pm\beta)l=k\pi\neq 0$. 而且正弦函数于余弦函数总是正交的, 
  即 $\int_{-l}^l\sin\alpha x\cos\beta x\equiv 0$。
\end{remark}


\begin{exercise}
  设弹簧一端固定, 一端在外力作用下做周期振动, 此时定解问题归结为
  \[\begin{cases}
    \frac{\partial^2u}{\partial t^2} = a^2 \frac{\partial^2u}{\partial x^2}, \\
    u(0,t) = 0, \quad u(l,t) = A\sin^2\omega t, \\
    u(x,0) = u_t(x,0) = 0.
  \end{cases}\]
  求解此问题.
\end{exercise}

\begin{solve}
  边界条件非齐次,故令$U(x,t)=\frac{x}{l}A\sin^2\omega t$,
  $V(x,t)=u(x,t)-U(x,t)$, 则 $V(x,t)$ 满足的定解问题是
  \[\begin{cases}
    \frac{\partial^2V}{\partial t^2}-a^2\frac{\partial^2V}{\partial x^2}
      = -\frac{2\omega^2A}{l}x\cos 2\omega t, \\
    V(0,t)=0,\quad V(l,t)=0, \\
    V(x,0)=0,\quad \frac{\partial V(x,0)}{\partial t} = 0.
  \end{cases}\]
  运用齐次化原理知该问题的解为
  \[u(x,t)=\sum_{k=1}^{\infty}\int_0^tB_k(\tau)\sin\frac{k\pi a}{l}(t-\tau)\diff\tau
    \cdot\sin\frac{k\pi}{l}x.\]
  其中
  \[\begin{split}
    B_k(\tau)
    & = \frac{2}{k\pi a}\int_0^lf(\xi,\tau)\sin\frac{k\pi}{l}\xi\diff\xi \\
    & = \frac{2}{k\pi a}\int_0^l-\frac{2\omega^2A}{l}\xi\cos2\omega\tau
      \sin\frac{k\pi}{l}\xi\diff\xi \\
    & = \frac{4(-1)^k\omega^2lA\cos2\omega\tau}{k^2\pi^2a}.\qedhere
  \end{split}\]
\end{solve}


\begin{exercise}[4]
  用分离变量法求解初边值问题:
  \[\begin{cases}
    u_{tt} - a^2 u_{xx} = g,\quad 0<x<l,\quad t>0, \\
    u|_{x=0} = u_x|_{x=l} = 0, \\
    u|_{t=0} = 0,\quad u_t|_{t=0} =\sin\frac{\pi x}{2l}.
  \end{cases}\]
  其中 $g$ 为常数.
\end{exercise}

\begin{solve}
  令 $v=u+\frac{g}{2a^2}x(x-2l)$,则
  \[\begin{cases}
    v_{tt}-a^2v_{xx}=0,0<x<l,t>0, \\
    v(0,t)=\frac{\partial v}{\partial x}(l,t)=0, \\
    v(x,0)=\frac{g}{2a^2}x(x-2l),\quad\frac{\partial v}{\partial t}(x,0)=\sin\frac{\pi x}{2l}.
  \end{cases}\]
  其通解为
  \begin{align*}
    u(x,t)
    ={} & \sum_{k=0}^{\infty}\left(A_n\cos\frac{(2k+1)\pi a}{2l}t+B_k\sin\frac{(2k+1)\pi a}{2l}t\right)\sin\frac{(2k+1)\pi}{2l}x \\
    & - \frac{g}{2a^2}x(x-2l),
  \end{align*}
  其中
  \[A_k=\frac{2}{l}\int_0^l\frac{g}{2a^2}x(x-2l)\sin\frac{(2k+1)\pi}{2l}x\diff x
    = -\frac{16l^2g}{(2k+1)^3a^2\pi^3},\]
  \[B_0=\frac{2l}{\pi a},\quad B_k=0\quad (k\geq 1).\]
\end{solve}


\begin{exercise}
  用分离变量法求下面问题的解:
  \[\begin{cases}
    \frac{\partial^2u}{\partial t^2} = a^2 \frac{\partial^2u}{\partial x^2} + b\sinh x, \\
    u|_{x=0} = u|_{x=l} = 0, \\
    u|_{t=0} = u_t|_{t=0} = 0.
  \end{cases}\]
\end{exercise}

\begin{solve}
  利用齐次化原理,方程的解为
  \[u(x,t)=\int_0^tW(x,t;\tau)\diff\tau,\]
  其中$W$为下面方程的解:
  \[\begin{cases}
    \frac{\partial^2W}{\partial t'^2}-a^2\frac{\partial^2W}{\partial x^2}=0,t'>0, \\
    W|_{t'=0} = 0,\quad\frac{\partial W}{\partial t'}|_{t'=0} = b\sinh x, \\
    W|_{x=0} = W|_{x=l} = 0.
  \end{cases}\]
  其通解为
  \[W(x,t;\tau) = \sum_{k=1}^{\infty}\left(A_k\cos\frac{k\pi a}{l}t'
    + B_k\sin\frac{k\pi a}{l}t'\right)\sin\frac{k\pi}{l}x,\]
  其中
  \[A_k = 0,\quad k=1,2,\cdots\]
  \[B_k = \frac{2}{k\pi a}\int_0^l b\sinh x\sin\frac{k\pi}{l}x\diff x
    = (-1)^{k+1}\frac{2bl\sinh l}{a(l^2+k^2\pi^2)}.\]
  故原方程解为
  \[ \boxed{u(x,t) = \frac{2bl^2\sinh l}{\pi a^2}
      \sum_{k=1}^{\infty}\frac{(-1)^{n+1}}{n(l^2+n^2\pi^2)}
      \left(1-\cos\frac{n\pi at}{l}\right)\sin\frac{n\pi x}{l}.}\qedhere \]
\end{solve}


\begin{exercise}
  Solve the following problem by separation of variables:
  \[\begin{cases}
    \frac{\partial^2 u}{\partial t^2} + 2b \frac{\partial u}{\partial t}
      = a^2 \frac{\partial^2u}{\partial x^2} \quad (b>0), \\
    u|_{x=0} = u|_{x=l} = 0, \\
    u|_{t=0} = \frac{h}{l}x,\ \frac{u}{t}\bigg|_{t=0} = 0.
  \end{cases}\]
\end{exercise}

\begin{solution}
  Let $u(x,t) = X(x)T(t)$, then
  \[\frac{T''+2bT'}{a^2T} = \frac{X''}{X} = -\lambda. \]

  First of all, $X'' + \lambda X = 0$.
  \begin{itemize}
    \item If $\lambda\leq 0$, there only exists trivial solution;
    \item If $\lambda>0$, then $\lambda_k = \bigl(\frac{k\pi}{l}\bigr)^2$
      and $X_k(x) = C_k\sin\frac{k\pi x}{l}$.
  \end{itemize}

  Thus $T$ satisfies the equation
  \[ T'' + 2bT' + \biggl(\frac{k\pi a}{l}\biggr)^2 T = 0, \]
  of which the characteristic equation is
  \[ \lambda^2 + 2b\lambda + \biggl(\frac{k\pi a}{l}\biggr)^2 = 0. \]

  We suppose that $b$ is sufficiently small such that
  $\Delta = 4b^2 - 4\bigl(\frac{k\pi a}{l}\bigr)^2 < 0$ for all $k\geq 1$, then
  \[\lambda = -b \pm \sqrt{\Bigl(\frac{k\pi a}{l}\Bigr)^2 - b^2}i, \]
  and
  \[ T_k(t) = e^{-bt} \biggl(A_k \sin \sqrt{\Bigl(\frac{k\pi a}{l}\Bigr)^2 - b^2} t
    + B_k \cos \sqrt{\Bigl(\frac{k\pi a}{l}\Bigr)^2 - b^2} t\biggr). \]
  Therefore,
  \[\boxed{u(x,t) = e^{-bt} \sum_{k=1}^\infty \sin\frac{k\pi x}{l}
      \biggl(A_k \sin \sqrt{\Bigl(\frac{k\pi a}{l}\Bigr)^2 - b^2} t
      + B_k \cos \sqrt{\Bigl(\frac{k\pi a}{l}\Bigr)^2 - b^2} t\biggr).}\]
  Combining with the initial value condition we have
  \[u|_{t=0} = \sum_{k=1}^\infty B_k \sin\frac{k\pi x}{l} = \frac{h}{l}x, \]
  and
  \[u_t|_{t=0} = -b\sum_{k=1}^\infty B_k \sin\frac{k\pi x}{l}
    + \sum_{k=1}^\infty A_k \sqrt{\Bigl(\frac{k\pi a}{l}\Bigr)^2-b^2}
      \sin\frac{k\pi x}{l} = 0,\]
  from which we solve that
  \[ \boxed{B_k = \frac{(-1)^{k+1}2h}{k\pi},\qquad
    A_k = \frac{(-1)^{k+1}2bh}{k\pi\sqrt{\bigl(\frac{k\pi a}{l}\bigr)^2-b^2}}.} \]
\end{solution}


\begin{exercise}[7]<补充题目>
  验证 $u(x,t) = \int_0^t W(x,t;\tau)\diff\tau$ 是初边值问题 (3.23)--(3.25) 的解.
\end{exercise}

\begin{proof}
  由于 $W(x,t;\tau)$ 满足条件
  \[\begin{cases}
    \frac{\partial^2W}{\partial t^2} - a^2 \frac{\partial^2W}{\partial x^2} = 0\quad (t>\tau), \\
    t = \tau: W=0,\,\frac{\partial W}{\partial t} = f(x,\tau), \\
    W(0,t;\tau) = W(l,t;\tau) = 0.
  \end{cases}\]
  故当 $t=0$ 时, $u(x,0) = 0$, $u_t(x,0) = W(x,0;0) = 0$.
  当 $x=0$ 时,
  \[u(0, t) = \int_0^t W(0,t;\tau)\diff\tau = 0.\]
  当 $x=l$ 时,
  \[u(l, t) = \int_0^t W(l,t;\tau)\diff\tau = 0.\]
  所以 $u(x,t)$ 满足初边值条件 (3.24), (3.25). 下面验证 $u(x,t)$ 满足方程 (3.23).
  由于 $u(x,t) = \int_0^t W(x,t;\tau)\diff\tau$, 故
  \[\frac{\partial u}{\partial t} = W(x,t;t) + \int_0^t \frac{\partial W}{\partial t}\diff\tau
    = \int_0^t \frac{\partial W}{\partial t}(x,t;\tau)\diff\tau.\]
  再求导可得
  \begin{align*}
    \frac{\partial^2u}{\partial t^2}
    & = \frac{\partial W}{\partial t}(x,t;t)
      + \int_0^t \frac{\partial^2W}{\partial t^2}(x,t;\tau)\diff\tau \\
    & = f(x,t) + \int_0^t \frac{\partial^2W}{\partial t^2}(x,t;\tau)\diff\tau.
  \end{align*}
  又因为
  \[\frac{\partial^2u}{\partial x^2} 
    = \int_0^t \frac{\partial^2W}{\partial x^2}(x,t;\tau)\diff\tau.\]
  于是
  \begin{align*}
    \frac{\partial^2u}{\partial t^2} - a^2 \frac{\partial^2u}{\partial x^2}
    & = f(x,t) + \int_0^t \biggl(\frac{\partial ^2W}{\partial t^2}(x,t;\tau)
      - a^2 \frac{\partial^2W}{\partial x^2}(x,t;\tau)\biggr)\diff\tau \\
    & = f(x,t).\qedhere
  \end{align*}
\end{proof}


\section{高维波动方程的柯西问题}

教材P26格林公式过程:
\begin{align*}
  \int_{\Gamma}T\frac{\partial u}{\partial\vec{n}}\diff s
  = \int_{\Gamma}(T\nabla u)\cdot\vec{n}\diff s
  & = \iint_{\Omega}\nabla\cdot(T\nabla u)\diff x\diff y \\
  & = \iint_{\Omega}T\Delta u\diff x\diff y \\
  & = \iint_{\Omega}T\left(\frac{\partial^2u}{\partial x^2}
    + \frac{\partial^2u}{\partial y^2}\right)\diff x\diff y.
\end{align*}

\begin{exercise}
  利用泊松公式求解波动方程的柯西问题:
  \begin{enumerate}[(1)]
    \item $\begin{cases}
            u_{tt} = a^2(u_{xx}+u_{yy}+u_{zz}), \\
            u|_{t=0}=0,\; u_t|_{t=0}=x^2+yz;
          \end{cases}$
    \item $\begin{cases}
            u_{tt} = a^2(u_{xx}+u_{yy}+u_{zz}), \\
            u|_{t=0}=x^3+y^2z,\; u_t|_{t=0}=0.
           \end{cases}$
  \end{enumerate}
\end{exercise}

\begin{solve}
  (1) 利用泊松公式知该柯西问题的解为:
  \begin{align*}
    u(x,y,x,t)
    ={} & \frac{\partial}{\partial t}\left(\frac{1}{4\pi a^2t}\iint_{S_{at}^M}0\diff S\right)
          + \frac{1}{4\pi a^2t}\iint_{S_{at}^M}(\xi^2+\eta\zeta)\diff S \\
    ={} & \frac{1}{4\pi a^2t}\int_0^{\pi}\int_0^{2\pi}(x+at\sin\theta\cos\varphi)^2 \\
        & + (y+at\sin\theta\sin\varphi)(z+at\cos\theta)
          a^2t^2\sin\theta\diff\varphi\diff\theta \\
    ={} & (x^2+	yz)t+\frac{a^2t^3}{3}. \qedhere
  \end{align*}
\end{solve}


\begin{exercise}
  试用降维法导出弦振动方程的达朗贝尔公式.
\end{exercise}

\begin{solve}
  考虑一维波动方程
  \[\begin{cases}
    u_{tt} = a^2 u_{xx}, \\
    u|_{t=0} = \varphi(x), \\
    u_t|_{t=0} = \psi(x).
  \end{cases}\]
  令 $\tilde{u}(x,y,z,t) = u(x,t)$, $\tilde{\varphi}(x,y,z) = \varphi(x)$,
  $\tilde{\psi}(x,y,z) = \psi(x)$, 则 $\tilde{u}$ 满足三维波动方程
  \[\begin{cases}
    \tilde{u}_{tt} = a^2 \bigl(\tilde{u}_{xx} + \tilde{u}_{yy} + \tilde{u}_{zz}\bigr), \\
    \tilde{u}|_{t=0} = \tilde{\varphi}, \\
    \tilde{u}_t|_{t=0} = \tilde{\psi}.
  \end{cases}\]
  由 Poisson 公式得
  \[
    \tilde{u}(x, y, z, t) = \frac{\partial}{\partial t}
      \biggl(\frac{1}{4\pi a^2t} \iint_{S_{at}^M} \tilde{\varphi}(\xi,\eta,\zeta)\diff S\biggr)
      + \frac{1}{4\pi a^2t} \iint_{S_{at}^M} \tilde{\psi}(\xi,\eta,\zeta)\diff S.
  \]
  于是
  \[
    u(x,t)
    = \frac{\partial}{\partial t}
      \biggl(\frac{1}{4\pi a^2t} \iint_{S_{at}^M} \varphi(\xi)\diff S\biggr)
      + \frac{1}{4\pi a^2t} \iint_{S_{at}^M} \psi(\xi)\diff S.
  \]
  采用球坐标 $\xi = x + r\cos\theta$, $\eta = y + r\sin\theta\cos\psi$,
  $\zeta = z + r\sin\theta\sin\phi$, 其中 $r=at$, $0\leq\theta\leq\pi$,
  $0\leq\phi < 2\pi$. 则
  \begin{align*}
    u(x,t)
    ={} & \frac{\partial}{\partial t}
          \biggl(\frac{1}{4\pi a} \int_0^{2\pi} \int_0^{\pi} \varphi(x+r\cos\theta)
          r\sin\theta \diff\theta\diff\phi\biggr) \\
        & + \frac{1}{4\pi a} \int_0^{2\pi} \int_0^{\pi} \psi(x+r\cos\theta)
          r\sin\theta \diff\theta\diff\phi  \\
    ={} & \frac{\partial}{\partial t}
          \biggl(-\frac{1}{2a} \int_0^{\pi} \varphi(x+r\cos\theta) 
          \diff (x+r\cos\theta)\biggr) \\
        &  - \frac{1}{2a} \int_0^{\pi} \psi(x+r\cos\theta) \diff (x+r\cos\theta) \\
    ={} & \frac{\partial}{\partial t}
          \biggl(-\frac{1}{2a} \int_{x+at}^{x-at} \varphi(\xi) \diff\xi\biggr)
          - \frac{1}{2a} \int_{x+at}^{x-at} \psi(\xi) \diff\xi \\
    ={} & \frac{1}{2}\bigl(\varphi(x+at) + \varphi(x-at)\bigr)
          + \frac{1}{2a}\int_{x-at}^{x+at} \psi(\xi)\diff\xi. \qedhere
  \end{align*}
\end{solve}


\begin{exercise}
  求解平面波动方程的柯西问题:
  \begin{enumerate}[(1)]
    \item $\begin{cases}
            u_{tt} = a^2(u_{xx}+u_{yy}), \\
            u|_{t=0}=x^2(x+y), \\
            u_t|_{t=0}=0.
          \end{cases}$
    \item $\begin{cases}
            u_{tt} - 3(u_{xx}+u_{yy}) = x^3+y^3, \\
            u|_{t=0}=0, \\
            u_t|_{t=0}=x^2.
          \end{cases}$
  \end{enumerate}
\end{exercise}

\begin{solve}
  (1) 根据二维波动方程柯西问题的泊松公式有
  \begin{align*}
    u(x,y,t)
    & = \frac{1}{2\pi a}\frac{\partial}{\partial t}\int_0^{at}\int_0^{2\pi}\frac{\varphi(x         
      + r\cos\theta,y+r\sin\theta)}{\sqrt{(at)^2-r^2}}r\diff\theta\diff r \\
    & = \frac{1}{2\pi a}\frac{\partial}{\partial t}\int_0^{at}
      \int_0^{2\pi}\frac{(x+r\cos\theta)^2
      (x+y+r\cos\theta+r\sin\theta)}{\sqrt{a^2t^2-r^2}}r\diff\theta\diff r \\
    & = \frac{1}{2\pi a}\frac{\partial}{\partial t}
      \int_0^{at}\frac{\pi\left[2x^2(x+y)+r^2(3x+y)\right]}{\sqrt{a^2t^2-r^2}}r\diff r \\
    & = \frac{1}{2\pi}\frac{\partial}{\partial t}\int_0^{\pi/2}
      \frac{2x^2(x+y)+a^2t^2\sin^2\beta(3x+y)}{at\cos\beta}
      at\sin\beta\cdot at\cos\beta \diff\beta \\
    & = \frac{1}{2a}\frac{\partial}{\partial t}\left(2x^2(x+y)at+\frac{2}{3}a^2t^2(3x+y)\right)\\
    & = x^2(x+y)+at^2(3x+y).\qedhere
  \end{align*}
\end{solve}


\begin{exercise}
  求二维波动方程的轴对称解 (即形如 $u=u(r,t)$ 的解, 其中 $r=\sqrt{x^2+y^2}$).
\end{exercise}

\begin{solve}
  由于 $\Delta u = u_{rr} + \frac{1}{r}u_r$, 故轴对称解满足方程
  \[u_{tt} = a^2\biggl(u_{rr}+\frac{1}{r}u_r\biggr).\]
  令 $u(r,t) = R(r)T(t)$, 代入上述方程得
  \[\frac{T''(t)}{a^2T(t)} = \frac{R''(r)+\frac{1}{r}R'(r)}{R(r)} = -\lambda^2
    \quad (\lambda>0).\]
  故
  \[T(t) = C_1\cos a\lambda t + C_2\sin a\lambda t,
    \quad R(r) = {\rm J}_0(\lambda r),\]
  其中 ${\rm J}_0$ 为 0 阶 Bessel 函数, 见附录 \uppercase\expandafter{\romannumeral3}.
\end{solve}


\begin{exercise}
  求解柯西问题:
  \[\begin{cases}
    u_{tt} = a^2(u_{xx}+u_{yy}) + c^2u, \\
    u|_{t=0} = \varphi(x,y), \\
    u_t|_{t=0} = \psi(x,y).
  \end{cases}\]
\end{exercise}

\begin{solve}
  令 $v(x,y,z,t) = \e^{\frac{cz}{a}} u(x,y,t)$, 则
  \[\begin{cases}
    v_{tt} = a^2(v_{xx} + v_{yy} + v_{zz}), \\
    v|_{t=0} = \e^{\frac{cz}{a}}\varphi(x,y), \\
    v_t|_{t=0} = \e^{\frac{cz}{a}}\psi(x,y).
  \end{cases}\]
  由三维波动方程柯西问题解的 Poisson 公式得
  \[v(x,y,z,t) = \frac{\partial}{\partial t}
    \biggl(\frac{1}{4\pi a^2t}\iint_{S_{at}^M} \e^{\frac{c\zeta}{a}}\varphi(\xi,\eta)\d S\biggr)
    + \frac{1}{4\pi a^2t}\iint_{S_{at}^M} \e^{\frac{c\zeta}{a}}\psi(\xi,\eta)\d S.\]
\end{solve}


\begin{exercise}[6]
  试用齐次化原理导出平面非齐次波动方程
  \[u_{tt} = a^2(u_{xx}+u_{yy})+f(x,y,t)\]
  在齐次初始条件
  \[\begin{cases}
    u|_{t=0} = 0, \\
    u_t|_{t=0} = 0
  \end{cases}\]
  下的求解公式.
\end{exercise}

\begin{solve}
  由齐次化原理$u(x,y,t)=\int_0^tw(x,y,t;\tau)\diff\tau$, 其中 $w(x,y,t;\tau)$为以下定解问题的解:
  \[\begin{cases}
    w_{tt} = a^2(w_{xx}+w_{yy})\quad (t>\tau), \\
    w|_{t=\tau} = 0,\quad w_t|_{t=\tau} = f(x,y,\tau).
  \end{cases}\]
  由二维波动方程柯西问题的泊松公式有:
  \[\begin{split}
    w(x,y,t;\tau)
    & = \frac{1}{2\pi a}\iint_{\Sigma_{at'}^M}
      \frac{f(\xi,\eta,\tau)}{\sqrt{(at')^2
      -(\xi-x)^2-(\eta-y)^2}}\diff\xi\diff\eta\quad(t'=t-\tau)\\
    & = \frac{1}{2\pi a}\iint_{\Sigma_{a(t-\tau)}^M}
      \frac{f(\xi,\eta,\tau)}{\sqrt{a^2(t-\tau)^2-(\xi-x)^2-(\eta-y)^2}}
      \diff\xi\diff\eta.
  \end{split}\]
  故
  \begin{align*}
    u(x,y,t)
    & = \int_0^tw(x,y,t;\tau)\diff\tau \\
    & = \frac{1}{2\pi a}\int_0^t\iint_{\Sigma_{a(t-\tau)}^M}
      \frac{f(\xi,\eta,\tau)}{\sqrt{a^2(t-\tau)^2-(\xi-x)^2-(\eta-y)^2}}
      \diff\xi\diff\eta\diff\tau.\qedhere
  \end{align*}
\end{solve}


\begin{exercise}
  用降维法求解上面的问题.
\end{exercise}

\begin{solve}
  令 $\tilde{u}(x,y,z,t) = u(x,y,t)$, $\tilde{f}(x,y,z,t)=f(x,y,t)$, 则
  \[\begin{cases}
    \tilde{u}_{tt} = a^2\bigl(\tilde{u}_{xx}+\tilde{u}_{yy}+\tilde{u}_{zz}\bigr)
      + \tilde{f}(x,y,z,t), \\
    \tilde{u}|_{t=0} = 0, \\
    \tilde{u}_t|_{t=0} = 0.
  \end{cases}\]
  上述问题的解已由教材 (4.47) 式给出, 即
  \[\tilde{u}(x,y,z,t)
    = \frac{1}{4\pi a^2}\iiint_{r\leq at} 
      \frac{\tilde{f}(\xi,\eta,\zeta,t-\frac{r}{a})}{r}\d V.\]
  故
  \begin{align*}
    u(x,y,t)
    & = \frac{1}{4\pi a^2}\iiint_{r\leq at} \frac{f(\xi,\eta,t-\frac{r}{a})}{r}\d V \\
    & = \frac{1}{4\pi a^2}\int_0^{at} 
      \iint_{S_r^M} \frac{f(\xi, \eta, t-\frac{r}{a})}{r}\d S_r \d r \\
    & = \frac{1}{2\pi a^2} \int_0^{at} \iint_{\Sigma_r^M} \frac{f(\xi,\eta,t-\frac{r}{a})}
      {\sqrt{r^2-(\xi-x)^2-(\eta-y)^2}}\d\xi\d\eta\d r.\qedhere
  \end{align*}
\end{solve}


\begin{exercise}[8]
  解非齐次方程的柯西问题:
  \[\begin{cases}
    u_{tt} = u_{xx}+u_{yy}+u_{zz}+2(y-t), \\
    u|_{t=0} = 0, \\
    u_t|_{t=0} = x^2+yz.
  \end{cases}\]
\end{exercise}

\begin{solve}
  利用叠加原理,考虑下面两个定解问题:
  \[(\star)
  \begin{cases}
    u_{tt}=u_{xx}+u_{yy}+u_{zz}, \\
    u|_{t=0}=0, \\
    u_t|_{t=0}=x^2+yz.
  \end{cases}
  (\star\star)
  \begin{cases}
    u_{tt} = u_{xx}+u_{yy}+u_{zz}+2(y-t), \\
    u|_{t=0}=0, \\
    u_t|_{t=0}=0.
  \end{cases}\]
  首先 $(\star)$ 的解为
  \[u_1(x,y,z,t) = \frac{1}{4\pi t}\iint_{S_t^M}(\xi^2+\eta\zeta)\diff S
    = (x^2+yz)t+\frac{1}{3}t^3.\]
  然后 $(\star\star)$ 的解为
  \[\begin{split}
    u_2(x,y,z,t)
    & = \frac{1}{4\pi}\iiint_{r\leq t}\frac{2(\eta-t+r)}{r}\diff V \\
    & = \frac{1}{4\pi}\int_0^t\iint_{S_r^M}\frac{2(\eta-t+r)}{r}\diff S\diff r \\
    & = \frac{1}{4\pi}\int_0^t\int_0^{\pi}\int_0^{2\pi}
      \frac{2(y+r\sin\theta\sin\varphi-t+r)}{r}r^2\sin\theta\diff\varphi\diff\theta\diff r \\
    & = -\frac{1}{3}t^3+yt^2.
  \end{split}\]
  所以原问题的解为
  \[u(x,y,z,t) = u_1(x,y,z,t)+u_2(x,y,z,t)=(x^2+yz)t+yt^2.\qedhere\]
\end{solve}


\section{波的传播与衰减}

\begin{exercise}
  试说明: 对一维波动方程所描述的波的传播过程一般具有后效现象.
\end{exercise}

\begin{solve}
  由教材 P12 分析知区间 $[x_1,x_2]$ 的影响区域为 $x_1-at\leq x\leq x_2+at$,
  故一旦扰动到达某一点,其对该点的影响将持续下去,因此一维波动方程所描述的波的传播过程具有后效现象.
\end{solve}


\begin{exercise}
  试说明: 对一维波动方程, 即使初始资料具有紧支集,
  当 $t\to +\infty$ 时其柯西问题的解没有衰减性.
\end{exercise}

\begin{solve}
  设初始资料 $\varphi$, $\psi$ 具有紧支集, 则存在一个常数 $\rho>0$,
  使得 $\varphi$, $\psi$ 在 $[-\rho,\rho]$ 外恒等于零,
  而在 $[-\rho,\rho]$ 内成立 $|\varphi|\leq C$, $|\psi|\leq C$, 那么对充分大的 $t$, 有
  \begin{align*}
    u(x,t)
    & = \frac{1}{2}(\varphi(x-at)
        +\varphi(x+at))+\frac{1}{2a}\int_{x-at}^{x+at}\psi(\xi)\diff\xi \\
    & = \frac{1}{2a}\int_{-\rho}^{\rho}\psi(\xi)\diff\xi= \text{const}.
  \end{align*}
  故一维波动方程柯西问题的解没有衰减性.
\end{solve}


\begin{exercise}
  设 $u$ 为初始资料 $\varphi$ 及 $\psi$ 具有紧支集的二维波动方程的解.
  试证明: 对任意固定的 $(x_0,y_0)\in \mathbb{R}^2$, 成立
  \[\lim_{t\to+\infty} u(x_0,y_0,t) = 0.\]
\end{exercise}

\begin{proof}
  因为初始资料 $\varphi$, $\psi$ 具有紧支集, 所以存在以原点为圆心,
  以某一个正常数 $\rho$ 为半径的圆 $\Sigma_{\rho}^O$ 
  使得 $\varphi,\psi$ 在 $\Sigma_{\rho}^O$ 外恒为零,
  在 $\Sigma_{\rho}^O$ 内满足 $|\varphi|\leq C,|\psi|\leq C$ ($C$是一个正常数),
  对于任意固定的 $M=(x_0,y_0)\in\mathbb{R}^2$
  和充分大的 $t$ (此时 $\Sigma_{\rho}^O\subset\Sigma_{at}^M$), 有
  \begin{align*}
    u(x_0,y_0,t) = {} 
    & \frac{1}{2\pi a}\biggl[\frac{\partial}{\partial t}
      \iint_{\Sigma_{at}^M}
      \frac{\varphi(\xi,\eta)\diff\xi\diff\eta}{\sqrt{a^2t^2-(\xi-x_0)^2-(\eta-y_0)^2}} \\
    & + \iint_{\Sigma_{at}^M}\frac{\psi(\xi,\eta)\diff\xi\diff\eta}
      {\sqrt{a^2t^2-(\xi-x_0)^2-(\eta-y_0)^2}}\biggr] \\
={} & \frac{1}{2\pi a}\bigg[\frac{\partial}{\partial t}\int_0^{at}\int_0^{2\pi}
      \frac{\varphi(x_0+r\cos\theta,y_0+r\sin\theta)}{\sqrt{a^2t^2-r^2}}r\diff\theta\diff r \\
    & +\int_0^{at}\int_0^{2\pi}\frac{\psi(x_0+r\cos\theta,y_0+r\sin\theta)}
      {\sqrt{a^2t^2-r^2}}r\diff\theta\diff r\bigg] \\
={} & \frac{1}{2\pi a}\bigg[\frac{\partial}{\partial t}
      \int_0^{\rho}\int_0^{2\pi}\frac{\varphi(x_0+r\cos\theta,y_0+r\sin\theta)}
      {\sqrt{a^2t^2-r^2}}r\diff\theta\diff r \\
    & + \int_0^{\rho}\int_0^{2\pi}\frac{\psi(x_0+r\cos\theta,y_0+r\sin\theta)}{\sqrt{a^2t^2-r^2}}r\diff\theta\diff r\bigg] \\
={} & \frac{1}{2\pi a}\bigg[\int_0^{\rho}\int_0^{2\pi}\frac{\partial}{\partial t}
      \frac{\varphi(x_0+r\cos\theta,y_0+r\sin\theta)}{\sqrt{a^2t^2-r^2}}r\diff\theta\diff r \\
    & + \int_0^{\rho}\int_0^{2\pi}\frac{\psi(x_0+r\cos\theta,y_0+r\sin\theta)}
      {\sqrt{a^2t^2-r^2}}r\diff\theta\diff r\bigg].
  \end{align*}
  故当 $t\to\infty$ 时,
  \[\begin{split}
  |u(x_0,y_0,t)|&\leq \frac{1}{2\pi a}\left[2\pi C\int_0^{\rho}-a^2t(a^2t^2-r^2)^{-\frac{3}{2}}r\diff r+2\pi C\int_0^{\rho}\frac{r}{\sqrt{a^2t^2-r^2}}\diff r\right]\\
  &=\frac{1}{2\pi a}\left[2\pi C\left(a-\frac{a^2t}{\sqrt{a^2t^2-\rho^2}}\right)+2\pi C(at-\sqrt{a^2t^2-\rho^2})\right]\to 0.
  \end{split}\]
  即
  \[\lim_{t\to+\infty}u(x_0,y_0,t) = 0.\qedhere\]
\end{proof}


\section{能量不等式,波动方程解的唯一性和稳定性}

\begin{note}[教材(6.15)式解释]
  第二个等式等价于
  \begin{align*}
        & \iint_{\Omega}(u_xu_{xt}+u_yu_{yt})\diff x\diff y
          +\iint_{\Omega}(u_{xx}u_t+u_{yy}u_t)\diff x\diff y \\
    ={} & \iint_{\Omega}\left[\frac{\partial}{\partial x}(u_xu_t)
          +\frac{\partial}{\partial y}(u_yu_t)\right]\diff x\diff y
          =\int_{\Gamma}u_t(\nabla u\cdot\vec{n})\diff s.
  \end{align*}
  而由格林公式得
  \[\begin{split}\int_{\Gamma}u_t(\nabla u)\cdot\vec{n}\diff s&=\iint_{\Omega}\nabla\cdot(u_t\nabla u)\diff x\diff y\\
  &=\iint_{\Omega}\left(\frac{\partial}{\partial x}\vec{i}+\frac{\partial}{\partial y}\vec{j}\right)\left(u_tu_x\vec{i}+u_tu_y\vec{j}\right)\diff x\diff y\\
  &=\iint_{\Omega}\left[\frac{\partial}{\partial x}(u_xu_t)+\frac{\partial}{\partial y}(u_yu_t)\right]\diff x\diff y\end{split}\]
\end{note}

\begin{note}[定理6.2]
  由能量不等式得
  \begin{align*}
    E(t)+E_0(t)
    \leq{} & C\left(E(0)+E_0(0)+\int_0^T
      \iint_{\Omega}(f_1-f_2)^2\diff x\diff y\diff t\right) \\
    ={} & C\biggl(\iint_{\Omega}\left[v_t^2+a^2\left(v_x^2+v_y^2\right)\right]\bigm|_{t=0}
          \diff x\diff y + \iint_{\Omega}v^2(x,y,0)\diff x\diff y \\
        & \quad + \int_0^T \iint_{\Omega}( f_1-f_2)^2\diff x\diff y\diff t\biggr) \\
    ={} & C\bigl(\|\psi_1-\psi_2\|^2+a^2\|\varphi_{1x}-\varphi_{2x}\|^2
          + a^2\|\varphi_{1y}-\varphi_{2y}\|^2 \\
        & + \|\varphi_1-\varphi_2\|^2+\|f_1-f_2\|^2\bigr).
  \end{align*}
  因此初边值问题的解关于初始值 $(\varphi,\psi)$ 和方程右端项 $f$ 在定理所述意义下是稳定的.
\end{note}


\begin{exercise}
  对受摩擦力作用且具固定端点的有界弦振动, 满足方程
  \[u_{tt} = a^2u_{xx} - cu_t,\]
  其中常数 $c>0$, 证明其能量是减少的, 并由此证明方程
  \[u_{tt} = a^2u_{xx} - cu_t + f\]
  的初边值问题解的唯一性以及关于初始条件及自由项的稳定性.
\end{exercise}

\begin{proof}
  能量 $E(t)=\int_0^l\left(u_t^2+a^2u_x^2\right)\diff x$, 关于 $t$ 求导得
  \begin{align*}
    \frac{\diff E(t)}{\diff t}
    & = 2\int_0^l\left(u_tu_{tt}+a^2u_xu_{xt}\right)\diff x \\
    & = 2\int_0^l\left[u_t\left(u_{tt}-a^2u_{xx}\right)
        + a^2\frac{\partial}{\partial x}(u_tu_x)\right]\diff x \\
    & = -2\int_0^lcu_t^2\diff x+2a^2u_tu_x|_0^l=-2\int_0^lcu_t^2\diff x\leq 0.
  \end{align*}
  因此其能量是减少的.
  
  为了证明方程 $u_{tt}=a^2u_{xx}-cu_t+f$ 的初边值问题的解的唯一性,
  只需要证明下面的齐次定解问题只有零解
  \[\begin{cases}
    u_{tt}=a^2u_{xx}-cu_t, \\
    u(0,t)=u(l,t)=0, \\
    u(x,0)=u_t(x,0)=0.
  \end{cases}\]
  由能量不等式得
  \[E(t)\leq E(0)=\int_0^l\left[u_t^2(x,0)+a^2u_x^2(x,0)\right]\diff x = 0.\]
  故$u_t=u_x=0\Rightarrow u(x,t)\equiv 0$.
  
  下面证明解对初始条件及自由项的稳定性:

  \textbullet 关于初始条件的稳定性. 记$E_0(t)=\int_0^lu^2\diff x$,则
  \[\frac{\diff E_0(t)}{\diff t}
    = 2\int_0^luu_t\diff x\leq\int_0^lu^2\diff x+\int_0^lu_t^2\diff x\leq E_0(t)+E(t),\]
  故
  \[E_0(t) \leq e^tE_0(0)+e^t\int_0^te^{-\tau}E(\tau)\diff\tau\leq e^tE_0(0)+E(0)(e^t-1).\]
  根据上式,当初值的均方模很小时,对固定的$T$, $0\leq t\leq T$时解的均方模也很小, 因此关于初始条件是稳定的.

  \textbullet 如有外力的作用, 此时定解问题为:
  \[\begin{cases}
    u_{tt}=a^2u_{xx}-cu_t+f, \\
    u(0,t)=u(l,t)=0, \\
    u(x,0)=u_t(x,0)=0.
  \end{cases}\]
  此时 $E(0)=E_0(0)=0$,且
  \[\begin{split}
    \frac{\diff E(t)}{\diff t}&=2\int_0^lu_t(-cu_t+f)\diff x=-2c\int_0^lu_t^2\diff x+2\int_0^lu_tf\diff x\\
    &\leq E(t)+\int_0^lf^2\diff x\\
    &\Rightarrow E(t)\leq C_0\left(E(0)+\int_0^T\int_0^lf^2\diff x\diff t\right)=C_0\int_0^T\int_0^lf^2\diff x\diff t\\
    &\Rightarrow E_0(t)\leq e^tE_0(0)+e^t\int_0^te^{-\tau}E(\tau)\diff\tau\leq A\int_0^T\int_0^lf^2\diff x\diff t
  \end{split}\]
  故关于自由项是稳定的.
\end{proof}


\begin{exercise}
  证明函数 $f(x,t)$ 在 $G\colon 0\leq x\leq l, 0\leq t\leq T$ 作微小改变时, 方程
  \[\frac{\partial^2 u}{\partial t^2}
    = \frac{\partial}{\partial x}\biggl(k(x)\frac{\partial u}{\partial x}\biggr)
      -q(x)u + f(x,t)\]
  (其中 $k(x)>0$, $q(x)>0$ 和 $f(x,t)$ 都是一些充分光滑的函数)
  具固定端点边界条件的初边值问题的解在 $G$ 内的改变也是很微小的.
\end{exercise}

\begin{proof}
  令
  \[E(t)=\int_0^l\left(u_t^2+k(x)u_x^2+q(x)u^2\right)\diff x,
    \quad E_0(t)=\int_0^lu^2\diff x.\]
  则
  \[\begin{split}
    E'(t)
    & = 2\int_0^l(u_tu_{tt}+ku_xu_{xt}+quu_t)\diff x \\
    & = 2\int_0^lu_t\bigl(u_{tt}-(ku_x)_x+qu\bigr)\diff x+2(ku_tu_x)|_0^l \\
    & = 2\int_0^lu_tf\diff x\leq E(t)+\int_0^lf^2\diff x.
  \end{split}\]
  从 $0$ 到 $t$ 积分得
  \[E(t)\leq e^tE(0)+e^t\int_0^te^{-\tau}\int_0^lf^2(x,\tau)\diff x\diff\tau.\]
  又
  \[E'_0(t) = 2\int_0^l uu_t \diff x\leq E_0(t) + E(t).\]
  故
  \[E_0(t)\leq e^t E_0(0) + e^t\int_0^t e^{-\tau} E(\tau) \diff\tau.\]
  设 $u(x,t)$ 为满足齐次初边值条件的解, 显然有 $E(0)=E_0(0)=0$.
  \[E_0(0) \leq C\int_0^T \int_0^l f^2 \diff x\diff t.\qedhere\]
\end{proof}


\begin{exercise}
  证明波动方程
  \[u_{tt} = a^2(u_{xx}+u_{yy}) + f(x,y,t)\]
  的自由项 $f$ 在 $L^2(K)$ 意义下作微小改变时, 对应的柯西问题的解 $u$ 在
  $L^2(K)$ 意义之下改变也是微小的, 其中 $K$ 是由 (6.30) 式所表示的锥体.
\end{exercise}

\begin{proof}
  作特征锥 $(x-x_0)^2+(y-y_0)^2\leq (R-at)^2$,
  记 $\Omega_t$ 为 $t=\text{const}$ 与锥的交截部分, 令
  \[E_1(\Omega_t)=\iint_{\Omega_t}\left(u_t^2+a^2(u_x^2+u_y^2)\right)\diff x\diff y,
    \quad E_0(\Omega_t)=\iint_{\Omega_t}u^2\diff x\diff y.\]
  关于 $E_1(\Omega_t)$ 求导并分布积分得
  \begin{align*}
    \frac{\diff E_1(\Omega_t)}{\diff t}
    ={} & 2\int_0^{R-at}\int_0^{2\pi r} u_t\left(u_{tt}-a^2(u_{xx}+u_{yy})\right)
          \diff s\diff r \\
        & + 2\int_{\Gamma_t} \left(a^2[u_xu_t\cos(n,x)+u_yu_t\cos(n,y)]
          - \frac{a}{2}[u_t^2+a^2(u_x^2+u_y^2)]\right)\diff s \\
    \leq{} & 2\iint_{\Omega_t} u_t f(x,y,t)\diff x\diff y
          \leq E_1(\Omega_t) + \iint_{\Omega_t} f^2 \diff x\diff y.
  \end{align*}
  记 $F(t)=\iint_{\Omega_t} f^2(x,y,t) \diff x\diff y$, 则
  \[E_1(\Omega_t) \leq e^t E_1(\Omega_0) + e^t \int_0^te^{-\tau}F(\tau)\diff\tau.\]
  \[\begin{split}
    \frac{\diff E_0(\Omega_t)}{\diff t}
    & = -a\int_{\Gamma_t}u^2\diff s+2\iint_{\Omega}uu_t\diff x\diff y\leq E_0(\Omega_t)
      + E_1(\Omega_t) \\
    & \Rightarrow E_0(\Omega_t)\leq e^tE_0(\Omega_0)
      + e^t\int_0^te^{-\tau}E_1(\Omega_{\tau})\diff\tau.
  \end{split}\]
  考虑柯西问题
  \[\begin{cases}
    u_{tt} = a^2(u_{xx}+u_{yy})+f(x,y,t) \\
    u|_{t=0} = u_t|_{t=0} = 0.
  \end{cases}\]
  此时 $E_0(\Omega_0) = E_1(\Omega_0) = 0$, 故
  \[E_0(\Omega_t)\leq e^t\int_0^t \int_0^{\tau} e^{-\xi}F(\xi) \diff\xi\diff\tau.\]
  对上式从 $0$ 到 $T=R/a$ 积分得
  \[\|u\|_{L^2(K)}^2\leq C\|f\|_{L^2(K)}^2.\qedhere\]
\end{proof}


\begin{exercise}
  固定端点有界弦的自由振动可以分解成各种不同固有频率的驻波(谐波)的叠加,
  试计算各个驻波的动能和位能, 并证明弦振动的总能量等于各个驻波能量的叠加.
  这个物理性质对应的数学事实是什么?
\end{exercise}

\begin{proof}
  由教材P21知此问题的解为
  \[u(x,t) = \sum_{k=1}^{\infty}u_k(x,t)
    = \sum_{k=1}^{\infty}N_k\cos(\omega_k t+\theta_k)\sin\frac{k\pi}{l}x.\]
  其中
  \[N_k = \sqrt{A_k^2+B_k^2}, \quad \omega_k=\frac{k\pi a}{l},
    \quad \cos\theta_k = \frac{A_k}{\sqrt{A_k^2+B_k^2}},
    \quad \sin\theta_k = \frac{B_k}{\sqrt{A_k^2+B_k^2}},\]
  \[A_k = \frac{2}{l}\int_0^l\varphi(\xi)\sin\frac{k\pi}{l}\xi\diff\xi,
  \quad B_k = \frac{2}{k\pi a}\int_0^l\psi(\xi)\sin\frac{k\pi}{l}\xi\diff\xi.\]
  第 $k$ 个驻波的能量为
  \[\begin{split}
    E_k
    & = \int_0^l\left((u_k)_t^2+a^2(u_k)_x^2\right)\diff x \\
    & = \omega_k^2N_k^2\int_0^l\left[\sin^2(\omega_kt+\theta_k)\sin^2\frac{k\pi}{l}x
        + \cos^2(\omega_kt+\theta_k)\cos^2\frac{k\pi}{l}x\right]\diff x \\
    & = \frac{1}{2}\omega_k^2N_k^2.
  \end{split}\]
  故
  \[\sum_{k=1}^{\infty} E_k = \frac{(\pi a)^2}{2l}\sum_{k=1}^{\infty}k^2N_k^2.\]
  另一方面,我们有
  \[u_t = -\sum_{k=1}^{\infty}\omega_kN_k\sin(\omega_kt+\theta_k)\sin\frac{k\pi}{l}x,\]
  \[u_x = \sum_{k=1}^{\infty}N_k\frac{k\pi}{l}\cos(\omega_kt+\theta_k)\cos\frac{k\pi}{l}x,\]
  故
  \begin{align*}
    E ={} & \int_0^l \left(u_t^2+a^2u_x^2\right)\diff x \\
      ={} & \sum_{k,j=1}^{\infty} \omega_k\omega_jN_kN_j\sin(\omega_kt
            +\theta_k)\sin(\omega_jt+\theta_j)
            \int_0^l \sin\frac{k\pi}{l}x\sin\frac{j\pi}{l}x\diff x \\
          & + \sum_{k,j=1}^{\infty} a^2\frac{kj\pi^2}{l^2}N_kN_j\cos(\omega_kt+\theta_k)
            \cos(\omega_jt+\theta_j) \int_0^l \cos\frac{k\pi}{l}x
            \cos\frac{j\pi}{l}x\diff x \\
      ={} & \frac{(\pi a)^2}{2l}\sum_{k=1}^{\infty}k^2N_k^2.
  \end{align*}
  因此
  \[E = \sum_{k=1}^{\infty}E_k.\]
  此事实反映了特征函数系的完备性, 即成立 Parseval 等式.
\end{proof}


\begin{exercise}
  考虑波动方程的第三类初边值问题
  \begin{align*}
    & u_{tt} - a^2(u_{xx}+u_{yy}) = 0,\quad t>0, (x,y)\in\varOmega, \\
    & u|_{t=0} = \varphi(x,y),\ u_t|_{t=0} = \psi(x,y), \\
    & \biggl(\frac{\partial u}{\partial \bm{n}}+\sigma u\biggr)\biggm|_{\varGamma} = 0,
  \end{align*}
  其中 $\sigma>0$ 是常数, $\varGamma$ 为 $\varOmega$ 的边界,
  $\bm{n}$ 是 $\varGamma$ 上的单位外法向量. 对于上述定解问题的解, 定义能量积分
  \[E(t) = \iint_{\varOmega} \bigl(u_t^2+a^2(u_x^2+u_y^2)\bigr)\diff x\diff y
    + a^2\int_{\varGamma} \sigma u^2\diff s,\]
  试证明 $E(t)$ 为常数, 并由此说明上述定解问题解的唯一性.
\end{exercise}

\begin{proof}
  直接求导得
  \begin{align*}
    \frac{\partial E(t)}{\partial t}
    ={} & 2\iint_{\varOmega} \bigl(u_t u_{tt} + a^2(u_x u_{xt} + u_y u_{yt})\bigr)
          \diff x\diff y + 2a^2\int_{\varGamma} \sigma uu_t \diff s \\
    ={} & 2\iint_{\varOmega} u_t(u_{tt}-a^2(u_{xx}+u_{yy}))
          + a^2(u_x u_t)_x + a^2(u_y u_t)_y \diff x\diff y \\
        & + 2a^2\int_{\varGamma} \sigma uu_t \diff s \\
    ={} & 2a^2\iint_{\varOmega} \div (u_t\nabla u) \diff x\diff y
          + 2a^2\int_{\varGamma} \sigma uu_t \diff s \\
    ={} & 2a^2\int_{\varGamma} u_t \biggl(\frac{\partial u}{\partial \bm{n}}+\sigma u\biggr)
          \diff s = 0,
  \end{align*}
  故 $E(t)$ 为常数. 由此立即可得解的唯一性.
\end{proof}
\chapter{热传导方程}

\section{热传导方程及其定解问题的导出}
\begin{exercise}[3]
	混凝土内部储藏着热量,称为水化热,在混凝土浇筑后该热量逐渐放出,放热速度和它所储藏的水化热成正比.以$Q(t)$表示它在单位体积中所储的热量, $Q_0$为初始时刻所储藏的热量,则$\displaystyle\frac{\d Q}{\d t}=-\beta Q$,其中$\beta$为正常数.又假设混凝土的比热为$c$,密度为$\rho$,热传导系数为$k$,求它在浇筑后温度$u$所满足的方程.
\end{exercise}

\begin{solve}
	将水化热视为一热源,由$\displaystyle\frac {\d Q}{dt}=-\beta Q$及$Q|_{t=0}=Q_0$,得$Q(t)=Q_0\e^{-\beta t}.$
	
	由假设,放热速度为$Q_0\beta \e^{\beta t}$,它就是单位时间所产生的热量,因此,由书65页(1.7)式得
	$$\frac{\partial u}{\partial t}=a^2\left(\frac{\partial^2u}{\partial x^2}+\frac{\partial^2u}{\partial y^2}+\frac{\partial^2u}{\partial z^2}\right)+\frac{\beta Q_0}{c\rho}\e^{-\beta t},$$
	其中,$\displaystyle a^2=\frac k{c\rho}.$
\end{solve}


\begin{exercise}[4]
	设一均匀的导线处在周围为常数温度$u_{0}$的介质中,试证:在常电流作用下导线的温度满足微分方程
	$$\frac{\partial u}{\partial t}=\frac{k}{c\rho}\frac{\partial^2u}{\partial x^2}-\frac{k_1P}{c\rho\omega}(u-u_0)+\frac{0.24i^2r}{c\rho\omega},$$
	其中$i$和$r$分别表示导体的电流强度及电阻系数, $P$表示横截面的周长, $\omega$	表示横截面的面积,而$k_{1}$表示导线对于介质的热交换系数.
\end{exercise}

\begin{proof}
	问题可视为有热源的杆的热传导问题,因此由65页(1.7)和(1.8)知方程取形式为
	$$\frac{\partial u}{\partial t}=a^2\frac{\partial^2u}{\partial x^2}+f(x,t),$$
	其中$a^2= \frac k{c\rho}$, $f(x,t)=\frac{F(x,t)}{c\rho}$, $F(x,t)$为单位体积单位时间产生的热量.
	
	由常电流$i$所产生$F_1(x,t)$为$0.24i^2r\omega^2$.因为单位长度的电阻为$\displaystyle\frac r{\omega}$,因此电流$i$做功为$\displaystyle i^2\frac r\omega$乘上功热当量单位长度产生的热量为$0.24i^2r/\omega$,其中0.24为功热当量.
	
	因此单位体积时间所产生的热量为$0.24i^2r/w^2$.由常温度的热交换所产生的,从本节习题2看出为$\displaystyle-\frac{4k_1}l(u-u_0)$,其中$l$为系杆直径,固有$\displaystyle\frac p\omega=\frac{\pi l}{\pi l^2/4}=\frac 4l$,代入得
	\[F_2(x,t)=\frac{-k_1p}{\omega}(u-u_0).\]
	因热源可叠加,故有$F(x,t)=F_1(x,t)+F_2(x,t)$,将所得代入$\displaystyle\frac{\partial u}{\partial t}=a^2\frac{\partial^2u}{\partial x^2}+f(x,t)$即得所求
	\[\frac{\partial u}{\partial t}=\frac{k}{c\rho}\frac{\partial^2u}{\partial x^2}-\frac{k_1p}{c\rho w}(u-u_0)+\frac{0.24i^2r}{c\rho \omega^2}.\qedhere\]
\end{proof}


\begin{exercise}[5]
	设物体表面的绝对温度为$u$,此时它向外界辐射出去的热量依斯特藩-玻尔兹曼(Stefan-Boltzmann)定律正比于$u^4$,即
	$$dQ=\sigma u^4\d S\d t.$$
	假设物体和周围介质之间只有热辐射而没有热传导,又假设物体周围介质的绝对温度为已知函数$f(x,y,z,t)$,试写出此时该物体热传导问题的边界条件.
\end{exercise}

\begin{solve}
	由假设,边界只有热辐射的热量交换,辐射出去的热量为
	$$dQ_1=\sigma u^4\big|_{\partial\Omega}\d S\d t,$$
	辐射进来的热量为
	$$dQ_2=\sigma f^4|_{\partial\Omega}\d S\d t.$$
	因此由热量的传导定律得边界条件为
	\[k\frac{\partial u}{\partial n}\bigg|_{\partial\Omega}=\sigma\left(u^4|_{\partial\Omega}-f^4|_{\partial\Omega}\right).\qedhere\]
\end{solve}
\section{初边值问题的分离变量法}


\begin{exercise}
  用分离变量法求下列定解问题的解:
  \[\begin{cases}
    \frac{\partial u}{\partial t} = a^2 \frac{\partial^2u}{\partial x^2}
      \quad (t>0, 0<x<\pi), \\
    u(0,t) = \frac{\partial u}{\partial x}(\pi,t) = 0\quad (t>0), \\
    u(x,0) = f(x)\quad (0<x<\pi).
  \end{cases}\]
\end{exercise}

\begin{solution}
  利用分离变量法, 设 $u(x,t)=X(x)T(t)$, 则
  \[X(x)T'(t) = a^2X''(x)T(t).\]
  由此得
  \begin{align}
    X''(x)+\lambda X(x) & = 0, \label{eq:c1} \\
    T'(t)+a^2\lambda T(t) & = 0. \label{eq:c2} 
  \end{align}
  由 \eqref{eq:c1} 及 $X(x)$ 满足的边界条件 $X(0)=0$, $X'(\pi)=0$, 得
  \begin{enumerate}[(i)]
    \item 当 $\lambda\leq 0$ 时只有零解;
    \item 当 $\lambda>0$ 时, $X(x)=A\cos\sqrt{\lambda}x+B\sin\sqrt{\lambda}x$,
      代入边界条件得固有值为 $$\lambda_k=\left(\frac{2k+1}{2}\right)^2,$$
      相应的固有函数 $$X_k(x)=B_k\sin\sqrt{\lambda_k}x=B_k\sin\frac{2k+1}{2}x.$$
  \end{enumerate}
  将 $\lambda=\lambda_k$ 代入 \eqref{eq:c2} 得 $T_k(t)=C_k\e^{-a^2\lambda_kt}$, 故
  \[u_k(x,t) = A_k\e^{-a^2\lambda_kt}\sin\frac{2k+1}{2}x,\quad u(x,t)
    = \sum_{k=0}^{\infty}u_k(x,t).\]
  利用初始条件得
  \[f(x) = \sum_{k=0}^{\infty}A_k\sin\frac{2k+1}{2}x\Rightarrow A_k
    = \frac{2}{\pi}\int_0^{\pi}f(\xi)\sin\frac{2k+1}{2}\xi\diff\xi.\]
  因此原问题的解为
  \[u(x,t) = \sum_{k=0}^{\infty}\frac{2}{\pi}\int_0^{\pi} f(\xi)\sin\frac{2k+1}{2}
    \xi\diff\xi\cdot \e^{-a^2(\frac{2k+1}2)^2t}\sin\frac{2k+1}{2}x. \qedhere\]
\end{solution}


\begin{exercise}
  用分离变量法求解热传导方程的初边值问题:
  \[\begin{cases}
    \displaystyle \frac{\partial u}{\partial t} = \frac{\partial^2 u}{\partial x^2} \quad (t>0,\, 0<x<1), \\
    u(x,0) = \begin{cases}
               x,     & 0 < x \leq\frac12, \\
               1 - x, & \frac12 < x < 1,
             \end{cases} \\
    u(0,t) = u(1,t) = 0 \quad (t>0).
  \end{cases}\]
\end{exercise}

\begin{solution}
  令 $u(x,t) = X(x)T(t)$, 得
  \[ \frac{X''(x)}{X(x)} = \frac{T'(t)}{T(t)} = -\lambda. \]
  首先, 函数$X(x)$满足以下条件
  \begin{equation}\label{eq:c3}
    \begin{cases}
      X''(x) + \lambda X(x) = 0, \\
      X(0) = X(1) = 0.
    \end{cases}
  \end{equation}
  解得固有值和固有函数为 $\lambda_k = (k\pi)^2$ and $X_k(x) = A_k\sin k\pi x$.

  于是 $T(t)$ 满足方程
  \[ T_k(t) + (k\pi)^2T_k(t) = 0, \]
  解得 $T_k(t) = B_k \e^{-(k\pi)^2t}$. 所以 $u(x,t)$ 的解为
  \begin{equation}\label{eq:c4}
    u(x,t) = \sum_{k=1}^\infty C_k \e^{-(k\pi)^2t} \sin k\pi x.
  \end{equation}
  利用初始条件得
  \[ \sum_{k=1}^\infty C_k \sin k\pi x = f(x) = \begin{cases}
    x,     & 0 < x \leq\frac12, \\
    1 - x, & \frac12 < x < 1.
  \end{cases} \]
  因此
  \begin{equation}\label{eq:c5}
    C_k = 2 \int_0^1 f(x) \sin k\pi x \d x = \frac{4}{k^2\pi^2} \sin\frac{k\pi}{2}.
  \end{equation}
  代入 \eqref{eq:c4} 可得
  \[ u(x,t) = \sum_{k=1}^\infty \frac{4}{k^2\pi^2} \sin\frac{k\pi}{2}
  \e^{-k^2\pi^2t} \sin k\pi x. \qedhere \]
\end{solution}


\begin{exercise}
  如果有一根长度为 $l$ 的均匀细棒, 其周围以及两端 $x=0$, $x=l$
  均为绝热, 初始温度分布为 $u(x,0)=f(x)$, 问以后的温度分布如何?
  且证明当 $f(x)$ 等于常数 $u_0$ 时, 恒有 $u(x,t)=u_0$.
\end{exercise}

\begin{solution}
  因为细棒的两端均为绝热,故根据傅里叶定律知$u_x|_{x=0}=u_x|_{x=l}=0$,此初边值问题为
  \[\begin{cases}
    u_t = a^2u_{xx}, \\
    u_x|_{x=0} = u_x|_{x=l} = 0, \\
    u|_{t=0} = f(x).
  \end{cases}\]
  直接解得
  \[ u(x,t)=\sum_{k=0}^{\infty}D_k\e^{-a^2\lambda_kt}\cos\frac{k\pi}{l}x,
    \quad \lambda_k=\left(\frac{k\pi}{l}\right)^2, \]
  其中
  \[ D_0 = \frac{1}{l}\int_0^lf(\xi) \d\xi,
    \quad D_k=\frac{2}{l}\int_0^lf(\xi)\cos\frac{k\pi}{l}\xi \d\xi
    \quad (k\geq1).\]
  当 $f(x)\equiv u_0$ 时, $D_0=u_0$, $D_k=0$ $(k=1,2,\cdots)$, 故 $ u(x,t)=u_0$.
\end{solution}


\begin{exercise}
  在区域 $t>0$, $0<x<l$ 中求解如下的定解问题:
  \[\begin{cases}
    \displaystyle\frac{\partial u}{\partial t} = a^2 \frac{\partial^2u}{\partial x^2} - \beta(u-u_0), \\
    u(0,t) = u(l,t) = u_0, \\
    u(x,0) = f(x),
  \end{cases}\]
  其中 $a$, $\beta$, $u_0$ 均为常数, $f(x)$ 为已知函数.
\end{exercise}

\begin{solution}
  作变量代换,令 $v(x,t)=(u-u_0)\e^{\beta t}$,则$v(x,t)$满足的定解问题为:
  \[\begin{cases}
    v_t=a^2v_{xx}, \\
    v(0,t)=v(l,t)=0, \\
    v(x,0)=f(x)-u_0.
  \end{cases}\]
  直接解得
  \[ v(x,t) = \sum_{k=1}^{\infty}A_k\e^{-a^2\lambda_kt}\sin\frac{k\pi}{l}x,
    \quad \lambda_k = \left(\frac{k\pi}{l}\right)^2, \]
  其中
  \[ A_k = \frac{2}{l}\int_0^l(f(\xi)-u_0)\sin\frac{k\pi}{l}\xi\diff\xi. \]
  因此
  \[u(x,t) = u_0 + \sum_{k=1}^{\infty}\frac{2}{l}
    \int_0^l(f(\xi)-u_0)\sin\frac{k\pi}{l}\xi\diff\xi
      \cdot \e^{-\left(\frac{a^2k^2\pi^2}{l^2}+\beta\right)t}\sin\frac{k\pi}{l}x. \qedhere \]
\end{solution}


\begin{exercise}
  长度为 $l$ 的均匀细杆的初始温度为 \qty{0}{\degreeCelsius}, 端点 $x=0$ 保持恒温 $u_0$,
  而在 $x=l$ 和侧面上, 热量可以发散到周围的介质中去, 介质的温度为 \qty{0}{\degreeCelsius},
  此时杆上的温度分布函数 $u(x,t)$ 满足下述定解问题:
  \[\begin{cases}
    \frac{\partial u}{\partial t} = a^2 \frac{\partial^2 u}{\partial x^2} - b^2 u, \\
    u(0,t) = u_0, \quad \Bigl(\frac{\partial u}{\partial x} + Hu\Bigr)\Big|_{x=l} = 0, \\
    u(x,0) = 0,
  \end{cases}\]
  其中 $a$, $b$, $H$ 均为常数, 试求出 $u(x,t)$.
\end{exercise}

\begin{solution}
    Let $u(x,t) = u_1(x,t) + u_2(x)$ in which $u_1$ satisfies the following 
    problem with homogeneous boundary condition
    \begin{equation}\label{eq:c6}
      \begin{cases}
        \frac{\partial u_1}{\partial t} = a^2 \frac{\partial^2 u_1}{\partial x^2} - b^2u_1, \\
        u_1(0,t) = \bigl(\frac{\partial u_1}{\partial x} + Hu_1\bigr)\big|_{x=l} = 0, \\
        u_1(x,0) = -u_2,
      \end{cases}
    \end{equation}
    and $u_2$ satisfies the following ordinary differential equation
    \begin{equation}\label{eq:c7}
      \begin{cases}
        a^2 \frac{\d^2 u_2}{\d x^2} - b^2 u_2 = 0, \\
        u_2(0) = u_0, \\
        \bigl(\frac{\d u_2}{\d x} + Hu_2\bigr)\big|_{x=l} = 0.
      \end{cases}
    \end{equation}
    We first solve equation \eqref{eq:c7}. Since $a^2 \frac{\d^2 u_2}{\d x^2} - b^2 u_2 = 0$,
    we have that
    \begin{equation}\label{eq:c8}
      u_2 = C_1 \e^{\frac{b}{a}x} + C_2 \e^{-\frac{b}{a}x}.
    \end{equation}
    By the boundary condition we get
    \[\begin{cases}
      C_1 + C_2 = 0, \\
      C_1\frac{b}{a}\e^{\frac{b}{a}l} - C_2 \frac{b}{a} \e^{-\frac{b}{a}l}
        + H C_1 \e^{\frac{b}{a}l} + H C_2 e^{-\frac{b}{a}l} = 0.
    \end{cases}\]
    Solving $C_1$, $C_2$ and substituting them into \eqref{eq:c8}, we have
    \begin{equation}\label{eq:c9}
      u_2 = u_0 \cosh \frac{b}{a}x - \frac{H\cosh\frac{b}{a}l + \frac{b}{a}\sinh\frac{b}{a}l}
        {H\sinh\frac{b}{a}l + \frac{b}{a}\cosh\frac{b}{a}l}
        u_0 \sinh\frac{b}{a}x.
    \end{equation}

    Now we solve equation \eqref{eq:c6}. Let $v = \e^{b^2 t}u_1$, then $v$ satisfies
    \begin{equation}\label{eq:c10}
      \begin{cases}
        \frac{\partial v}{\partial t} = a^2 \frac{\partial^2 v}{\partial x^2}, \\
        v(0,t) = \bigl(\frac{\partial v}{\partial x} + Hv\bigr)\big|_{x=l} = 0, \\
        v(x,0) = -u_2.
      \end{cases}      
    \end{equation}
    The procedure of solving this problem is actually the same as that in the textbook
    from Page 51 to 54.
    Denote $v(x,t) = X(x)T(t)$, then
    \[ \frac{X''(x)}{X(x)} = \frac{T'(t)}{a^2 T(t)} = -\lambda \]
    for some constant $\lambda$. First of all, $X(x)$ satisfies
    \begin{equation}\label{eq:c11}
      \begin{cases}
        X''(x) + \lambda X(x) = 0, \\
        X(0) = X'(l) + HX(l) = 0.
      \end{cases}
    \end{equation}
    \begin{itemize}
      \item If $\lambda\leq 0$, there only exists trivial solution $X\equiv 0$;
      \item If $\lambda>0$,
        \begin{equation}\label{eq:c12}
          X(x) = A\cos\sqrt{\lambda}x + B\sin\sqrt{\lambda}x.
        \end{equation}
        Combining with the boundary conditions we have
        \begin{equation}\label{eq:c13}
          X_k(x) = B_k \sin\sqrt{\lambda_k} x,
        \end{equation}
        where $(\lambda_k)_{k\geq 1}$ is the sequence of positive solutions to
        $\sqrt\lambda + H\tan\sqrt\lambda l = 0$.
    \end{itemize}

    On the other hand, $T(t)$ satisfies
    \begin{equation}\label{eq:c14}
      T'(t) + a^2\lambda_k T(t) = 0,
    \end{equation}
    from which we get
    \begin{equation}\label{eq:c15}
      T_k(t) = C_k \e^{-a^2\lambda_k t}.
    \end{equation}
    Therefore according to \eqref{eq:c13} and \eqref{eq:c15} we can write $v(x,t)$ as
    \begin{equation}\label{eq:c16}
      v(x,t) = \sum_{k=1}^\infty A_k e^{-a^2\lambda_k t} \sin\sqrt{\lambda_k} x.
    \end{equation}
    Finally we need to utilize the initial value condition to get
    \begin{equation}\label{eq:c17}
      v(x,0) = \sum_{k=1}^\infty A_k \sin\sqrt{\lambda_k} x = -u_0.
    \end{equation}
    Since
    \begin{equation}\label{eq:c18}
      \bigl(\sin\sqrt{\lambda_m}x, \sin\sqrt{\lambda_n}x\bigr)_{L^2}
       = \delta_{mn} \biggl(\frac{l}{2} + \frac{H}{2(H^2+\lambda_m)}\biggr)
       =: \delta_{mn} \alpha_m,
    \end{equation}
    we have
    \begin{equation}\label{eq:c19}
      A_k = -\frac{1}{\alpha_k} \int_0^l u_0(x) \sin\sqrt{\lambda_k} x \d x.
    \end{equation}
    Hence
    \begin{equation}\label{eq:c20}
      v(x,t) = -\sum_{k=1}^\infty \frac{1}{\alpha_k}
        \int_0^l u_0(x) \sin\sqrt{\lambda_k}x \d x \cdot
        \e^{-a^2\lambda_k t} \sin\sqrt{\lambda_k} x
    \end{equation}
    and
    \begin{equation}\label{eq:c21}
      u_1(x,t) = -\sum_{k=1}^\infty \frac{1}{\alpha_k}
        \int_0^l u_0(x) \sin\sqrt{\lambda_k}x \d x \cdot
        \e^{-(a^2\lambda_k + b^2)t} \sin\sqrt{\lambda_k} x
    \end{equation}
    Finally we conclude that
    \begin{equation}\label{eq:c22}
      u(x,t) = u_1(x,t) + u_2(x),
    \end{equation}
    where $u_1$ and $u_2$ are given by \eqref{eq:c21} and \eqref{eq:c9}
    respectively.
\end{solution}


\begin{exercise}
	半径为$a$的半圆形平板,其表面绝热,在板的圆周边界上保持常温$u_0$,而在直径边界上保持常温$u_1$,求圆板稳恒状态(即与时间$t$无关的状态)的温度分布.
\end{exercise}

\begin{solve}
	引入极坐标,求稳恒状态的温度分布化为解定解问题
	$$\begin{aligned}
		&u_{rr}+\frac{1}{r}u_{r}+\frac{1}{r^{2}}u_{\theta\theta}=0,\quad0<r<a,\quad0\leq\theta\leq\pi,\\
		&=u_{1},\quad u|_{\theta=\pi}=u_{1},\\
		&u|_{r=a}=u_{0},\quad u|_{r=0}=C;
	\end{aligned}$$
	引入变量代换$u=u_1+v(r,\theta)$,则$v(r,\theta)$满足
	$$\begin{aligned}
		&v_{rr}+\frac{1}{r}v_{r}+\frac{1}{r^{2}}v_{\theta\theta}=0,\\
		&v|_{\theta=0}=0,\quad v|_{\theta=\pi}=0,\\
		&v|_{r=a}=u_{0}-u_{1},\quad v|_{r=0}=C.
	\end{aligned}$$
	设$v=R(r)\Phi(\theta)$代入方程及边值得
	$$R''+\frac{1}{r}R'\Phi+\frac{1}{r^{2}}R\Phi''=0,$$
	移项后得
	$$\left\{\begin{array}{l}
		\varPhi''+\lambda\varPhi=0,\quad\varPhi(0)=\varPhi(\pi)=0,\\
		r^2R''+rR'-\lambda R=0.
	\end{array}\right.$$
	直接计算得固有值与固有函数分别为
	$$\lambda_n=n^2,\quad\varPhi(\theta)=\sin n\theta,\quad n=1,2...,$$
	对应的$R$满足方程
	$$r^2R''+rR'-n^2R=0.$$
	设$R=r^a$代入得$a= \pm n$, 即$R= r^n$,或$R=r^{-n}$为两个线性无关的特解
	因此通解为
	$$R_n(r)=C_nr^n+D_nr^{-n},$$
	由自然边界条件$v|_r=0$有界知$R(r)$在$r=0$处有界. 因此, $D_n=0$,那么,
	$$v(r,\theta)=\sum_{n=0}^\infty C_nr^n\sin n\theta.$$
	再由边界条件得
	$$C_n=\frac{2}{\pi a^n}\int_0^\pi(u_0-u_1)\sin n\theta d\theta=\frac{2u_0-2u_1}{n\pi a^n}\left(1-(-1)^n\right),$$
	所以
	\[u(r,\theta)=u_1+\sum_{n=0}^\infty\frac{2r^n(u_0-u_1)}{n\pi a^n}\left[1-(-1)^n\right]\sin n\theta.\qedhere\]
\end{solve}

\begin{exercise}
	设$u(x,t)$是$(0,\frac\pi2)\times(0,\infty)$中初边值问题
	$$\begin{cases}&u_t=u_{xx}\\&u(0,t)=1,\quad u(\frac{\pi}{2},t)=4\\&u(x,0)=\cos^4x+4\sin^5x\end{cases}$$
	的解,求$\lim_t\to\infty u(x,t)$
\end{exercise}

\begin{solve}
	作代换$v(x,t)=u(x,t)-(\frac{6}{\pi}x+1)$
	则原问题化为
	$$\begin{cases}&v_t=v_{xx}\\&v(0,t)=v(\frac{\pi}{2},t)=0\\&v(x,0)=\cos^4x+4\sin^5x-\frac{6}{\pi}x-1\end{cases}$$
	由分离变量法,设$v(x,t)=X(x)T(t)$, 其中$X^{\prime\prime}+\lambda X=0, T^{\prime}+\lambda T=0$, $\lambda$为常数.
	
	当$\lambda\leq0$时, $X(x)$只有零解.
	
	当$\lambda>0$时, $X(x)=C_1\cos\sqrt{\lambda}x+C_2\sin\sqrt{\lambda}x$.
	代入边界条件, 解得
	$$\lambda_{k}=(2k+1)^{2}, k=0,1,2,\ldots$$
	则$X(x)=C_{2}\sin{(2k+1)}x$.
	将$\lambda_k$代入$T^\prime+\lambda T=0$, 解得相应的$T_k(t)$. 则
	\[v(x,t)=\sum_{k=0}^{\infty}A_{k}\e^{-(2k+1)^{2}t}\sin{(2k+1)x}\]
	代入初值条件,得
	$$\sum_{k=0}^\infty A_k\sin{(2k+1)x}=\cos^4x+4\sin^5x-\frac6\pi x-1=f(x)$$
	
	其中$A_k$由$f(x)$的Fourier展开式决定:
	\[A_{k} =\sqrt{\frac4\pi}\left\langle f(x),\sqrt{\frac4\pi}\sin{(2k+1)x}\right\rangle =\frac4\pi\int_0^{\frac\pi2}f(x)\sin{(2k+1)x}\d x\]
	故
	\[\begin{aligned}&&u(x,t)& =v(x,t)+\frac6\pi x+1 \\
	&&&=\frac6\pi x+1+\sum_{k=0}^\infty\frac4\pi\int_0^{\frac\pi2}f(x)\sin{(2k+1)x}\d x\cdot \e^{-(2k+1)^2t}\sin{(2k+1)x},
	\end{aligned}\]
	则
	\[\lim_{t\to\infty}u(x,t)=\frac{6}{\pi}x+1.\qedhere\]
\end{solve}
\section{柯西问题}

\begin{note}[性质3 (频域定理)的证明]
	设$F_1(x)=F[f_1(x)],F_2(x)=F[f_2(x)]$,则
	\[\begin{aligned}
		F^{-1}[F_1*F_2]&=F^{-1}\left[\int_{-\infty}^{\infty}F_1(x-t)F_2(t)\d t\right]\\
		&=\frac{1}{2\pi}\int_{-\infty}^{\infty}\e^{\upi\lambda x}\d\lambda\int_{-\infty}^{\infty}F_1(\lambda-t)F_2(t)\d t\\
		&=\frac{1}{2\pi}\int_{-\infty}^{\infty}F_2(t)\e^{\upi tx}\d t\int_{-\infty}^{\infty}F_1(\lambda-t)\e^{\upi(\lambda-t)x}\d\lambda\\
		&=\frac{1}{2\pi}\int_{-\infty}^{\infty}F_2(t)\e^{\upi tx}\d t\int_{-\infty}^{\infty}F_1(\xi)\e^{\upi\xi x}\d\xi\\
		&=2\pi f_1(x)f_2(x),
	\end{aligned}
	\]
	即
	\[F[f_1\cdot f_2]=\frac{1}{2\pi}F[f_1]*F[f_2].\]
\end{note}
\begin{exercise}
  求下列函数的 Fourier 变换:
  \begin{enumerate}[(1),left=\parindent]
    \item $\displaystyle\e^{-\eta x^2}\quad(\eta>0)$;
    \item $\displaystyle\e^{-a|x|}\quad(a>0)$;
    \item $\displaystyle\frac{x}{(a^2+x^2)^k},\quad\displaystyle\frac{1}{(a^2+x^2)^k}\quad$ ($a>0$, $k$ 为自然数).
  \end{enumerate}
\end{exercise}

\begin{solution}
  (1) 直接计算得
  \begin{align*}
    \widehat{\e^{-\eta x^2}}(\xi)
    & = \int_{-\infty}^{\infty} \e^{-\eta x^2}\cdot \e^{-\upi x\xi}\diff x
      = \e^{-\frac{\xi^2}{4\eta}} \int_{-\infty}^{\infty}
      \e^{-\eta\left(\xi+\frac{\upi\xi}{2\eta}\right)^2} \diff x \\
    & = \e^{-\frac{\xi^2}{4\eta}} \int_{-\infty}^{\infty}
      \e^{-y^2} \frac{1}{\sqrt{\eta}} \diff y
      = \biggl(\frac{\pi}{\eta}\biggr)^{1/2} \e^{-\frac{\xi^2}{4\eta}}.
    \end{align*}

  (2) 直接计算得
  \begin{align*}
    \widehat{\e^{-a|x|}} (\xi)
    & = \int_{-\infty}^{\infty} \e^{-a|x|}\cdot\e^{-\upi x\xi} \diff x \\
    & = 2\int_0^{\infty} \e^{-ax} \cos\xi x \diff x = \frac{2a}{a^2+\xi^2}.
  \end{align*}
  
	(3) $\displaystyle f(x)=\frac 1{(a^{2}+ x^{2})^{k}}$的Fourier变换为
	$$\tilde{f}(\lambda)=\int_{-\infty}^\infty\frac{\e^{-\upi\lambda x}}{(a^2+x^2)^k}\d x.$$
	将这个积分视为复平面上的积分,设$\displaystyle g(z)=\frac{\e^{-\upi\lambda z}}{(a^2+z^2)^k}$,  则$\tilde{f}(\lambda)$即为$g(z)$在实轴$\varGamma$上的积分.不妨设复平面中的上半平面$\varOmega$为实轴$\varGamma$所围成的区域,容易发现$g(z)$在$\varOmega$中有奇点$z=a\upi$.
  
	由留数定理,
  	$$\tilde{f}(\lambda)=\int_{\Gamma}g(z)\mathrm{d}z=2\pi\upi\operatorname*{Res}_{z=a\upi}g(z).$$
  	又奇点$z=a\upi$为$k$阶奇点,设
  	$$g(z)=\frac{\frac{e^{-\upi\lambda z}}{(z+a\upi)^k}}{(z-a\upi)^k}=\frac{\varphi(z)}{(z-a\upi)^k},$$
  	则
  	$$\begin{aligned}
  	\operatorname*{Res}_{z=a\upi}g(z)&=\frac{\varphi^{(k-1)}(a\upi)}{(k-1)!} \\
  	&=\frac1{(k-1)!}\left[\sum_{m=0}^{k-1}C_{k-1}^m\left((z+a\upi)^{-k}\right)^{(m)}\cdot\left(\e^{-\upi\lambda z}\right)^{(k-m-1)}\right]\Bigg|_{z=a\upi} \\
  	&=\frac1{(k-1)!}\left[\sum_{m=0}^{k-1}C_{k-1}^m\left(\frac{(-1)^m(k+m-1)!}{(k-1)!(2a\upi)^{k+m}}\right)\left((-\upi\lambda)^{k-m-1}\cdot\e^{a\lambda}\right)\right] \\
  	&=\frac1{(k-1)!}\sum_{m=0}^{k-1}\frac{(k+m-1)!(-1)^{k-m-1}}{\upi m!(k-m-1)!(2a)^{k+m}}\cdot\lambda^{k-m-1}\e^{a\lambda}.
  	\end{aligned}$$
  	则
  	$$\tilde{f}(\lambda)=\frac{2\pi\upi}{(k-1)!}\sum_{m=0}^{k-1}\frac{(k+m-1)!(-1)^{k-m-1}}{m!(k-m-1)!\upi(2a)^{k+m}}\cdot\lambda^{k-m-1}\e^{a\lambda}.$$
  	类似地,
  	$$\begin{aligned}
  		F\left[\frac x{(a^2+x^2)^k}\right]
  		&=-\frac{1}{\upi}\frac \d{\d\lambda}F\left[\frac1{(a^2+x^2)^k}\right] \\
  		&=-\frac{2\pi}{(k-1)!}\sum_{m=0}^{k-2}\frac{(k+m-1)!(-1)^{k-m-1}}{m!(k-m-1)!(2a)^{k+m}} \\
  		&\times\left[(k-m-1)\lambda^{k-m-2}\e^{a\lambda}+a\lambda^{k-m-1}\e^{a\lambda}\right]-\frac{\pi(2k-2)!}{[(k-1)!]^2}(2a)^{-2k+2}\e^{a\lambda} \\
  		&=-\frac{\pi\upi(2k-2)!}{\left[(k-1)!\right]^2}(2a)^{-2k+2}\e^{a\lambda}-\frac{2\pi}{(k-1)!}\sum_{m=0}^{k-2}\frac{(k+m-1)!(-1)^{k-m-1}}{m!(k-m-1)!(2a)^{k+m}} \\
  		&\times(2a)^{-k-m}(-1)^{k-m-1}\lambda^{k-m-2}\e^{a\lambda}(a\lambda+k-m-1).
  	\end{aligned}$$
\end{solution}


\begin{exercise}
  证明: 当 $f(x)$ 在 $(-\infty,\infty)$ 上绝对可积时, $F[f]$ 为连续函数.
\end{exercise}

\begin{proof}
  记 $\displaystyle F[f]=\int_{-\infty}^{\infty} f(\xi)\e^{-\upi\lambda\xi}\diff\xi = \tilde{f}(\lambda)$,则
  \[\begin{split}
    \bigl|\tilde{f}(\lambda+h)-\tilde{f}(\lambda)\bigr|
    & = \left|\int_{-\infty}^{\infty}f(\xi)
        \left(\e^{-\upi(\lambda+h)\xi} - \e^{-\upi\lambda\xi}\right)\diff\xi\right| \\
    & \leq \int_{-\infty}^{\infty}|f(\xi)|\cdot|\e^{-\upi h\xi}-1|\diff\xi\to 0
      \quad(\text{as } h\to 0),
  \end{split}\]
  故 $F[f]$ 为连续函数.
\end{proof}


\begin{exercise}
	用傅里叶变换求解三维热传导方程的柯西问题
	\[\begin{cases}
		\displaystyle\frac{\partial u}{\partial t} = a^2 \Bigl(\frac{\partial^2 u}{\partial x^2}+\frac{\partial^2 u}{\partial y^2}+\frac{\partial^2 u}{\partial z^2}\Bigr), \\
		u|_{t=0} = \varphi(x,y,z).
	\end{cases}\]
\end{exercise}

\begin{solve}
	对齐次方程及初始条件均作用关于 $x,y,z$ 的Fourier变换,记 $$F[u(x,y,z,t)]=\tilde{u}(\lambda,\mu,\nu,t),$$
	$$F[\varphi(x,y,z)]=\tilde{\varphi}(\lambda,\mu,\nu).$$
	对方程两边关于$x,y,z$进行Fourier变换,就得到
	$$\frac{\d\widetilde{u}}{\d t} = -a^{2}(\lambda^2+\mu^2+\nu^2)\widetilde{u},$$
	类似地,
	$$\tilde{u}(\lambda,\mu,\nu,0) = \tilde{\varphi}(\lambda,\mu,\nu,0).$$
	这是带参数$\lambda,\mu,\nu$的常微分方程的Cauthy问题,它的解为
	$$\tilde{u}(\lambda,\mu,\nu,t)=\tilde{\varphi}\e^{-a^2(\lambda^2+\mu^2+\nu^2)t}.$$
	函数$\e^{-a^2(\lambda^2+\mu^2+\nu^2)t}$的Fourier逆变换为
	\begin{align*}
		F^{-1}\left[\e^{-a^2(\lambda^2+\mu^2+\nu^2)t}\right]&=\left(\frac{1}{2\pi}\right)^3{\iiint}_{\mathbb{R}^3}\e^{-a^2\left(\lambda^2+\mu^2+\nu^2\right)t}\cdot\e^{\upi(\lambda x+\mu y+\nu z)}\d\lambda\d\mu\d\nu\\
		&= {\left(\frac{1}{2\pi} \right)}^{3}\e^{{-}\frac{x^2+y^2+z^2}{4a^2t}}{\left( \sqrt{\frac{\pi}{a^2t}} \right)}^3\\
		&= {\left(\frac{1}{2a\sqrt{\pi t}}\right)}^3\e^{-\frac{x^2+y^2+z^2}{4a^2t}}.
	\end{align*}
	故可得Cauthy问题的解为
	\[u={\left(\frac{1}{2a\sqrt{\pi t}} \right)}^3{{\iiint}}_{{\mathbb{R}}^{3}}\varphi(\xi,\eta,\zeta)\e^{-\frac{{(x {-} \xi)}^{2} + {(y-\eta)}^2+{(z-\zeta)}^2}{4a^2t}}\d\zeta\d\eta\d\zeta.\qedhere\]
\end{solve}

\begin{exercise}[4]
  证明 (3.29) 所表示的函数满足非齐次方程 (3.15) 以及初始条件 (3.16).
\end{exercise}

\begin{proof}
  It suffices to verify that $u(x,t)$ given by (3.28) is the solution to
  the Cauchy problem (3.23)--(3.24). Denote the heat kernel by
  \[ \varPhi(x,t) = \frac{1}{2a\sqrt{\pi t}} \e^{-\frac{x^2}{4a^2t}}, \]
  which satisfies $\partial_t\varPhi = a^2\partial_{xx}\varPhi$. Then
  the $u(x,t)$ in (3.28) can be rewritten as
  \begin{equation}\label{eq:d1}
    u(x,t) = \int_0^t\int_{-\infty}^\infty \varPhi(x-\xi, t-\tau) f(\xi,\tau) \d\xi\d\tau.
  \end{equation}
  Making a change of variables, we have
  \begin{equation}\label{eq:d2}
    u(x,t) = \int_0^t\int_{-\infty}^\infty \varPhi(\xi,\tau) f(x-\xi,t-\tau) \d\xi\d\tau.
  \end{equation}
  Hence
  \begin{align}
    \frac{\partial u}{\partial t}
    & = \int_0^t\int_{-\infty}^\infty \varPhi(\xi,\tau) f_t(x-\xi,t-\tau) \d\xi\d\tau \notag \\
    &   \quad + \int_{-\infty}^\infty \varPhi(\xi,t)f(x-\xi,0) \d\xi, \\
    \frac{\partial^2u}{\partial x^2}
    & = \int_0^t \int_{-\infty}^\infty \varPhi(\xi,\tau) f_{xx}(x-\xi,t-\tau) \d\xi\d\tau.
  \end{align}
  So
  \begin{align*}
    \frac{\partial u}{\partial t} - a^2\frac{\partial^2u}{\partial x^2}
    & = \int_0^t \int_{-\infty}^\infty \varPhi(\xi,\tau) (\partial_t - a^2\partial_{xx})
        f(x-\xi,t-\tau) \d\xi\d\tau \\
    &   \quad + \int_{-\infty}^\infty \varPhi(\xi,t) f(x-\xi,0) \d\xi \\
    & = \int_0^\epsilon \int_{-\infty}^\infty \varPhi(\xi,\tau) (\partial_t - a^2\partial_{xx})
        f(x-\xi,t-\tau) \d\xi\d\tau \\
    &   \quad + \int_\epsilon^t \int_{-\infty}^\infty \varPhi(\xi,\tau) (-\partial_\tau - a^2\partial_{\xi\xi})
        f(x-\xi,t-\tau) \d\xi\d\tau \\
    &   \quad + \int_{-\infty}^\infty \varPhi(\xi,t) f(x-\xi,0) \d\xi \\
    & =: I_\epsilon + J_\epsilon + K.
  \end{align*}
  We suppose that $f$ is $C^1$ in $t$ and $C^2$ in $x$ and has compact support.
  Then
  \begin{equation}\label{eq:d3}
    |I_\epsilon| \leq \bigl(\|f_t\|_\infty + a^2 \|f_{xx}\|_\infty\bigr)
      \int_0^\epsilon \int_{-\infty}^\infty \varPhi(\xi,\tau) \d\xi\d\tau \leq C\epsilon.
  \end{equation}
  On the other hand, integrate by parts to get
  \begin{equation}\label{eq:d4}
    \begin{aligned}
      J_\epsilon
      & = \int_\epsilon^t \int_{-\infty}^\infty (\partial_\tau - a^2\partial_{\xi\xi}) \varPhi(\xi,\tau)
          f(x-\xi,t-\tau) \d\xi\d\tau \\
      &   \quad + \int_{-\infty}^\infty \varPhi(\xi,\epsilon) f(x-\xi,t-\epsilon) \d\xi \\
      &   \quad - \int_{-\infty}^\infty \varPhi(\xi,t) f(x-\xi,0) \d\xi \\
      & = \int_{-\infty}^\infty \varPhi(\xi,\epsilon) f(x-\xi,t-\epsilon) \d\xi - K.
    \end{aligned}
  \end{equation}
  Therefore
  \begin{equation}\label{eq:d5}
    \frac{\partial u}{\partial t} - a^2\frac{\partial^2u}{\partial x^2}
      = \lim_{\epsilon\to 0} \int_{-\infty}^\infty \varPhi(\xi,\epsilon) f(x-\xi,t-\epsilon) \d\xi
      = f(x,t).
  \end{equation}
  The initial value condition (3.24) is obvious.
\end{proof}


\begin{exercise}[5]
  求解热传导方程 (3.17) 的柯西问题, 已知
  \begin{enumerate}[(1)]
    \item $u|_{t=0} = \sin x$,
    \item 用延拓法求解半有界直线上的热传导方程 (3.17), 假设
      \[\begin{cases}
        u(x,0) = \varphi(x)\quad (0<x<\infty), \\
        u(0,t) = 0.
      \end{cases}\]
  \end{enumerate}
\end{exercise}

\begin{solution}
  (1) 由泊松公式知
  \[\begin{split}
    u(x,t)
    & = \frac{1}{2a\sqrt{\pi t}} \int_{-\infty}^{\infty}
        \sin\xi\cdot \e^{-\frac{(x-\xi)^2}{4a^2t}}\diff\xi \\
    & = \frac{1}{2a\sqrt{\pi t}}\cdot 2a\sqrt{t}\int_{-\infty}^{\infty}
        \sin(x-2a\sqrt{t}\zeta)\cdot \e^{-\zeta^2}\diff\zeta \\
    & = \frac{1}{\sqrt{\pi}} \int_{-\infty}^{\infty}
        \sin x\cos 2a\sqrt{t}\zeta\cdot \e^{-\zeta^2}\diff\zeta \\
    & = \frac{2\sin x}{\sqrt{\pi}} \int_0^{\infty}
        \cos 2a\sqrt{t}\zeta\cdot \e^{-\zeta^2}\diff\zeta \\
    & = \e^{-a^2t}\sin x.
  \end{split}\]

  (2) 对 $\varphi(x)$ 作奇延拓, 即
  \[\varPhi(x) = \begin{cases}
    \varphi(x),   & x\geq 0, \\
    -\varphi(-x), & x<0.
  \end{cases}\]
  求解如下 Cauchy 问题
  \[\begin{cases}
    u_t = a^2 u_{xx}, \\
    u|_{t=0} = \varPhi(x).
  \end{cases}\]
  得
  \begin{align*}
    u(x,t)
    & = \frac{1}{2a\sqrt{\pi t}} \int_{-\infty}^{\infty}
        \varPhi(\xi) \e^{-\frac{(x-\xi)^2}{4a^2t}} \diff\xi \\
    & = \frac{1}{2a\sqrt{\pi t}} \biggl[
        \int_0^{\infty} \varphi(\xi) \e^{-\frac{(x-\xi)^2}{4a^2t}} \diff\xi
        + \int_{-\infty}^0 -\varphi(-\xi) \e^{-\frac{(x-\xi)^2}{4a^2t}}\diff\xi\biggr] \\
    & = \frac{1}{a\sqrt{\pi t}} \int_0^{\infty} \varphi(\xi)
        \e^{-\frac{x^2+\xi^2}{4a^2 t}}\sinh \frac{x\xi}{2a^2 t}\diff\xi. \qedhere
  \end{align*}
\end{solution}


\begin{exercise}
	证明函数
	$$v(x,y,t;\xi,\eta,\tau)=\frac{1}{4\pi a^2(t-\tau)}\e^{-\frac{(x-\xi)^2+(y-\eta)^2}{4a^2(t-\tau)}}$$
	对于变量$(x,y,t)$满足方程
	$$\dfrac{\partial v}{\partial t}=a^2\left(\dfrac{\partial^2v}{\partial x^2}+\dfrac{\partial^2v}{\partial y^2}\right),$$
	而对于变量$(\xi,\eta,\tau)$满足方程
	$$\frac{\partial v}{\partial\tau}+a^2\left(\frac{\partial^2v}{\partial\xi^2}+\frac{\partial^2v}{\partial\eta^2}\right)=0.$$
\end{exercise}

\begin{proof}
	直接验证即可.因为
	$$\frac{\partial v}{\partial t}=\frac{1}{4\pi a}\left[-\frac{1}{(t-\tau)^{2}}+\frac{(x-\xi)^{2}+(y-\eta)^{2}}{4a^{2}(t-\tau)^{3}}\right]\e^{-\frac{(x-\xi)^{2}+(y-\eta)^{2}}{4a^{2}(t-\tau)}},$$
	$$\frac{\partial v}{\partial x}=\frac{1}{4\pi a^{2}}\frac{-2(x-\xi)}{4a^{2}(t-\tau)^{2}}e^{-\frac{(x-\xi)^{2}+(y-\eta)^{2}}{4a^{2}(t-\tau)}},$$
	$$\frac{\partial^{2}v}{\partial x^{2}}=\frac{1}{4\pi a^{2}}\frac{1}{(t-\tau)^{2}}\left[\frac{-1}{2a^{2}}+\frac{(x-\xi)^{2}}{4a^{4}(t-\tau)}\right]e^{-\frac{(x-\xi)^{2}+(y-\eta)^{2}}{4a^{2}(t-\tau)}}.$$
	同理
	$$\frac{\partial^2v}{\partial y^2}=\frac{1}{4\pi a^2}\frac{1}{(t-\tau)^2}\left[\frac{-1}{2a^2}+\frac{(y-\eta)^2}{4a^4(t-\tau)}\right]e^{-\frac{(x-\xi)^2+(y-\eta)^2}{4a^2(t-\tau)}},$$
	所以
	$$\frac{\partial^2v}{\partial x^2}+\frac{\partial^2v}{\partial y^2}=\frac{1}{4\pi a^2}\left[-\frac{1}{a^2(t-\tau)^2}+\frac{(x-\xi)^2+(y-\eta)^2}{4a^4(t-\tau)^3}\right]e^{-\frac{(x-\xi)^2+(y-\eta)^2}{4a^2(t-\tau)}}=\frac{1}{a^2}\frac{\partial v}{\partial t}.$$
	类似地,
	$$\dfrac{\partial v}{\partial t}=-\dfrac{\partial v}{\partial\tau},\quad\dfrac{\partial v}{\partial\xi}=-\dfrac{\partial v}{\partial x},\quad\dfrac{\partial^2v}{\partial\xi^2}=\dfrac{\partial^2v}{\partial x^2},\quad\dfrac{\partial^2v}{\partial\eta^2}=\dfrac{\partial^2v}{\partial y^2},$$
	所以
	$$\frac{\partial v}{\partial\tau}+a^2\left(\frac{\partial^2v}{\partial\xi^2}+\frac{\partial^2v}{\partial\eta^2}\right)=0.$$
\end{proof}
\begin{exercise}[7]
  证明: 如果 $u_1(x,t)$, $u_2(y,t)$ 分别是下述两个定解问题的解:
  \[\begin{cases}
    \displaystyle\frac{\partial u_1}{\partial t} = a^2 \frac{\partial^2u_1}{\partial x^2}, \\
    u_1|_{t=0} = \varphi_1(x);
  \end{cases}
  \qquad
  \begin{cases}
    \displaystyle\frac{\partial u_2}{\partial t} = a^2 \frac{\partial^2u_2}{\partial y^2}, \\
    u_2|_{t=0} = \varphi_2(y).
  \end{cases}\]
  则 $u(x,y,t) = u_1(x,t)u_2(y,t)$ 是定解问题
  \[\begin{cases}
    \displaystyle\frac{\partial u}{\partial t} = 
      a^2\Bigl(\frac{\partial^2u}{\partial x^2} + \frac{\partial^2u}{\partial y^2}\Bigr), \\
    u|_{t=0} = \varphi_1(x) \varphi_2(y)
  \end{cases}\]
  的解.
\end{exercise}

\begin{proof}
  直接验证即可. 因为
  	$$\frac{\partial u}{\partial t}=\frac{\partial u_1}{\partial t}u_2+\frac{\partial u_2}{\partial t}u_1,$$
  	所以
  	$$a^{2}\left(\frac{\partial^{2}u}{\partial t^{2}}+\frac{\partial^{2}u}{\partial y^{2}}\right)=a^{2}\frac{\partial^{2}u_{1}}{\partial t^{2}}u_{2}+a^{2}u_{1}\frac{\partial^{2}u_{2}}{\partial y^{2}}=\frac{\partial u_{1}}{\partial t}u_{2}+u_{1}\frac{\partial u_{2}}{\partial t}=\frac{\partial u}{\partial t}.$$
  	又初始条件满足
  	\[u|_{t=0}=u_{1}|_{t=0}\cdot u_{2}|_{t=0}=\varphi_{1}(x)\varphi_{2}(y).\qedhere\]
\end{proof}


\begin{exercise}[8]
  导出下列热传导方程柯西问题解的表达式:
  \[\begin{cases}
    \displaystyle\frac{\partial u}{\partial t} 
      = a^2\Bigl(\frac{\partial^2u}{\partial x^2} +
                  \frac{\partial^2u}{\partial y^2}\Bigr), \\
   \displaystyle u|_{t=0} = \sum_{i=1}^n \alpha_i(x)\beta_i(y).
  \end{cases}\]
\end{exercise}

\begin{solution}
  由叠加原理与上题结果或直接应用 Fourier 变换可得解为
  \[u(x,y,t) = \frac{1}{4a^2\pi t} \sum_{i=1}^n \int_{-\infty}^{\infty}
    \int_{-\infty}^{\infty} \alpha_i(\xi) \beta_i(\eta)
      \exp\biggl(-\frac{(x-\xi)^2+(y-\eta)^2}{4a^2 t}\biggr) \diff\xi\diff\eta.\qedhere\]
\end{solution}


\begin{exercise}[9]
  验证:二维热传导方程柯西问题
  \[\begin{cases}
    \displaystyle\frac{\partial u}{\partial t} = a^2 
      \Bigl(\frac{\partial^2 u}{\partial x^2} + \frac{\partial^2 u}{\partial y^2}\Bigr), \\
    u|_{t=0} = \varphi(x,y)
  \end{cases}\]
  解的表达式为
  \[u(x,y,t) = \frac{1}{4\pi a^2t} \int_{-\infty}^{\infty}
    \int_{-\infty}^{\infty} \varphi(\xi,\eta) \e^{-\frac{(x-\xi)^2+(y-\eta)^2}{4a^2t}}
    \diff\xi\diff\eta.\]
\end{exercise}

\begin{proof}
  本习题应该添加假设: $\varphi(x,y)$有界, 因为
  \[u(x,y,t) = \frac{1}{4\pi a^2t}\int_{-\infty}^{\infty}
    \int_{-\infty}^{\infty}\varphi(\xi,\eta)
      \e^{-\frac{(x-\xi)^2+(y-\eta)^2}{4a^2t}}\diff\xi\diff\eta,\]
  所以
  \[\begin{split}
    \frac{\partial u}{\partial t}
    ={} & \frac{-1}{4\pi a^2t^2}\int_{-\infty}^{\infty}\int_{-\infty}^{\infty}
      \varphi(\xi,\eta) \e^{-\frac{(x-\xi)^2+(y-\eta)^2}{4a^2t}}\diff\xi\diff\eta \\
        & + \frac{1}{4\pi a^2t}\int_{-\infty}^{\infty}\int_{-\infty}^{\infty}
          \varphi(\xi,\eta) \e^{-\frac{(x-\xi)^2+(y-\eta)^2}{4a^2t}}
          \cdot\frac{(x-\xi)^2+(y-\eta)^2}{4a^2t^2}\diff\xi\diff\eta.
  \end{split}\]
  又
  \[\frac{\partial u}{\partial x} = 
    \frac{1}{4\pi a^2t}\int_{-\infty}^{\infty}\int_{-\infty}^{\infty}
    \varphi(\xi,\eta) \e^{-\frac{(x-\xi)^2+(y-\eta)^2}{4a^2t}}
    \cdot\frac{-(x-\xi)}{2a^2t}\diff\xi\diff\eta.\]
  故
  \[\frac{\partial^2u}{\partial x^2} =
    \frac{1}{4\pi a^2t}\int_{-\infty}^{\infty}\int_{-\infty}^{\infty}
    \varphi(\xi,\eta) \e^{-\frac{(x-\xi)^2+(y-\eta)^2}{4a^2t}}
    \left(\frac{-1}{2a^2t} + \frac{(x-\xi)^2}{4a^4t^2}\right) \diff\xi\diff\eta.\]
  显然 $\displaystyle\frac{\partial^2u}{\partial y^2}$ 的结果形式同 $\displaystyle\frac{\partial^2u}{\partial x^2}$, 故
  \begin{align*}
    & a^2\left(\frac{\partial^2u}{\partial x^2}+\frac{\partial^2u}{\partial y^2}\right) \\
    ={} & \frac{1}{4\pi t}\int_{-\infty}^{\infty}\int_{-\infty}^{\infty}
          \varphi(\xi,\eta) \e^{-\frac{(x-\xi)^2+(y-\eta)^2}{4a^2t}}
          \left(\frac{-1}{a^2t}+\frac{(x-\xi)^2+(y-\eta)^2}{4a^4t^2}\right)\diff\xi\diff\eta.
  \end{align*}
  对比可知
  \[\frac{\partial u}{\partial t}
    = a^2\left(\frac{\partial^2u}{\partial x^2}+\frac{\partial^2u}{\partial y^2}\right).\]
  
  对于初值的检验可对照教材P61的方法,下面不妨简单叙述一下.
  要证明当 $t\to 0$, $x\to x_0$, $y\to y_0$ 时,
  $u(x,y,t)\to\varphi(x_0,y_0)$, 令 $\displaystyle\zeta=\frac{x-\xi}{2a\sqrt{t}}$,
  $\displaystyle\theta=\frac{y-\eta}{2a\sqrt{t}}$,则
  \[u(x,y,t) =
    \frac{1}{\pi}\int_{-\infty}^{\infty}\int_{-\infty}^{\infty}
    \varphi(x-2a\sqrt{t}\zeta,y-2a\sqrt{t}\theta) \e^{-(\zeta^2+\theta^2)}
    \diff\zeta\diff\theta.\]
  而
  \[\varphi(x_0,y_0) =
    \frac{1}{\pi}\int_{-\infty}^{\infty}\int_{-\infty}^{\infty}
    \varphi(x_0,y_0) \e^{-(\zeta^2+\theta^2)} \diff\zeta\diff\theta.\]
  故
  \[u(x,y,t) - \varphi(x_0,y_0) =
    \frac{1}{\pi} \int_{-\infty}^{\infty}\int_{-\infty}^{\infty}
    \left[\varphi(x-2a\sqrt{t}\zeta,y-2a\sqrt{t}\theta)-\varphi(x_0,y_0)\right]
    \e^{-(\zeta^2+\theta^2)}\diff\zeta\diff\theta.\]
  将 $(\zeta,\theta)$ 平面用正方形 (四个顶点为$(\pm N,\pm N)$) 分成两个部分.
  在正方形内部, 利用$\varphi(x,y)$的连续性控制, 在正方形的外部,
  用积分$\displaystyle\int_{-\infty}^{\infty}\int_{-\infty}^{\infty}e^{-(\zeta^2+\theta^2)}\diff\zeta\diff\theta$ 可以任意小以及$\varphi(x,y)$是有界的来进行控制即可证明.
\end{proof}


\section{极值原理,定解问题解的唯一性和稳定性}

\begin{exercise}
  证明方程 $\displaystyle\frac{\partial u}{\partial t} = a^2 \frac{\partial^2 u}{\partial x^2} + cu (c\geq 0)$ 具有狄利克雷边界条件的初边值问题解的唯一性和稳定性.
\end{exercise}

\begin{proof}
  设$u(x,t)$满足的定解问题为
  \[\begin{cases}
    u_t = a^2 u_{xx} + cu, \\
    u(x,0) = \varphi(x), \\
    u(\alpha,t) = \mu_1(t),\,u(\beta,t)=\mu_2(t).
  \end{cases}\]
  则令 $v(x,t)=u(x,t) \e^{-ct}$, 可得 $v(x,t)$ 满足的定解问题为
  \[\begin{cases}
    v_t = a^2v_{xx}, \\
    v(x,0) = \varphi(x), \\
    v(\alpha,t) = \mu_1(t)\e^{-ct},\,v(\beta,t) = \mu_2(t)\e^{-ct}.
  \end{cases}\]
  由定理4.2知上述定解问题的解是唯一的且稳定的,
  记为 $v=v_0(x,t)$, 则原定解问题的解为$u = u_0(x,t) = \e^{ct}v_0(x,t)$, 显然也是唯一的且稳定的.
\end{proof}


\begin{exercise}
  利用热传导方程极值原理的方法, 证明:满足方程$\displaystyle \frac{\partial^2 u}{\partial x^2}+\frac{\partial^2 u}{\partial y^2}=0$的函数在有界区域上的最大值不会超过
  它在边界上的最大值.
\end{exercise}

\begin{proof}
  记有界闭区域为 $\varOmega$, 其边界为 $\varGamma$.
  设 $u(x,y)$ 在 $\varOmega$ 上的最大值为 $M$, 在 $\varGamma$ 上的最大值为 $m$.
  假设在区域内部存在某点 $(x^*,y^*)$ 使得
  \[u(x^*,y^*) = M > m.\]
  作辅助函数
  \[V(x,y) = u(x,y)+\frac{M-m}{4l^2}\left[(x-x^*)^2+(y-y^*)^2\right],\]
  其中 $l$ 为 $\varOmega$ 的直径. 由于在 $\varGamma$ 上
  \[V(x,y) < m+\frac{M-m}{4}=\frac M4+\frac 34m=\theta M\quad (0<\theta<1),\]
  而
  \[V(x^*,y^*)=u(x^*,y^*)=M,\]
  因此, 函数 $V(x,y)$ 与$u(x,y)$ 一样, 不在$\varGamma$ 上取到最大值. 设 $V(x,y)$ 在区域 $\varOmega$ 内部某点 $(x_1,y_1)$ 取得最大值, 在这个点应有 $\displaystyle\frac{\partial^2 V}{\partial x^2}\leq0, \frac{\partial^2 V}{\partial y^2}\leq0$, 因此在点$(x^*,y^*)$处
  \[\frac{\partial^2 V}{\partial x^2}+\frac{\partial^2 V}{\partial y^2}\leq0;\]
  但是
  \[\frac{\partial^2 V}{\partial x^2}+\frac{\partial^2 V}{\partial y^2}=\frac{\partial^2 u}{\partial x^2}+\frac{M-m}{2l^2}+\frac{\partial^2 u}{\partial y^2}+\frac{M-m}{2l^2}=\frac{M-m}{l^2}>0,\]
  这就得到矛盾, 故成立 $M=m$.
\end{proof}


\begin{exercise}
  导出初边值问题
  \[\begin{cases}
    u_t - a^2 u_{xx} = f(x,t), \\
    u|_{x=0} = \mu_1(t),\quad
      \displaystyle\left(\frac{\partial u}{\partial x} + hu\right)\bigg|_{x=l} = \mu_2(t)\quad (h>0), \\
    u|_{t=0} = \varphi(x)
  \end{cases}\]
  的解 $u(x,t)$ 在 $R_T=\{0\leq t\leq T, 0\leq x\leq l\}$ 中满足估计
  \[u(x,t) \leq \e^{\lambda T} \max\biggl\{
    0, \max_{0\leq x\leq l} \varphi(x),
    \max_{0\leq t\leq T} \biggl(\e^{-\lambda t}\mu_1(t),
      \frac{\e^{-\lambda t}\mu_2(t)}{h}\biggr),
      \frac{1}{\lambda} \max_{R_T} (\e^{-\lambda t}f)
  \biggr\},\]
  其中 $\lambda>0$ 为任意正常数.
\end{exercise}

\begin{proof}
  Let $v = \e^{-\lambda t}u$, then $v$ satisfies
  \[\begin{cases}
    v_t-a^2v_{xx}+\lambda v = \e^{-\lambda t}f(x,t), \\
    v(x,0) = \varphi(x), \\
    v(0,t) = \e^{-\lambda t}\mu_1(t),\ (v_x+hv)|_{x=l} = \e^{-\lambda t}\mu_2(t).
  \end{cases}\]
  It suffices to consider the case when $u$ takes its positive maximum value at some point $(x_0,t_0)$.
  Suppose that $(x_0,t_0)$ is not on the parabolic boundary, then
  $v_t(x_0,t_0)\geq 0$, $v_{xx}(x_0,t_0)\leq 0$ and $v(x_0,t_0)>0$. So
  \begin{equation}\label{eq:e1}
    (v_t - a^2v_{xx} + \lambda v)\big|_{(x_0,t_0)} > 0.
  \end{equation}
  By choosing $\lambda$ large enough, we can let
  \begin{equation}\label{eq:e2}
    \e^{-\lambda t_0} f(x_0,t_0) \to 0.
  \end{equation}
  Thus $v_t-a^2v_{xx}+\lambda v = \e^{-\lambda t}f(x,t)$ could not hold at $(x_0,t_0)$
  and so $(x_0,t_0)$ is on the parabolic boundary.
  \begin{enumerate}[(i)]
    \item If $(x_0,t_0)\in \{x=0\}$, then
      \begin{equation}\label{eq:e3}
        v(x,t) \leq \max_{0\leq t\leq T} \e^{-\lambda t} \mu_1(t).
      \end{equation}
    \item If $(x_0,t_0)\in \{t=0\}$, then
      \begin{equation}\label{eq:e4}
        v(x,t) \leq \max_{0\leq x\leq l} \varphi(x).
      \end{equation}
    \item If $(x_0,t_0)\in \{x=l\}$, then
      \begin{equation}\label{eq:e5}
        v(x,t) \leq \max_{0\leq t\leq T} \frac{\e^{-\lambda t}\mu_2(t)}{h}.
      \end{equation}
  \end{enumerate}
  On the other hand, since $v_t(x_0,t_0)-a^2v_{xx}(x_0,t_0) \geq 0$, we have that
  \begin{equation}\label{eq:e6}
    \lambda v(x_0,t_0) \leq \e^{-\lambda t_0} f(x_0,t_0) \leq \max_{R_T} (\e^{-\lambda t}f).
  \end{equation}
  Combining \eqref{eq:e3}, \eqref{eq:e4}, \eqref{eq:e5} and \eqref{eq:e6} we conclude that
  \begin{equation}\label{eq:e7}
    v(x,t) \leq \max\biggl(0, \max_{0\leq x\leq l} \varphi(x),
      \max_{0\leq t\leq T} \biggl(\e^{-\lambda t}\mu_1(t), \frac{\e^{-\lambda t}\mu_2(t)}{h}\biggr),
      \frac{1}{\lambda} \max_{R_T} (\e^{-\lambda t}f)\biggr).
  \end{equation}
  The final estimate for $u(x,t)$ follows directly from \eqref{eq:e7}.
\end{proof}


\begin{exercise}
	证明无界区域上热传导方程的极值原理:设$u(x,t)$在带形区域$\{(x,t)|x\in\mathbb{R},0\leq t\leq T\}$上连续有界
	当$0<t<T$时满足热传导方程$u_t-a^2u_{xx}=0$,则
	$$\sup_{0\leq t\leq T,x\in \mathbb{R}}u(x,t)=\sup_{x\in \mathbb{R}}u(x,0),$$
	$$\inf_{0\leq t\leq T,x\in \mathbb{R}}u(x,t)=\inf_{x\in \mathbb{R}}u(x,0).$$
\end{exercise}

\begin{proof}
	记$\sup_x\in Ru(x,0)=A,|u(x,t)|\leq2B$
	
	构造函数$v(x,t)=\frac{4B-2A}{L^{2}}[\frac{(x-x_{0})^{2}}{2}+a^{2}t]+A$, 则$v_t=a^2v_{xy}$
	
	考虑区域 $R_0=\{(x,t)|\overset{L}{\operatorname*{\operatorname*{0}}}\leq t\leq t_{0},\tilde{|}x-x_{0}|\leq L\}$,则
	$$v(x,0)=\frac{4B-2A}{L^2}[\frac{(x-x_0)^2}{2}]+A\geq A$$
	$$v(x_0\pm L,t)=\frac{4B-2A}{L^2}\left[\frac{L^2}{2}+a^2t\right]+A\geq2B\geq u(x_0\pm L,t)$$
	由极值原理,$v(x,t)\geq u(x,t)$在$R_0$上恒成立.
	
	令 $t\to \infty$, 得 $u( x_{0}, y_{0}) \leq \lim _{L\to \infty }v( x_{0}, y_{0}) = A$.
	故
	$$\sup_{0\leq t\leq T,x\in\mathbb{R}}u(x,t)=\sup_{x\in\mathbb{R}}u(x,0).$$
	同理,
	\[\inf_{0\leq t\leq T,x\in\mathbb{R}}u(x,t)=\inf_{x\in\mathbb{R}}u(x,0).\qedhere\]
\end{proof}


\begin{exercise}[6]
	设$u(x,t)$是$\{0\leq x\leq l,0\leq t\leq T\}$中边值问题
	$$\begin{cases}&u_t=u_{xx}+f(x)\\&u|_{x=0}=u|_{x=l}=0\\&u|_{t=0}=0\end{cases}$$
	的经典解,其中$f(x)\leq0$在$0\leq x\leq l$上成立. 试证明: 对任意的$x_0\in(0,l)$, 函数$u(x_0,t)$关于 $t$ 是非增的.
\end{exercise}

\begin{proof}
	[教材 86 页注]
	由定理4.1的证明可见, 若 $u$ 是非齐次热传导方程$u_t-u_{xx}=f$的解, 且$f\leq0$, 则仍成立$\displaystyle\max_{R_T}u=\max_{\varGamma_T}u$.

	由极值原理, 上述问题的解的最大值只在抛物边界处取到,即为 $0$.
	
	任意的$x_0\in(0,l), t_0\in(0,T), u(x_0,t_0)<0=u(x_0,0)$.
	
	则显然$u(x_0,t)$ 关于 $t$ 是非增的.
\end{proof}
\newpage
\section{解的渐近性态}

\begin{exercise}
  证明方程
  \[\begin{cases}
    u_t - a^2 u_{xx} = 0, \\
    u|_{x=0} = u|_{x=l} = 0, \\
    u|_{t=0} = \varphi(x)
  \end{cases}\]
  的解当 $t\to +\infty$ 时指数地衰减于零, 其中 $\varphi\in C^2$,
  且 $\varphi(0) = \varphi(l) = 0$.
\end{exercise}

\begin{proof}
  运用分离变量法求得定解问题的解为
  \[u(x,t) = \sum_{k=1}^{\infty} A_k \e^{-\frac{k^2\pi^2a^2}{l^2}t}\sin\frac{k\pi}{l}x.\]
  其中 $\displaystyle A_k=\frac{2}{l} \int_0^l \varphi(\xi) \sin\frac{k\pi}{l}\xi\diff\xi$,
  由 $\varphi$ 有界知 $\exists C_1>0$, 使得 $|A_k|\leq C_1$, 故当 $t>1$ 时,
  \[\begin{split}
  |u(x,t)|
  & \leq C_1\sum_{k=1}^{\infty} \e^{-\frac{k^2\pi^2a^2}{l^2}t} \\
  & = C_1\biggl(1+\sum_{k=2}^{\infty}e^{-\frac{(k^2-1)\pi^2a^2}{l^2}t}\biggr)
      \e^{-\frac{\pi^2a^2}{l^2}t} \\
  & \leq C_1\biggl(1+\sum_{k=2}^{\infty}e^{-\frac{(k^2-1)\pi^2a^2}{l^2}}\biggr)
      \e^{-\frac{\pi^2a^2}{l^2}t} \\
  & < C\e^{-\frac{\pi^2a^2}{l^2}t}.
  \end{split}\]
  因此解当 $t\to+\infty$ 时指数地衰减于零.
\end{proof}


\begin{exercise}
  证明: 当 $\varphi(x,y)$ 为 $\mathbb{R}^2$ 上的有界连续函数, 且 $\varphi\in L^1(\mathbb{R}^2)$
  时, 二维热传导方程柯西问题的解, 当 $t\to +\infty$ 时, 以 $t^{-1}$ 衰减率趋于零.
\end{exercise}

\begin{proof}
  \[\begin{split}
    |u(x,y,t)|
    & = \left|\frac{1}{4\pi a^2t} \iint_{\mathbb{R}^2} \varphi(\xi,\eta)
        \e^{-\frac{(x-\xi)^2+(y-\eta)^2}{4a^2t}}\diff\xi\diff\eta\right| \\
    & \leq \frac{1}{4\pi a^2t} \iint_{\mathbb{R}^2} |\varphi(x,y)|
        \e^{-\frac{(x-\xi)^2+(y-\eta)^2}{4a^2t}}\diff\xi\diff\eta \\
    & \leq Ct^{-1},
  \end{split}\]
  其中 $C$ 是仅与 $a$ 和 $\|\varphi\|_{L^1(\mathbb{R}^2)}$ 有关的正常数.
\end{proof}
\chapter{调和方程}

\section{建立方程,定解条件}

\begin{exercise}
  设 $u(x_1,\cdots,x_n)=f(r)$ (其中 $r=\sqrt{x_1^2+\cdots+x_n^2}$)
  是 $n$ 维调和函数 $\Bigl( \text{即满足方程\ }
  \frac{\partial^2u}{\partial x_1^2}+\cdots+\frac{\partial^2u}{\partial x_n^2}=0\Bigr)$,
  试证明
  \[f(r) = c_1+\frac{c_2}{r^{n-2}}\quad (n\geq 2),\]
  \[f(r) = c_1+c_2\ln\frac{1}{r}\quad (n=2),\]
  其中 $c_1$, $c_2$ 为任意常数.
\end{exercise}

\begin{proof}
  因为 $\frac{\partial u}{\partial x_i}=\frac{\diff f}{\diff r}\frac{x_i}{r}$,所以
  \[\frac{\partial^2u}{\partial x_i^2} =
    \frac{\diff^2f}{\diff r^2}\frac{x_i^2}{r^2}
    + \frac{\diff f}{\diff r}\frac{r^2-x_i^2}{r^3},\]
  故
  \[\sum_{i=1}^n\frac{\partial^2u}{\partial x_i^2}
    = \frac{\diff^2f}{\diff r^2}+\frac{\diff f}{\diff r}\frac{n-1}{r}
    = f''(r)+\frac{n-1}{r}f'(r)=0.\]
  由上式得
  \[f'=cr^{1-n}.\]
  \begin{enumerate}[(i)]
    \item $n = 2$ 时, $f(r)=c_1+c_2\ln r=c_1+c_2\ln\frac{1}{r}$,
    \item $n\neq 2$ 时, $f(r)=c_1+\frac{c_2}{2-n}r^{2-n}=c_1+\frac{c_2}{r^{n-2}}$,
  \end{enumerate}
  其中 $c_1$, $c_2$ 为任意常数.
\end{proof}


\begin{exercise}
  证明: 拉普拉斯算子在球坐标$(r,\theta,\varphi)$下可以写成
  \[\Delta u
    = \frac{1}{r^2}\frac{\partial}{\partial r}\left(r^2\frac{\partial u}{\partial r}\right)
    + \frac{1}{r^2\sin\theta}\frac{\partial}{\partial\theta}
      \left(\sin\theta\frac{\partial u}{\partial\theta}\right)
    +\frac{1}{r^2\sin^2\theta}\frac{\partial^2u}{\partial\varphi^2}.\]
\end{exercise}

\begin{proof}
  球坐标变换及其逆变换为
  \[\begin{cases}
    x = r\sin\theta\cos\varphi, \\
    y = r\sin\theta\sin\varphi, \\
    z = r\cos\theta;
  \end{cases}\Rightarrow
  \begin{cases}
    r = \sqrt{x^2+y^2+z^2}, \\
    \theta = \arccos\frac{z}{\sqrt{x^2+y^2+z^2}}, \\
    \varphi = \arctan\frac{y}{x}.
  \end{cases}\]
  通过链式法则可得
  \[\frac{\partial u}{\partial x}
    = \frac{\partial u}{\partial r}\frac{\partial r}{\partial x}
      + \frac{\partial u}{\partial\theta}\frac{\partial\theta}{\partial x}
      + \frac{\partial u}{\partial\varphi}\frac{\partial\varphi}{\partial x}.\]
  \[\frac{\partial^2u}{\partial x^2}
    = \frac{\partial^2u}{\partial r^2}\left(\frac{\partial r}{\partial x}\right)^2
      + \frac{\partial u}{\partial r}\frac{\partial^2r}{\partial x^2}
      + \frac{\partial^2u}{\partial\theta^2}\left(\frac{\partial\theta}{\partial x}\right)^2
      + \frac{\partial u}{\partial\theta}\frac{\partial^2\theta}{\partial x^2}
      + \frac{\partial^2u}{\partial\varphi^2}\left(\frac{\partial\varphi}{\partial x}\right)^2
      + \frac{\partial u}{\partial\varphi}\frac{\partial^2\varphi}{\partial x^2}.\]
  由逆变换公式求得 (所有求导项并未在下面完全列出, 因为很多项的形式是一样的)
  \[\frac{\partial r}{\partial x}
    = \frac{x}{r},\quad
    \frac{\partial^2r}{\partial x^2} = \frac{r^2-x^2}{r^3}.\]
  \[\frac{\partial\theta}{\partial x}
    = \frac{zx}{r^2\sqrt{x^2+y^2}},\quad
    \frac{\partial\theta}{\partial z}
    = \frac{-\sqrt{x^2+y^2}}{r^2}.\]
  \[\frac{\partial^2\theta}{\partial x^2}
    = \frac{zr^2y^2-2zx^2(x^2+y^2)}{r^4(x^2+y^2)^{3/2}},\quad
    \frac{\partial^2\theta}{\partial z^2}=\frac{2z\sqrt{x^2+y^2}}{r^4}\]
  \[\frac{\partial\varphi}{\partial x} = \frac{-y}{x^2+y^2},\quad
    \frac{\partial\varphi}{\partial y} = \frac{x}{x^2+y^2},\quad
    \frac{\partial\varphi}{\partial z}=0.\]
  \[\frac{\partial^2\varphi}{\partial x^2} = \frac{2xy}{(x^2+y^2)^2},\quad
    \frac{\partial^2\varphi}{\partial y^2}=\frac{-2xy}{(x^2+y^2)^2}.\]
  故
  \[\begin{split}
  \Delta u
    & = \frac{\partial^2u}{\partial r^2}\left[\left(\frac{\partial r}{\partial x}\right)^2
        + \left(\frac{\partial r}{\partial y}\right)^2
        + \left(\frac{\partial r}{\partial z}\right)^2\right]
        + \frac{\partial u}{\partial r}\left(\frac{\partial^2r}{\partial x^2}
        + \frac{\partial^2r}{\partial y^2}
        + \frac{\partial^2r}{\partial z^2}\right) \\
    & + \frac{\partial^2u}{\partial\theta^2}
        \left[\left(\frac{\partial\theta}{\partial x}\right)^2
        + \left(\frac{\partial\theta}{\partial y}\right)^2
        + \left(\frac{\partial\theta}{\partial z}\right)^2\right]
        + \frac{\partial u}{\partial\theta}\left(\frac{\partial^2\theta}{\partial x^2}
        + \frac{\partial^2\theta}{\partial y^2}+\frac{\partial^2\theta}{\partial z^2}\right) \\
    & + \frac{\partial^2u}{\partial\varphi^2}
        \left[\left(\frac{\partial\varphi}{\partial x}\right)^2
        + \left(\frac{\partial\varphi}{\partial y}\right)^2
        + \left(\frac{\partial\varphi}{\partial z}\right)^2\right]
        + \frac{\partial u}{\partial\varphi}\left(\frac{\partial^2\varphi}{\partial x^2}
        + \frac{\partial^2\varphi}{\partial y^2}
        + \frac{\partial^2\varphi}{\partial z^2}\right) \\
    & = \frac{\partial^2u}{\partial r^2}+\frac{\partial u}{\partial r}\frac{2}{r}
        + \frac{\partial^2u}{\partial\theta^2}\frac{1}{r^2}
        + \frac{\partial u}{\partial\theta}\frac{z}{r^2\sqrt{x^2+y^2}}
        + \frac{\partial^2u}{\partial\varphi^2}\frac{1}{x^2+y^2} \\
    & = \frac{1}{r^2}\frac{\partial}{\partial r}\left(r^2\frac{\partial u}{\partial r}\right)
        + \frac{1}{r^2\sin\theta}\frac{\partial}{\partial\theta}
        \left(\sin\theta\frac{\partial u}{\partial\theta}\right)
        + \frac{1}{r^2\sin^2\theta}\frac{\partial^2u}{\partial\varphi^2}.
  \end{split}\]
  证毕.
\end{proof}



\begin{exercise}
  证明: 拉普拉斯算子在柱坐标$(r,\theta,z)$下可以写成
  \[\Delta u
    = \frac{1}{r}\frac{\partial}{\partial r}\left(r\frac{\partial u}{\partial r}\right)
    + \frac{1}{r^2}\frac{\partial^2u}{\partial\theta^2}+\frac{\partial^2u}{\partial z^2}.\]
\end{exercise}

\begin{proof}
  柱坐标变换为
  \[\begin{cases}
    x = r\cos\theta, \\
    y = r\sin\theta, \\
    z = z.
  \end{cases}\]
  故
  \[\frac{\partial u}{\partial r}
    = \frac{\partial u}{\partial x}\cos\theta
      + \frac{\partial u}{\partial y}\sin\theta,\quad
    \frac{\partial u}{\partial\theta}
    = -\frac{\partial u}{\partial x}r\sin\theta
      + \frac{\partial u}{\partial y}r\cos\theta.\]
  \[\begin{split}
  \frac{\partial^2u}{\partial r^2}
    ={} & \left[\frac{\partial}{\partial x}\left(\frac{\partial u}{\partial x}\right)\cos\theta
          + \frac{\partial}{\partial y}\left(\frac{\partial u}{\partial x}\right)\sin\theta\right]\cos\theta \\
        & + \left[\frac{\partial}{\partial x}\left(\frac{\partial u}{\partial y}\right)\cos\theta
          + \frac{\partial}{\partial y}\left(\frac{\partial u}{\partial y}\right)\sin\theta\right]\sin\theta \\
    ={} & \frac{\partial^2u}{\partial x^2}\cos^2\theta
          + 2\frac{\partial^2u}{\partial x\partial y}\sin\theta\cos\theta
          + \frac{\partial^2u}{\partial y^2}\sin^2\theta.
  \end{split}\]
  \[\begin{split}
  \frac{\partial^2u}{\partial\theta^2}=
    & -\frac{\partial u}{\partial x}r\cos\theta-\frac{\partial u}{\partial y}r\sin\theta
      - \left[\frac{\partial}{\partial x}\left(\frac{\partial u}{\partial x}\right)(-r\sin\theta)
      + \frac{\partial}{\partial y}\left(\frac{\partial u}{\partial x}\right)r\cos\theta\right]r\sin\theta \\
    & + \left[\frac{\partial}{\partial x}\left(\frac{\partial u}{\partial y}\right)(-r\sin\theta)
      + \frac{\partial}{\partial y}\left(\frac{\partial u}{\partial y}\right)r\cos\theta\right]r\cos\theta \\
    ={} & -\frac{\partial u}{\partial x}r\cos\theta-\frac{\partial u}{\partial y}r\sin\theta
      + \frac{\partial^2u}{\partial\theta^2}r^2\sin^2\theta-2\frac{\partial^2u}{\partial x\partial y}r^2\sin\theta\cos\theta+\frac{\partial^2u}{\partial y^2}r^2\cos^2\theta.
  \end{split}\]
  故
  \begin{align*}
    \frac{1}{r}\frac{\partial}{\partial r}\left(r\frac{\partial u}{\partial r}\right)
      + \frac{1}{r^2}\frac{\partial^2u}{\partial\theta^2}
      + \frac{\partial^2u}{\partial z^2}
    & = \frac{1}{r}\frac{\partial u}{\partial r}+\frac{\partial^2u}{\partial r^2}
      + \frac{1}{r^2}\frac{\partial^2u}{\partial\theta^2} + \frac{\partial^2u}{\partial z^2} \\
    & = \frac{\partial^2u}{\partial x^2}+\frac{\partial^2u}{\partial y^2}
      + \frac{\partial^2u}{\partial z^2}=\Delta u. \qedhere
  \end{align*}
\end{proof}



\begin{exercise}
  证明下列函数都是调和函数:
  \begin{enumerate}[(1)]
    \item $ax+by+c$ ($a,b,c$ 为常数);
    \item $x^2-y^2$和$2xy$;
    \item $x^3-3xy^2$和$3x^2y-y^2$;
  \end{enumerate}
\end{exercise}

\begin{proof}
  直接验证即可.
\end{proof}



\begin{exercise}
  证明用极坐标表示的下列函数都满足调和方程:
  \begin{enumerate}[(1)]
    \item $\ln r$ 和 $\theta$;
    \item $r^n\cos n\theta$ 和 $r^n\sin n\theta$ ($n$为常数);
    \item $r\ln r\cos\theta-r\theta\sin\theta$ 和 $r\ln r\sin\theta+r\theta\cos\theta$.
  \end{enumerate}
\end{exercise}

\begin{proof}
  极坐标下的 Laplace 算子为
  \[\Delta u = \frac{1}{r}\frac{\partial}{\partial r}\left(r\frac{\partial u}{\partial r}\right)
    + \frac{1}{r^2}\frac{\partial^2u}{\partial\theta^2}.\]
  代入验证 $\Delta u=0$ 即可.
\end{proof}


\begin{exercise}
  用分离变量法求解由下述调和方程的第一边值问题所描述的矩阵平板 ($0\leq x\leq a,0\leq y\leq b$)
  上的稳定温度分布:
  \[\begin{cases}
    \frac{\partial^2u}{\partial x^2}+\frac{\partial^2u}{\partial y^2}=0,\\
    u(0,y)=u(a,y)=0,\\
    u(x,0)=\sin\frac{\pi x}{a},u(x,b)=0.
  \end{cases}\]
\end{exercise}

\begin{solve}
  令 $u(x,y)=X(x)Y(y)$, 代入 $\Delta u=0$ 得
  \[\frac{X''(x)}{X(x)} = -\frac{Y''(y)}{Y(y)} = \lambda.\]
  由于 $u(x,0) = X(x)Y(0) = \sin\frac{\pi x}{a}$, 故 $X(x) = C\sin\frac{\pi x}{a}$ 且求导得
  \[\frac{X''(x)}{X(x)} = \lambda = -\biggl(\frac{\pi}{a}\biggr)^2.\]
  所以
  \[Y''(y) - \biggl(\frac{\pi}{a}\biggr)^2 Y(y) = 0.\]
  解得
  \[Y(y) = C_1 \e^{\frac{\pi}{a}y} + C_2 \e^{-\frac{\pi}{a}y},\]
  结合边界条件 $Y(b) = 0$ 得
  \[Y(y) = C_3 \Bigl(\e^{\frac{(y-b)\pi}{a}} - \e^{\frac{(b-y)\pi}{a}}\Bigr)
    = 2C_3 \sinh\frac{(y-b)\pi}{a}.\]
  于是
  \[u(x,y) = X(x)Y(y) = C_4 \sinh\frac{(y-b)\pi}{a} \sin\frac{\pi x}{a}.\]
  结合 $u(x,0) = \sin\frac{\pi x}{a}$, 得
  \[u(x,y) = \frac{\sinh \frac{(y-b)\pi}{a}}{\sin h \frac{-b\pi}{a}} \sin\frac{\pi x}{a}.\qedhere\]
\end{solve}


\begin{exercise}
  在膜型扁壳渠道闸门的设计中, 为了考察闸门在水压力作用下的受力情况, 要在矩形区域
  $0\leq x\leq a$, $0\leq y\leq b$ 上求解如下的非齐次调和方程的边值问题:
  \[\begin{cases}
    \Delta u = py+q\quad (p<0, q>0\text{\ 常数}), \\
    \displaystyle\frac{\partial u}{\partial x}\biggm|_{x=0} = 0,\ u|_{x=a} = 0, \\
    u|_{y=0,y=b} = 0.
  \end{cases}\]
  试求解之.
\end{exercise}

\begin{solution}
  令 $v = u + (x^2-a^2)(fy+g)$, 通过选取 $f = -p/2$, $g = -q/2$, 则 $v$ 满足方程
  \[\begin{cases}
    \Delta v = 0, \\
    v_x|_{x=0} = v|_{x=a} = 0, \\
    v|_{y=0} = -\frac{q}{2}(x^2-a^2) = \alpha(x), \\
    v|_{y=b} = -\frac{1}{2}(bp+q)(x^2-a^2) = \beta(x).
  \end{cases}\]
  再利用分离变量法求解即可.
\end{solution}


\begin{exercise}
  举例说明在二维 Laplace 方程的 Dirichlet 外问题中, 如对解 $u(x,y)$ 不加在无穷远处
  为有界的限制, 那么定解问题的解不是唯一的.
\end{exercise}

\begin{proof}
  考虑区域 $\varOmega = \{(x,y)\mid x^2+y^2 > 1\}$ 以及相应的 Dirichlet 外问题
  \[\Delta u = 0\text{ in }\varOmega,\quad u = 1\text{ on }\partial\varOmega.\]
  显然 $u\equiv 1$ 和 $u = c\ln\frac{1}{r}+1$ 都为对应的解.
\end{proof}


\begin{exercise}
  设
  \[J(u) = \iiint_{\varOmega} \frac12 \biggl[\biggl(\frac{\partial u}{\partial x}^2\biggr)
    + \biggl(\frac{\partial u}{\partial y}^2\biggr)
    + \biggl(\frac{\partial u}{\partial z}^2\biggr)\biggr] \diff x\diff y\diff z
    + \iint_{\varGamma} \biggl(\frac12 \sigma u^2 - gu\biggr) \diff s,\]
  变分问题的提法为: 求 $u\in V$, 使
  \[J(u) = \min_{v\in V} J(v),\]
  其中 $V = C^2(\varOmega)\cap C^1(\closure{\varOmega})$.
  试导出与此变分问题等价的边值问题, 并证明它们的等价性.
\end{exercise}

\begin{proof}
  等价的边值问题为
  \[\begin{cases}
    \Delta u = 0, \\
    \Bigl(\frac{\partial u}{\partial \bm{n}} + \sigma u\Bigr)\Bigm|_{\varGamma} = g.
  \end{cases}\]
\end{proof}



\section{格林公式及其应用}

\begin{exercise}
  证明 (2.7) 式对于 $M_0$ 在 $\varOmega$ 外与 $\Gamma$ 上的情形成立.
\end{exercise}

\begin{proof}
  当 $M_0$ 在 $\varOmega$ 外时, $v=\frac{1}{r_{M_0M}}$ 在区域 $\varOmega$ 内无奇异点, 故由格林第二公式得
  \[\iiint_{\varOmega} \left(u\Delta\frac{1}{r}-\frac{1}{r}\Delta u\right)\diff V
    = \iint_{\Gamma}\left(u\frac{\partial}{\partial\vec{n}}\left(\frac{1}{r}\right)
      - \frac{1}{r}\frac{\partial u}{\partial\vec{n}}\right) \diff S.\]
  在 $\varOmega$ 内 $\Delta u=0$, $\Delta\frac{1}{r}=0$, 故
  \[-\iint_{\Gamma}\left(u\frac{\partial}{\partial\vec{n}}\left(\frac{1}{r}\right)
    - \frac{1}{r}\frac{\partial u}{\partial\vec{n}}\right)\diff S = 0.\]

  当 $M_0$ 在 $\Gamma$ 上时, 将以 $M_0$ 为球心, 
  以充分小正数 $\varepsilon$ 为半径的球与 $\varOmega$ 相交的部分记为 $K_{\varepsilon}$,
  将 $K_{\varepsilon}$ 的包含于 $\varOmega$ 内的边界记为 $\Gamma_{\varepsilon}$,
  且记 $\partial(\varOmega\backslash K_{\varepsilon})-\Gamma_{\varepsilon}=\Gamma_{\varepsilon}'$,
  则由格林第二公式得
  \[0 = \iiint_{\varOmega\backslash K_{\varepsilon}}
    \left(u\Delta\frac{1}{r} - \frac{1}{r}\Delta u\right)\diff V
    = \iint_{\Gamma_{\varepsilon}\bigcup\Gamma_{\varepsilon}'} \left(u\frac{\partial}{\partial\vec{n}}
      \left(\frac{1}{r}\right)-\frac{1}{r}\frac{\partial u}{\partial\vec{n}}\right) \diff S.\]
  故
  \[-\iint_{\Gamma_{\varepsilon}'} \left(u\frac{\partial}{\partial\vec{n}}\left(\frac{1}{r}\right)
    -\frac{1}{r}\frac{\partial u}{\partial\vec{n}}\right) \diff S
    = \iint_{\Gamma_{\varepsilon}}\left(u\frac{\partial}{\partial\vec{n}}
      \left(\frac{1}{r}\right) - \frac{1}{r}\frac{\partial u}{\partial\vec{n}}\right)\diff S.\]
  而
  \[\iint_{\Gamma_{\varepsilon}} \left(u\frac{\partial}{\partial\vec{n}}\left(\frac{1}{r}\right)
    -\frac{1}{r}\frac{\partial u}{\partial\vec{n}}\right) \diff S
    = \iint_{\Gamma_{\varepsilon}}\left(\frac{u}{\varepsilon^2}
      - \frac{1}{\varepsilon}\frac{\partial u}{\partial\vec{n}}\right) \diff S.\]
  令 $\varepsilon\to 0$, 注意到 $\Gamma$ 充分光滑
  (这意味着面积 $S(\Gamma_{\varepsilon})\to 2\pi\varepsilon^2$), 所以有
  \[\lim_{\varepsilon\to 0} \iint_{\Gamma_{\varepsilon}}
    \left(\frac{u}{\varepsilon^2}-\frac{1}{\varepsilon}\frac{\partial u}{\partial\vec{n}}\right)\diff S
    = 2\pi u(M_0).\]
  同时 $\lim_{\varepsilon\to0}\Gamma_{\varepsilon}'=\Gamma$, 因此最终得到
  \[-\iint_{\Gamma} \left(u\frac{\partial}{\partial\vec{n}}\left(\frac{1}{r}\right)-\frac{1}{r}
    \frac{\partial u}{\partial\vec{n}}\right)\diff S = 2\pi u(M_0). \qedhere\]
\end{proof}


\begin{exercise}
  若函数 $u(x,y)$ 是单位圆周上的调和函数, 又它在单位圆周上的数值已知为 $u=\sin\theta$,
  其中 $\theta$ 表示极角, 问函数 $u$ 在原点之值等于多少?
\end{exercise}

\begin{proof}
  由平均值公式知原点之值为
  \[u(O) = \frac{1}{2\pi}\int_{\Gamma}\sin\theta\diff s
    = \frac{1}{2\pi}\int_0^{2\pi}\sin\theta\diff\theta = 0.\qedhere\]
\end{proof}


\begin{exercise}[4]
  证明: 当 $u(M)$ 在闭曲面 $\Gamma$ 的外部调和, 并且在无穷远处成立
  \[u(M) = O\biggl(\frac{1}{r_{OM}}\biggr),\quad
    \frac{\partial u}{\partial r} = O\biggl(\frac{1}{r_{OM}^2}\biggr)\quad
    (r_{OM}\to\infty),\]
  而 $M_0$ 是 $\Gamma$ 外任意一点, 则公式 (2.6) 仍成立.
\end{exercise}

\begin{proof}
  取以 $M_0$ 为球心, 以 $R$ (充分大)为半径的球 $K_R$ 使其包含曲面 $\Gamma$,
  并记该球去掉闭曲面 $\Gamma$ 内部区域后得到的部分为 $\varOmega_R$. 将 $K_R$ 的边界记为 $\Gamma_R$,
  再取以 $M_0$ 为球心, 以 $\epsilon$ 为半径的球 $K_{\epsilon}$ 使其完全包含在区域 $\varOmega_R$中,
  将 $K_{\epsilon}$ 的边界记为 $\Gamma_{\epsilon}$.
  取 $r=r_{MM_0}$, 则由格林第二公式得
  \begin{equation}
    \begin{aligned}
    0 & = \iiint_{\varOmega_R\backslash K_{\epsilon}}\left(u\Delta\frac{1}{r}
        - \frac{1}{r}\Delta u\right)\diff V \\
      & = \iint_{\Gamma\bigcup\Gamma_R\bigcup\Gamma_{\epsilon}}\left(u\frac{\partial}{\partial\vec{n}}\left(\frac{1}{r}\right)-\frac{1}{r}\frac{\partial u}{\partial\vec{n}}\right)\diff S.
    \end{aligned} \tag{$\star$}
  \end{equation}
  因为 $u(M)=O(\frac{1}{r})$, $\frac{\partial u}{\partial r}=O(\frac{1}{r^2})$ $(r\to\infty)$,
  所以当 $R\to +\infty$ 时,
  \[\iint_{\Gamma_R}\left(u\frac{\partial}{\partial\vec{n}}\left(\frac{1}{r}\right)
    - \frac{1}{r}\frac{\partial u}{\partial\vec{n}}\right)\diff S
    = \iint_{\Gamma_R}\left(\frac{-u}{R^2}-\frac{1}{R}\frac{\partial u}{\partial r}\right)\diff S = 0.\]
  又因为当 $\epsilon\to 0$ 时,
  \[\iint_{\Gamma_{\epsilon}}\left(u\frac{\partial}{\partial\vec{n}}\left(\frac{1}{r}\right)
    - \frac{1}{r}\frac{\partial u}{\partial\vec{n}}\right)\diff S
    = \iint_{\Gamma_{\epsilon}}\left(\frac{u}{\epsilon^2}
    + \frac{1}{\epsilon}\frac{\partial u}{\partial r}\right)\diff S
    \to 4\pi u(M_0).\]
  在 ($\star$) 式中令 $R\to +\infty$, $\epsilon\to 0$, 即得
  \[u(M_0)
  =-\frac{1}{4\pi}\iint_{\Gamma}\left(u\frac{\partial}{\partial\vec{n}}\left(\frac{1}{r}\right)
  -\frac{1}{r}\frac{\partial u}{\partial\vec{n}}\right)\diff S. \qedhere\]
\end{proof}


\begin{exercise}
  证明调和方程 Dirichlet 外问题解的稳定性.
\end{exercise}

\begin{proof}
  在闭曲面 $\Gamma$ 上给定两个函数 $f,f^*$, 并且在$\Gamma$上满足$|f-f^*|\leq\epsilon$,
  设 $u,u^*$ 是相应的狄利克雷外问题的解, 以 $\Gamma_R$ 表示半径为 $R$ 的球面, 令$v=u-u^*$, 因为
  \[\lim_{r\to\infty}v(x,y,z)=0.\]
  所以存在 $R_0$, 使得在 $\Gamma_{R_0}$ 及其外部满足 $|v|\leq\epsilon$,
  在 $\Gamma$ 和 $\Gamma_{R_0}$ 围成的有界区域中,
  利用极值原理知 $|v|\leq\epsilon$, 故在 $\Gamma$ 的外部满足 $|v|\leq\epsilon$,
  由此证明了狄利克雷外问题的解是稳定的.
\end{proof}


\begin{exercise}
  对于二阶偏微分方程
  \[\sum_{i,j=1}^n a_{ij} u_{x_ix_j} + \sum_{i=1}^n b_i u_{x_i} + cu = 0,\]
  其中 $a_{ij}$, $b_i$, $c$ ($i,j=1,\dots,n$) 均为常数. 假设存在常数 $\lambda>0$, 使得
  \[\sum_{i,j=1}^n a_{ij}\xi_i\xi_j \geq \lambda |\xi|^2,\quad \forall\xi\in\mathbb{R}^n.\]
  又设 $c>0$, 证明极值原理: 若 $u$ 在 $\varOmega$ 中满足方程, 在 $\varOmega\cup\Gamma$ 上连续,
  则 $u$ 不能在 $\varOmega$ 的内部达到正的最大值或负的最小值.
\end{exercise}

\begin{proof}
  假设 $u$ 在点 $x_0\in\varOmega$ 处达到正的最大值, 则
  \[\nabla u(x_0) = 0,\qquad (D_{ij} u(x_0))\leq 0.\]
  因此
  \[\sum_{i,j=1}^n a_{ij} u_{x_ix_j}(x_0) \leq 0,
    \quad \sum_{i=1}^n b_i u_{x_i}(x_0) = 0,\quad cu(x_0)<0.\]
  从而
  \[\sum_{i,j=1}^n a_{ij} u_{x_ix_j}(x_0) + \sum_{i=1}^n b_i u_{x_i}(x_0) + cu(x_0) <0,\]
  与条件矛盾.
\end{proof}


\section{格林函数}

\[u(M_0) = \iint_{\Gamma}\left[\frac{1}{4\pi r_{M_0M}}\frac{\partial u}{\partial\vec{n}}-u\frac{\partial}{\partial\vec{n}}\frac{1}{4\pi r_{M_0M}}\right]\diff S_M\]
\[\iint_{\Gamma}\left(g\frac{\partial u}{\partial\vec{n}}-u\frac{\partial g}{\partial\vec{n}}\right)\diff S_M=0\]
相减得
\[u(M_0)=\iint_{\Gamma}\left(G\frac{\partial u}{\partial\vec{n}}-u\frac{\partial G}{\partial\vec{n}}\right)\diff S_M,\mbox{其中}G(M,M_0)=\frac{1}{4\pi r_{M_0M}}-g(M,M_0)\]

\begin{note}
  格林函数的五点性质的证明.

  \begin{property}
    格林函数 $G(M, M_0)$ 除 $M=M_0$ 一点外处处满足方程 (1.1), 而当 $M\to M_0$ 时,
    $G(M, M_0)$ 趋于无穷大, 其阶数和 $\frac{1}{4\pi r_{M_0M}}$ 相同.
  \end{property}

  \begin{proof}
    除了点$M=M_0$外,$\frac{1}{4\pi r_{M_0M}}$调和,又因为$g(M,M_0)$在$\varOmega$内调和,故$G(M,M_0)$除了$M=M_0$外处处调和,由极值原理知$g(M,M_0)$在$\varOmega$上有界,故
    \[\lim_{M\to M_0}G(M,M_0)=\lim_{M\to M_0}\left(\frac{1}{4\pi r_{M_0M}}-g(M,M_0)\right)=\infty\mbox{且和}\frac{1}{4\pi r_{M_0M}}\mbox{同阶}\]
  \end{proof}

  \begin{property}
    在边界上格林函数 $G(M, M_0)$ 恒等于零.
  \end{property}

  \begin{proof}
    由$g(M,M_0)$的定义知$G(M,M_0)|_{\Gamma}=0$.
  \end{proof}

  \begin{property}
    在区域 $\varOmega$ 上成立着不等式:
    \[0 < G(M, M_0) < \frac{1}{4\pi r_{M_0M}}.\]
  \end{property}
  
  \begin{proof}
    \[0<G(M,M_0)<\frac{1}{4\pi r_{M_0M}}\Leftrightarrow 0<g(M,M_0)<\frac{1}{4\pi r_{M_0M}}\]
    由极值原理知$g(M,M_0)>0$是显然的,下面证明$g(M,M_0)<\frac{1}{4\pi r_{M_0M}}$:

    取$\delta$足够小使得$\frac{1}{4\pi r_{M_0M}}>g(M,M_0)$在$B(M_0,\delta)$上成立(这样的$\delta$显然是可以取到的),记$D=\varOmega\backslash\overline{B(M_0,\delta)}$,则$\frac{1}{4\pi r_{M_0M}}-g(M,M_0)$在$D$上调和,且
    \[\min_{\partial D}\left(\frac{1}{4\pi r_{M_0M}}-g(M,M_0)\right)=0\]
    故由极值原理知在$D$上成立$\frac{1}{4\pi r_{M_0M}}>g(M,M_0)$,从而在$\varOmega$上成立$\frac{1}{4\pi r_{M_0M}}>g(M,M_0)$.
  \end{proof}

  \begin{property}
    格林函数 $G(M, M_0)$ 在自变量 $M$ 及参变量 $M_0$ 之间具有对称性, 即设
    $M_1$, $M_2$ 为区域中的亮两点, 则
    \[G(M_1, M_2) = G(M_2, M_1).\]
  \end{property}
  
  \begin{proof}
    记$D_{\epsilon}=\varOmega\backslash(B(M_1,\epsilon)\bigcup B(M_2,\epsilon))$,
    再记$w(M)=G(M,M_2),v(M)=G(M,M_1)$,则我们要证明$w(M_1)=v(M_2)$,
    由定义知$w(M)$和$v(M)$都是$D_{\epsilon}$上的调和函数,故由格林公式得
    \[\iint_{\partial D_{\epsilon}}\left(w\frac{\partial v}{\partial\vec{n}}-v\frac{\partial w}{\partial\vec{n}}\right)\diff S=0\]
    故
    \[\iint_{\partial B(M_1,\epsilon)}\left(w\frac{\partial v}{\partial\vec{n}}-v\frac{\partial w}{\partial\vec{n}}\right)\diff S=\iint_{\partial B(M_2,\epsilon)}\left(v\frac{\partial w}{\partial\vec{n}}-w\frac{\partial v}{\partial\vec{n}}\right)\diff S\]
    当$\epsilon\to0+$时,在$\partial B(M_1,\epsilon)$上,$v=O(\frac{1}{\epsilon}),\frac{\partial v}{\partial\vec{n}}=O(\frac{1}{\epsilon^2}),\frac{\partial v}{\partial\vec{n}}=-\frac{1}{4\pi\epsilon^2}+\mbox{有界量}$,故
    \[\iint_{\partial B(M_1,\epsilon)}w\frac{\partial v}{\partial\vec{n}}\diff S=\iint_{\partial B(M_1,\epsilon)}-\frac{1}{4\pi\epsilon^2}w\diff S+\iint_{\partial B(M_1,\epsilon)}\mbox{有界量}\cdot w\diff S\to-w(M_1)\]取极限$\epsilon\to0+$,即得$w(M_1)=v(M_2)$.
  \end{proof}

  \begin{property}
    $\iint_{\varGamma} \frac{\partial G(M, M_0)}{\partial n} \diff S_M = -1$.
  \end{property}
  
  \begin{proof}
    设 $\Gamma_{\epsilon}$ 是以 $M_0$为球心,
    以 $\epsilon$ 为半径的球面, 并且其包含在 $\varOmega$ 当中, 则
    \[\begin{split}
      \iint_{\Gamma}\frac{\partial G(M,M_0)}{\partial\vec{n}}\diff S_M
      & = \iint_{\Gamma}\frac{\partial}{\partial\vec{n}}\left(\frac{1}{4\pi r_{M_0M}}\right)\diff S_M-\iint_{\Gamma}\frac{\partial g(M,M_0)}{\partial\vec{n}}\diff S_M \\
      & = \iint_{\Gamma}\frac{\partial}{\partial\vec{n}}
          \left(\frac{1}{4\pi r_{M_0M}}\right)\diff S_M\quad (\text{Theorem 2.1}) \\
      & = \iint_{\Gamma_{\epsilon}}\frac{\partial}{\partial\vec{n}}
          \left(\frac{1}{4\pi r_{M_0M}}\right)\diff S_M\quad (\text{Theorem 2.1}) \\
      & = \iint_{\Gamma_{\epsilon}}\frac{-1}{4\pi\epsilon^2}\diff S_M = -1.
    \end{split}\]
    另法:考虑定解问题(老师上课时解法)
    \[\begin{cases}
    \Delta u=0(in\;\varOmega)\\u|_{\Gamma=1}
    \end{cases}\]
    由极值原理知解为$u\equiv1$,代入(3.4)式即得结论
  \end{proof}

  二维情形圆的格林函数取为$G(M,M_0)=\frac{1}{2\pi}\left(\ln\frac{1}{r_{M_0M}}-\ln\frac{R}{\rho_0}\frac{1}{r_{M_1M}}\right)$也是为了使得$\int_{\Gamma}\frac{\partial G(M,M_0)}{\partial\vec{n}}\diff s=-1$成立,证明思路同上.
\end{note}



\begin{exercise}[5]
  求半圆区域上狄利克雷问题的格林函数.
\end{exercise}

\begin{solve}
  设上半圆区域为 $D$, 半径为$R$, $A(x_0,y_0)\in D,M(x,y)\in D$,
  点 $A$ 关于边界圆的反演点为 $A_1(R^2\frac{x_0}{\sqrt{x_0^2+y_0^2}}$,
  $R^2\frac{y_0}{\sqrt{x_0^2+y_0^2}})$,
  记 $A$ 与 $A_1$ 关于 $x$ 轴的对称点分别为 $A_2$, $A_3$,
  再记 $MA=r$, $MA_1=r_1$, $MA_2=r_2$, $MA_3=r_3$, 则格林函数为
  \[G(M,M_0)=\frac{1}{2\pi}\ln\frac{r_1r_2}{rr_3}.\qedhere\]
\end{solve}


\begin{exercise}[7]
  证明二维调和函数的可去奇点定理:
  若 $A$ 是调和函数 $u(M)$ 的孤立奇点, 在 $A$ 点邻域中成立着
  \[u(M) = o\biggl(\ln\frac{1}{r_{AM}}\biggr),\]
  则此时可以重新定义 $u(M)$ 在 $M=A$ 的值, 使它在 $A$ 点也是调和的.
\end{exercise}

\begin{proof}
  设 $K$ 是以 $A$ 为圆心, 以$R$为半径的圆,并且其完全包含在$A$的那个所考察的邻域中,
  以 $u$ 在 $K$ 上的值为边界条件, 在 $K$ 内求拉普拉斯方程的解,记为 $v$,
  下面证明在整个 $K$ 内除了点 $A$ 外 $u=v$, 令 $w=u-v$, 则
  \[\lim_{M\to A}\frac{w(M)}{\ln\frac{1}{r_{AM}}}=0.\]
  此外在圆 $K$ 的边界上 $w=0$. 作函数
  \[w_{\epsilon}(M) = \epsilon\left(\ln\frac{1}{r_{AM}}-\ln\frac{1}{R}\right).\]
  具有如下的性质: (1)在$K$的边界上$w_{\epsilon}(M)=0$;在$K$内$w_{\epsilon}(M)>0$.
  (2) 在$r=\delta$和$r=R$所包围的同心圆环上是调和函数,这里$\delta$是任意小的正数.
  因为
  \[\lim_{M\to A}\frac{w_{\epsilon}(M)}{\ln\frac{1}{r_{AM}}}=\epsilon,\]
  所以存在 $\delta'$, 使得当 $r_{AM}<\delta'$时
  \[\frac{w_{\epsilon}(M)}{\ln\frac{1}{r_{AM}}}>\frac{\epsilon}{2}.\]
  又因为
  \[\lim_{M\to A}\frac{w(M)}{\ln\frac{1}{r_{AM}}}=0,\]
  所以存在 $\delta''$,使得当$r_{AM}<\delta''$时
  \[\left|\frac{w(M)}{\ln\frac{1}{r_{AM}}}\right|<\frac{\epsilon}{2}\]
  令$\delta=\min(\delta',\delta'')$,则在圆$r_{AM}=\delta$上有
  \[\left|\frac{w(M)}{\ln\frac{1}{r_{AM}}}\right|
    \leq\frac{w_{\epsilon}(M)}{\ln\frac{1}{r_{AM}}}\Rightarrow|w(M)|\leq w_{\epsilon}(M).\]
  故由极值原理推论2知对于 $r=\delta$ 和 $r=R$ 所包围的同心圆环中的任意一点 $M$, 成立
  \[|w(M)|\leq w_{\epsilon}(M).\]
  令 $\epsilon\to0$, 即得 $w(M)=0,\forall M\in K\backslash A$.
\end{proof}


\begin{exercise}
  证明: 如果三维调和函数 $u(M)$ 在奇点 $A$ 附近表示成 $\frac{N}{r_{AM}^{\alpha}}$,
  其中常数 $0<\alpha\leq 1$, 而 $N$ 是不为零的光滑函数, 则当 $M\to A$ 时它趋于
  无穷大的阶数必与 $\frac{1}{r_{AM}}$ 同阶, 即 $\alpha=1$.
\end{exercise}

\begin{proof}
  假设$\alpha<1$,则
  \[\lim_{M\to A}r_{AM}u(M) = \lim_{M\to A}Nr_{AM}^{1-\alpha} = 0.\]
  由可去奇点定理知 $A$ 是可去奇点, 矛盾, 故 $\alpha=1$.
\end{proof}


\begin{exercise}
  Try to find a function $u$ such that it is harmonic in a circle of radius $\alpha$
  and takes the following values on the boundary $C$:
  \begin{enumerate}[(1)]
    \item $u|_C = A\cos\varphi$;
    \item $u|_C = A + B\sin\varphi$.
  \end{enumerate}
\end{exercise}

9.解:利用泊松公式(3.13)式
\[u(\rho_0,\varphi_0)=\frac{1}{2\pi}\int_0^{2\pi}\frac{(R^2-\rho_0^2)f(\varphi)}{R^2+\rho_0^2-2R\rho_0\cos(\varphi-\varphi_0)}\diff\varphi\]
(1)\[\begin{split}
u(\rho_0,\varphi_0)&=\frac{1}{2\pi}\int_0^{2\pi}\frac{(R^2-\rho_0^2)A\cos(\varphi)}{R^2+\rho_0^2-2R\rho_0\cos(\varphi-\varphi_0)}\diff\varphi\\
&=\frac{A(R^2-\rho_0^2)}{2\pi}\int_0^{2\pi}\frac{\cos(\varphi)}{R^2+\rho_0^2-2R\rho_0\cos(\varphi-\varphi_0)}\diff\varphi\\
&=\frac{A(R^2-\rho_0^2)}{2\pi}\int_{0}^{2\pi}\frac{\cos(\theta+\varphi_0)}{R^2+\rho_0^2-2R\rho_0\cos\theta}\diff\theta(\mbox{换元}\theta=\varphi-\varphi_0)\\
&=\frac{A(R^2-\rho_0^2)}{2\pi}\int_{-\pi}^{\pi}\frac{\cos\theta\cos\varphi_0-\sin\theta\sin\varphi_0}{R^2+\rho_0^2-2R\rho_0\cos\theta}\diff\theta\\
&=\frac{A(R^2-\rho_0^2)}{2\pi}\int_{-\pi}^{\pi}\frac{\cos\theta\cos\varphi_0}{R^2+\rho_0^2-2R\rho_0\cos\theta}\diff\theta\\
&=\frac{2A(R^2-\rho_0^2)\cos\varphi_0}{2\pi}\int_0^{\pi}\frac{\cos\theta}{R^2+\rho_0^2-2R\rho_0\cos\theta}\diff\theta\\
&=\frac{2A(R^2-\rho_0^2)\cos\varphi_0}{2\pi}\frac{\pi\rho_0}{R(R^2-\rho_0^2)}\\
&=\frac{A}{R}\rho_0\cos\varphi_0=\frac{A}{\alpha}\rho_0\cos\varphi_0
\end{split}\]
故\[u(\rho,\varphi)=\frac{A}{\alpha}\rho\cos\varphi\]
(2)因为$u|_C=A+B\sin\varphi=A+B\cos(\varphi-\frac{\pi}{2})$,故由叠加原理及(1)中结果知
\[u(\rho,\varphi)=A+\frac{B}{\alpha}\rho\cos\left(\varphi-\frac{\pi}{2}\right)=A+\frac{B}{\alpha}\rho\sin\varphi\]


\begin{exercise}
  Derive the solution of the Dirichlet problem of 2-dimensional Laplace
  equation in the half plane:
  \[\begin{cases}
    \Delta u = u_{xx} + u_{yy} = 0, \quad y>0, \\
    u|_{y=0} = f(x).
  \end{cases}\]
\end{exercise}

\begin{solve}
  \[G(M,M_0)=\frac{1}{2\pi}\left[\ln\frac{1}{\sqrt{(x-x_0)^2+(y-y_0)^2}}-\ln\frac{1}{\sqrt{(x-x_0)^2+(y+y_0)^2}}\right]\]
  注意到$\frac{\partial}{\partial\vec{n}}=-\frac{\partial}{\partial y}$,故
  \[\begin{split}
  u(x_0,y_0)&=-\int_{\Gamma}f(x)\frac{\partial G(M,M_0)}{\partial\vec{n}}\diff x\\
  &=\frac{1}{2\pi}\int_{-\infty}^{\infty}f(x)\frac{\partial}{\partial y}\left[\ln\frac{1}{\sqrt{(x-x_0)^2+(y-y_0)^2}}-\ln\frac{1}{\sqrt{(x-x_0)^2+(y+y_0)^2}}\right]\Bigg|_{y=0}\diff x\\
  &=\frac{1}{2\pi}\int_{-\infty}^{\infty}f(x)\left(-\frac{1}{2}\frac{2(y-y_0)}{(x-x_0)^2+(y-y_0)^2}+\frac{1}{2}\frac{2(y+y_0)}{(x-x_0)^2+(y+y_0)^2}\right)\bigg|_{y=0}\diff x\\
  &=\frac{1}{\pi}\int_{-\infty}^{\infty}f(x)\frac{y_0}{(x-x_0)^2+y_0^2}\diff x
  \end{split}\]
  因此二维调和方程在半平面上的狄利克雷问题的解为:
  \[u(x,y)=\frac{1}{\pi}\int_{-\infty}^{\infty}\frac{f(\xi)y}{(\xi-x)^2+y^2}\diff\xi. \qedhere\]
\end{solve}


\section{强极值原理,第二边值问题解的唯一性}

\begin{exercise}
  试用强极值原理来证明极值原理: 对不恒等于常数的调和函数$u(x,y,z)$,
  其在区域 $\varOmega$ 的任何内点上的值不可能达到它在 $\varOmega$ 上的上界或下界.
\end{exercise}

\begin{proof}
  假设调和函数 $u(x,y,z)$ 不恒等于常数, 且在区域 $\varOmega$ 内部某点达最小值 $m$, 记
  \[E = \{ M\in\varOmega \mid u(M) = m\}.\]
  则由 $u$ 的连续性知 $E$ 是相对闭集, 由于 $u$ 不恒为常数, 故 $\varOmega\setminus E$ 为非空开集,
  取点 $M_0\in\varOmega\setminus E$ 使得 $\dist(M_0, E) < \dist(M_0, \partial\varOmega)$.
  取以点 $M_0$ 为球心, 以 $\dist(M_0, E)$ 为半径的球 $B$, 取 $M_1\in \partial B\cap E$.
  对于 $B$ 内任一点 $M$ 均有 $u(M)>u(M_1)$, 故由强极值原理
  \[\frac{\partial u}{\partial\vec{\nu}}\bigg|_{M_1} > 0,\]
  其中 $\vec{\nu}$ 与 $B$ 在点 $M_1$ 处的内法线方向成锐角,
  但是由于 $M_1$ 是 $\varOmega$ 内部的最小值点, 故对于任意方向 $\vec{l}$ 均有
  \[\frac{\partial u}{\partial\vec{l}}\bigg|_{M_1} = 0.\]
  矛盾.
\end{proof}


\begin{exercise}
  利用极值原理和强极值原理证明: 当区域 $\varOmega$ 的边界 $\varGamma$
  满足定理 4.2 中的条件时, 调和方程第三边值问题
  \[\biggl(\frac{\partial u}{\partial \bm{n}} + \sigma u\biggr)\bigg|_{\varGamma}
    = f \quad (\sigma>0)\]
  的解的唯一性.
\end{exercise}

\begin{proof}
  只需要证明满足边界条件 $(\frac{\partial u}{\partial\bm{n}}+\sigma u)|_{\varGamma}=0$ 的只有零解即可,
  下面分两种情况讨论.

  对于第三边值问题的内问题: 假设 $u$ 不恒为常数, 则由极值原理知 $u$ 在 $\varGamma$ 上取得最大值和最小值,
  记在 $M_1$ 处取到最小值, 在 $M_2$ 处取得最大值, 则
  \[\frac{\partial u}{\partial\bm{n}}(M_1) + \sigma u(M_1) = 0
    \Rightarrow u(M_1) = -\frac{1}{\sigma}\frac{\partial u}{\partial\bm{n}}(M_1)>0.\]
  \[\frac{\partial u}{\partial\bm{n}}(M_2) + \sigma u(M_2) = 0
    \Rightarrow u(M_2) = -\frac{1}{\sigma}\frac{\partial u}{\partial\bm{n}}(M_2)<0.\]
  故 $u(M_2)<u(M_1)$, 矛盾, 故假设不成立, 所以
  \[u\equiv C\Rightarrow\frac{\partial u}{\partial\bm{n}}\bigg|_{\varGamma} = 0
    \Rightarrow u|_{\varGamma}=0\Rightarrow u\equiv 0.\]
  
  对于第三边值问题的外问题: 记边界 $\varGamma$ 的外部为 $\varOmega'$,
  假设存在 $M_0\in\varOmega'$, 使得 $u(M_0)>0$, 由于 $\lim_{M\to\infty}u(M)=0$,
  故存在充分大的 $R$, 使得在 $\varGamma_R = \{r=R\}$ 上成立 $|u|<u(M_0)$,
  则由极值原理知 $u$ 的最大值只能在 $\varGamma$ 上取, 设最大值点为 $M_1\in\varGamma$, 则
  \[\frac{\partial u}{\partial\bm{n}}\bigg|_{M_1}+\sigma u(M_1)>0,\]
  与边界条件相矛盾, 假设 $u(M_0)<0$ 同样可以导出矛盾, 故$u\equiv 0$.
\end{proof}


\begin{exercise}
  说明在证明强极值原理过程中, 不可能作出一个满足条件 (1) 和 (3) 的辅助函数 $v(x,y,z)$,
  使它在整个球 $\closure{B}_R = \{x^2+y^2+z^2 \leq R^2\}$ 内满足 $\Delta v>0$.
\end{exercise}

\begin{proof}
  若在 $B_R$ 上有 $\Delta v>0$, 则
  \[\max_{\closure{B}_R} v = \max_{\partial B_R} v = 0.\]
  又因为 $\frac{\partial v}{\partial r} < 0$, 所以
  \[\min_{\closure{B}_R} v = \min_{\partial B_R} v = 0.\]
  结合二者即得 $v\equiv 0$, 与 $\Delta v>0$ 矛盾.
\end{proof}


\begin{exercise}
  设 $\varOmega$ 为 $\mathbb{R}^3$ 中有界区域, 边界为 $\varGamma$, $u$ 为定解问题
  \[\begin{cases}
    - \Delta u + cu = f, \\
    \biggl(\frac{\partial u}{\partial \bm{n}} + \sigma u\biggr)\biggm|_{\partial\varOmega} = g
  \end{cases}\]
  的解, 其中 $c,f,g,\sigma>0$, 证明在 $\closure{\varOmega}$ 上 $u>0$.
\end{exercise}


\begin{exercise}[6]
  对于一般的椭圆型方程
  \[\sum_{i,j=1}^n a_{ij}\frac{\partial^2u}{\partial x_i\partial x_j}
    + \sum_{i=1}^n b_i \frac{\partial u}{\partial x_i} + cu = 0,\]
  其中矩阵 $(a_{ij})$ 正定, 即存在常数 $\alpha>0$ 使得
  \[\sum_{i,j=1}^n a_{ij}\xi_i\xi_j \geq \alpha|\xi|^2\quad
    \forall\xi\in \mathbb{R}^n.\]
  又设 $c\leq 0$, 试证明它的解也成立着强极值原理.
  也就是说, 如果 $u(M)$ 在球 $|x|<R$ 内满足上述方程, 在闭球 $|x|\leq R$ 上连续,
  在球面上一点 $M_0$ 处取到非正的最小值, 且在该点沿 $\bm{\nu}$ 方向的方向导数
  $\frac{\partial u}{\partial \bm{\nu}}$ 存在, 其中 $\bm{\nu}$ 与球的内法线方向成锐角,
  则在 $M_0$ 点有 $\frac{\partial u}{\partial \bm{\nu}}>0$.
\end{exercise}

\begin{proof}
  若 $u$ 在球面上一点 $M_0$ 取非正的最小值, 即 $u(M_0)\leq 0$,
  且对球内任一点 $M$ 有 $u(M)>u(M_0)$, 因此在 $M_0$ 点有
  \[\frac{\partial u}{\partial\bm{\nu}}\geq 0.\]
  现在需要证明上式中的等号不能成立, 构造函数
  \[v(x) = \e^{-a\sum_{i=1}^nx_i^2} - \e^{-aR^2} = \e^{-a|x|^2} - \e^{-aR^2},\]
  其中 $a$ 为待定的正常数, 则 $v$ 满足如下性质:
  \begin{enumerate}[(1)]
    \item 在球面 $|x|=R$ 上 $v=0$;
    \item 通过适当选取 $a$, 在区域 $D=\{ R/2 \leq |x| \leq R\}$ 内
      \[Lv = \sum_{i,j=1}^na_{ij}\frac{\partial^2v}{\partial x_i\partial x_j}
        + \sum_{i=1}^nb_i\frac{\partial v}{\partial x_i}+cv > 0.\]
      事实上,因为
      \[\frac{\partial v}{\partial x_i} = -2a x_i \e^{-a|x|^2},\]
      \[\frac{\partial^2v}{\partial x_i\partial x_j} = 4a^2 x_i x_j 
        \e^{-a|x|^2},\quad i\neq j,\]
      \[\frac{\partial^2v}{\partial x_i^2} = 4a^2 x_i^2 \e^{-a|x|^2}
        -2a \e^{-a|x|^2}.\]
      所以
      \begin{align*}
      Lv
      & = 4a^2\left(\sum_{i,j=1}^na_{ij}x_ix_j\right) \e^{-a|x|^2}
        - 2a \sum_{i=1}^n (b_ix_i + a_{ii}) \e^{-a|x|^2} \\
      & \quad + c \left(\e^{-a|x|^2} - \e^{-aR^2}\right) \\
      & = \e^{-a|x|^2} \biggl\{ 4a^2\sum_{i,j=1}^n a_{ij} x_i x_j
        - 2a\sum_{i=1}^n (b_i x_i + a_{ii}) + c \left(1 - \e^{-a(R^2-|x|^2)}\right)\biggr\}
      \end{align*}
      因为 $\sum_{i,j=1}^n a_{ij} x_i x_j \geq \alpha |x|^2 \geq \frac{\alpha R^2}{4} > 0$,
      故当 $a$ 充分大时在区域 $D$ 内 $Lv > 0$.
    \item $v$ 沿球的半径方向$\frac{\partial v}{\partial r}<0$.
        于是 $\frac{\partial v}{\partial\bm{\nu}} > 0$.
  \end{enumerate}

  作函数
  \[\tilde{u}(M)=\varepsilon v(M)+u(M_0).\]
  在 $M_0$ 点有 $\frac{\partial\tilde{u}}{\partial\bm{\nu}}
  = \varepsilon\frac{\partial v}{\partial\bm{\nu}}>0$,
  令函数
  \[w(M) := u(M)-\tilde{u}(M) = u(M)-\varepsilon v(M)-u(M_0).\]
  
  在区域 $D$ 上考察 $w(M)$:
  \begin{enumerate}[(1)]
    \item $Lw=Lu-\varepsilon Lv-Lu(M_0)=-\varepsilon Lv-cu(M_0)<0$;
    \item 在 $|x| = R/2$ 上由于 $u(M)>u(M_0)$,
      取 $\varepsilon$ 足够小可使得 $w(M)>0$;
    \item 在 $|x| = R$ 上 $v=0$, $u(M)>u(M_0)$, 故 $w(M)\geq 0$.
  \end{enumerate}
  现在证明在整个区域 $D$ 上 $w\geq 0$, 假设存在 $M_1\in D$, 使得 $w(M_1) < 0$, 于是
  \[cw(M_1)\geq 0,\quad \frac{\partial w}{\partial x_i}\bigg|_{M_1} = 0,
    \quad\left(\frac{\partial^2w}{\partial x_i\partial x_j}\right)\bigg|_{M_1}\text{ 非负定}.\]
  又 $a_{ij}=\sum_{r=1}^n g_{ri}g_{rj}$, 故
  \[\sum_{i,j=1}^na_{ij}\frac{\partial^2w}{\partial x_i\partial x_j}
    \bigg|_{M_1}=\sum_{r=1}^n\sum_{i,j=1}^n\frac{\partial^2w}{\partial x_i\partial x_j}
    \bigg|_{M_1}g_{ri}g_{rj}\geq 0,\]
  因此 $Lw|_{M_1}\geq 0$, 与(1)矛盾, 因此在 $D$ 内 $w(M)\geq w(M_0)$, 故
  \[\frac{\partial w}{\partial\bm{\nu}}\bigg|_{M_0} \geq 0.\]
  从而
  \[\frac{\partial u}{\partial\bm{\nu}} > 0. \qedhere\]
\end{proof}
\chapter{二阶线性偏微分方程的分类与总结}

\section{二阶线性方程的分类}

\begin{exercise}
  证明: 两个自变量的二阶线性方程经过自变量的可逆变换后, 其类型不会改变,
  即变换后 $\Delta = a_{12}^2 - a_{11}a_{22}$ 的符号不变.
\end{exercise}


\begin{proof}
  因为
  \[\begin{cases}
  \bar{a}_{11}=a_{11}\xi_x^2+2a_{12}\xi_x\xi_y+a_{22}\xi_y^2\\
  \bar{a}_{12}=a_{11}\xi_x\eta_x+a_{12}(\xi_x\eta_y+\xi_y\eta_x)+a_{22}\xi_x\eta_y\\
  \bar{a}_{22}=a_{11}\eta_x^2+2a_{12}\eta_x\eta_y+a_{22}\eta_y^2
  \end{cases}\]
  所以
  \[\begin{split}\overline{\Delta}&=\bar{a}_{12}^2-\bar{a}_{11}\bar{a}_{22}\\
  &=a_{12}^2(\xi_x\eta_y+\xi_y\eta_x)^2-4a_{12}^2\xi_x\xi_y\eta_x\eta_y+2a_{11}a_{22}\xi_x\xi_y\eta_x\eta_y-a_{11}a_{22}(\xi_x^2\eta_y^2+\xi_y^2\eta_x^2)\\
  &=(a_{12}^2-a_{11}a_{22})(\xi_x\eta_y-\xi_y\eta_x)^2\\
  &=\Delta\cdot\left[\frac{D(\xi,\eta)}{D(x,y)}\right]^2
  \end{split}\]
  故$\Delta$与$\overline{\Delta}$的符号相同.
\end{proof}


\begin{exercise}[3]
  化下列方程为标准形式:
  \begin{enumerate}[(1)]
    \item $u_{xx} + 4u_{xy} 5u_{yy} + u_x + 2u_y = 0$;
    \item $x^2 u_{xx} + 2xy u_{xy} + y^2 u_{yy} = 0$;
    \item $u_{xx} + yu_{yy} = 0$;
    \item $u_{xx} - 2\cos x u_{xy} - (3 + \sin^2 x)u_{yy} - yu_y = 0$;
    \item $(1+x^2)u_{xx} + (1+y^2)u_{yy} + xu_x + yu_y = 0$;
  \end{enumerate}
\end{exercise}

\begin{solve}
  (1) $u_{xx}+4u_{xy}+5u_{yy}+u_x+2u_y=0$.
  $\Delta=4-5=-1<0$, 故方程为椭圆型.
  特征方程为 $\diff y^2-4\diff x\diff y+5\diff x^2=0\Rightarrow\frac{\diff y}{\diff x}=2\pm i\Rightarrow y=(2\pm i)x+C$,取$y=(2+i)x+C$, 即$y-2x-ix=C$. 令
  \[\begin{cases}
  \xi=2x-y\\\eta=x
  \end{cases}\]
  则\[\begin{cases}
  u_x=2u_{\xi}+u_{\eta}\\
  u_y=-u_{\xi}\\
  u_{xx}=2(2u_{\xi\xi}+u_{\xi\eta})+2u_{\xi\eta}+u_{\eta\eta}=4u_{\xi\xi}+4u_{\xi\eta}+u_{\eta\eta}\\
  u_{yy}=-(-u_{\xi\xi})=u_{\xi\xi}\\
  u_{xy}=-(2u_{\xi\xi}+u_{\xi\eta})
  \end{cases}\]
  代入原方程即得标准形式为
  \[u_{\xi\xi}+u_{\eta\eta}+u_{\eta} = 0.\]

  (2) $x^2u_{xx}+2xyu_{xy}+y^2u_{yy}=0$.
  $\Delta=x^2y^2-x^2y^2=0$, 故方程为抛物型.
  特征方程为 $x^2\diff y^2-2xy\diff x\diff y+y^2\diff x^2=0\Rightarrow y=Cx$. 令
  \[\begin{cases}
  \xi=\frac{y}{x}\\\eta=x
  \end{cases}\]
  则\[\begin{cases}
  u_x=-\frac{y}{x^2}u_{\xi}+u_{\eta}\\
  u_y=\frac{1}{x}u_{\xi}\\
  u_{xx}=\frac{2y}{x^3}u_{\xi}-\frac{y}{x^2}\left(-\frac{y}{x^2}u_{\xi\xi}+u_{\xi\eta}\right)-\frac{y}{x^2}u_{\eta\xi}+u_{\eta\eta}=\frac{2y}{x^3}u_{\xi}+\frac{y^2}{x^4}u_{\xi\xi}-\frac{2y}{x^2}u_{\xi\eta}+u_{\eta\eta}\\
  u_{yy}=\frac{1}{x^2}u_{\xi\xi}\\
  u_{xy}=-\frac{1}{x^2}u_{\xi}+\frac{1}{x}\left(-\frac{y}{x^2}u_{\xi\xi}+u_{\xi\eta}\right)=-\frac{1}{x^2}u_{\xi}-\frac{y}{x^3}u_{\xi\xi}+\frac{1}{x}u_{\xi\eta}
  \end{cases}\]
  代入原方程即得标准形式为
  \[x^2u_{\eta\eta}=0\Rightarrow u_{\eta\eta} = 0.\]

  (3) $u_{xx}+yu_{yy}=0$.
  $\Delta=-y$,故$y>0$时为椭圆型,$y=0$时为抛物型,$y<0$时为双曲型.
  特征方程为 $\diff y^2+y\diff x^2=0$.

  (i) $y>0$ 时, $\frac{\diff y}{\diff x}=\pm\sqrt{y}i$,
  取 $\frac{\diff y}{\diff x}=\sqrt{y}i\Rightarrow 2\sqrt{y}-ix=C$,令
  \[\begin{cases}
  \xi=2\sqrt{y}\\\eta=-x
  \end{cases}\]
  则\[\begin{cases}
  u_x=-u_{\eta}\\
  u_{xx}=u_{\eta\eta}\\
  u_{y}=\frac{1}{\sqrt{y}}u_{\xi}\\
  u_{yy}=-\frac{1}{2}y^{-3/2}u_{\xi}+\frac{1}{y}u_{\xi\xi}
  \end{cases}\]
  代入原方程即得标准形式为
  \[u_{\eta\eta}+u_{\xi\xi}-\frac{1}{\xi}u_{\xi} = 0.\]

  (ii) $y=0$ 时, $u_{xx}=0$即为标准形式.

  (iii) $y<0$ 时, $\frac{\diff y}{\diff x}=\pm\sqrt{-y}\Rightarrow 2\sqrt{-y}\pm x=C$,令
  \[\begin{cases}
  \xi=2\sqrt{-y}+x\\\eta=2\sqrt{-y}-x
  \end{cases}\]
  则
  \[\begin{cases}
  u_x=u_{\xi}-u_{\eta}\\
  u_{xx}=u_{\xi\xi}-2u_{\xi\eta}+u_{\eta\eta}\\
  u_{y}=\frac{-1}{\sqrt{-y}}(u_{\xi}+u_{\eta})\\
  u_{yy}=-\frac{1}{2}(-y)^{-3/2}(u_{\xi}+u_{\eta})-\frac{1}{y}u_{\xi\xi}-\frac{1}{y}u_{\eta\eta}-\frac{2}{y}u_{\xi\eta}
  \end{cases}\]
  代入原方程即得标准形式为
  \[u_{\xi\eta}-\frac{1}{2(\xi+\eta)}(u_{\xi}+u_{\eta}) = 0.\]

  (4) $u_{xx}-2\cos xu_{xy}-(3+\sin^2x)u_{yy}-yu_y=0$.
  $\Delta=\cos^2x+3+\sin^2x=4>0$,故方程为双曲型.
  特征方程为 $\diff y^2+2\cos x\diff x\diff y-(3+\sin^2x)\diff x^2\Rightarrow\frac{\diff y}{\diff x}=-\cos x\pm2\Rightarrow y+\sin x\pm2x=C\Rightarrow y+\sin x\pm 2x=C$.
  令\[\begin{cases}
  \xi=y+\sin x+2x\\\eta=y+\sin x-2x
  \end{cases}\]
  则
  \[\begin{cases}
  u_x=(\cos x+2)u_{\xi}+(\cos x-2)u_{\eta}\\
  u_y=u_{\xi}+u_{\eta}\\
  u_{xx}=-\sin x(u_{\xi}+u_{\eta})+(\cos x+2)^2u_{\xi\xi}+(\cos x-2)^2u_{\eta\eta}+2(\cos^2x-4)u_{\xi\eta}\\
  u_{yy}=u_{\xi\xi}+2u_{\xi\eta}+u_{\eta\eta}\\
  u_{xy}=(\cos x+2)u_{\xi\xi}+2\cos xu_{\xi\eta}+(\cos x-2)u_{\eta\eta}
  \end{cases}\]
  代入原方程即得标准形式为
  \[u_{\xi\eta}+\frac{\xi+\eta}{32}(u_{\xi}+u_{\eta}) = 0.\]

  (5) $(1+x^2)u_{xx}+(1+y^2)u_{yy}+xu_x+yu_y=0$.
  $\Delta=-(1+x^2)(1+y^2)<0$,故方程为椭圆型.
  特征方程为$(1+x^2)\diff y^2+(1+y^2)\diff x^2=0\Rightarrow\frac{\diff y}{\diff x}=\pm\sqrt{\frac{1+y^2}{1+x^2}}i\Rightarrow\frac{\diff y}{\sqrt{1+y^2}}=\pm i\frac{\diff x}{\sqrt{1+x^2}}\Rightarrow\ln(y+\sqrt{1+y^2})\pm i\ln(x+\sqrt{1+x^2})=C$\\
  令\[\begin{cases}
  \xi=\ln(y+\sqrt{1+y^2})\\\eta=\ln(x+\sqrt{1+x^2})
  \end{cases}\]
  则
  \[\begin{cases}
  u_x=\frac{1}{\sqrt{1+x^2}}u_{\eta}\\
  u_y=\frac{1}{\sqrt{1+y^2}}u_{\xi}\\
  u_{xx}=-x(1+x^2)^{-3/2}u_{\eta}+\frac{1}{1+x^2}u_{\eta\eta}\\
  u_{yy}=-y(1+y^2)^{-3/2}u_{\xi}+\frac{1}{1+y^2}u_{\xi\xi}
  \end{cases}\]
  代入原方程即得标准形式为
  \[u_{\xi\xi}+u_{\eta\eta}=0\]
\end{solve}


\begin{exercise}
  证明: 两个自变量的二阶常系数双曲型方程或椭圆型方程一定可以经过自变量
  及未知函数的可逆变换
  \[u = \e^{\lambda\xi + \mu\eta} v\]
  将它化成
  \[v_{\xi\xi} \pm v_{\eta\eta} + cv = f\]
  的形式.
\end{exercise}

\begin{proof}
  已知两个自变量的二阶常系数双曲型方程或椭圆型方程可以通过可逆变换化为标准形式:
  \[u_{\xi\xi}\pm u_{\eta\eta}+au_{\xi}+bu_{\eta}+cu+f=0\]
  下面以椭圆型方程为例,因为$u=e^{\lambda\xi+\mu\eta}v$,所以
  \[\begin{cases}
  u_{\xi}=e^{\lambda\xi+\mu\eta}(\lambda v+v_{\xi})\\
  u_{\eta}=e^{\lambda\xi+\mu\eta}(\mu v+v_{\eta})\\
  u_{\xi\xi}=e^{\lambda\xi+\mu\eta}(v_{\xi\xi}+2\lambda v_{\xi}+\lambda^2v)\\
  u_{\eta\eta}=e^{\lambda\xi+\mu\eta}(v_{\eta\eta}+2\mu v_{\eta}+\mu^2v)
  \end{cases}\]
  故
  \[\begin{split}
  u_{\xi\xi}&+u_{\eta\eta}+au_{\xi}+bu_{\eta}+cu+f\\
  &=e^{\lambda\xi+\mu\eta}\big[v_{\xi\xi}+v_{\eta\eta}+(2\lambda+a)v_{\xi}+(2\mu+b)v_{\eta}+(\lambda^2+\mu^2+a\lambda+b\mu+c)v\big]+f\\
  &=0
  \end{split}\]
  令\[\lambda=-\frac{a}{2},\mu=-\frac{b}{2},c_1=c-\frac{a^2}{4}-\frac{b^2}{4},f_1=-fe^{-(\lambda\xi+\mu\eta)}\]
  即可将原方程化简为
  \[v_{\xi\xi}+v_{\eta\eta}+c_1v=f_1\]
  对于双曲型方程的情形可以进行类似证明.
\end{proof}


\section{二阶线性方程的特征理论}

\begin{exercise}
  对波动方程 $u_{tt} - a^2(u_{xx}+u_{yy}) = 0$, 求过直线 $l: t = 0, y = 2x$ 的特征平面.
\end{exercise}

\begin{solve}
  特征方程为
  \[\alpha_0^2-a^2(\alpha_1^1+\alpha_2^2) = 0.\]
  又因为$\alpha_0^2+\alpha_1^2+\alpha_2^2=1$, 所以解出
  \[\alpha_0 = \pm\frac{a}{\sqrt{1+a^2}},
    \quad \alpha_1=\frac{\cos\theta}{\sqrt{1+a^2}},
    \quad \alpha_2=\frac{\sin\theta}{\sqrt{1+a^2}}.\]
  故过直线 $t=0,y=2x$ 的特征平面为: $at+\cos\theta\cdot(x-x_0)+\sin\theta\cdot(y-2x_0)=0$,
  其中$\theta$, $x_0$为参数.
\end{solve}


\begin{exercise}
  证明: 经过可逆的坐标变换 $x_i = f_i(y_1, \ldots, y_n)$ $(i=1, \ldots, n)$,
  原方程的特征曲面变为经变换后的新方程的特征曲面,
  即特征曲面关于可逆坐标变换具有不变形.
\end{exercise}

\begin{proof}
  考虑二阶线性方程
  \[\sum_{i,j=1}^nA_{ij}\frac{\partial^2u}{\partial x_i\partial x_j}+\sum_{i=1}^nB_i\frac{\partial u}{\partial x_i}+Cu=F\]
  设$G(x_1,\cdots,x_n)=0$为其特征曲面,则
  \[\sum_{i,j=1}^nA_{ij}\frac{\partial G}{\partial x_i}\frac{\partial G}{\partial x_j}=0\]
  经过可逆的坐标变换:$x_i=f_i(y_1,\cdots,y_n)$,有
  \[\frac{\partial u}{\partial x_i}=\sum_{l=1}^n\frac{\partial u}{\partial y_l}\frac{\partial y_l}{\partial x_i}\]
  \[\frac{\partial^2u}{\partial x_i\partial x_j}=\sum_{k,l=1}^n\frac{\partial^2u}{\partial y_l\partial y_k}\frac{\partial y_l}{\partial x_i}\frac{\partial y_k}{\partial x_j}+\sum_{l=1}^n\frac{\partial u}{\partial y_l}\frac{\partial^2y_l}{\partial x_i\partial x_j}\]
  将上面两式代入原方程得
  \[\sum_{i,j=1}^nA_{ij}\left(\sum_{k,l=1}^n\frac{\partial^2u}{\partial y_l\partial y_k}\frac{\partial y_l}{\partial x_i}\frac{\partial y_k}{\partial x_j}+\sum_{l=1}^n\frac{\partial u}{\partial y_l}\frac{\partial^2y_l}{\partial x_i\partial x_j}\right)+\sum_{i=1}^nB_i\left(\sum_{l=1}^n\frac{\partial u}{\partial y_l}\frac{\partial y_l}{\partial x_i}\right)+Cu=F\]
  整理上式并简记一次偏导数项得
  \[\sum_{k,l=1}^n\left(\sum_{i,j=1}^nA_{ij}\frac{\partial y_l}{\partial x_i}\frac{\partial y_k}{\partial x_j}\right)\frac{\partial^2u}{\partial y_l\partial y_k}+\sum_{l=1}^n\widetilde{B}_l\frac{\partial u}{\partial y_l}+Cu=F\]
  设$G^*(y_1,\cdots,y_n)$为其特征曲面,则需满足
  \[\sum_{k,l=1}^n\left(\sum_{i,j=1}^nA_{ij}\frac{\partial y_l}{\partial x_i}\frac{\partial y_k}{\partial x_j}\right)\frac{\partial G^*}{\partial y_k}\frac{\partial G^*}{\partial y_l}=0\]
  另一方面,对原方程的特征曲面经过可逆变换后的特征曲面为:
  \[G(x_1,\cdots,x_n)=G(f_1(y_1,\cdots,y_n),\cdots,f_n(y_1,\cdots,y_n))=G_1(y_1,\cdots,y_n)\]
  故
  \[\sum_{i,j=1}^nA_{ij}\frac{\partial G}{\partial x_i}\frac{\partial G}{\partial x_j}=0\Rightarrow\sum_{i,j=1}^nA_{ij}\left(\sum_{l=1}^n\frac{\partial G_1}{\partial y_l}\frac{\partial y_l}{\partial x_i}\right)\left(\sum_{k=1}^n\frac{\partial G_1}{\partial y_k}\frac{\partial y_k}{\partial x_j}\right)=\sum_{k,l=1}^n\left(\sum_{i,j=1}^nA_{ij}\frac{\partial y_l}{\partial x_i}\frac{\partial y_k}{\partial x_j}\right)\frac{\partial G_1}{\partial y_k}\frac{\partial G_1}{\partial y_l}=0\]
  对比即得$G^*=G_1$,即特征曲面关于可逆坐标变换具有不变性.
\end{proof}



\section{三类方程的比较}


\begin{exercise}
  证明热传导方程
  \[\frac{\partial u}{\partial t} = a^2 \frac{\partial^2 u}{\partial x^2}\]
  的初边值问题
  \[\begin{cases}
    u(0, t) = u(l, t) = 0, \\
    u(x, 0) = \varphi(x)
  \end{cases}\]
  的解关于自变量 $x$ ($0<x<l$) 和 $t$ ($t>0$) 可进行任意次微分.
\end{exercise}

\begin{proof}
  利用分离变量法得该初边值问题的解为
  \[u(x,t) = \sum_{n=1}^{\infty}C_ne^{-\frac{n^2\pi^2a^2}{l^2}t}\sin\frac{n\pi}{l}x,\]
  其中 $C_n=\frac{2}{l}\int_0^l\varphi(x)\sin\frac{n\pi}{l}x\diff x$,
  $|C_n|\leq M$, 只需要证明级数逐项微分任意次后仍然是绝对且一致收敛即可,
  对 $t$ 微分 $\alpha$ 次, 对 $x$ 微分 $\beta$ 次需要级数
  \[\sum_{n=1}^{\infty}C_n\left(-\frac{n^2\pi^2a^2}{l^2}\right)^{\alpha}\left(\frac{n\pi}{l}\right)^{\beta}\left(\sin\frac{n\pi}{l}x\right)^{(\beta)}e^{-\frac{n^2\pi^2a^2}{l^2}t}\]
  绝对且一致收敛,而当$t\geq t_0>0$时,上述级数以
  \[\sum_{n=1}^{\infty}M\left(\frac{n\pi a}{l}\right)^{2\alpha}\left(\frac{n\pi}{l}\right)^{\beta}e^{-\frac{n^2\pi^2a^2}{l^2}t_0}\]
  为优级数,易知此级数收敛,故原级数绝对且一致收敛.
\end{proof}


\section{先验估计}

\begin{exercise}
  设 $u(x_1,\cdots,x_n)$ 在区域 $\Omega$ 上非负, 且满足不等式
  \[\sum_{i,j=1}^n a_{ij}(x) u_{x_ix_j} + \sum_{i=1}^n b_i(x) u_{x_i} + c(x)u\geq 0,\]
  其中 $a_{ij}$, $b_i$, $c$ 在 $\overline{\Omega}$ 上具有一阶连续偏导数,
  满足 (4.38) 式, 且 $c(x)\leq 0$,
  证明极值原理 $\max_{\overline{\Omega}}u=\max_{\Gamma}u$ 成立.
\end{exercise}

\begin{proof}
  See \emph{Elliptic Partial Differential Equations} (Han Qing \& Lin Fanghua) 
  Lemma 2.1 and Theorem 2.3.
\end{proof}


\begin{exercise}[3]
  在 $Q_T = (0,l)\times (0,T)$ 中考察下列初边值问题
  \begin{align*}
    & u_{tt} - a^2 u_{xx} + b(x,t)u_x + b_0(x,t)u_x + c(x,t)u = f(x,t), \\
    & u|_{x=0} = 0,\qquad (u_x + ku)|_{x=l} = 0, \\
    & u|_{t=0} = \varphi(x), \qquad u_t|_{t=0} = \psi(x),
  \end{align*}
  证明其解的唯一性及稳定性.
\end{exercise}

\begin{proof}
  令
\end{proof}


\begin{exercise}
  建立下列初边值问题的能量估计式:
  \[u_t - \Delta u + \sum_{i=1}^n b_i(x,t) u_{x_i} + c(x,t)u = f(x,t),\]
  \[\frac{\partial u}{\partial n}\bigg|_{\varGamma} = 0,
    \qquad u|_{t=0} = \varphi(x).\]
\end{exercise}

\begin{proof}
  任取 $T>0$,下面在$[0,T]$上建立能量估计式, 记$E(t)=\int_{\Omega}u^2(x,t)\diff x$,
  则$E'(t)=2\int_{\Omega}uu_t\diff x$,代入原方程得
  \[E'(t)=2\int_{\Omega}u\Delta u\diff x-2\sum_{i=1}^n\int_{\Omega}b_iuu_{x_i}\diff x-2\int_{\Omega}cu^2\diff x+2\int_{\Omega}uf\diff x\]
  由格林公式及边界条件得
  \[\begin{split}
  \int_{\Omega}u\Delta u\diff x&=\int_{\Omega}\sum_{k=1}(uu_{x_k})_{x_k}\diff x-\int_{\Omega}|\nabla u|^2\diff x\\
  &=\int_{\partial\Omega}\sum_{k=1}^nuu_{x_k}\cos(\vec{n},x_k)\diff x-\int_{\Omega}|\nabla u|^2\diff x\\
  &=\int_{\partial\Omega}u\frac{\partial u}{\partial\vec{n}}\diff x-\int_{\Omega}|\nabla u|^2\diff x=-\int_{\Omega}|\nabla u|^2\diff x
  \end{split}\]
  再设$|b(x,t)|,|c(x,t)|$在$\overline{R}_T=\overline{\Omega}\times[0,T]$上的最大值为$M$,记$C_0=\frac{M}{2}\max(1,\frac{1}{a^2})$,利用加权平均值不等式$2ab\leq\epsilon a^2+\frac{1}{\epsilon}b^2(\epsilon>0)$得
  \[\begin{split}
  E'(t)&=-2\int_{\Omega}|\nabla u|^2\diff x+2M\sum_{i=1}^n\int_{\Omega}|uu_{x_i}|\diff x+2M\int_{\Omega}u^2\diff x+2\int_{\Omega}|uf|\diff x\\
  &\leq-2\int_{\Omega}|\nabla u|^2\diff x+M\left(\epsilon\int_{\Omega}|\nabla u|^2\diff x+\frac{n}{\epsilon}\int_{\Omega}u^2\diff x\right)+(2M+1)\int_{\Omega}u^2\diff x+\int_{\Omega}f^2\diff x
  \end{split}\]
  取$\epsilon=\frac{1}{M}$,并记$\widehat{C}=nM^2+2M+1$,则
  \[\begin{split}
  E'(t)&\leq-\int_{\Omega}|\nabla u|^2\diff x+(nM^2+2M+1)\int_{\Omega}u^2\diff x+\int_{\Omega}f^2\diff x\\
  &\leq-\int_{\Omega}|\nabla u|^2\diff x+\widehat{C}\int_{\Omega}u^2\diff x+\int_{\Omega}f^2\diff x
  \end{split}\]
  由Gronwall不等式得
  \[\begin{split}
  E(t)&\leq e^{\widehat{C}t}E(0)-\int_0^te^{\widehat{C}(t-s)}\diff s\int_{\Omega}|\nabla u|^2\diff x+\int_0^te^{\widehat{C}(t-s)}\diff s\int_{\Omega}f^2\diff x\\
  &\leq e^{\widehat{C}t}E(0)-\int_0^t\diff s\int_{\Omega}|\nabla u|^2\diff x+\int_0^te^{\widehat{C}(t-s)}\diff s\int_{\Omega}f^2\diff x
  \end{split}\]
  故
  \[\int_{\Omega}u^2(x,t)\diff x+\int_0^t\diff s\int_{\Omega}|\nabla u|^2\diff x\leq e^{\widehat{C}t}\left(\int_{\Omega}\varphi^2(x)\diff x+\int_0^t\diff s\int_{\Omega}f^2\diff x\right)\]
\end{proof}


\begin{exercise}
  考察初边值问题
  \begin{align*}
    & \Delta u + \sum_{i=1}^{n} b_i(x) u_{x_i} + c(x)u = f, \\
    & \frac{\partial u}{\partial n}\bigg|_{\varGamma} = 0.
  \end{align*}
  试证当 $c(x)$ 充分负时, 其解在能量模意义下的稳定性.
\end{exercise}

\begin{proof}
  在方程两边同时乘以 $u$ 并在 $\Omega$ 上积分得
  \[\int_{\Omega}fu\diff x 
    = \int_{\Omega}\left(cu^2+\sum_{i=1}^nb_iu_{x_i}u+u\Delta u\right)\diff x.\]
  利用格林公式及边界条件得
  \[\begin{split}
  \int_{\Omega}u\Delta u\diff x
  &=\int_{\Omega}\sum_{k=1}^n(uu_{x_k})_{x_k}\diff x-\int_{\Omega}|\nabla u|^2\diff x\\
  &=\int_{\partial\Omega}\sum_{k=1}^nuu_{x_k}\cos(\vec{n},x_k)\diff x-\int_{\Omega}|\nabla u|^2\diff x\\
  &=\int_{\partial\Omega}u\frac{\partial u}{\partial\vec{n}}\diff x-\int_{\Omega}|\nabla u|^2\diff x=-\int_{\Omega}|\nabla u|^2\diff x
  \end{split}\]
  故
  \[\int_{\Omega}fu\diff x\leq-\int_{\Omega}|\nabla u|^2\diff x+\int_{\Omega}\sum b_iu_{x_i}u\diff x+\int_{\Omega}cu^2\diff x\]
  记$M=\max_{1\leq i\leq n}\max_{x\in\Omega}|b_i(x)|$,则
  \[\begin{split}
  \int_{\Omega}|\nabla u|^2\diff x-\int_{\Omega}cu^2\diff x
  &\leq\sum_{i=1}^n\int_{\Omega}b_iu_{x_i}u\diff x-\int_{\Omega}fu\diff x\\
  &\leq2M\int_{\Omega}\sum_{i=1}^n|u_{x_i}u|\diff x+\int_{\Omega}|fu|\diff x\\
  &\leq2M\left(\frac{\epsilon}{2}\int_{\Omega}|\nabla u|^2\diff x+\frac{n}{2\epsilon}\int_{\Omega}u^2\diff x\right)+\int_{\Omega}\left(\frac{1}{2}u^2+\frac{1}{2}f^2\right)\diff x
  \end{split}\]
  取$\epsilon=\frac{1}{2M}$,则
  \[\int_{\Omega}|\nabla u|^2\diff x-\int_{\Omega}cu^2\diff x\leq\frac{1}{2}\int_{\Omega}|\nabla u|^2\diff x+\left(2nM+\frac{1}{2}\right)\int_{\Omega}u^2\diff x+\frac{1}{2}\int_{\Omega}f^2\diff x\]
  令$\gamma_0=2nM^2+1$,则当$c(x)\leq-\gamma_0$时,有
  \[\int_{\Omega}|\nabla u|^2\diff x+\left(2nM^2+1\right)\int_{\Omega}u^2\diff x\leq\int_{\Omega}|\nabla u|^2\diff x-\int_{\Omega}cu^2\diff x\]
  将上面两式结合,得
  \[\int_{\Omega}|\nabla u|^2\diff x+(2nM^2+1)\int_{\Omega}u^2\diff x\leq\frac{1}{2}\int_{\Omega}|\nabla u|^2\diff x+\left(2nM^2+\frac{1}{2}\right)\int_{\Omega}u^2\diff x+\frac{1}{2}\int_{\Omega}f^2\diff x\]
  所以
  \[\frac{1}{2}\int_{\Omega}|\nabla u|^2\diff x+\frac{1}{2}\int_{\Omega}u^2\diff x\leq\frac{1}{2}\int_{\Omega}f^2\diff x\Rightarrow\int_{\Omega}(|\nabla u|^2+u^2)\diff x\leq C\int_{\Omega}f^2\diff x\]
\end{proof}
\chapter{一阶偏微分方程组}
\section{引言}
1.解:令$p=\frac{\partial u}{\partial t},q_1=\frac{\partial u}{\partial x},q_2=\frac{\partial u}{\partial y},q_3=\frac{\partial u}{\partial z}$,则
\[\begin{cases}
\frac{\partial p}{\partial t}=a^2\left(\frac{\partial q_1}{\partial x}+\frac{\partial q_2}{\partial y}+\frac{\partial q_3}{\partial z}\right)&(1)\\
\frac{\partial p}{\partial x}=\frac{\partial q_1}{\partial t},\frac{\partial p}{\partial y}=\frac{\partial q_2}{\partial t},\frac{\partial p}{\partial z}=\frac{\partial q_3}{\partial t}&(2)\\
t=0:p=\psi,q_1=\frac{\partial\varphi}{\partial x},q_2=\frac{\partial\varphi}{\partial y},q_3=\frac{\partial\varphi}{\partial z}&(3)
\end{cases}\]
原方程的解显然满足新方程,而如果新方程的解为$(p,q_1,q_2,q_3)$,则
\[u(x,y,z,t)=\varphi(x,y,z)+\int_{(0,0,0,0)}^{(x,y,x,t)}p\diff t+q_1\diff x+q_2\diff y+q_3\diff z\]
是原方程的解,其中条件(2)确保了积分与路径无关,故积分定义是合理的
\section{两个自变量的一阶线性偏微分方程组的特征理论}
1.解:特征线为$x=\int_0^ta(x,\tau)\diff\tau$\\\\
2.解:(1)令$x=\frac{1}{2}(y+z),t=\frac{1}{2}(y-z)$,则
\[u_y=u_xx_y+u_tt_y=\frac{1}{2}(u_x+u_t)=0\]
故$u=\Phi(z)=\Phi(x-t)$,令$t=0$得$\Phi(x)=\varphi(x)$,故解为$u=\varphi(x-t)$\\
(2)令$x=\frac{1}{2}(y+z),t=\frac{1}{2}(y-z)$,则
\[u_y=\frac{1}{2}(u_x+u_t)=\frac{1}{2}u\]
故$u=Ce^y+\Phi(z)=Ce^{x+t}+\Phi(x-t)$,令$t=0$得$Ce^x+\Phi(x)=\varphi(x)$,故解为
\[u=Ce^{x+t}-Ce^{x-t}+\varphi(x-t)\]
\newline
3.解:
\[A=\begin{pmatrix}
a&-b\\b&a
\end{pmatrix}\Rightarrow(\lambda-a)^2+b^2=0\]
故当$b=0$时为双曲型,当$b\neq0$时为椭圆型\\\\
4.解:\\
(1)\[A=\begin{pmatrix}
1+\sin x&2\\0&0\end{pmatrix}\Rightarrow\lambda=0\mbox{或}\lambda=1+\sin x\]
相应的特征向量为$(-2,1+\sin x)^T,(c,0)^T$,作变换
\[\begin{cases}
u=-2v_1+cv_2\\
v=(1+\sin x)v_1
\end{cases}\]
则得
\[\begin{cases}
(1+\sin x)\frac{\partial v_1}{\partial t}-2v_1+cv_2=0\\
c(1+\sin x)\frac{\partial v_2}{\partial t}+c(1+\sin x)^2\frac{\partial v_2}{\partial x}+x(1+\sin x)-4v_1+2cv_2=0
\end{cases}\]
(2)\[A=\begin{pmatrix}
x&1\\a^2&x
\end{pmatrix}\Rightarrow(\lambda-x)^2-a^2=0\Rightarrow\lambda=x\pm a\]
对应特征向量为$(-1,a)^T,(1,a)^T$,作变换
\[\begin{cases}
u=-v_1+v_2\\
v=av_1+av_2
\end{cases}\]
则得
\[\begin{cases}
\frac{\partial v_1}{\partial t}=(x-a)\frac{\partial v_1}{\partial x}\\
\frac{\partial v_2}{\partial t}=(x+a)\frac{\partial v_2}{\partial x}
\end{cases}\]
(3)\[A=
\begin{pmatrix}
6&5&0\\5&6&0\\-1&0&2
\end{pmatrix}\Rightarrow
\begin{vmatrix}
\lambda-6&-5&0\\-5&\lambda-6&0\\1&0&\lambda-2
\end{vmatrix}=0\Rightarrow(\lambda-2)(\lambda^2-12\lambda+11)=0\Rightarrow\lambda=1,2,11\]
对应特征向量为$(1,-1,1)^T,(0,0,1)^T,(9,9,-1)^T$,作变换
\[\begin{cases}
u_1=v_1+9v_3\\
u_2=-v_1+9v_3\\
u_3=v_1+v_2-v_3
\end{cases}\]
\[R=\begin{pmatrix}
1&0&9\\-1&0&9\\1&1&-1
\end{pmatrix}
\Rightarrow
R^{-1}=
\begin{pmatrix}
1/2&-1/2&0\\-4/9&5/9&1\\1/18&1/18&0
\end{pmatrix}\]
又因为
\[B=\begin{pmatrix}
0&0&0\\-2&0&0\\1&-2/3&-1
\end{pmatrix},C=\begin{pmatrix}
0\\0\\0
\end{pmatrix}\]
所以
\[R^{-1}BR=\begin{pmatrix}
1&0&9\\-4/9&-1&-6\\-1/9&0&-1
\end{pmatrix},R^{-1}C=\begin{pmatrix}
0\\0\\0
\end{pmatrix}\]
即得对角型方程组
\[\begin{cases}
\frac{\partial v_1}{\partial t}+\frac{\partial v_1}{\partial x}+v_1+9v_3=0\\
\frac{\partial v_2}{\partial t}+2\frac{\partial v_2}{\partial x}-\frac{4}{9}v_1-v_2-6v_3=0\\
\frac{\partial v_3}{\partial t}+11\frac{\partial v_3}{\partial x}-\frac{1}{9}v_1-v_3=0
\end{cases}\]\\\\
5.\textit{Proof}:原方程为
\[\frac{\partial U}{\partial t}+A\frac{\partial U}{\partial x}+BU+C=0\]
未知函数作可逆线性变换$U=RV$后,有
\[R\frac{\partial V}{\partial t}+AR\frac{\partial V}{\partial x}+\left(\frac{\partial R}{\partial t}+A\frac{\partial R}{\partial x}+BR\right)V+C=0\]
原特征方程为$\det(A-\lambda I)=0$,新的特征方程仍然不变,因此方程组在每一点的特征线方向保持不变
% \chapter{广义函数}
\section{基本空间}
1.\textit{Proof}:(反证法)若存在$x_0\in[a,b],s.t.f(x_0)\neq0$,不妨设$f(x_0)>0$,则由连续性知
\[\exists\delta>0,s.t.f(x)>0,\forall x\in(x_0-\delta,x_0+\delta)\cap[a,b]\]
考虑$[a,b]$内紧集$[x_0-\delta/2,x_0+\delta/2]\cap[a,b]$,知存在$\varphi\in C_0^{\infty}([a,b])$,使得$\varphi\geq0$且$\varphi$在该紧集上取1,在$(x_0-\delta,x_0+\delta)$之外取0.将$\varphi$限制在$(a,b)$上知$\varphi\in C_0^{\infty}((a,b))$,但是
\[\int f(x)\varphi(x)\diff x\geq\int_{x_0-\delta/2}^{x_0+\delta/2}f(x)\varphi(x)\diff x>0\]
矛盾,故假设不成立,故$f(x)\equiv0$\\
4.\textit{Proof}:因为
\[\begin{split}
\int|f_{\epsilon}(x)-f(x)|\diff x
&=\int\left|\int(f(x-\tau)-f(x))\varphi_{\epsilon}(\tau)\diff\tau\right|\diff x\\
&\leq\int\left|\int|f(x-\tau)-f(x)||\varphi_{\epsilon}(\tau)|\diff\tau\right|\diff x\\
&=\iint|f(x-\tau)-f(x)|\diff x|\varphi_{\epsilon}(\tau)|\diff\tau\\
&=\iint|f(x-\epsilon s)-f(x)|\diff x\frac{|\varphi(s)|}{\|\varphi\|_{L_1}}\diff s
\end{split}\]
又
\[\int|f(x-\epsilon s)-f(x)|\diff x\leq 2\|f\|_{L_1}\]
\[\lim_{\epsilon\to0}\int|f(x-\epsilon s)-f(x)|\diff x=0\]
由勒贝格控制收敛定理得$\lim_{\epsilon\to0}\|f_{\epsilon}(x)-f(x)\|_{L_1}=0$\\\\
5.\textit{Proof}:\\
以$y$代$x-ty$,则
\[\begin{split}
\partial_{x_i}(t^ku(x,t))
&=\partial_{x_i}\left(t^{k-1}\frac{1}{t^n}\int v(y)\varphi\left(\frac{x-y}{t}\right)\diff y\right)\\
&=t^{k-1}\int v(y)\partial_{x_i}\varphi\left(\frac{x-y}{t}\right)\diff y\\
&=t^{k-1}\int v(x-ty)\partial_{x_i}\varphi(y)\diff y
\end{split}\]
\[\begin{split}
\partial_t(t^ku(x,t))&=\partial_t\left(t^{k-1-n}\int v(y)\varphi\left(\frac{x-y}{t}\right)\diff y\right)\\
&=t^{k-1}\int v(x-ty)((k-n)\varphi(y)-\sum_iy_i\partial_{y_i}\varphi(y))\diff y
\end{split}\]
\newline
6.\textit{Proof}:因为
\[\partial^{\alpha}f_{\epsilon}(x)=\int\partial_x^{\alpha}\frac{1}{c\epsilon^n}\varphi\left(\frac{x-y}{t}\right)\diff y=\frac{1}{c\epsilon^n}\frac{1}{\epsilon^{|\alpha|}}\int\varphi^{(\alpha)}\left(\frac{x-y}{t}\right)\diff y\]
所以
\[|\partial^{\alpha}f_{\epsilon(x)}|\leq\frac{1}{c\epsilon^n}\frac{1}{\epsilon^{|\alpha|}}\int\left|\varphi^{(\alpha)}\left(\frac{x-y}{t}\right)\right|\diff y\]
可以命
\[C(\alpha,n)=\frac{1}{c\epsilon^n}\int\left|\varphi^{(\alpha)}\left(\frac{x-y}{t}\right)\right|\diff y=\frac{1}{c}\int\left|\varphi^{|\alpha|}{\xi}\right|\diff\xi\]
\section{广义函数及其基本运算}
1.\textit{Proof}:$\forall\varphi\in\mathscr{D}(\Omega)$,有
\[
\left\langle\frac{\partial^2u}{\partial x_i\partial x_j},\varphi\right\rangle=-\left\langle\frac{\partial u}{\partial x_i},\frac{\partial\varphi}{\partial x_j}\right\rangle
=\left\langle u,\frac{\partial^2\varphi}{\partial x_j\partial x_i}\right\rangle
=-\left\langle\frac{\partial u}{\partial x_j},\frac{\partial\varphi}{\partial x_i}\right\rangle
=\left\langle\frac{\partial^2u}{\partial x_j\partial x_i},\varphi\right\rangle
\]
故\[\frac{\partial^2u}{\partial x_i\partial x_j}=\frac{\partial^2u}{\partial x_j\partial x_i}\]
\newline
2.\textit{Proof}:$\forall\varphi\in\mathscr{D}(\Omega)$,有
\[\left\langle\frac{\partial}{\partial x_i}(au),\varphi\right\rangle=-\left\langle au,\frac{\partial\varphi}{\partial x_i}\right\rangle=-\left\langle u,a\frac{\partial\varphi}{\partial x_i}\right\rangle=\left\langle u,\varphi\frac{\partial a}{\partial x_i}-\frac{\partial(a\varphi)}{\partial x_i}\right\rangle\]
因为
\[-\left\langle u,\frac{\partial(a\varphi)}{\partial x_i}\right\rangle=\left\langle\frac{\partial u}{\partial x_i},a\varphi\right\rangle=\left\langle a\frac{\partial u}{\partial x_i},\varphi\right\rangle\]
所以
\[\left\langle\frac{\partial}{\partial x_i}(au),\varphi\right\rangle=\left\langle\frac{\partial a}{\partial x_i}u+a\frac{\partial u}{\partial x_i},\varphi\right\rangle\]
故
\[\frac{\partial}{\partial x_i}(au)=a\frac{\partial u}{\partial x_i}+\frac{\partial a}{\partial x_i}u\]
\newline
3.\textit{Proof}:\\
(1)$\forall\varphi\in\mathscr{D}(\Omega)$,有
\[\left\langle\check{\delta},\varphi\right\rangle=\left\langle\delta,\check{\varphi}\right\rangle=\check{\varphi}(0)=\varphi(0)=\left\langle\delta,\varphi\right\rangle\]
\[\left\langle\check{c},\varphi\right\rangle=\left\langle c,\check{\varphi}\right\rangle=\int c\varphi(-x)\diff x=\int c\varphi(x)\diff x=\left\langle c,\varphi\right\rangle\]
故$\delta$和$c$都是偶广义函数\\
(2)设$u$为偶广义函数,$\forall\varphi\in\mathscr{D}(\Omega)$,有
\[\left\langle\check{\left(\frac{\partial u}{\partial x_j}\right)},\varphi\right\rangle=\left\langle\frac{\partial u}{\partial x_j},\check{\varphi}\right\rangle=-\left\langle u,\frac{\partial\check{\varphi}}{\partial x_j}\right\rangle=\left\langle u,\frac{\partial\varphi}{\partial x_j}\right\rangle=-\left\langle\frac{\partial u}{\partial x_j},\varphi\right\rangle\]
故$\frac{\partial u}{\partial x_j}$为奇广义函数\\
(3)$\forall\varphi\in\mathscr{D}(\Omega)$,有
\[\left\langle\check{(u+\check{u})},\varphi\right\rangle\]

\end{document}