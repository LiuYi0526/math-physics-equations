\chapter{热传导方程}

\section{热传导方程及其定解问题的导出}
\begin{exercise}[3]
	混凝土内部储藏着热量,称为水化热,在混凝土浇筑后该热量逐渐放出,放热速度和它所储藏的水化热成正比.以$Q(t)$表示它在单位体积中所储的热量, $Q_0$为初始时刻所储藏的热量,则$\displaystyle\frac{\d Q}{\d t}=-\beta Q$,其中$\beta$为正常数.又假设混凝土的比热为$c$,密度为$\rho$,热传导系数为$k$,求它在浇筑后温度$u$所满足的方程.
\end{exercise}

\begin{solve}
	将水化热视为一热源,由$\displaystyle\frac {\d Q}{dt}=-\beta Q$及$Q|_{t=0}=Q_0$,得$Q(t)=Q_0\e^{-\beta t}.$
	
	由假设,放热速度为$Q_0\beta \e^{\beta t}$,它就是单位时间所产生的热量,因此,由书65页(1.7)式得
	$$\frac{\partial u}{\partial t}=a^2\left(\frac{\partial^2u}{\partial x^2}+\frac{\partial^2u}{\partial y^2}+\frac{\partial^2u}{\partial z^2}\right)+\frac{\beta Q_0}{c\rho}\e^{-\beta t},$$
	其中,$\displaystyle a^2=\frac k{c\rho}.$
\end{solve}


\begin{exercise}[4]
	设一均匀的导线处在周围为常数温度$u_{0}$的介质中,试证:在常电流作用下导线的温度满足微分方程
	$$\frac{\partial u}{\partial t}=\frac{k}{c\rho}\frac{\partial^2u}{\partial x^2}-\frac{k_1P}{c\rho\omega}(u-u_0)+\frac{0.24i^2r}{c\rho\omega},$$
	其中$i$和$r$分别表示导体的电流强度及电阻系数, $P$表示横截面的周长, $\omega$	表示横截面的面积,而$k_{1}$表示导线对于介质的热交换系数.
\end{exercise}

\begin{proof}
	问题可视为有热源的杆的热传导问题,因此由65页(1.7)和(1.8)知方程取形式为
	$$\frac{\partial u}{\partial t}=a^2\frac{\partial^2u}{\partial x^2}+f(x,t),$$
	其中$a^2= \frac k{c\rho}$, $f(x,t)=\frac{F(x,t)}{c\rho}$, $F(x,t)$为单位体积单位时间产生的热量.
	
	由常电流$i$所产生$F_1(x,t)$为$0.24i^2r\omega^2$.因为单位长度的电阻为$\displaystyle\frac r{\omega}$,因此电流$i$做功为$\displaystyle i^2\frac r\omega$乘上功热当量单位长度产生的热量为$0.24i^2r/\omega$,其中0.24为功热当量.
	
	因此单位体积时间所产生的热量为$0.24i^2r/w^2$.由常温度的热交换所产生的,从本节习题2看出为$\displaystyle-\frac{4k_1}l(u-u_0)$,其中$l$为系杆直径,固有$\displaystyle\frac p\omega=\frac{\pi l}{\pi l^2/4}=\frac 4l$,代入得
	\[F_2(x,t)=\frac{-k_1p}{\omega}(u-u_0).\]
	因热源可叠加,故有$F(x,t)=F_1(x,t)+F_2(x,t)$,将所得代入$\displaystyle\frac{\partial u}{\partial t}=a^2\frac{\partial^2u}{\partial x^2}+f(x,t)$即得所求
	\[\frac{\partial u}{\partial t}=\frac{k}{c\rho}\frac{\partial^2u}{\partial x^2}-\frac{k_1p}{c\rho w}(u-u_0)+\frac{0.24i^2r}{c\rho \omega^2}.\qedhere\]
\end{proof}


\begin{exercise}[5]
	设物体表面的绝对温度为$u$,此时它向外界辐射出去的热量依斯特藩-玻尔兹曼(Stefan-Boltzmann)定律正比于$u^4$,即
	$$dQ=\sigma u^4\d S\d t.$$
	假设物体和周围介质之间只有热辐射而没有热传导,又假设物体周围介质的绝对温度为已知函数$f(x,y,z,t)$,试写出此时该物体热传导问题的边界条件.
\end{exercise}

\begin{solve}
	由假设,边界只有热辐射的热量交换,辐射出去的热量为
	$$dQ_1=\sigma u^4\big|_{\partial\Omega}\d S\d t,$$
	辐射进来的热量为
	$$dQ_2=\sigma f^4|_{\partial\Omega}\d S\d t.$$
	因此由热量的传导定律得边界条件为
	\[k\frac{\partial u}{\partial n}\bigg|_{\partial\Omega}=\sigma\left(u^4|_{\partial\Omega}-f^4|_{\partial\Omega}\right).\qedhere\]
\end{solve}
\section{初边值问题的分离变量法}


\begin{exercise}
  用分离变量法求下列定解问题的解:
  \[\begin{cases}
    \frac{\partial u}{\partial t} = a^2 \frac{\partial^2u}{\partial x^2}
      \quad (t>0, 0<x<\pi), \\
    u(0,t) = \frac{\partial u}{\partial x}(\pi,t) = 0\quad (t>0), \\
    u(x,0) = f(x)\quad (0<x<\pi).
  \end{cases}\]
\end{exercise}

\begin{solution}
  利用分离变量法, 设 $u(x,t)=X(x)T(t)$, 则
  \[X(x)T'(t) = a^2X''(x)T(t).\]
  由此得
  \begin{align}
    X''(x)+\lambda X(x) & = 0, \label{eq:c1} \\
    T'(t)+a^2\lambda T(t) & = 0. \label{eq:c2} 
  \end{align}
  由 \eqref{eq:c1} 及 $X(x)$ 满足的边界条件 $X(0)=0$, $X'(\pi)=0$, 得
  \begin{enumerate}[(i)]
    \item 当 $\lambda\leq 0$ 时只有零解;
    \item 当 $\lambda>0$ 时, $X(x)=A\cos\sqrt{\lambda}x+B\sin\sqrt{\lambda}x$,
      代入边界条件得固有值为 $$\lambda_k=\left(\frac{2k+1}{2}\right)^2,$$
      相应的固有函数 $$X_k(x)=B_k\sin\sqrt{\lambda_k}x=B_k\sin\frac{2k+1}{2}x.$$
  \end{enumerate}
  将 $\lambda=\lambda_k$ 代入 \eqref{eq:c2} 得 $T_k(t)=C_k\e^{-a^2\lambda_kt}$, 故
  \[u_k(x,t) = A_k\e^{-a^2\lambda_kt}\sin\frac{2k+1}{2}x,\quad u(x,t)
    = \sum_{k=0}^{\infty}u_k(x,t).\]
  利用初始条件得
  \[f(x) = \sum_{k=0}^{\infty}A_k\sin\frac{2k+1}{2}x\Rightarrow A_k
    = \frac{2}{\pi}\int_0^{\pi}f(\xi)\sin\frac{2k+1}{2}\xi\diff\xi.\]
  因此原问题的解为
  \[u(x,t) = \sum_{k=0}^{\infty}\frac{2}{\pi}\int_0^{\pi} f(\xi)\sin\frac{2k+1}{2}
    \xi\diff\xi\cdot \e^{-a^2(\frac{2k+1}2)^2t}\sin\frac{2k+1}{2}x. \qedhere\]
\end{solution}


\begin{exercise}
  用分离变量法求解热传导方程的初边值问题:
  \[\begin{cases}
    \displaystyle \frac{\partial u}{\partial t} = \frac{\partial^2 u}{\partial x^2} \quad (t>0,\, 0<x<1), \\
    u(x,0) = \begin{cases}
               x,     & 0 < x \leq\frac12, \\
               1 - x, & \frac12 < x < 1,
             \end{cases} \\
    u(0,t) = u(1,t) = 0 \quad (t>0).
  \end{cases}\]
\end{exercise}

\begin{solution}
  令 $u(x,t) = X(x)T(t)$, 得
  \[ \frac{X''(x)}{X(x)} = \frac{T'(t)}{T(t)} = -\lambda. \]
  首先, 函数$X(x)$满足以下条件
  \begin{equation}\label{eq:c3}
    \begin{cases}
      X''(x) + \lambda X(x) = 0, \\
      X(0) = X(1) = 0.
    \end{cases}
  \end{equation}
  解得固有值和固有函数为 $\lambda_k = (k\pi)^2$ and $X_k(x) = A_k\sin k\pi x$.

  于是 $T(t)$ 满足方程
  \[ T_k(t) + (k\pi)^2T_k(t) = 0, \]
  解得 $T_k(t) = B_k \e^{-(k\pi)^2t}$. 所以 $u(x,t)$ 的解为
  \begin{equation}\label{eq:c4}
    u(x,t) = \sum_{k=1}^\infty C_k \e^{-(k\pi)^2t} \sin k\pi x.
  \end{equation}
  利用初始条件得
  \[ \sum_{k=1}^\infty C_k \sin k\pi x = f(x) = \begin{cases}
    x,     & 0 < x \leq\frac12, \\
    1 - x, & \frac12 < x < 1.
  \end{cases} \]
  因此
  \begin{equation}\label{eq:c5}
    C_k = 2 \int_0^1 f(x) \sin k\pi x \d x = \frac{4}{k^2\pi^2} \sin\frac{k\pi}{2}.
  \end{equation}
  代入 \eqref{eq:c4} 可得
  \[ u(x,t) = \sum_{k=1}^\infty \frac{4}{k^2\pi^2} \sin\frac{k\pi}{2}
  \e^{-k^2\pi^2t} \sin k\pi x. \qedhere \]
\end{solution}


\begin{exercise}
  如果有一根长度为 $l$ 的均匀细棒, 其周围以及两端 $x=0$, $x=l$
  均为绝热, 初始温度分布为 $u(x,0)=f(x)$, 问以后的温度分布如何?
  且证明当 $f(x)$ 等于常数 $u_0$ 时, 恒有 $u(x,t)=u_0$.
\end{exercise}

\begin{solution}
  因为细棒的两端均为绝热,故根据傅里叶定律知$u_x|_{x=0}=u_x|_{x=l}=0$,此初边值问题为
  \[\begin{cases}
    u_t = a^2u_{xx}, \\
    u_x|_{x=0} = u_x|_{x=l} = 0, \\
    u|_{t=0} = f(x).
  \end{cases}\]
  直接解得
  \[ u(x,t)=\sum_{k=0}^{\infty}D_k\e^{-a^2\lambda_kt}\cos\frac{k\pi}{l}x,
    \quad \lambda_k=\left(\frac{k\pi}{l}\right)^2, \]
  其中
  \[ D_0 = \frac{1}{l}\int_0^lf(\xi) \d\xi,
    \quad D_k=\frac{2}{l}\int_0^lf(\xi)\cos\frac{k\pi}{l}\xi \d\xi
    \quad (k\geq1).\]
  当 $f(x)\equiv u_0$ 时, $D_0=u_0$, $D_k=0$ $(k=1,2,\cdots)$, 故 $ u(x,t)=u_0$.
\end{solution}


\begin{exercise}
  在区域 $t>0$, $0<x<l$ 中求解如下的定解问题:
  \[\begin{cases}
    \displaystyle\frac{\partial u}{\partial t} = a^2 \frac{\partial^2u}{\partial x^2} - \beta(u-u_0), \\
    u(0,t) = u(l,t) = u_0, \\
    u(x,0) = f(x),
  \end{cases}\]
  其中 $a$, $\beta$, $u_0$ 均为常数, $f(x)$ 为已知函数.
\end{exercise}

\begin{solution}
  作变量代换,令 $v(x,t)=(u-u_0)\e^{\beta t}$,则$v(x,t)$满足的定解问题为:
  \[\begin{cases}
    v_t=a^2v_{xx}, \\
    v(0,t)=v(l,t)=0, \\
    v(x,0)=f(x)-u_0.
  \end{cases}\]
  直接解得
  \[ v(x,t) = \sum_{k=1}^{\infty}A_k\e^{-a^2\lambda_kt}\sin\frac{k\pi}{l}x,
    \quad \lambda_k = \left(\frac{k\pi}{l}\right)^2, \]
  其中
  \[ A_k = \frac{2}{l}\int_0^l(f(\xi)-u_0)\sin\frac{k\pi}{l}\xi\diff\xi. \]
  因此
  \[u(x,t) = u_0 + \sum_{k=1}^{\infty}\frac{2}{l}
    \int_0^l(f(\xi)-u_0)\sin\frac{k\pi}{l}\xi\diff\xi
      \cdot \e^{-\left(\frac{a^2k^2\pi^2}{l^2}+\beta\right)t}\sin\frac{k\pi}{l}x. \qedhere \]
\end{solution}


\begin{exercise}
  长度为 $l$ 的均匀细杆的初始温度为 \qty{0}{\degreeCelsius}, 端点 $x=0$ 保持恒温 $u_0$,
  而在 $x=l$ 和侧面上, 热量可以发散到周围的介质中去, 介质的温度为 \qty{0}{\degreeCelsius},
  此时杆上的温度分布函数 $u(x,t)$ 满足下述定解问题:
  \[\begin{cases}
    \frac{\partial u}{\partial t} = a^2 \frac{\partial^2 u}{\partial x^2} - b^2 u, \\
    u(0,t) = u_0, \quad \Bigl(\frac{\partial u}{\partial x} + Hu\Bigr)\Big|_{x=l} = 0, \\
    u(x,0) = 0,
  \end{cases}\]
  其中 $a$, $b$, $H$ 均为常数, 试求出 $u(x,t)$.
\end{exercise}

\begin{solution}
    Let $u(x,t) = u_1(x,t) + u_2(x)$ in which $u_1$ satisfies the following 
    problem with homogeneous boundary condition
    \begin{equation}\label{eq:c6}
      \begin{cases}
        \frac{\partial u_1}{\partial t} = a^2 \frac{\partial^2 u_1}{\partial x^2} - b^2u_1, \\
        u_1(0,t) = \bigl(\frac{\partial u_1}{\partial x} + Hu_1\bigr)\big|_{x=l} = 0, \\
        u_1(x,0) = -u_2,
      \end{cases}
    \end{equation}
    and $u_2$ satisfies the following ordinary differential equation
    \begin{equation}\label{eq:c7}
      \begin{cases}
        a^2 \frac{\d^2 u_2}{\d x^2} - b^2 u_2 = 0, \\
        u_2(0) = u_0, \\
        \bigl(\frac{\d u_2}{\d x} + Hu_2\bigr)\big|_{x=l} = 0.
      \end{cases}
    \end{equation}
    We first solve equation \eqref{eq:c7}. Since $a^2 \frac{\d^2 u_2}{\d x^2} - b^2 u_2 = 0$,
    we have that
    \begin{equation}\label{eq:c8}
      u_2 = C_1 \e^{\frac{b}{a}x} + C_2 \e^{-\frac{b}{a}x}.
    \end{equation}
    By the boundary condition we get
    \[\begin{cases}
      C_1 + C_2 = 0, \\
      C_1\frac{b}{a}\e^{\frac{b}{a}l} - C_2 \frac{b}{a} \e^{-\frac{b}{a}l}
        + H C_1 \e^{\frac{b}{a}l} + H C_2 e^{-\frac{b}{a}l} = 0.
    \end{cases}\]
    Solving $C_1$, $C_2$ and substituting them into \eqref{eq:c8}, we have
    \begin{equation}\label{eq:c9}
      u_2 = u_0 \cosh \frac{b}{a}x - \frac{H\cosh\frac{b}{a}l + \frac{b}{a}\sinh\frac{b}{a}l}
        {H\sinh\frac{b}{a}l + \frac{b}{a}\cosh\frac{b}{a}l}
        u_0 \sinh\frac{b}{a}x.
    \end{equation}

    Now we solve equation \eqref{eq:c6}. Let $v = \e^{b^2 t}u_1$, then $v$ satisfies
    \begin{equation}\label{eq:c10}
      \begin{cases}
        \frac{\partial v}{\partial t} = a^2 \frac{\partial^2 v}{\partial x^2}, \\
        v(0,t) = \bigl(\frac{\partial v}{\partial x} + Hv\bigr)\big|_{x=l} = 0, \\
        v(x,0) = -u_2.
      \end{cases}      
    \end{equation}
    The procedure of solving this problem is actually the same as that in the textbook
    from Page 51 to 54.
    Denote $v(x,t) = X(x)T(t)$, then
    \[ \frac{X''(x)}{X(x)} = \frac{T'(t)}{a^2 T(t)} = -\lambda \]
    for some constant $\lambda$. First of all, $X(x)$ satisfies
    \begin{equation}\label{eq:c11}
      \begin{cases}
        X''(x) + \lambda X(x) = 0, \\
        X(0) = X'(l) + HX(l) = 0.
      \end{cases}
    \end{equation}
    \begin{itemize}
      \item If $\lambda\leq 0$, there only exists trivial solution $X\equiv 0$;
      \item If $\lambda>0$,
        \begin{equation}\label{eq:c12}
          X(x) = A\cos\sqrt{\lambda}x + B\sin\sqrt{\lambda}x.
        \end{equation}
        Combining with the boundary conditions we have
        \begin{equation}\label{eq:c13}
          X_k(x) = B_k \sin\sqrt{\lambda_k} x,
        \end{equation}
        where $(\lambda_k)_{k\geq 1}$ is the sequence of positive solutions to
        $\sqrt\lambda + H\tan\sqrt\lambda l = 0$.
    \end{itemize}

    On the other hand, $T(t)$ satisfies
    \begin{equation}\label{eq:c14}
      T'(t) + a^2\lambda_k T(t) = 0,
    \end{equation}
    from which we get
    \begin{equation}\label{eq:c15}
      T_k(t) = C_k \e^{-a^2\lambda_k t}.
    \end{equation}
    Therefore according to \eqref{eq:c13} and \eqref{eq:c15} we can write $v(x,t)$ as
    \begin{equation}\label{eq:c16}
      v(x,t) = \sum_{k=1}^\infty A_k e^{-a^2\lambda_k t} \sin\sqrt{\lambda_k} x.
    \end{equation}
    Finally we need to utilize the initial value condition to get
    \begin{equation}\label{eq:c17}
      v(x,0) = \sum_{k=1}^\infty A_k \sin\sqrt{\lambda_k} x = -u_0.
    \end{equation}
    Since
    \begin{equation}\label{eq:c18}
      \bigl(\sin\sqrt{\lambda_m}x, \sin\sqrt{\lambda_n}x\bigr)_{L^2}
       = \delta_{mn} \biggl(\frac{l}{2} + \frac{H}{2(H^2+\lambda_m)}\biggr)
       =: \delta_{mn} \alpha_m,
    \end{equation}
    we have
    \begin{equation}\label{eq:c19}
      A_k = -\frac{1}{\alpha_k} \int_0^l u_0(x) \sin\sqrt{\lambda_k} x \d x.
    \end{equation}
    Hence
    \begin{equation}\label{eq:c20}
      v(x,t) = -\sum_{k=1}^\infty \frac{1}{\alpha_k}
        \int_0^l u_0(x) \sin\sqrt{\lambda_k}x \d x \cdot
        \e^{-a^2\lambda_k t} \sin\sqrt{\lambda_k} x
    \end{equation}
    and
    \begin{equation}\label{eq:c21}
      u_1(x,t) = -\sum_{k=1}^\infty \frac{1}{\alpha_k}
        \int_0^l u_0(x) \sin\sqrt{\lambda_k}x \d x \cdot
        \e^{-(a^2\lambda_k + b^2)t} \sin\sqrt{\lambda_k} x
    \end{equation}
    Finally we conclude that
    \begin{equation}\label{eq:c22}
      u(x,t) = u_1(x,t) + u_2(x),
    \end{equation}
    where $u_1$ and $u_2$ are given by \eqref{eq:c21} and \eqref{eq:c9}
    respectively.
\end{solution}


\begin{exercise}
	半径为$a$的半圆形平板,其表面绝热,在板的圆周边界上保持常温$u_0$,而在直径边界上保持常温$u_1$,求圆板稳恒状态(即与时间$t$无关的状态)的温度分布.
\end{exercise}

\begin{solve}
	引入极坐标,求稳恒状态的温度分布化为解定解问题
	$$\begin{aligned}
		&u_{rr}+\frac{1}{r}u_{r}+\frac{1}{r^{2}}u_{\theta\theta}=0,\quad0<r<a,\quad0\leq\theta\leq\pi,\\
		&=u_{1},\quad u|_{\theta=\pi}=u_{1},\\
		&u|_{r=a}=u_{0},\quad u|_{r=0}=C;
	\end{aligned}$$
	引入变量代换$u=u_1+v(r,\theta)$,则$v(r,\theta)$满足
	$$\begin{aligned}
		&v_{rr}+\frac{1}{r}v_{r}+\frac{1}{r^{2}}v_{\theta\theta}=0,\\
		&v|_{\theta=0}=0,\quad v|_{\theta=\pi}=0,\\
		&v|_{r=a}=u_{0}-u_{1},\quad v|_{r=0}=C.
	\end{aligned}$$
	设$v=R(r)\Phi(\theta)$代入方程及边值得
	$$R''+\frac{1}{r}R'\Phi+\frac{1}{r^{2}}R\Phi''=0,$$
	移项后得
	$$\left\{\begin{array}{l}
		\varPhi''+\lambda\varPhi=0,\quad\varPhi(0)=\varPhi(\pi)=0,\\
		r^2R''+rR'-\lambda R=0.
	\end{array}\right.$$
	直接计算得固有值与固有函数分别为
	$$\lambda_n=n^2,\quad\varPhi(\theta)=\sin n\theta,\quad n=1,2...,$$
	对应的$R$满足方程
	$$r^2R''+rR'-n^2R=0.$$
	设$R=r^a$代入得$a= \pm n$, 即$R= r^n$,或$R=r^{-n}$为两个线性无关的特解
	因此通解为
	$$R_n(r)=C_nr^n+D_nr^{-n},$$
	由自然边界条件$v|_r=0$有界知$R(r)$在$r=0$处有界. 因此, $D_n=0$,那么,
	$$v(r,\theta)=\sum_{n=0}^\infty C_nr^n\sin n\theta.$$
	再由边界条件得
	$$C_n=\frac{2}{\pi a^n}\int_0^\pi(u_0-u_1)\sin n\theta d\theta=\frac{2u_0-2u_1}{n\pi a^n}\left(1-(-1)^n\right),$$
	所以
	\[u(r,\theta)=u_1+\sum_{n=0}^\infty\frac{2r^n(u_0-u_1)}{n\pi a^n}\left[1-(-1)^n\right]\sin n\theta.\qedhere\]
\end{solve}

\begin{exercise}
	设$u(x,t)$是$(0,\frac\pi2)\times(0,\infty)$中初边值问题
	$$\begin{cases}&u_t=u_{xx}\\&u(0,t)=1,\quad u(\frac{\pi}{2},t)=4\\&u(x,0)=\cos^4x+4\sin^5x\end{cases}$$
	的解,求$\lim_t\to\infty u(x,t)$
\end{exercise}

\begin{solve}
	作代换$v(x,t)=u(x,t)-(\frac{6}{\pi}x+1)$
	则原问题化为
	$$\begin{cases}&v_t=v_{xx}\\&v(0,t)=v(\frac{\pi}{2},t)=0\\&v(x,0)=\cos^4x+4\sin^5x-\frac{6}{\pi}x-1\end{cases}$$
	由分离变量法,设$v(x,t)=X(x)T(t)$, 其中$X^{\prime\prime}+\lambda X=0, T^{\prime}+\lambda T=0$, $\lambda$为常数.
	
	当$\lambda\leq0$时, $X(x)$只有零解.
	
	当$\lambda>0$时, $X(x)=C_1\cos\sqrt{\lambda}x+C_2\sin\sqrt{\lambda}x$.
	代入边界条件, 解得
	$$\lambda_{k}=(2k+1)^{2}, k=0,1,2,\ldots$$
	则$X(x)=C_{2}\sin{(2k+1)}x$.
	将$\lambda_k$代入$T^\prime+\lambda T=0$, 解得相应的$T_k(t)$. 则
	\[v(x,t)=\sum_{k=0}^{\infty}A_{k}\e^{-(2k+1)^{2}t}\sin{(2k+1)x}\]
	代入初值条件,得
	$$\sum_{k=0}^\infty A_k\sin{(2k+1)x}=\cos^4x+4\sin^5x-\frac6\pi x-1=f(x)$$
	
	其中$A_k$由$f(x)$的Fourier展开式决定:
	\[A_{k} =\sqrt{\frac4\pi}\left\langle f(x),\sqrt{\frac4\pi}\sin{(2k+1)x}\right\rangle =\frac4\pi\int_0^{\frac\pi2}f(x)\sin{(2k+1)x}\d x\]
	故
	\[\begin{aligned}&&u(x,t)& =v(x,t)+\frac6\pi x+1 \\
	&&&=\frac6\pi x+1+\sum_{k=0}^\infty\frac4\pi\int_0^{\frac\pi2}f(x)\sin{(2k+1)x}\d x\cdot \e^{-(2k+1)^2t}\sin{(2k+1)x},
	\end{aligned}\]
	则
	\[\lim_{t\to\infty}u(x,t)=\frac{6}{\pi}x+1.\qedhere\]
\end{solve}
\section{柯西问题}

\begin{note}[性质3 (频域定理)的证明]
	设$F_1(x)=F[f_1(x)],F_2(x)=F[f_2(x)]$,则
	\[\begin{aligned}
		F^{-1}[F_1*F_2]&=F^{-1}\left[\int_{-\infty}^{\infty}F_1(x-t)F_2(t)\d t\right]\\
		&=\frac{1}{2\pi}\int_{-\infty}^{\infty}\e^{\upi\lambda x}\d\lambda\int_{-\infty}^{\infty}F_1(\lambda-t)F_2(t)\d t\\
		&=\frac{1}{2\pi}\int_{-\infty}^{\infty}F_2(t)\e^{\upi tx}\d t\int_{-\infty}^{\infty}F_1(\lambda-t)\e^{\upi(\lambda-t)x}\d\lambda\\
		&=\frac{1}{2\pi}\int_{-\infty}^{\infty}F_2(t)\e^{\upi tx}\d t\int_{-\infty}^{\infty}F_1(\xi)\e^{\upi\xi x}\d\xi\\
		&=2\pi f_1(x)f_2(x),
	\end{aligned}
	\]
	即
	\[F[f_1\cdot f_2]=\frac{1}{2\pi}F[f_1]*F[f_2].\]
\end{note}
\begin{exercise}
  求下列函数的 Fourier 变换:
  \begin{enumerate}[(1),left=\parindent]
    \item $\displaystyle\e^{-\eta x^2}\quad(\eta>0)$;
    \item $\displaystyle\e^{-a|x|}\quad(a>0)$;
    \item $\displaystyle\frac{x}{(a^2+x^2)^k},\quad\displaystyle\frac{1}{(a^2+x^2)^k}\quad$ ($a>0$, $k$ 为自然数).
  \end{enumerate}
\end{exercise}

\begin{solution}
  (1) 直接计算得
  \begin{align*}
    \widehat{\e^{-\eta x^2}}(\xi)
    & = \int_{-\infty}^{\infty} \e^{-\eta x^2}\cdot \e^{-\upi x\xi}\diff x
      = \e^{-\frac{\xi^2}{4\eta}} \int_{-\infty}^{\infty}
      \e^{-\eta\left(\xi+\frac{\upi\xi}{2\eta}\right)^2} \diff x \\
    & = \e^{-\frac{\xi^2}{4\eta}} \int_{-\infty}^{\infty}
      \e^{-y^2} \frac{1}{\sqrt{\eta}} \diff y
      = \biggl(\frac{\pi}{\eta}\biggr)^{1/2} \e^{-\frac{\xi^2}{4\eta}}.
    \end{align*}

  (2) 直接计算得
  \begin{align*}
    \widehat{\e^{-a|x|}} (\xi)
    & = \int_{-\infty}^{\infty} \e^{-a|x|}\cdot\e^{-\upi x\xi} \diff x \\
    & = 2\int_0^{\infty} \e^{-ax} \cos\xi x \diff x = \frac{2a}{a^2+\xi^2}.
  \end{align*}
  
	(3) $\displaystyle f(x)=\frac 1{(a^{2}+ x^{2})^{k}}$的Fourier变换为
	$$\tilde{f}(\lambda)=\int_{-\infty}^\infty\frac{\e^{-\upi\lambda x}}{(a^2+x^2)^k}\d x.$$
	将这个积分视为复平面上的积分,设$\displaystyle g(z)=\frac{\e^{-\upi\lambda z}}{(a^2+z^2)^k}$,  则$\tilde{f}(\lambda)$即为$g(z)$在实轴$\varGamma$上的积分.不妨设复平面中的上半平面$\varOmega$为实轴$\varGamma$所围成的区域,容易发现$g(z)$在$\varOmega$中有奇点$z=a\upi$.
  
	由留数定理,
  	$$\tilde{f}(\lambda)=\int_{\Gamma}g(z)\mathrm{d}z=2\pi\upi\operatorname*{Res}_{z=a\upi}g(z).$$
  	又奇点$z=a\upi$为$k$阶奇点,设
  	$$g(z)=\frac{\frac{e^{-\upi\lambda z}}{(z+a\upi)^k}}{(z-a\upi)^k}=\frac{\varphi(z)}{(z-a\upi)^k},$$
  	则
  	$$\begin{aligned}
  	\operatorname*{Res}_{z=a\upi}g(z)&=\frac{\varphi^{(k-1)}(a\upi)}{(k-1)!} \\
  	&=\frac1{(k-1)!}\left[\sum_{m=0}^{k-1}C_{k-1}^m\left((z+a\upi)^{-k}\right)^{(m)}\cdot\left(\e^{-\upi\lambda z}\right)^{(k-m-1)}\right]\Bigg|_{z=a\upi} \\
  	&=\frac1{(k-1)!}\left[\sum_{m=0}^{k-1}C_{k-1}^m\left(\frac{(-1)^m(k+m-1)!}{(k-1)!(2a\upi)^{k+m}}\right)\left((-\upi\lambda)^{k-m-1}\cdot\e^{a\lambda}\right)\right] \\
  	&=\frac1{(k-1)!}\sum_{m=0}^{k-1}\frac{(k+m-1)!(-1)^{k-m-1}}{\upi m!(k-m-1)!(2a)^{k+m}}\cdot\lambda^{k-m-1}\e^{a\lambda}.
  	\end{aligned}$$
  	则
  	$$\tilde{f}(\lambda)=\frac{2\pi\upi}{(k-1)!}\sum_{m=0}^{k-1}\frac{(k+m-1)!(-1)^{k-m-1}}{m!(k-m-1)!\upi(2a)^{k+m}}\cdot\lambda^{k-m-1}\e^{a\lambda}.$$
  	类似地,
  	$$\begin{aligned}
  		F\left[\frac x{(a^2+x^2)^k}\right]
  		&=-\frac{1}{\upi}\frac \d{\d\lambda}F\left[\frac1{(a^2+x^2)^k}\right] \\
  		&=-\frac{2\pi}{(k-1)!}\sum_{m=0}^{k-2}\frac{(k+m-1)!(-1)^{k-m-1}}{m!(k-m-1)!(2a)^{k+m}} \\
  		&\times\left[(k-m-1)\lambda^{k-m-2}\e^{a\lambda}+a\lambda^{k-m-1}\e^{a\lambda}\right]-\frac{\pi(2k-2)!}{[(k-1)!]^2}(2a)^{-2k+2}\e^{a\lambda} \\
  		&=-\frac{\pi\upi(2k-2)!}{\left[(k-1)!\right]^2}(2a)^{-2k+2}\e^{a\lambda}-\frac{2\pi}{(k-1)!}\sum_{m=0}^{k-2}\frac{(k+m-1)!(-1)^{k-m-1}}{m!(k-m-1)!(2a)^{k+m}} \\
  		&\times(2a)^{-k-m}(-1)^{k-m-1}\lambda^{k-m-2}\e^{a\lambda}(a\lambda+k-m-1).
  	\end{aligned}$$
\end{solution}


\begin{exercise}
  证明: 当 $f(x)$ 在 $(-\infty,\infty)$ 上绝对可积时, $F[f]$ 为连续函数.
\end{exercise}

\begin{proof}
  记 $\displaystyle F[f]=\int_{-\infty}^{\infty} f(\xi)\e^{-\upi\lambda\xi}\diff\xi = \tilde{f}(\lambda)$,则
  \[\begin{split}
    \bigl|\tilde{f}(\lambda+h)-\tilde{f}(\lambda)\bigr|
    & = \left|\int_{-\infty}^{\infty}f(\xi)
        \left(\e^{-\upi(\lambda+h)\xi} - \e^{-\upi\lambda\xi}\right)\diff\xi\right| \\
    & \leq \int_{-\infty}^{\infty}|f(\xi)|\cdot|\e^{-\upi h\xi}-1|\diff\xi\to 0
      \quad(\text{as } h\to 0),
  \end{split}\]
  故 $F[f]$ 为连续函数.
\end{proof}


\begin{exercise}
	用傅里叶变换求解三维热传导方程的柯西问题
	\[\begin{cases}
		\displaystyle\frac{\partial u}{\partial t} = a^2 \Bigl(\frac{\partial^2 u}{\partial x^2}+\frac{\partial^2 u}{\partial y^2}+\frac{\partial^2 u}{\partial z^2}\Bigr), \\
		u|_{t=0} = \varphi(x,y,z).
	\end{cases}\]
\end{exercise}

\begin{solve}
	对齐次方程及初始条件均作用关于 $x,y,z$ 的Fourier变换,记 $$F[u(x,y,z,t)]=\tilde{u}(\lambda,\mu,\nu,t),$$
	$$F[\varphi(x,y,z)]=\tilde{\varphi}(\lambda,\mu,\nu).$$
	对方程两边关于$x,y,z$进行Fourier变换,就得到
	$$\frac{\d\widetilde{u}}{\d t} = -a^{2}(\lambda^2+\mu^2+\nu^2)\widetilde{u},$$
	类似地,
	$$\tilde{u}(\lambda,\mu,\nu,0) = \tilde{\varphi}(\lambda,\mu,\nu,0).$$
	这是带参数$\lambda,\mu,\nu$的常微分方程的Cauthy问题,它的解为
	$$\tilde{u}(\lambda,\mu,\nu,t)=\tilde{\varphi}\e^{-a^2(\lambda^2+\mu^2+\nu^2)t}.$$
	函数$\e^{-a^2(\lambda^2+\mu^2+\nu^2)t}$的Fourier逆变换为
	\begin{align*}
		F^{-1}\left[\e^{-a^2(\lambda^2+\mu^2+\nu^2)t}\right]&=\left(\frac{1}{2\pi}\right)^3{\iiint}_{\mathbb{R}^3}\e^{-a^2\left(\lambda^2+\mu^2+\nu^2\right)t}\cdot\e^{\upi(\lambda x+\mu y+\nu z)}\d\lambda\d\mu\d\nu\\
		&= {\left(\frac{1}{2\pi} \right)}^{3}\e^{{-}\frac{x^2+y^2+z^2}{4a^2t}}{\left( \sqrt{\frac{\pi}{a^2t}} \right)}^3\\
		&= {\left(\frac{1}{2a\sqrt{\pi t}}\right)}^3\e^{-\frac{x^2+y^2+z^2}{4a^2t}}.
	\end{align*}
	故可得Cauthy问题的解为
	\[u={\left(\frac{1}{2a\sqrt{\pi t}} \right)}^3{{\iiint}}_{{\mathbb{R}}^{3}}\varphi(\xi,\eta,\zeta)\e^{-\frac{{(x {-} \xi)}^{2} + {(y-\eta)}^2+{(z-\zeta)}^2}{4a^2t}}\d\zeta\d\eta\d\zeta.\qedhere\]
\end{solve}

\begin{exercise}[4]
  证明 (3.29) 所表示的函数满足非齐次方程 (3.15) 以及初始条件 (3.16).
\end{exercise}

\begin{proof}
  It suffices to verify that $u(x,t)$ given by (3.28) is the solution to
  the Cauchy problem (3.23)--(3.24). Denote the heat kernel by
  \[ \varPhi(x,t) = \frac{1}{2a\sqrt{\pi t}} \e^{-\frac{x^2}{4a^2t}}, \]
  which satisfies $\partial_t\varPhi = a^2\partial_{xx}\varPhi$. Then
  the $u(x,t)$ in (3.28) can be rewritten as
  \begin{equation}\label{eq:d1}
    u(x,t) = \int_0^t\int_{-\infty}^\infty \varPhi(x-\xi, t-\tau) f(\xi,\tau) \d\xi\d\tau.
  \end{equation}
  Making a change of variables, we have
  \begin{equation}\label{eq:d2}
    u(x,t) = \int_0^t\int_{-\infty}^\infty \varPhi(\xi,\tau) f(x-\xi,t-\tau) \d\xi\d\tau.
  \end{equation}
  Hence
  \begin{align}
    \frac{\partial u}{\partial t}
    & = \int_0^t\int_{-\infty}^\infty \varPhi(\xi,\tau) f_t(x-\xi,t-\tau) \d\xi\d\tau \notag \\
    &   \quad + \int_{-\infty}^\infty \varPhi(\xi,t)f(x-\xi,0) \d\xi, \\
    \frac{\partial^2u}{\partial x^2}
    & = \int_0^t \int_{-\infty}^\infty \varPhi(\xi,\tau) f_{xx}(x-\xi,t-\tau) \d\xi\d\tau.
  \end{align}
  So
  \begin{align*}
    \frac{\partial u}{\partial t} - a^2\frac{\partial^2u}{\partial x^2}
    & = \int_0^t \int_{-\infty}^\infty \varPhi(\xi,\tau) (\partial_t - a^2\partial_{xx})
        f(x-\xi,t-\tau) \d\xi\d\tau \\
    &   \quad + \int_{-\infty}^\infty \varPhi(\xi,t) f(x-\xi,0) \d\xi \\
    & = \int_0^\epsilon \int_{-\infty}^\infty \varPhi(\xi,\tau) (\partial_t - a^2\partial_{xx})
        f(x-\xi,t-\tau) \d\xi\d\tau \\
    &   \quad + \int_\epsilon^t \int_{-\infty}^\infty \varPhi(\xi,\tau) (-\partial_\tau - a^2\partial_{\xi\xi})
        f(x-\xi,t-\tau) \d\xi\d\tau \\
    &   \quad + \int_{-\infty}^\infty \varPhi(\xi,t) f(x-\xi,0) \d\xi \\
    & =: I_\epsilon + J_\epsilon + K.
  \end{align*}
  We suppose that $f$ is $C^1$ in $t$ and $C^2$ in $x$ and has compact support.
  Then
  \begin{equation}\label{eq:d3}
    |I_\epsilon| \leq \bigl(\|f_t\|_\infty + a^2 \|f_{xx}\|_\infty\bigr)
      \int_0^\epsilon \int_{-\infty}^\infty \varPhi(\xi,\tau) \d\xi\d\tau \leq C\epsilon.
  \end{equation}
  On the other hand, integrate by parts to get
  \begin{equation}\label{eq:d4}
    \begin{aligned}
      J_\epsilon
      & = \int_\epsilon^t \int_{-\infty}^\infty (\partial_\tau - a^2\partial_{\xi\xi}) \varPhi(\xi,\tau)
          f(x-\xi,t-\tau) \d\xi\d\tau \\
      &   \quad + \int_{-\infty}^\infty \varPhi(\xi,\epsilon) f(x-\xi,t-\epsilon) \d\xi \\
      &   \quad - \int_{-\infty}^\infty \varPhi(\xi,t) f(x-\xi,0) \d\xi \\
      & = \int_{-\infty}^\infty \varPhi(\xi,\epsilon) f(x-\xi,t-\epsilon) \d\xi - K.
    \end{aligned}
  \end{equation}
  Therefore
  \begin{equation}\label{eq:d5}
    \frac{\partial u}{\partial t} - a^2\frac{\partial^2u}{\partial x^2}
      = \lim_{\epsilon\to 0} \int_{-\infty}^\infty \varPhi(\xi,\epsilon) f(x-\xi,t-\epsilon) \d\xi
      = f(x,t).
  \end{equation}
  The initial value condition (3.24) is obvious.
\end{proof}


\begin{exercise}[5]
  求解热传导方程 (3.17) 的柯西问题, 已知
  \begin{enumerate}[(1)]
    \item $u|_{t=0} = \sin x$,
    \item 用延拓法求解半有界直线上的热传导方程 (3.17), 假设
      \[\begin{cases}
        u(x,0) = \varphi(x)\quad (0<x<\infty), \\
        u(0,t) = 0.
      \end{cases}\]
  \end{enumerate}
\end{exercise}

\begin{solution}
  (1) 由泊松公式知
  \[\begin{split}
    u(x,t)
    & = \frac{1}{2a\sqrt{\pi t}} \int_{-\infty}^{\infty}
        \sin\xi\cdot \e^{-\frac{(x-\xi)^2}{4a^2t}}\diff\xi \\
    & = \frac{1}{2a\sqrt{\pi t}}\cdot 2a\sqrt{t}\int_{-\infty}^{\infty}
        \sin(x-2a\sqrt{t}\zeta)\cdot \e^{-\zeta^2}\diff\zeta \\
    & = \frac{1}{\sqrt{\pi}} \int_{-\infty}^{\infty}
        \sin x\cos 2a\sqrt{t}\zeta\cdot \e^{-\zeta^2}\diff\zeta \\
    & = \frac{2\sin x}{\sqrt{\pi}} \int_0^{\infty}
        \cos 2a\sqrt{t}\zeta\cdot \e^{-\zeta^2}\diff\zeta \\
    & = \e^{-a^2t}\sin x.
  \end{split}\]

  (2) 对 $\varphi(x)$ 作奇延拓, 即
  \[\varPhi(x) = \begin{cases}
    \varphi(x),   & x\geq 0, \\
    -\varphi(-x), & x<0.
  \end{cases}\]
  求解如下 Cauchy 问题
  \[\begin{cases}
    u_t = a^2 u_{xx}, \\
    u|_{t=0} = \varPhi(x).
  \end{cases}\]
  得
  \begin{align*}
    u(x,t)
    & = \frac{1}{2a\sqrt{\pi t}} \int_{-\infty}^{\infty}
        \varPhi(\xi) \e^{-\frac{(x-\xi)^2}{4a^2t}} \diff\xi \\
    & = \frac{1}{2a\sqrt{\pi t}} \biggl[
        \int_0^{\infty} \varphi(\xi) \e^{-\frac{(x-\xi)^2}{4a^2t}} \diff\xi
        + \int_{-\infty}^0 -\varphi(-\xi) \e^{-\frac{(x-\xi)^2}{4a^2t}}\diff\xi\biggr] \\
    & = \frac{1}{a\sqrt{\pi t}} \int_0^{\infty} \varphi(\xi)
        \e^{-\frac{x^2+\xi^2}{4a^2 t}}\sinh \frac{x\xi}{2a^2 t}\diff\xi. \qedhere
  \end{align*}
\end{solution}


\begin{exercise}
	证明函数
	$$v(x,y,t;\xi,\eta,\tau)=\frac{1}{4\pi a^2(t-\tau)}\e^{-\frac{(x-\xi)^2+(y-\eta)^2}{4a^2(t-\tau)}}$$
	对于变量$(x,y,t)$满足方程
	$$\dfrac{\partial v}{\partial t}=a^2\left(\dfrac{\partial^2v}{\partial x^2}+\dfrac{\partial^2v}{\partial y^2}\right),$$
	而对于变量$(\xi,\eta,\tau)$满足方程
	$$\frac{\partial v}{\partial\tau}+a^2\left(\frac{\partial^2v}{\partial\xi^2}+\frac{\partial^2v}{\partial\eta^2}\right)=0.$$
\end{exercise}

\begin{proof}
	直接验证即可.因为
	$$\frac{\partial v}{\partial t}=\frac{1}{4\pi a}\left[-\frac{1}{(t-\tau)^{2}}+\frac{(x-\xi)^{2}+(y-\eta)^{2}}{4a^{2}(t-\tau)^{3}}\right]\e^{-\frac{(x-\xi)^{2}+(y-\eta)^{2}}{4a^{2}(t-\tau)}},$$
	$$\frac{\partial v}{\partial x}=\frac{1}{4\pi a^{2}}\frac{-2(x-\xi)}{4a^{2}(t-\tau)^{2}}e^{-\frac{(x-\xi)^{2}+(y-\eta)^{2}}{4a^{2}(t-\tau)}},$$
	$$\frac{\partial^{2}v}{\partial x^{2}}=\frac{1}{4\pi a^{2}}\frac{1}{(t-\tau)^{2}}\left[\frac{-1}{2a^{2}}+\frac{(x-\xi)^{2}}{4a^{4}(t-\tau)}\right]e^{-\frac{(x-\xi)^{2}+(y-\eta)^{2}}{4a^{2}(t-\tau)}}.$$
	同理
	$$\frac{\partial^2v}{\partial y^2}=\frac{1}{4\pi a^2}\frac{1}{(t-\tau)^2}\left[\frac{-1}{2a^2}+\frac{(y-\eta)^2}{4a^4(t-\tau)}\right]e^{-\frac{(x-\xi)^2+(y-\eta)^2}{4a^2(t-\tau)}},$$
	所以
	$$\frac{\partial^2v}{\partial x^2}+\frac{\partial^2v}{\partial y^2}=\frac{1}{4\pi a^2}\left[-\frac{1}{a^2(t-\tau)^2}+\frac{(x-\xi)^2+(y-\eta)^2}{4a^4(t-\tau)^3}\right]e^{-\frac{(x-\xi)^2+(y-\eta)^2}{4a^2(t-\tau)}}=\frac{1}{a^2}\frac{\partial v}{\partial t}.$$
	类似地,
	$$\dfrac{\partial v}{\partial t}=-\dfrac{\partial v}{\partial\tau},\quad\dfrac{\partial v}{\partial\xi}=-\dfrac{\partial v}{\partial x},\quad\dfrac{\partial^2v}{\partial\xi^2}=\dfrac{\partial^2v}{\partial x^2},\quad\dfrac{\partial^2v}{\partial\eta^2}=\dfrac{\partial^2v}{\partial y^2},$$
	所以
	$$\frac{\partial v}{\partial\tau}+a^2\left(\frac{\partial^2v}{\partial\xi^2}+\frac{\partial^2v}{\partial\eta^2}\right)=0.$$
\end{proof}
\begin{exercise}[7]
  证明: 如果 $u_1(x,t)$, $u_2(y,t)$ 分别是下述两个定解问题的解:
  \[\begin{cases}
    \displaystyle\frac{\partial u_1}{\partial t} = a^2 \frac{\partial^2u_1}{\partial x^2}, \\
    u_1|_{t=0} = \varphi_1(x);
  \end{cases}
  \qquad
  \begin{cases}
    \displaystyle\frac{\partial u_2}{\partial t} = a^2 \frac{\partial^2u_2}{\partial y^2}, \\
    u_2|_{t=0} = \varphi_2(y).
  \end{cases}\]
  则 $u(x,y,t) = u_1(x,t)u_2(y,t)$ 是定解问题
  \[\begin{cases}
    \displaystyle\frac{\partial u}{\partial t} = 
      a^2\Bigl(\frac{\partial^2u}{\partial x^2} + \frac{\partial^2u}{\partial y^2}\Bigr), \\
    u|_{t=0} = \varphi_1(x) \varphi_2(y)
  \end{cases}\]
  的解.
\end{exercise}

\begin{proof}
  直接验证即可. 因为
  	$$\frac{\partial u}{\partial t}=\frac{\partial u_1}{\partial t}u_2+\frac{\partial u_2}{\partial t}u_1,$$
  	所以
  	$$a^{2}\left(\frac{\partial^{2}u}{\partial t^{2}}+\frac{\partial^{2}u}{\partial y^{2}}\right)=a^{2}\frac{\partial^{2}u_{1}}{\partial t^{2}}u_{2}+a^{2}u_{1}\frac{\partial^{2}u_{2}}{\partial y^{2}}=\frac{\partial u_{1}}{\partial t}u_{2}+u_{1}\frac{\partial u_{2}}{\partial t}=\frac{\partial u}{\partial t}.$$
  	又初始条件满足
  	\[u|_{t=0}=u_{1}|_{t=0}\cdot u_{2}|_{t=0}=\varphi_{1}(x)\varphi_{2}(y).\qedhere\]
\end{proof}


\begin{exercise}[8]
  导出下列热传导方程柯西问题解的表达式:
  \[\begin{cases}
    \displaystyle\frac{\partial u}{\partial t} 
      = a^2\Bigl(\frac{\partial^2u}{\partial x^2} +
                  \frac{\partial^2u}{\partial y^2}\Bigr), \\
   \displaystyle u|_{t=0} = \sum_{i=1}^n \alpha_i(x)\beta_i(y).
  \end{cases}\]
\end{exercise}

\begin{solution}
  由叠加原理与上题结果或直接应用 Fourier 变换可得解为
  \[u(x,y,t) = \frac{1}{4a^2\pi t} \sum_{i=1}^n \int_{-\infty}^{\infty}
    \int_{-\infty}^{\infty} \alpha_i(\xi) \beta_i(\eta)
      \exp\biggl(-\frac{(x-\xi)^2+(y-\eta)^2}{4a^2 t}\biggr) \diff\xi\diff\eta.\qedhere\]
\end{solution}


\begin{exercise}[9]
  验证:二维热传导方程柯西问题
  \[\begin{cases}
    \displaystyle\frac{\partial u}{\partial t} = a^2 
      \Bigl(\frac{\partial^2 u}{\partial x^2} + \frac{\partial^2 u}{\partial y^2}\Bigr), \\
    u|_{t=0} = \varphi(x,y)
  \end{cases}\]
  解的表达式为
  \[u(x,y,t) = \frac{1}{4\pi a^2t} \int_{-\infty}^{\infty}
    \int_{-\infty}^{\infty} \varphi(\xi,\eta) \e^{-\frac{(x-\xi)^2+(y-\eta)^2}{4a^2t}}
    \diff\xi\diff\eta.\]
\end{exercise}

\begin{proof}
  本习题应该添加假设: $\varphi(x,y)$有界, 因为
  \[u(x,y,t) = \frac{1}{4\pi a^2t}\int_{-\infty}^{\infty}
    \int_{-\infty}^{\infty}\varphi(\xi,\eta)
      \e^{-\frac{(x-\xi)^2+(y-\eta)^2}{4a^2t}}\diff\xi\diff\eta,\]
  所以
  \[\begin{split}
    \frac{\partial u}{\partial t}
    ={} & \frac{-1}{4\pi a^2t^2}\int_{-\infty}^{\infty}\int_{-\infty}^{\infty}
      \varphi(\xi,\eta) \e^{-\frac{(x-\xi)^2+(y-\eta)^2}{4a^2t}}\diff\xi\diff\eta \\
        & + \frac{1}{4\pi a^2t}\int_{-\infty}^{\infty}\int_{-\infty}^{\infty}
          \varphi(\xi,\eta) \e^{-\frac{(x-\xi)^2+(y-\eta)^2}{4a^2t}}
          \cdot\frac{(x-\xi)^2+(y-\eta)^2}{4a^2t^2}\diff\xi\diff\eta.
  \end{split}\]
  又
  \[\frac{\partial u}{\partial x} = 
    \frac{1}{4\pi a^2t}\int_{-\infty}^{\infty}\int_{-\infty}^{\infty}
    \varphi(\xi,\eta) \e^{-\frac{(x-\xi)^2+(y-\eta)^2}{4a^2t}}
    \cdot\frac{-(x-\xi)}{2a^2t}\diff\xi\diff\eta.\]
  故
  \[\frac{\partial^2u}{\partial x^2} =
    \frac{1}{4\pi a^2t}\int_{-\infty}^{\infty}\int_{-\infty}^{\infty}
    \varphi(\xi,\eta) \e^{-\frac{(x-\xi)^2+(y-\eta)^2}{4a^2t}}
    \left(\frac{-1}{2a^2t} + \frac{(x-\xi)^2}{4a^4t^2}\right) \diff\xi\diff\eta.\]
  显然 $\displaystyle\frac{\partial^2u}{\partial y^2}$ 的结果形式同 $\displaystyle\frac{\partial^2u}{\partial x^2}$, 故
  \begin{align*}
    & a^2\left(\frac{\partial^2u}{\partial x^2}+\frac{\partial^2u}{\partial y^2}\right) \\
    ={} & \frac{1}{4\pi t}\int_{-\infty}^{\infty}\int_{-\infty}^{\infty}
          \varphi(\xi,\eta) \e^{-\frac{(x-\xi)^2+(y-\eta)^2}{4a^2t}}
          \left(\frac{-1}{a^2t}+\frac{(x-\xi)^2+(y-\eta)^2}{4a^4t^2}\right)\diff\xi\diff\eta.
  \end{align*}
  对比可知
  \[\frac{\partial u}{\partial t}
    = a^2\left(\frac{\partial^2u}{\partial x^2}+\frac{\partial^2u}{\partial y^2}\right).\]
  
  对于初值的检验可对照教材P61的方法,下面不妨简单叙述一下.
  要证明当 $t\to 0$, $x\to x_0$, $y\to y_0$ 时,
  $u(x,y,t)\to\varphi(x_0,y_0)$, 令 $\displaystyle\zeta=\frac{x-\xi}{2a\sqrt{t}}$,
  $\displaystyle\theta=\frac{y-\eta}{2a\sqrt{t}}$,则
  \[u(x,y,t) =
    \frac{1}{\pi}\int_{-\infty}^{\infty}\int_{-\infty}^{\infty}
    \varphi(x-2a\sqrt{t}\zeta,y-2a\sqrt{t}\theta) \e^{-(\zeta^2+\theta^2)}
    \diff\zeta\diff\theta.\]
  而
  \[\varphi(x_0,y_0) =
    \frac{1}{\pi}\int_{-\infty}^{\infty}\int_{-\infty}^{\infty}
    \varphi(x_0,y_0) \e^{-(\zeta^2+\theta^2)} \diff\zeta\diff\theta.\]
  故
  \[u(x,y,t) - \varphi(x_0,y_0) =
    \frac{1}{\pi} \int_{-\infty}^{\infty}\int_{-\infty}^{\infty}
    \left[\varphi(x-2a\sqrt{t}\zeta,y-2a\sqrt{t}\theta)-\varphi(x_0,y_0)\right]
    \e^{-(\zeta^2+\theta^2)}\diff\zeta\diff\theta.\]
  将 $(\zeta,\theta)$ 平面用正方形 (四个顶点为$(\pm N,\pm N)$) 分成两个部分.
  在正方形内部, 利用$\varphi(x,y)$的连续性控制, 在正方形的外部,
  用积分$\displaystyle\int_{-\infty}^{\infty}\int_{-\infty}^{\infty}e^{-(\zeta^2+\theta^2)}\diff\zeta\diff\theta$ 可以任意小以及$\varphi(x,y)$是有界的来进行控制即可证明.
\end{proof}


\section{极值原理,定解问题解的唯一性和稳定性}

\begin{exercise}
  证明方程 $\displaystyle\frac{\partial u}{\partial t} = a^2 \frac{\partial^2 u}{\partial x^2} + cu (c\geq 0)$ 具有狄利克雷边界条件的初边值问题解的唯一性和稳定性.
\end{exercise}

\begin{proof}
  设$u(x,t)$满足的定解问题为
  \[\begin{cases}
    u_t = a^2 u_{xx} + cu, \\
    u(x,0) = \varphi(x), \\
    u(\alpha,t) = \mu_1(t),\,u(\beta,t)=\mu_2(t).
  \end{cases}\]
  则令 $v(x,t)=u(x,t) \e^{-ct}$, 可得 $v(x,t)$ 满足的定解问题为
  \[\begin{cases}
    v_t = a^2v_{xx}, \\
    v(x,0) = \varphi(x), \\
    v(\alpha,t) = \mu_1(t)\e^{-ct},\,v(\beta,t) = \mu_2(t)\e^{-ct}.
  \end{cases}\]
  由定理4.2知上述定解问题的解是唯一的且稳定的,
  记为 $v=v_0(x,t)$, 则原定解问题的解为$u = u_0(x,t) = \e^{ct}v_0(x,t)$, 显然也是唯一的且稳定的.
\end{proof}


\begin{exercise}
  利用热传导方程极值原理的方法, 证明:满足方程$\displaystyle \frac{\partial^2 u}{\partial x^2}+\frac{\partial^2 u}{\partial y^2}=0$的函数在有界区域上的最大值不会超过
  它在边界上的最大值.
\end{exercise}

\begin{proof}
  记有界闭区域为 $\varOmega$, 其边界为 $\varGamma$.
  设 $u(x,y)$ 在 $\varOmega$ 上的最大值为 $M$, 在 $\varGamma$ 上的最大值为 $m$.
  假设在区域内部存在某点 $(x^*,y^*)$ 使得
  \[u(x^*,y^*) = M > m.\]
  作辅助函数
  \[V(x,y) = u(x,y)+\frac{M-m}{4l^2}\left[(x-x^*)^2+(y-y^*)^2\right],\]
  其中 $l$ 为 $\varOmega$ 的直径. 由于在 $\varGamma$ 上
  \[V(x,y) < m+\frac{M-m}{4}=\frac M4+\frac 34m=\theta M\quad (0<\theta<1),\]
  而
  \[V(x^*,y^*)=u(x^*,y^*)=M,\]
  因此, 函数 $V(x,y)$ 与$u(x,y)$ 一样, 不在$\varGamma$ 上取到最大值. 设 $V(x,y)$ 在区域 $\varOmega$ 内部某点 $(x_1,y_1)$ 取得最大值, 在这个点应有 $\displaystyle\frac{\partial^2 V}{\partial x^2}\leq0, \frac{\partial^2 V}{\partial y^2}\leq0$, 因此在点$(x^*,y^*)$处
  \[\frac{\partial^2 V}{\partial x^2}+\frac{\partial^2 V}{\partial y^2}\leq0;\]
  但是
  \[\frac{\partial^2 V}{\partial x^2}+\frac{\partial^2 V}{\partial y^2}=\frac{\partial^2 u}{\partial x^2}+\frac{M-m}{2l^2}+\frac{\partial^2 u}{\partial y^2}+\frac{M-m}{2l^2}=\frac{M-m}{l^2}>0,\]
  这就得到矛盾, 故成立 $M=m$.
\end{proof}


\begin{exercise}
  导出初边值问题
  \[\begin{cases}
    u_t - a^2 u_{xx} = f(x,t), \\
    u|_{x=0} = \mu_1(t),\quad
      \displaystyle\left(\frac{\partial u}{\partial x} + hu\right)\bigg|_{x=l} = \mu_2(t)\quad (h>0), \\
    u|_{t=0} = \varphi(x)
  \end{cases}\]
  的解 $u(x,t)$ 在 $R_T=\{0\leq t\leq T, 0\leq x\leq l\}$ 中满足估计
  \[u(x,t) \leq \e^{\lambda T} \max\biggl\{
    0, \max_{0\leq x\leq l} \varphi(x),
    \max_{0\leq t\leq T} \biggl(\e^{-\lambda t}\mu_1(t),
      \frac{\e^{-\lambda t}\mu_2(t)}{h}\biggr),
      \frac{1}{\lambda} \max_{R_T} (\e^{-\lambda t}f)
  \biggr\},\]
  其中 $\lambda>0$ 为任意正常数.
\end{exercise}

\begin{proof}
  Let $v = \e^{-\lambda t}u$, then $v$ satisfies
  \[\begin{cases}
    v_t-a^2v_{xx}+\lambda v = \e^{-\lambda t}f(x,t), \\
    v(x,0) = \varphi(x), \\
    v(0,t) = \e^{-\lambda t}\mu_1(t),\ (v_x+hv)|_{x=l} = \e^{-\lambda t}\mu_2(t).
  \end{cases}\]
  It suffices to consider the case when $u$ takes its positive maximum value at some point $(x_0,t_0)$.
  Suppose that $(x_0,t_0)$ is not on the parabolic boundary, then
  $v_t(x_0,t_0)\geq 0$, $v_{xx}(x_0,t_0)\leq 0$ and $v(x_0,t_0)>0$. So
  \begin{equation}\label{eq:e1}
    (v_t - a^2v_{xx} + \lambda v)\big|_{(x_0,t_0)} > 0.
  \end{equation}
  By choosing $\lambda$ large enough, we can let
  \begin{equation}\label{eq:e2}
    \e^{-\lambda t_0} f(x_0,t_0) \to 0.
  \end{equation}
  Thus $v_t-a^2v_{xx}+\lambda v = \e^{-\lambda t}f(x,t)$ could not hold at $(x_0,t_0)$
  and so $(x_0,t_0)$ is on the parabolic boundary.
  \begin{enumerate}[(i)]
    \item If $(x_0,t_0)\in \{x=0\}$, then
      \begin{equation}\label{eq:e3}
        v(x,t) \leq \max_{0\leq t\leq T} \e^{-\lambda t} \mu_1(t).
      \end{equation}
    \item If $(x_0,t_0)\in \{t=0\}$, then
      \begin{equation}\label{eq:e4}
        v(x,t) \leq \max_{0\leq x\leq l} \varphi(x).
      \end{equation}
    \item If $(x_0,t_0)\in \{x=l\}$, then
      \begin{equation}\label{eq:e5}
        v(x,t) \leq \max_{0\leq t\leq T} \frac{\e^{-\lambda t}\mu_2(t)}{h}.
      \end{equation}
  \end{enumerate}
  On the other hand, since $v_t(x_0,t_0)-a^2v_{xx}(x_0,t_0) \geq 0$, we have that
  \begin{equation}\label{eq:e6}
    \lambda v(x_0,t_0) \leq \e^{-\lambda t_0} f(x_0,t_0) \leq \max_{R_T} (\e^{-\lambda t}f).
  \end{equation}
  Combining \eqref{eq:e3}, \eqref{eq:e4}, \eqref{eq:e5} and \eqref{eq:e6} we conclude that
  \begin{equation}\label{eq:e7}
    v(x,t) \leq \max\biggl(0, \max_{0\leq x\leq l} \varphi(x),
      \max_{0\leq t\leq T} \biggl(\e^{-\lambda t}\mu_1(t), \frac{\e^{-\lambda t}\mu_2(t)}{h}\biggr),
      \frac{1}{\lambda} \max_{R_T} (\e^{-\lambda t}f)\biggr).
  \end{equation}
  The final estimate for $u(x,t)$ follows directly from \eqref{eq:e7}.
\end{proof}


\begin{exercise}
	证明无界区域上热传导方程的极值原理:设$u(x,t)$在带形区域$\{(x,t)|x\in\mathbb{R},0\leq t\leq T\}$上连续有界
	当$0<t<T$时满足热传导方程$u_t-a^2u_{xx}=0$,则
	$$\sup_{0\leq t\leq T,x\in \mathbb{R}}u(x,t)=\sup_{x\in \mathbb{R}}u(x,0),$$
	$$\inf_{0\leq t\leq T,x\in \mathbb{R}}u(x,t)=\inf_{x\in \mathbb{R}}u(x,0).$$
\end{exercise}

\begin{proof}
	记$\sup_x\in Ru(x,0)=A,|u(x,t)|\leq2B$
	
	构造函数$v(x,t)=\frac{4B-2A}{L^{2}}[\frac{(x-x_{0})^{2}}{2}+a^{2}t]+A$, 则$v_t=a^2v_{xy}$
	
	考虑区域 $R_0=\{(x,t)|\overset{L}{\operatorname*{\operatorname*{0}}}\leq t\leq t_{0},\tilde{|}x-x_{0}|\leq L\}$,则
	$$v(x,0)=\frac{4B-2A}{L^2}[\frac{(x-x_0)^2}{2}]+A\geq A$$
	$$v(x_0\pm L,t)=\frac{4B-2A}{L^2}\left[\frac{L^2}{2}+a^2t\right]+A\geq2B\geq u(x_0\pm L,t)$$
	由极值原理,$v(x,t)\geq u(x,t)$在$R_0$上恒成立.
	
	令 $t\to \infty$, 得 $u( x_{0}, y_{0}) \leq \lim _{L\to \infty }v( x_{0}, y_{0}) = A$.
	故
	$$\sup_{0\leq t\leq T,x\in\mathbb{R}}u(x,t)=\sup_{x\in\mathbb{R}}u(x,0).$$
	同理,
	\[\inf_{0\leq t\leq T,x\in\mathbb{R}}u(x,t)=\inf_{x\in\mathbb{R}}u(x,0).\qedhere\]
\end{proof}


\begin{exercise}[6]
	设$u(x,t)$是$\{0\leq x\leq l,0\leq t\leq T\}$中边值问题
	$$\begin{cases}&u_t=u_{xx}+f(x)\\&u|_{x=0}=u|_{x=l}=0\\&u|_{t=0}=0\end{cases}$$
	的经典解,其中$f(x)\leq0$在$0\leq x\leq l$上成立. 试证明: 对任意的$x_0\in(0,l)$, 函数$u(x_0,t)$关于 $t$ 是非增的.
\end{exercise}

\begin{proof}
	[教材 86 页注]
	由定理4.1的证明可见, 若 $u$ 是非齐次热传导方程$u_t-u_{xx}=f$的解, 且$f\leq0$, 则仍成立$\displaystyle\max_{R_T}u=\max_{\varGamma_T}u$.

	由极值原理, 上述问题的解的最大值只在抛物边界处取到,即为 $0$.
	
	任意的$x_0\in(0,l), t_0\in(0,T), u(x_0,t_0)<0=u(x_0,0)$.
	
	则显然$u(x_0,t)$ 关于 $t$ 是非增的.
\end{proof}
\newpage
\section{解的渐近性态}

\begin{exercise}
  证明方程
  \[\begin{cases}
    u_t - a^2 u_{xx} = 0, \\
    u|_{x=0} = u|_{x=l} = 0, \\
    u|_{t=0} = \varphi(x)
  \end{cases}\]
  的解当 $t\to +\infty$ 时指数地衰减于零, 其中 $\varphi\in C^2$,
  且 $\varphi(0) = \varphi(l) = 0$.
\end{exercise}

\begin{proof}
  运用分离变量法求得定解问题的解为
  \[u(x,t) = \sum_{k=1}^{\infty} A_k \e^{-\frac{k^2\pi^2a^2}{l^2}t}\sin\frac{k\pi}{l}x.\]
  其中 $\displaystyle A_k=\frac{2}{l} \int_0^l \varphi(\xi) \sin\frac{k\pi}{l}\xi\diff\xi$,
  由 $\varphi$ 有界知 $\exists C_1>0$, 使得 $|A_k|\leq C_1$, 故当 $t>1$ 时,
  \[\begin{split}
  |u(x,t)|
  & \leq C_1\sum_{k=1}^{\infty} \e^{-\frac{k^2\pi^2a^2}{l^2}t} \\
  & = C_1\biggl(1+\sum_{k=2}^{\infty}e^{-\frac{(k^2-1)\pi^2a^2}{l^2}t}\biggr)
      \e^{-\frac{\pi^2a^2}{l^2}t} \\
  & \leq C_1\biggl(1+\sum_{k=2}^{\infty}e^{-\frac{(k^2-1)\pi^2a^2}{l^2}}\biggr)
      \e^{-\frac{\pi^2a^2}{l^2}t} \\
  & < C\e^{-\frac{\pi^2a^2}{l^2}t}.
  \end{split}\]
  因此解当 $t\to+\infty$ 时指数地衰减于零.
\end{proof}


\begin{exercise}
  证明: 当 $\varphi(x,y)$ 为 $\mathbb{R}^2$ 上的有界连续函数, 且 $\varphi\in L^1(\mathbb{R}^2)$
  时, 二维热传导方程柯西问题的解, 当 $t\to +\infty$ 时, 以 $t^{-1}$ 衰减率趋于零.
\end{exercise}

\begin{proof}
  \[\begin{split}
    |u(x,y,t)|
    & = \left|\frac{1}{4\pi a^2t} \iint_{\mathbb{R}^2} \varphi(\xi,\eta)
        \e^{-\frac{(x-\xi)^2+(y-\eta)^2}{4a^2t}}\diff\xi\diff\eta\right| \\
    & \leq \frac{1}{4\pi a^2t} \iint_{\mathbb{R}^2} |\varphi(x,y)|
        \e^{-\frac{(x-\xi)^2+(y-\eta)^2}{4a^2t}}\diff\xi\diff\eta \\
    & \leq Ct^{-1},
  \end{split}\]
  其中 $C$ 是仅与 $a$ 和 $\|\varphi\|_{L^1(\mathbb{R}^2)}$ 有关的正常数.
\end{proof}